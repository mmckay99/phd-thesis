% Replace "oneside" with "twoside" to set the gutter correctly for two-sided printing.
% Add "nogutter" option for digital copy (without binding offsets).
% If printed copy is twoside, use [twoside,nogutter] for digital copy.
\documentclass[oneside,nogutter]{style/glasgowthesis}
% \documentclass[twoside]{style/glasgowthesis}
\makeatletter
\newcommand*{\DetectThesisClassDigitalMode}{%
    \IfSubStr{@classoptionslist}{nogutter=true}{true}{false}%
}
\makeatother

% incredible fix by Joseph Wright https://tex.stackexchange.com/a/447004 which corrects an issue I think is caused by subscriptcorrection option to newtxmath in combination with labels with underscores (e.g. bibliography entries)
\makeatletter
\edef\savedcodes{\catcode`\noexpand\_=\the\catcode`\_}
\edef\@tempa{\csname opt@newtxmath.sty\endcsname}
\def\@tempb{{subscriptcorrection}}
\expandafter\expandafter\expandafter
  \in@\expandafter\@tempb\expandafter{\@tempa}
\ifin@
  \catcode`\_=12 %
\fi
\makeatother

% this code can help with labels changed? lost source
% \makeatletter
% \def\@testdef #1#2#3{%
%   \def\reserved@a{#3}\expandafter \ifx \csname #1@#2\endcsname
%  \reserved@a  \else
% \typeout{^^Jlabel #2 changed:^^J%
% \meaning\reserved@a^^J%
% \expandafter\meaning\csname #1@#2\endcsname^^J}%
% \@tempswatrue \fi}

% Some suggestions of Andrew T. Young from https://web.archive.org/web/20220714055623/https://aty.sdsu.edu/bibliog/latex/LaTeXtoPDF.html - they seem good to me!
\renewcommand{\topfraction}{0.9} % max fraction of floats at top
\renewcommand{\bottomfraction}{0.8}	% max fraction of floats at bottom
%   Parameters for TEXT pages (not float pages):
\setcounter{topnumber}{2}
\setcounter{bottomnumber}{2}
\setcounter{totalnumber}{4} % 2 may work better
\setcounter{dbltopnumber}{2} % for 2-column pages
\renewcommand{\dbltopfraction}{0.9}	% fit big float above 2-col. text
\renewcommand{\textfraction}{0.07} % allow minimal text w. figs
% Parameters for FLOAT pages (not text pages):
\renewcommand{\floatpagefraction}{0.7} % require fuller float pages
% N.B.: floatpagefraction MUST be less than topfraction !!
\renewcommand{\dblfloatpagefraction}{0.7} % require fuller float pages

% \widowpenalty10000
% \clubpenalty10000

% this makes the list of algorithms look like the list of figures
% https://tex.stackexchange.com/a/374686
% \usepackage{etoolbox}
\newlength\insanelyhugespacebetweenchapters
\newlength\insanelyhugespacebetweenalgorithms
\setlength{\insanelyhugespacebetweenchapters}{15pt}
\setlength{\insanelyhugespacebetweenalgorithms}{5pt}
\makeatletter
\patchcmd{\@chapter}% <cmd>
  {\chaptermark{#1}}% <search>
  {\chaptermark{#1}%
  \addtocontents{loa}{\protect\addvspace{\insanelyhugespacebetweenchapters}}}% replace
  {}{}% <success><failure>
\makeatother
\AtEndEnvironment{algorithm}{%
  \addtocontents{loa}{\protect\addvspace{\insanelyhugespacebetweenalgorithms}}%
}

\usepackage[pagebackref,
            pdftex,
            pdfusetitle,
            linktoc=all]{hyperref}
\renewcommand*{\backref}[1]{}
\renewcommand*{\backrefalt}[4]{
    \ifcase #1 (not cited)%
          \or (cited on page~#2)%
          \else (cited on pages~#2)%
    \fi%
    }

\usepackage[noend]{algpseudocode}
\usepackage[chapter]{algorithm}
\usepackage{complexity}
\usepackage{graphics}
\usepackage{graphicx}
\usepackage{url}
\usepackage{mcaption}
\usepackage[font=scriptsize]{subfig}
\usepackage{bookmark}
\usepackage{enumitem}
\usepackage{color}
\usepackage{changebar}
\usepackage{soul}
\usepackage{afterpage}
\usepackage[most]{tcolorbox}
\usepackage{xcolor,colortbl,bm,graphicx,adjustbox}
\usepackage{thmtools}
\usepackage{thm-restate}
\usepackage{xspace}
\usepackage{listofitems}
\usepackage{tabularx}
\usepackage{multirow}
\usepackage{bigdelim}
\usepackage{booktabs}
\usepackage{relsize}
\usepackage{mathtools}
\usepackage{readarray}
\usepackage{longtable}

% \setlength{\headheight}{28pt}

\declaretheorem[name=Theorem, numberwithin=chapter]{thm}
\declaretheorem[name=Lemma, numberwithin=chapter]{lem}
\declaretheorem[name=Proposition, numberwithin=chapter]{prop}
\declaretheorem[style=remark, numbered=no, name=Proof sketch]{proofsketch}
\declaretheorem[style=remark, numberwithin=chapter, name=Definition]{definition}
\declaretheorem[name=Model, numberwithin=chapter]{model}
\declaretheorem[name=Problem, numberwithin=chapter]{pr}
\declaretheorem[name=Observation, numberwithin=chapter]{observation}
\declaretheorem[name=Corollary, numberwithin=chapter]{cor}

% used for my own symbol solution, adapted from https://tex.stackexchange.com/a/350947
\newcommand{\mysymbolfirstusedefinition}[2]{%
    \phantomsection%
    \label{#1}%
    \hypertarget{#1}{#2}%
}
\newcommand{\mysymbolpageref}[1]{%
    \hyperlink{#1}{\pageref{#1}}
}

\declaretheoremstyle[
notebraces={}{},
headpunct={},
headformat={\NAME\ \NUMBER.~\NOTE},
headfont=\bfseries, 
postheadhook = {\hspace*{0pt}\vspace*{-3pt}},
% postheadspace = \newline, 
spaceabove = 0.5cm, 
spacebelow = 0.5cm]{myproblemstyle}
\declaretheorem[style=myproblemstyle, numberwithin=chapter, name=Problem]{myproblem}

\declaretheoremstyle[
notebraces={}{},
headpunct={},
headformat={\NAME\ \NUMBER.~\NOTE},
headfont=\itshape, 
postheadhook = {\hspace*{0pt}\vspace*{-3pt}},
% postheadspace = \newline, 
spaceabove = 0.5cm, 
spacebelow = 0.5cm]{mydefinitionstyle}
\declaretheorem[style=mydefinitionstyle, numberwithin=chapter, name=Definition]{mydefinition}

\declaretheoremstyle[
notebraces={}{},
headpunct={},
headformat={\NAME\ \NUMBER~{\normalfont(\cite{HedonicGamesHOCSC})}.~\NOTE},
headfont=\itshape, 
postheadhook = {\hspace*{0pt}\vspace*{-3pt}},
% postheadspace = \newline, 
spaceabove = 0.5cm, 
spacebelow = 0.5cm]{mydefinitionofhedonicgamestyle}
\declaretheorem[style=mydefinitionofhedonicgamestyle, numberwithin=chapter, numberlike=mydefinition, name=Definition]{mydefinitionofhedonicgame}

% tells relsize package what the smallest acceptable font size is
\renewcommand\RSsmallest{5.2pt}
\newcommand{\algorithmfont}[1]{\textsf{\relscale{0.9}#1}}

\algdef{S}[WHILE]{WhileNoDo}[1]{\algorithmicwhile\ #1}%

\newcommand{\myoplus}{\oplus}

\newcommand{\inp}{\textit{Input: }} 
\newcommand{\outp}{\textit{Output: }}
\newcommand{\ques}{\textit{Question: }}
\newcommand{\instance}{\textit{Instance: }}
\newcommand{\solution}{\textit{Solution: }}
\newcommand{\measure}{\textit{Measure: }}

\newcommand{\vdkr}{\texorpdfstring{VDK\textsubscript{$r$}}{VDKr}\xspace}
\newcommand{\edkr}{\texorpdfstring{EDK\textsubscript{$r$}}{VDKr}\xspace}
\newcommand{\vdktwo}{\texorpdfstring{VDK\textsubscript{$2$}}{VDK2}\xspace}
\newcommand{\edktwo}{\texorpdfstring{EDK\textsubscript{$2$}}{EDK2}\xspace}
\newcommand{\vdkthree}{\texorpdfstring{VDK\textsubscript{$3$}}{VDK3}\xspace}
\newcommand{\edkthree}{\texorpdfstring{EDK\textsubscript{$3$}}{EDK3}\xspace}
\newcommand{\vdkfour}{\texorpdfstring{VDK\textsubscript{$4$}}{VDK4}\xspace}
\newcommand{\edkfour}{\texorpdfstring{EDK\textsubscript{$4$}}{EDK4}\xspace}
\newcommand{\vdkfive}{\texorpdfstring{VDK\textsubscript{$5$}}{VDK5}\xspace}
\newcommand{\edkfive}{\texorpdfstring{EDK\textsubscript{$5$}}{EDK5}\xspace}

\def\mistfvariant/{MIS-3-TF}
\def\maxtwosatthreeshort/{M2S3}
\def\maxtwosatthree/{Max $\text{2SAT}^{\leq 3}$}
\def\porschenxsatvariant/{$\text{X3SAT}_{+}^{\,=3}$}

\def\iwjnomaxdegreetwofamily/{$\mathcal{I}^\star$}

\def\glossarychaptertitle/{Glossary of abbreviations}

% partially copied from the source of tocloft so the glossary matches the toc
\makeatletter
\newcommand \myglossarydotfill {\leavevmode \cleaders \hbox{$\m@th\mkern 4.5mu\hbox{.}\mkern 4.5mu$}\hfill}
\makeatother

% \hbox{$\m@th\mkern #1 mu\hbox{\cftdot}\mkern #1 mu$}\hfill

% makes the vertical spacing between fracs nicer
% https://tex.stackexchange.com/a/55180
\usepackage{mathtools}
\makeatletter
\newlength\minalignvsep
\def\align@preamble{%
  &\hfil
    \setboxz@h{\@lign$\m@th\displaystyle{##}$}%
    \ifnum\row@>\@ne
    \ifdim\ht\z@>\ht\strutbox@
    \dimen@\ht\z@
    \advance\dimen@\minalignvsep
    \ht\strutbox\dimen@
    \fi\fi
    \strut@
    \ifmeasuring@\savefieldlength@\fi
    \set@field
    \tabskip\z@skip
  &\setboxz@h{\@lign$\m@th\displaystyle{{}##}$}%
    \ifnum\row@>\@ne
    \ifdim\ht\z@>\ht\strutbox@
    \dimen@\ht\z@
    \advance\dimen@\minalignvsep
    \ht\strutbox@\dimen@
    \fi\fi
    \strut@
    \ifmeasuring@\savefieldlength@\fi
    \set@field
    \hfil
    \tabskip\alignsep@
}
\makeatother
\minalignvsep.3em

% define mystrut and undermat
% https://tex.stackexchange.com/q/374985
\newcommand*\mystrut[1]{\vrule width0pt height0pt depth#1\relax}
\newcommand\undermat[2]{%	
  \makebox[0pt][l]{$\smash{\underbrace{\mystrut{1em}\phantom{%	
    \begin{matrix}#2\end{matrix}}}_{\text{$#1$}}}$}#2}
    
% allow math mode commas to line break
% https://tex.stackexchange.com/a/19100
\AtBeginDocument{%
  \mathchardef\mathcomma\mathcode`\,
  \mathcode`\,="8000 
}
{\catcode`,=\active
  \gdef,{\mathcomma\discretionary{}{}{}}
}

% made this myself to allow for indented multiline assignments in algorithms
% based on various stackoverflow answers
\usepackage{silence}
\WarningFilter{etex}{Extended allocation}
\usepackage{linegoal}
\makeatletter
\algnewcommand{\myhackyalgorithmbox}[1]{ \begin{tabularx}{\linegoal}[t]{@{}X@{}} #1\end{tabularx}}
\makeatother

% line comments in pseudocode
\makeatletter
\algnewcommand{\LineComment}[1]{\State \(\triangleright\) \myhackyalgorithmbox{#1}
}
\makeatother

\algrenewcommand\algorithmicindent{1.0em}

% using this in algorithms to control multiline blocks of text, see https://tex.stackexchange.com/a/432874 and its usage in the algorithms
\newcommand{\myalgorithmparboxwidth}{\textwidth-\algorithmicindent-\widthof{$\mathcal{Q} \gets$}}

% made this myself
\newcommand{\paranote}[2] {
\begin{tcolorbox}[enhanced, parbox=false, colback=#1, drop shadow]
    \begin{minipage}{\textwidth}
        {\sffamily Note: #2}
    \end{minipage}
\end{tcolorbox}
\medskip
}

% make it so \paragraphs are numbered and get a newline after
\usepackage{titlesec}
\titleformat{\paragraph}
{\normalfont\normalsize\bfseries}{\theparagraph}{1em}{}
\titlespacing*{\paragraph}
{0pt}{3.25ex plus 1ex minus .2ex}{1.5ex plus .2ex}
\setcounter{secnumdepth}{5}

% define "struts", as suggested by Claudio Beccari in
% a piece in TeX and TUG News, Vol. 2, 1993.
% https://tex.stackexchange.com/questions/65127/extra-vertical-space-after-hline-causes-a-gap-in-the-right-border-of-an-array
\newcommand\Tstrut{\rule{0pt}{2.6ex}}         % = `top' strut
\newcommand\Bstrut{\rule[-0.9ex]{0pt}{0pt}}   % = `bottom' strut

\usepackage{tikz}
\usetikzlibrary{arrows,decorations.markings,decorations.pathreplacing,patterns,matrix,calc,positioning,backgrounds,arrows.meta,shapes,decorations.markings,fadings,fpu}

\newcommand\solutionconceptsdiagramarrow{-Straight Barb[length=1.4mm]}

\tikzset{edge from parent/.append style={\solutionconceptsdiagramarrow}}

% https://tex.stackexchange.com/a/529159/194703, this helps avoid a particular 'dimension too large' error
\makeatletter
\tikzset{use fpu reciprocal/.code={%
\def\pgfmathreciprocal@##1{%
    \begingroup
    \pgfkeys{/pgf/fpu=true,/pgf/fpu/output format=fixed}%
    \pgfmathparse{1/##1}%
    \pgfmath@smuggleone\pgfmathresult
    \endgroup
}}}%
\makeatother

\definecolor{enclosure_color}{rgb}{0.6,0.6.,0.6}
\definecolor{figurecolourschemewt1_adjusted}{rgb}{0.8,0.8,0.8}
\definecolor{figurecolourschemewt1}{rgb}{0.5,0.5,0.5}
\definecolor{figurecolourschemewt2}{rgb}{0.5,0.0,0.5}
\definecolor{figurecolourschemewt3}{rgb}{0,0,0}
\definecolor{figurecolourschemewt4}{rgb}{0,0,1}
\definecolor{figurecolourschemewt5}{rgb}{0,1,0}
\definecolor{figurecolourschemewt6}{rgb}{1,0,0}

\newcommand{\figurecolorschemewtonename}{grey}
\newcommand{\figurecolorschemewttwoname}{purple}
\newcommand{\figurecolorschemewtthreename}{black}
\newcommand{\figurecolorschemewtfourname}{blue}
\newcommand{\figurecolorschemewtfivename}{green}
\newcommand{\figurecolorschemewtsixname}{red}

\newcommand{\myfirstchoicearrow}{%
    % filled arrow
    % \begin{scope}[yscale=0.7, xscale=1.1]
    %     \filldraw (0.0,0.0) -- (-0.125cm, 0.15cm) -- (-0.125cm, -0.15cm) -- (0.0, 0.0);
    % \end{scope}
    \draw (0.0,0.0) -- (-0.125cm, 0.15cm);
    \draw (0.0, 0.0) -- (-0.125cm, -0.15cm);
}

\newcommand{\myvaluationarrow}{%
    \draw (0.0,0.0) -- (-0.125cm, 0.15cm);
    \draw (0.0, 0.0) -- (-0.125cm, -0.15cm);
}

\newcommand{\myvaluationarrowreversed}{%
\begin{scope}[xscale=-1.0]
\myvaluationarrow
\end{scope}
}

\tikzset{firstchoicearrow/.style={decoration={
  markings,
  mark=at position .5 with {\myfirstchoicearrow}
  }
  ,postaction={decorate}}}

\tikzset{farrow/.style={decoration={
  markings,
  mark=at position 1.3cm with {\myvaluationarrow},
  }
  ,postaction={decorate}}}

\tikzset{darrow/.style={decoration={
  markings,
  mark=at position .2 with {\myvaluationarrowreversed},
  mark=at position .8 with {\myvaluationarrow},
  }
  ,postaction={decorate}}}

\tikzset{darrow12/.style={decoration={
  markings,
  mark=at position {0.2*\pgfdecoratedpathlength} with {\myvaluationarrowreversed},
  mark=at position {0.8*\pgfdecoratedpathlength-0.4mm} with {\myvaluationarrow},
  mark=at position {0.8*\pgfdecoratedpathlength+0.4mm} with {\myvaluationarrow},
  }
  ,postaction={decorate}}}
  
\tikzset{darrow22/.style={decoration={
  markings,
  mark=at position {0.2*\pgfdecoratedpathlength-0.4mm} with {\myvaluationarrowreversed},
  mark=at position {0.2*\pgfdecoratedpathlength+0.4mm} with {\myvaluationarrowreversed},
  mark=at position {0.8*\pgfdecoratedpathlength-0.4mm} with {\myvaluationarrow},
  mark=at position {0.8*\pgfdecoratedpathlength+0.4mm} with {\myvaluationarrow},
%   mark=between positions 0.8 and 1.0 step 3mm with {\arrow{angle 90[width=2.5mm]}},
  }
  ,postaction={decorate}}}

\tikzset{-->-/.style={decoration={
  markings,
  mark=at position .8 with {\myvaluationarrow}},postaction={decorate}}}
  
\tikzset{-<--/.style={decoration={
  markings,
  mark=at position .8 with {\myvaluationarrow}},postaction={decorate}}}
  
\tikzset{->-/.style={decoration={
  markings,
  mark=at position .5 with {\myvaluationarrow} },postaction={decorate}}}

\tikzset{--->/.style={decoration={
  markings,
  mark=at position 1 with {\myvaluationarrow} },postaction={decorate}}}

\tikzset{
  nat/.style     = {fill=white,draw=none,ellipse,minimum size=0.3cm,inner sep=1pt},
}

% this is to fix an issue with tikz angles
% see https://tex.stackexchange.com/a/260833
\makeatletter
\tikzset{
  reset label anchor/.code={%
    \let\tikz@auto@anchor=\pgfutil@empty
    \def\tikz@anchor{#1}
  },
  reset label anchor/.default=center,
  every label/.append style={reset label anchor}
}
\makeatother

% convex hull stuff (used for dashedcontainerthing), note requires \usepackage{readarray} https://tex.stackexchange.com/questions/70999/highlight-a-group-of-nodes-in-a-tikz-tree/71000#71000
\pgfdeclarelayer{background}
\pgfsetlayers{background,main}
\newcommand{\convexpath}[2]{
[   
    create hullnodes/.code={
        \global\edef\namelist{#1}
        \foreach [count=\counter] \nodename in \namelist {
            \global\edef\numberofnodes{\counter}
            \node at (\nodename) [draw=none,name=hullnode\counter] {};
        }
        \node at (hullnode\numberofnodes) [name=hullnode0,draw=none] {};
        \pgfmathtruncatemacro\lastnumber{\numberofnodes+1}
        \node at (hullnode1) [name=hullnode\lastnumber,draw=none] {};
    },
    create hullnodes
]
($(hullnode1)!#2!-90:(hullnode0)$)
\foreach [
    evaluate=\currentnode as \previousnode using \currentnode-1,
    evaluate=\currentnode as \nextnode using \currentnode+1
    ] \currentnode in {1,...,\numberofnodes} {
  let
    \p1 = ($(hullnode\currentnode)!#2!-90:(hullnode\previousnode)$),
    \p2 = ($(hullnode\currentnode)!#2!90:(hullnode\nextnode)$),
    \p3 = ($(\p1) - (hullnode\currentnode)$),
    \n1 = {atan2(\y3,\x3)},
    \p4 = ($(\p2) - (hullnode\currentnode)$),
    \n2 = {atan2(\y4,\x4)},
    \n{delta} = {-Mod(\n1-\n2,360)}
  in 
    {-- (\p1) arc[start angle=\n1, delta angle=\n{delta}, radius=#2] -- (\p2)}
}
-- cycle
}
\newcommand\dashedcontainerthing[1]{
\readlist*\nodes{#1}%
\begin{scope}[on background layer]
    \ifthenelse{\nodeslen>1}{
        \draw[thick, densely dashed] \convexpath{#1}{0.353cm};
        \fill[white] \convexpath{#1}{9.6pt};
    }{
        \node[draw, thick, circle, densely dashed, minimum size=20pt] (anonymousnode) at (\nodes[1]) {};
    }
\end{scope}
}

\newtcolorbox{ipmodelbox}[1][]{ams align,
    colback=white, %yellow!10!white,
    colframe=black, %red!50!black,
    top=4pt,bottom=12pt,left=-32pt,right=18pt,
    boxsep=0pt,
    boxrule=0.5pt,
    arc=0pt,
    outer arc=0pt,
    code={}
}

% Fancy Header and Footer
\usepackage{fancyhdr}                    
\pagestyle{fancy}                       
\fancyfoot{}                            
\renewcommand{\headrulewidth}{0.2pt}

\ifdefined\DetectThesisClassDigitalMode
    \fancyhead[L,RO]{\bfseries\thepage}    
    \fancyhead[LO]{\bfseries\nouppercase{\rightmark}}
\else
    % Page number (boldface): left-even, right-odd
    \fancyhead[LE,RO]{\bfseries\thepage}    
    % Chapter in the right on even pages
    %\fancyhead[RE]{\bfseries\nouppercase{\leftmark}}
    
    % Section in the left on odd pages
    \fancyhead[LO]{\bfseries\nouppercase{\rightmark}}     
\fi

\setlength{\headheight}{15pt}        % adjust for fancyhdr warning
% \pagestyle{fancy}                    % clear all header and footer fields
% \fancyhf{}
% \fancyhead[L]{\slshape \rightmark}   % put section heading left
% \fancyhead[R]{\thepage}              % put page number right

% % redefine "plain" to fix page numbering (on first page of chapters)
\fancypagestyle{plain} { %
   \fancyhf{}                           % clear all header and footer fields
   \fancyhead[L]{\slshape \rightmark}  % no section heading for these pages
   \fancyhead[R]{\bfseries\thepage}
   \renewcommand{\headrulewidth}{0pt}   % no headrule for these pages
   \renewcommand{\footrulewidth}{0pt}
}

% checkmarks
\usepackage{pifont}% http://ctan.org/pkg/pifont
\newcommand{\xmark}{\ding{55}}%
\newcommand{\note}[1]{\emph{\textbf{Note:}[#1]}}

% \usepackage[color=gray!2]{draftwatermark}
% \SetWatermarkText{wien}

% hides the contents of proof environments
% \usepackage{environ}
% \NewEnviron{killcontents}{}
% \let\proof\killcontents
% \let\endproof\endkillcontents

\newcommand{\showacknowledgements}{}
\ifdefined\showacknowledgements%
    \usepackage{fetamont}
    \usepackage[T1]{fontenc}
    \newcommand{\ackfont}[1]{{\itshape\ffmfamily\relscale{1.22} #1}}
\fi

\begin{document}

\pagestyle{empty}
\pagenumbering{gobble}
\setcounter{tocdepth}{2}

\newcommand{\ThesisTitle}{Algorithmic Aspects of Fixed-Size~Coalition Formation}
\newcommand{\AuthorName}{Michael McKay}
% \input{writing_conventions}
\title{\ThesisTitle}
\author{\AuthorName}
\date{\today}

\maketitle
\cleardoublepage
\chapter*{Abstract}

We study algorithmic aspects of models in which a set of agents is to be organised into coalitions of a fixed size. Such models can be viewed as a type of hedonic game, coalition formation game, or multidimensional matching problem. We mostly consider models in which coalitions have size three and are formalisms of \emph{Three-Dimensional Roommates} (3DR). Models of 3DR are characterised by a combination of the system by which agents have preferences over coalitions, and the solution concept (e.g.\ stability). Since the computational problems associated with 3DR are typically hard, we consider approximate solutions and restricted cases, with the aim of characterising the boundary between tractable and intractable variants.

Part of our contribution relates to two new models of 3DR, which involve two existing systems of preferences called $\mathscr{B}$- and $\mathscr{W}$-preferences. In each model, we consider the existence of matchings that are stable. We show that the related decision problems are $\NP$\nobreakdash-complete and devise approximation algorithms for corresponding optimisation problems. 

In a model of 3DR with additively separable preferences, we investigate stable matchings and envy-free matchings, for three successively weaker definitions of envy-freeness. We consider restrictions on the agents' preferences including symmetric, binary, and ternary valuations. We identify dichotomies based on these restrictions and provide a comprehensive complexity classification. Interestingly, we identify a general trend that, for successively weaker solution concepts, existence and polynomial-time solvability hold under successively weaker preference restrictions.

We also consider a variant of 3DR known as \emph{Three-Dimensional Stable Matching with Cyclic preferences} (3-DSM-CYC), which has been of independent interest. It was recently shown that finding a stable matching is $\NP$-hard, so we consider a related optimisation problem and present an approximation algorithm based on serial dictatorship. We also consider a situation in which the preferences of some agents are sufficiently similar to some master list, and show that the approximation ratio of this algorithm can be improved in relation to a specific similarity measure.

Finally, we consider a problem in graph theory that generalises the notion of assigning agents to coalitions of a fixed size. Rather than organising a set of agents, the problem is to find a maximum-cardinality set of $r$-cliques in an undirected graph subject to that set being either vertex disjoint or edge disjoint, for a fixed integer $r \geq 3$. This general problem is known as the \emph{$K_r$-packing problem}. Here we study the restriction of this problem in which the graph has a fixed maximum degree $\Delta$. It is known for $r=3$ that the vertex-disjoint (edge-disjoint) variant is solvable in linear time if $\Delta=3$ ($\Delta=4$) but $\APX$-hard if $\Delta \geq 4$ ($\Delta \geq 5$). We generalise these results to an arbitrary but fixed $r \geq 3$, and provide a full complexity classification for both the vertex- and edge-disjoint variants in graphs of maximum degree $\Delta$, for all $r \geq 3$.
\newpage
\renewcommand{\listalgorithmname}{List of Pseudocode}
\tableofcontents
% \listoftables
% \listoftheorems
\listoffigures
\listofalgorithms
\newpage
\chapter*{Acknowledgements}
\ifdefined\showacknowledgements

In the course of my studies I have been supported by more people than can be listed.

David Manlove has been an outstanding supervisor. I have benefited enormously from his superlative knowledge of algorithmics and his meticulous (latterly virtual) blue pen. Most importantly however, he has been a constant source of enthusiasm and encouragement.

Many thanks to my two examiners, Martin Gairing and Ciaran McCreesh, for taking the time to read my thesis in full and for their helpful and constructive feedback.

I am very grateful to the EPSRC for funding my degree, and to the Graduate School and the School of Computing Science for providing other funding, training, and administrative support.

I am thankful to my collaborator \'Agnes Cseh for working with me. Our meetings and correspondence have been both enjoyable and hugely valuable. Thanks also to my office-mate Peace Ayegba, who kindly read through my thesis draft and made suggestions to improve its style.

Thanks to each and every one of my other office-mates, fellow PhD students, colleagues, and other friends in the School of Computing. I hope your time at Glasgow will be (or has been!) as fun as mine.

Thank you also to my family and friends outside work, for all their encouragement and moral support. 

Finally, a special thank you to my partner Ailsa for her unwavering support ---\ackfont{Qapla'!}
\fi
\newpage
\chapter*{Declaration}

I certify that the thesis presented here for examination for a Ph.D.\ degree of the University of Glasgow is my own work, with the following exceptions:
\begin{itemize}
    \item Chapter~\ref{c:three_dsm_cyc} includes joint work with Klaus Heeger. In particular, it was Klaus' idea that led to a significant improvement of the approximation ratio (in Theorem~\ref{thm:three_dsm_cyc_unrestricted}) and a collaborative brainstorm that produced the approximation algorithm in the setting of master lists (shown in Theorem~\ref{thm:three_dsm_cyc_ml}).
    \item The instance described in Theorem~\ref{thm:threed_sr_b_ifnis15thensmdoesnotexist}, presented in Chapter~\ref{c:threed_sr_b}, is the product of a joint effort with Sofia Simola. 
    \item Chapter~\ref{c:threed_efr_as} includes joint work with \'Agnes Cseh, conducted part-time from December 2020 to May 2022. \'Agnes provided expert guidance and advice throughout the project. In terms of our new results, it was her ideas that led to the polynomial-time algorithm and $\NP$-completeness reduction related to envy (now shown in Theorem~\ref{thm:threed_efr_as_regularenvy_npcomplete}). We worked jointly on the algorithm and reduction for weakly justified envy, which I then formalised and wrote up (in Theorem~\ref{thm:threed_efr_as_wjef_npcomplete}). I developed the $\NP$-completeness reduction for weakly justified envy (shown in Theorem~\ref{thm:threed_efr_as_wjef_algowjpathscycles}) and proved the results relating to justified envy (shown in Theorems~\ref{thm:threed_efr_as_jef_binary_algorithm}--\ref{thm:threed_efr_as_jef_symmetric_6_npcomplete}).  Everything else was a combined effort.
\end{itemize}

My supervisor, David Manlove, has supported this research throughout, providing general guidance and critically reviewing technical material. 

\chapter*{Research output}

The following papers are a result of the research presented in this thesis:
\begin{itemize}
    \item M.\ McKay and D.\ Manlove. The Three-Dimensional Stable Roommates Problem with Additively Separable Preferences. In \emph{Proceedings of SAGT '21: The 14\textsuperscript{th} International Symposium on Algorithmic Game Theory}. LNCS, vol.\ 12885, pp.\ 266--280. Springer (2021). % A full version is available as Technical Report number 2107.04368, arXiv (2021) from \url{https://arxiv.org/abs/2107.04368}.
    This paper contains some of the results presented in Chapter~\ref{c:threed_sr_as}.
    
    \item \'A. Cseh, M.\ McKay, and D.\ Manlove. Envy-freeness in 3D Hedonic Games. Submitted to \emph{Autonomous Agents and Multi-Agent Systems}, 2023. A preprint is available as Technical Report number 2209.07440 (2022) on arXiv, from \url{https://arxiv.org/abs/2209.07440}. This paper contains the results presented in Chapter~\ref{c:threed_efr_as}.
    
    \item M.\ McKay and D.\ Manlove. Packing $K_r$s in bounded degree graphs. Submitted to \emph{Discrete Applied Mathematics}, 2022. A preprint is available as Technical Report number 2209.03684 (2022) on arXiv, from \url{https://arxiv.org/abs/2209.03684}. This paper contains the results presented in Chapter~\ref{c:kr_packing}.
\end{itemize}
\cleardoublepage

% Reset page numbering
\pagestyle{fancy}
\pagenumbering{arabic}
\setcounter{page}{1}

% Number chapter bookmarks
\bookmarksetup{numbered}

\chapter{Introduction}
\label{c:intro}

\section{About}
% In this thesis we study the algorithmic aspects of models in which agents form coalitions, and in particular coalitions of a fixed size.

% Such models can represent practical scenarios, such as a set of autonomous robots who must form work teams, or a class of students who are to be matched into pairs by their teacher. 

% In such scenarios, there may be some criteria by which one possible outcome, for example of robots in work teams or students into pairs, is better than another. It may be that each autonomous robot has a unique capability, so the robots should be programmed to form work teams such that the probability of the overall goal being achieved is maximised. Alternatively, each student may have a preference over possible partners, and the teacher's aim is to ensure as many students are satisfied as possible.

% In the former scenario, a robotics engineer might need to program the robots to form work teams in a way that is mathematically optimal. The solution to this problem is an efficient algorithm, which can be implemented collectively in the software of each robot. In a new environment, the robots can then run the algorithm and form a set of work teams that is optimal.

% If the class is relatively small, the teacher may be able to pair students informally. Even in a large class, it may

% n the latter scenario, 


% be acceptable to pair students up in an informal way.


% , such that at least a majority of the students are satisfied. Such an approach may be 

% In fact, it is possible to model both scenarios in a formal, mathematical way. 


% Alternatively, they might be purely theoretical models, involving abstract agents that form coalitions based on some  

% They appear in a variety of fields, with a variety of names, and often the aspects of academic interest are different. For example, research in game theory studies models in which autonomous agents form coalitions cooperatively in order to achieve a desirable outcome \cite{aumann1964bargaining,Peters2008}. This type of research can provide insights into how humans make decisions in the real world. Other research in economics involves a market of agents that form coalitions in order to divide a common budget \cite{DominikPetersThesis} or trade a commodity \cite{SS74}. This research can also inform solutions for real-life problems of resource allocation and fair division. 

% The contribution of this thesis is to the interdisciplinary area of computational social choice, in which the focus is on the computational aspects of such models \cite{HedonicGamesHOCSC}. For example, we might ask if there exists a practical algorithm that, given a set of agents, can produce an outcome that is somehow optimal. Such an algorithm might be of practical use to decision makers. Alternatively, if we can show that some problem is $\NP$-hard then it is likely that no such algorithm exists. More generally, computational results might also help us understand more about the model itself.

% % The models that we study involve a set of agents that form coalitions. or states that form economic agreements. 

% Specifically, we focus on models in which a set of agents is partitioned into disjoint coalitions of a fixed size. We call such a partition a \emph{matching}. For example, such a model might represent a set of students that are matched into pairs, or a set of cooperative robots that are matched into work teams of size three. An algorithm, or mechanism, that matches students into pairs or robots into work teams in a somehow-optimal way could be useful in practice.





% We assume that each agent has a preference only between the possible coalitions that they might belong to. 






% Such models are called \emph{hedonic} because each agent cares only about their own coalition \cite{HedonicGamesHOCSC}. In fact, we assume that the preferences of the agents are systematic and can be represented formally. For example, it may be that each agent assigns a numerical valuation to every other agent, and prefers one coalition to another if the sum of their valuations of the other agents in the former coalition is higher than in the latter coalition. These valuations could represent a degree of amity between two students or the combined efficiency of two robots when they are assigned to the same work team. We refer to such a system as a \emph{system of preference representation}.

% % As well as a fixed coalition size, most of the problem models we consider in this thesis have two other key attributes. 
% % First, each agent has a preference over the possible coalitions in which they may belong to. Usually in the problem a \emph{system of preference representation} is specified. For example, it may be that agents assign a numerical valuation to each other and compare two coalitions based on the sum of their valuations of the other agents in each coalition.

% Our research focuses on the existence of matchings in which the agents are somehow satisfied. The precise definition of a satisfactory matching is known as the \emph{solution concept}. For example, for some specific set of students we could ask if there exists a matching of the students into pairs of partners in which each student's partner is, to some degree, a friend. Alternatively, we could ask if there exists a matching of the robots into work teams such that the total efficiency of the robot workforce is maximised. More generally, we might ask if there are restricted cases in which a satisfactory matching is bound to exist.

% We might also ask if a satisfactory matching can be found by an efficient algorithm, or consider the problem of verifying that a given matching is satisfactory. The solutions to such problems can often be applied in practice, to assign agents into coalitions \cite{IJangHedonicGamesRobots2018}, or to predict the behaviour of autonomous agents in the real world \cite{LRRSS15}.

% % From the literature, many of these types of problems are very likely to be intractable, even in very restricted cases. Recent research in thus area thus considers approximate solutions, restricted cases, or fixed parameters. 

% % Our main focus is a class of problem models, involving coalitions of size three, that we call \emph{Three-Dimensional Roommates} (3DR). 

% % In this thesis, we consider a number of different models that relate to coalition formation, and for each one consider some related computational problems. 

% % The models of coalition formation that we focus on here have been described as \emph{hedonic coalition formation games} (or \emph{hedonic games} for short) \cite{DG80, BJ02, Haj06}. The term `game' could be misleading, since often no competition is involved or assumptions are made relating to the agents' behaviour. The term `hedonic' means that we assume agents preference between two coalition partitions depends only on the evaluation of their own coalition. 

% % Hedonic games are very closely related to the study of \emph{matching under preferences} \cite{KMR15_DO_NOT_USE_THERE_IS_A_MORE_GENERAL_ONE, AMUP}. This topic also involves models of agents with preferences. In this case agents might have preferences over a fixed set of resources that must be distributed. Some research has considered the problem of constructing an matching of resources to agents respecting the agents' preferences in a particular way. Other research models agents with preferences over other agents. For example, it may be that model involves producing a set of pairs of agents based on the agents' preferences \cite{Irv85}. This particular problem is a form of hedonic game. Matching under preferences links to many practical applications. The seminal work on the topic considered a model of potential students applying to universities \cite{GS62}. Other real-life applications include assigning junior doctors to hospitals or children to schools \cite{ZZZ18}.

% % The fields of hedonic games and matching under preferences are intertwined, and my research involves theory and terminology from both. In particular, my research has focused on models that can be viewed as hedonic games with a constraint on the size of possible coalitions. These models also generalise the well-known \emph{Stable Roommates problem} \cite{Irv85, GS62} that belongs to the area of matching under preferences.

In this thesis we study the algorithmic aspects of models in which agents form coalitions.

Sometimes, these models can be applied in practice. For example, consider a set of cooperative robots that may form work teams in order to maximise their total efficiency (in terms of some specific criteria). In such an application, research in this area can provide practical tools and techniques to organise the robots in an optimal way \cite{IJangHedonicGamesRobots2018}. Other models are less directly applicable, but by studying them we can make insights into how autonomous agents, such as humans, behave in the real world \cite{aumann1964bargaining,Peters2008}. Even purely abstract research involving agents that form coalitions has led to interesting and sometimes unexpected theoretical results.

The models that we focus on involve a set of agents which is to be partitioned into disjoint \emph{coalitions} of a fixed size. We call such a partition a \emph{matching}. For example, this might represent a set of robot workers who will organise themselves into work teams, where each work team contains three robots. We assume that each agent has \emph{preferences} between the possible coalitions that they might belong to. For example, it may be that for practical reasons each robot can only evaluate its own individual efficiency in a specific work team. Each robot therefore assigns a numerical score to each of the work teams it may belong to, where a higher score indicates that a given robot is more effective in that work team. Alternatively, such a model might represent a set of students in a class who must be assigned to pairs by their teacher. Each student might list their classmates in order from the most preferred to least preferred.

In these models, a central idea is the existence of a matching that satisfies some specific criteria, or is somehow optimal. For example, for a specific set of robots, we might seek a matching in which the sum of the individual robots' efficiency scores is maximised. Alternatively, for a specific set of students we might ask whether it is possible to pair the students such that no two students both prefer each other to their respectively assigned partners. We refer to these criteria as \emph{solution concepts}. The solution concept in the latter example, which involves a set of agents that may deviate away from their assigned coalitions, is a type of \emph{stability}. Stability is a central concept in the theory of coalition formation in general and this thesis in particular. 

% We assume that each agent has a preference only between the possible coalitions that they might belong to. Such models are called \emph{hedonic} because each agent cares only about their own coalition \cite{HedonicGamesHOCSC}. In fact, we assume that the preferences of the agents are systematic and can be represented formally. For example, it may be that each agent assigns a numerical valuation to every other agent, and prefers one coalition to another if the sum of their valuations of the other agents in the former coalition is higher than in the latter coalition. These valuations could represent a degree of amity between two students or the combined efficiency of two robots when they are assigned to the same work team. We refer to such a system as a \emph{system of preference representation}.

The specific contribution of this thesis relates to the algorithmic aspects of these models. For example, one of our new results is an algorithm that could be used to construct a matching of robots into work teams that satisfies a type of stability. For a slightly different solution concept we show that, for a given set of robots, a matching that satisfies that solution concept does not necessarily exist. Furthermore, deciding whether a given set of robots can be matched in such a way that satisfies that solution concept is $\NP$-complete. In contrast, we show that if the robots' scores are all restricted in a certain way then such a matching must always exist, and can be found by an efficient algorithm. 



% that we consider can be related directly consider a model that represents the cooperative robots. In this thesis we 





% For example, this algorithm could then be used by the robot workers in order to find a set of work teams in which their collective efficiency is maximised. Alternatively, we might show for a given model that certain problems are, in a specific sense, intractable. In a real life setting, this tells us that certain types of algorithm may not work well in practice.


% Such an algorithm might be of practical use to decision makers. Alternatively, if we can show that some problem is $\NP$-hard then it is likely that no such algorithm exists. More generally, computational results might also help us understand more about the model itself.

% The contribution of this thesis is 

% lead to interesting


% It may be that each robot has a unique capability, so the problem is to program the robots to form work teams such that the probability of achieving the goal is maximised. 

% in which our new results provide  
% . For example,  such as a set of autonomous robots who must form work teams, or a class of students who are to be matched into pairs by their teacher. 

% Such models appear in a variety of fields, with a variety of names, and often the aspects of academic interest are different. For example, research in game theory studies models in which autonomous agents form coalitions cooperatively in order to achieve a desirable outcome . This type of research can provide insights into how humans make decisions in the real world. Other research in economics involves a market of agents that form coalitions in order to divide a common budget \cite{DominikPetersThesis} or trade a commodity \cite{SS74}. This research can also inform solutions for real-life problems of resource allocation and fair division. 



% The models that we study involve a set of agents that form coalitions. or states that form economic agreements. 


% For example, this might represent a set of $60$ robot workers who must organise themselves into exactly $10$ work teams, each of size three. Alternatively, it could represent a class of $30$ students that must be organised by their teacher into $15$ pairs. They also involve...











% In the robot example, for practical reasons it may be that the robots can only evaluate their own individual efficiency. Each robot therefore assigns a numerical score to each of the work teams it may belong to, where a higher score indicates that robot is more effective in that work team. 


% As well as a fixed coalition size, most of the problem models we consider in this thesis have two other key attributes. 
% First, each agent has a preference over the possible coalitions in which they may belong to. Usually in the problem a \emph{system of preference representation} is specified. For example, it may be that agents assign a numerical valuation to each other and compare two coalitions based on the sum of their valuations of the other agents in each coalition.

% Our research focuses on the existence of matchings in which the agents are somehow satisfied. The precise definition of a satisfactory matching is known as the \emph{solution concept}. For example, for some specific set of students we could ask if there exists a matching of the students into pairs of partners in which each student's partner is, to some degree, a friend. Alternatively, we could ask if there exists a matching of the robots into work teams such that the total efficiency of the robot workforce is maximised. More generally, we might ask if there are restricted cases in which a satisfactory matching is bound to exist.

% We might also ask if a satisfactory matching can be found by an efficient algorithm, or consider the problem of verifying that a given matching is satisfactory. The solutions to such problems can often be applied in practice, to assign agents into coalitions \cite{IJangHedonicGamesRobots2018}, or to predict the behaviour of autonomous agents in the real world \cite{LRRSS15}.

\section{Thesis statement}
\label{c:intro:thesisstatement}
% Many algorithmic problems related to Hedonic Games are believed to be computationally intractable while others are solvable in polynomial time. 
% In the context of hedonic games, imposing constraints on the possible sizes of coalitions significantly alters the structure and complexity of associated algorithmic problems. 
% V2
% Computational problems of to coalition formation with fixed size coalitions do appear to be hard in general. Nevertheless, some problems are solvable in polynomial time if either the preferences are sufficiently restricted or the solution concept is sufficiently weak. In fact, for successively weaker solution concepts, existence and polynomial-time solvability hold under successively weaker preference restrictions. 

% Although most computational problems in which agents must be assigned to coalitions of fixed size are $\NP$-hard in general, many of them can be solved in polynomial time if either the agents' preferences are sufficiently restricted or the solution concept is sufficiently weak.
% In fact, for successively weaker solution concepts, existence and polynomial-time solvability hold under successively weaker preference restrictions.
Computational problems involving agents that form coalitions are generally $\NP$-hard, and in particular when those coalitions must have a fixed size. Nevertheless, there exist natural models of fixed-size coalition formation in which optimal or near-optimal matchings can be found using efficient algorithms.

\section{Contribution}
Our main contribution relates to a family of models of fixed-size coalition formation known as \emph{Three-Dimensional Roommates} (\mysymbolfirstusedefinition{symboldef:threedr}{3DR}). The defining characteristic of a model of 3DR is that there are a set of $3n$ agents that must be partitioned into $n$ triples, which we call a \emph{matching}, and each agent has some kind of preference between the triples to which they may belong. Working within the framework of 3DR, we systematically study a set of fixed-size coalition formation problems.

Many real-world scenarios involve coalitions of restricted size, which provides a strong rationale for our study \cite{Sless18}. We motivate our study of 3DR in particular with the fact that results relating to coalitions of size three often generalise directly to problems involving other restrictions on coalition size. Even when this is not the case, it may be that they at least give an indication of what to expect in more general models. For this reason, much of the existing research also focused specifically on models in which coalitions must have size three.

We begin by reviewing the literature in and around coalition formation, including both models involving restricted coalition sizes and models in which coalitions need not have a fixed size. We tie related concepts and theory together from across computing science, economics, and game theory.

In our first technical contribution, we consider a variant of 3DR known as \emph{Three-Dimensional Stable Matching with Cyclic preferences} (3-DSM-CYC), which has been of independent interest. In 3-DSM-CYC, each agent has one of three types, sometimes termed \emph{man}, \emph{woman}, and \emph{dog}, which have a cyclic order. Any feasible triple must contain exactly one agent of each type. There are $n$ agents of each type and each agent has a preference list only over the agents of the next type. A matching is \emph{stable} if there exists no triple $t$ in which each agent in $t$ prefers the triple $t$ to their assigned triple in the matching. It was recently shown, contrary to previous conjectures, that a given instance of 3-DSM-CYC need not contain a stable matching and that the associated decision problem is $\NP$-complete~\cite{Plaxton3DSMCYCJournal}. We thus consider the approximability of a closely related optimisation problem, in which the objective is to construct a matching that is, in terms of a specific measure, as stable as possible. To our knowledge, this work is the first investigation into the approximability of 3-DSM-CYC. We first show that an existing algorithm for another three-dimensional matching problem, which is closely related to 3-DSM-CYC, can be used to construct a $9/4$-approximation algorithm. Improving this approximation, we then present a $6/5$-approximation algorithm based on serial dictatorship. We then consider a restriction of 3-DSM-CYC in which the preferences of all agents of at least one type are in some way similar, using a specific similarity measure. Specifically, we suppose the preference lists of all agents of at least one type are derived from a master preference list and consider the maximum \emph{Kendall tau distance} \cite{KendallTauCitation} between the master list and the list of any agent of that type. We show that if this distance is sufficiently small then as it is further reduced the approximation ratio decreases from $6/5$ to $1$.

Next, we define two new models of 3DR that involve so-called $\mathscr{B}$- and $\mathscr{W}$-preferences, which we name 3DR-B and 3DR-W. Using $\mathscr{B}$- ($\mathscr{W}$-) preferences, each agent has a strict preference list over the other agents and compares two triples based only on the most-preferred (least-preferred) member of each triple. We consider in both 3DR-B and 3DR-W the existence of matchings that are stable. We first show that it is $\NP$-complete to decide if a given instance (of agents and preferences) of either model contains a stable matching. Interestingly, this contrasts with the existence of polynomial-time algorithms in two analogous models in which coalitions may have any size \cite{CR01, CH04}. For both 3DR-B and 3DR-W, we also consider a closely related optimisation problem in which the objective is to construct a matching that is, in terms of a specific measure, as stable as possible. We show that an existing result leads to a $9/4$-approximation algorithm in both models and a simple algorithm based on serial dictatorship gives a $3/2$-approximation for 3DR-B.

We then formalise a model of 3DR with additively separable preferences, which we call 3DR-AS. In this model, each agent provides a numerical valuation of every other agent and compares triples based on the sum of the valuations of the other two agents in each triple. We investigate in 3DR-AS the existence of stable matchings as well as matchings that are envy-free, meaning there exists no pair of agents where the one would prefer to swap places with the other. In fact, we consider three successively weaker formalisms of this notion, namely envy-freeness, weakly justified envy-freeness, and justified envy-freeness. We consider the computational problems of deciding if such a matching exists, and constructing one if so. In particular, we study these problems in a setting where the agents' valuations are restricted. We consider various restrictions involving binary, ternary, and symmetric valuations. We provide a full complexity classification and identify dichotomies in terms of these restrictions. Interestingly, we identify a general trend that shows, for successively weaker solution concepts, existence and polynomial-time solvability hold under three successively weaker restrictions on the agents' preferences. 

Building on our new result that any instance of 3DR-AS with binary and symmetric preferences must contain a stable matching, we also consider a related optimisation problem in which the objective is to construct a stable matching in such an instance with maximum utilitarian welfare, i.e.\ the total sum of agents' valuations of their assigned partners is maximised. We devise a $2$-approximation algorithm for this optimisation problem.

Finally, we consider a problem in graph theory that generalises the notion of assigning agents to coalitions of a fixed size. Rather than partitioning a set of agents, the problem is to find a maximum-cardinality set of $r$-cliques in an undirected graph, subject to that set being either vertex disjoint or edge disjoint, for a fixed integer $r \geq 3$. This general problem is known as the \emph{$K_r$-packing problem}. Here we study the restriction of this problem in which the graph has a fixed maximum degree $\Delta$. It is known for $r=3$ that the vertex-disjoint (edge-disjoint) variant is solvable in linear time if $\Delta=3$ ($\Delta=4$) but $\APX$-hard if $\Delta \geq 4$ ($\Delta \geq 5$) \cite{caprara_packing_2002}. In other words, there exists some fixed constants $\varepsilon > 1$ and $\varepsilon' > 1$ such that no polynomial-time $\varepsilon$-approximation algorithm exists for the vertex-disjoint variant if $\Delta \geq 4$; and no polynomial-time $\varepsilon'$-approximation algorithm exists for the edge-disjoint variant if $\Delta \geq 5$. We generalise these results to an arbitrary but fixed $r \geq 3$, and provide a full complexity classification for both the vertex- and edge-disjoint variants in graphs of maximum degree $\Delta$, for every $r \geq 3$. Specifically, we show that the vertex-disjoint problem is solvable in linear time if $\Delta < 3r/2 - 1$, solvable in polynomial time if $\Delta < 5r/3 - 1$, and $\APX$-hard if $\Delta \geq \lceil 5r/3 \rceil - 1$. We also show that if $r\geq 6$ then these implications also hold for the edge-disjoint problem. If $r < 6$, then the edge-disjoint problem is solvable in linear time if $\Delta < 3r/2 - 1$, solvable in polynomial time if $\Delta \leq 2r - 2$, and $\APX$-hard if $\Delta > 2r - 2$.


\section{Structure}
The remainder of this thesis is structured as follows:

\begin{itemize}
    \item In Chapter~\ref{c:lit_review}, we give an overview of literature involving coalition formation, hedonic games, and fixed-size coalitions, setting the scene for our subsequent results.
    
    \item In Chapter~\ref{c:three_dsm_cyc}, we introduce 3-DSM-CYC and present new results on the approximability of a related optimisation problem, both in the general case and in a restriction of the problem involving master lists.
    
    \item In Chapter~\ref{c:threed_sr_b}, we define 3DR-B and present new results concerning the existence of stable matchings in instances of 3DR-B and on the approximability of a related optimisation problem.
    
    \item In Chapter~\ref{c:threed_sr_w}, we define 3DR-W and present new results concerning the existence of stable matchings in instances of 3DR-W and on the approximability of a related optimisation problem.
    
    \item In Chapter~\ref{c:threed_sr_as}, we define 3DR-AS and present new results concerning the existence of stable matchings in 3DR-AS, under various restrictions on the agents' valuations.
    
    \item In Chapter~\ref{c:threed_efr_as}, we present new results concerning the existence of envy-free, weakly justified envy-free, and justified envy-free matchings in instances of 3DR-AS, under various restrictions on the agents' valuations.
    
    \item In Chapter~\ref{c:kr_packing}, we present new results on the complexity of the $K_r$-packing problem in graphs of fixed maximum degree.
    
    \item In Chapter~\ref{c:conclusion}, we summarise the contribution of this thesis and discuss some directions for future work.
\end{itemize}
\chapter{Literature review}
\label{c:lit_review}

\section{Introduction}

Models involving agents that form coalitions appear in a variety of settings, and terminology varies depending on the context and application. In this chapter we give an overview of the existing models in which a set of agents is to be partitioned into disjoint coalitions based somehow on the preferences of those agents, with a focus on models related to fixed-size coalitions and in particular Three-Dimensional Roommates (3DR). We shall present models that appear across the literature of computing science, economics, and game theory. For many concepts, the notation and terminology used varies between fields, even when the underlying concept is the same.

In Section~\ref{sec:lit_review_hedonicgames}, we cover models of coalition formation in which coalitions need not have a fixed size. Models of this type are almost always defined as \emph{hedonic games}, which are the subject of a sizeable area of research. Many of the concepts and terminology associated with hedonic games are also used in other models of coalition formation, some of which we consider subsequently. For this reason we consider these models first.

In Section~\ref{sec:lit_review_matchingunderpreferences}, we cover models of coalition formation in which a restriction exists on the sizes of possible coalitions. In most of the existing models of this type, for example in the \emph{Stable Roommates problem} (SR), the restriction is that any feasible coalition has some fixed size. The terminology used in the literature to refer to these models can vary. For the sake of consistency, we standardise terminology. For example, we use the term ``matching'', meaning a set of coalitions, while other authors use the terms \emph{assignment}~\cite{NH91} or \emph{partition}~\cite{CR01} for a similar purpose.

% Such models include classical models of matching under preferences, such as the Stable Roommates problem and its three-dimensional generalisation.

% \section{Coalitions of arbitrary size}
% \label{sec:lit_review_arbitrarysize}

\section{Hedonic games and stable partitions}
\label{sec:lit_review_hedonicgames}

A well-studied model of coalition formation is the \emph{hedonic game}. The defining characteristic of a hedonic game is \emph{hedonic preferences}. Generally speaking, hedonic preferences mean that ``every agent only cares about which agents are in its coalition, but does not care how agents in other coalitions are grouped together''~\cite{HedonicGamesHOCSC,DG80}. This contrasts with other so-called co-operative games that involve divisible goods, which are shared among the agents based on the coalitions formed. \emph{The Handbook of Computational Social Choice}~\cite{HedonicGamesHOCSC} defines a hedonic game as follows:
%
\begin{mydefinitionofhedonicgame}[Hedonic game]
\begin{adjustwidth}{8pt}{8pt}
\label{def:lit_review_hedonicgamehocs}
Let $N$ be a finite set of agents. A \emph{coalition} is a non-empty subset of $N$. Let $\mathcal{N}_i = \{ S \subseteq N : i \in S\}$ be the set of all coalitions (subsets of $N$) that include agent $i\in N$. A \emph{coalition partition} $\pi$ is a partition of the agent set $N$ into disjoint coalitions. A \emph{hedonic game} or \emph{hedonic coalition formation game} is a pair $(\mathcal{N}, \succsim)$, where $\succsim$ is a \emph{preference profile} that specifies for every agent $i\in N$ a reflexive, complete, and transitive relation $\succsim_i$ on $\mathcal{N}_i$. We call $\succsim_i$ a \emph{preference relation}.
\end{adjustwidth}
\end{mydefinitionofhedonicgame}
%
An \emph{instance} of a hedonic game is a specific set of agents and a preference profile $\succsim$. The exact representation of $\succsim$ can vary. For example, one possibility is that for each agent $i$, $\succsim_i$ is an ordered list containing every possible coalition in $\mathcal{N}_i$. For some partition $\pi$ and some agent $i$, we denote by $\pi(i)$ the coalition containing $i$. For any two agents $i$ and $j$, if $j \in \pi(i)$ then we say that $i$ is a \emph{partner} of $j$. For some agent $i$ and any two coalitions $S$ and $T$ in $\mathcal{N}_i$, we write $S \sim_i T$ if $S \succsim_i T$ and $T \succsim_i S$ and write $S \succ_i T$ if $S \succsim_i T$ and $T \not\succsim_i S$. If $S \succsim_i T$ then we say that $i$ \emph{weakly prefers} $S$ to $T$. If $S \sim_i T$ then we say that $i$ is \emph{indifferent} between $S$ to $T$. If $S \succ_i T$ then we say that $i$ \emph{strictly prefers} $S$ to $T$. For some agent $i$ and any coalition $S$, if $S \succsim_i T$ for any other coalition $T$ then we say that $S$ is one of $i$'s \emph{most-preferred} coalitions. Each agent's preference between two coalition partitions depends only on their assigned coalition in each, so for any agent $i$ and coalition partitions $\pi$ and $\pi'$, we write $\pi \succsim_i \pi'$ if $\pi(i) \succsim_i \pi'(i)$, $\pi \succ_i \pi'$ if $\pi(i) \succ_i \pi'(i)$, and $\pi \sim_i \pi'$ if $\pi(i) \sim_i \pi'(i)$.

In a given hedonic game, we may ask if there exists a coalition partition in which each agent is satisfied. In formal terms, we may consider the existence of a coalition partition that meets some fixed \emph{solution concept}. One well-studied solution concept is \emph{individual rationality}. We say that coalition partition $\pi$ is \emph{individually rational} if there exists no agent $i$ who strictly prefers the \emph{individual coalition} $\{i\}$ to $\pi(i)$. Some other relevant solution concepts are:
\begin{itemize}
    \item \emph{Perfection}. A coalition partition is \emph{perfect} if each agent belongs to one of its most-preferred coalitions~\cite{ABH11}. This is a very strong solution concept and in general, a given hedonic game need not contain a perfect partition~\cite{HedonicGamesHOCSC}.
    \item \emph{Pareto optimality}. Given coalition partitions $\pi$ and $\pi'$, we say $\pi'$ \textit{Pareto dominates} $\pi$ if $\pi' \succsim_i \pi$ for each agent $i \in N$ and there exists some agent $j\in N$ where $\pi' \succ_j \pi$. A coalition partition $\pi$ is \textit{Pareto optimal} if no coalition partition exists that Pareto dominates $\pi$~\cite{AzizLang2016}.
    \item \emph{Core stability} (also known as \emph{stability}). We say coalition partition $\pi$ is \emph{core stable} if no set of agents $S\subseteq N$ exists such that $S \succ_i \pi(i)$ for each agent $i \in S$. In other words, no set of agents have a common incentive to deviate from the coalition partition and form a new coalition. If such a set $S$ exists then we call it a \emph{blocking coalition}.
    \item \emph{Envy-freeness}. Suppose we are given some coalition partition $\pi$. If there exists some pair of agents $i, j \in N$ where $\pi(i) \neq \pi(j)$ and $i$ strictly prefers $(\pi(j) \setminus \{ j \}) \cup \{ i \}$ to $\pi(i)$ then we say that $i$ \emph{has envy for} $j$. We say that $\pi$ is \emph{envy-free} if no such pair exists~\cite{AZIZ2013316}.
    \item \emph{Justified envy-freeness}. Suppose we are given some coalition partition $\pi$. If there exists some pair of agents $i, j \in N$ where $i$ envies $j$ and $(\pi(j) \setminus \{ j \}) \cup i \succ_k  \pi(j)$ for each agent $k \in \pi(j) \setminus \{ j \}$, then we say that $i$ \emph{has justified envy for} $j$. In other words, $i$ envies $j$ and each partner of $j$ in $\pi$ would be strictly better off if $i$ were to replace $j$ in $\pi(j)$. We say that $\pi$ is \emph{justified envy-free} if no such pair exists~\cite{BY19}.
    \item \emph{Weakly justified envy-freeness}. Suppose we are given some coalition partition $\pi$. If there exists some pair of agents $i, j \in N$ where $i$ envies $j$ and $(\pi(j) \setminus \{ j \}) \cup i \succsim_k  \pi(j)$ for each agent $k \in \pi(j) \setminus \{ j \}$, then we say that $i$ \emph{has weakly justified envy for} $j$. In other words, $i$ envies $j$ and each partner of $j$ in $\pi$ would be weakly better off if $i$ were to replace $j$ in $\pi(j)$. We say that $\pi$ is \emph{weakly justified envy-free} if no such pair exists~\cite{BY19}.
\end{itemize}
Some solution concepts generalise others. For example, by definition any coalition partition that is core stable must also be individually rational, and any perfect partition must also satisfy all of the other concepts that we have described. In fact, a remarkable hierarchy of solution concepts exists~\cite{HedonicGamesHOCSC,AZIZ2013316,BY19}. Part of this hierarchy, involving the seven solution concepts described so far, is illustrated in Figure~\ref{fig:lit_review_hgsolutionconcepts}.

\begin{figure}
    \centering
    \begin{tikzpicture}[thick, every edge/.style = {draw, -to}]
% \node[draw=none] (casenumber) at (-1.5, 3.0) {\emph{Case 7}};
% \draw[help lines,step=0.5] (0,0) grid (14,4);
\begin{scope}[every node/.style={inner sep=8pt}, style={sibling distance=30mm, level distance=20mm}, every edge/.style = {darrow}]
  \node {perfect}
    child {node {Pareto optimal}}
    child {node[yshift=-1.6cm] (cs) {core stable}
      child {node[yshift=-0.8cm] {individually rational}}
    }
    child {node[xshift=1.4cm] {envy-free}
        child {node[yshift=0.25cm] {weakly justified envy-free}
            child {node (jef) {justified envy-free}}
        }
    };
\end{scope}
\draw[\solutionconceptsdiagramarrow] (cs) -- (jef);
\begin{scope}

\end{scope}
\end{tikzpicture}

% [edge from parent path={(\tikzparentnode.south west) -- (\tikzchildnode.west)}]
    \caption[Part of the known hierarchy of solution concepts in hedonic games]{Part of the known hierarchy of solution concepts in hedonic games. In the diagram, an arrow points from one concept to another if any partition that satisfies the former must also satisfy the latter. Adapted from the \emph{Handbook of Computational Social Choice}~\cite{HedonicGamesHOCSC}.}
    \label{fig:lit_review_hgsolutionconcepts}
\end{figure}

A wide range of solution concepts have been proposed and studied in the setting of hedonic games. Many (such as Pareto optimality) have roots in game theory and economics~\cite{HedonicGamesHOCSC}. In a well-cited 2013 article, Aziz et al.~\cite{AZIZ2013316} considered a variety of solution concepts and applied them to a specific type of hedonic game. One such concept is \emph{popularity}. They defined a popular partition $\pi$ as a partition in which for every alternative partition $\pi'$, the number of agents that prefer $\pi'$ to $\pi$ is at least the number of agents that prefer $\pi'$ to $\pi$. We shall discuss later (in Sections~\ref{sec:lit_review_multidimensionalmatching} and~\ref{sec:lit_review_othermodels}) some other research that involves popular partitions.

In some hedonic games, it is assumed that the agents' preferences have some additional structure, which dictates the representation of the preference profile $\succsim$. For example, it may be that a hedonic game models agents in the real world whose preference between coalitions is based on some underlying preference over the agents themselves. The exact representation of $\succsim$ is also important when considering computational aspects of hedonic games.

The simplest system of preference representation is to suppose that each agent has a \emph{preference list} over the $2^{|N|-1}$ possible coalitions in $\mathcal{N}_i$, in order from the most-preferred to the least-preferred. This system is known as \emph{lists of coalitions} (\mysymbolfirstusedefinition{symboldef:lc}{LCs})~\cite{EW09}. In the literature, unless otherwise specified, it is usually assumed that agents provide LCs. Alternatively, agents might list only the coalitions from the most-preferred to the individual coalition, which is known as \emph{Individually Rational Lists of Coalitions} (\mysymbolfirstusedefinition{symboldef:irlc}{IRLCs})~\cite{EW09}. IRLCs might be used if the solution concept implies individual rationality, since no agent $i$ need ever consider a coalition partition less-preferred than the individual coalition $\{i\}$.

LCs and IRLCs involve \emph{ordinal} preferences, meaning that they describe a preference between alternatives but do not quantify the extent that one alternative may be preferred over another~\cite{AMUP}. If preferences do involve such a quantification then they are \emph{cardinal}. For example, we could assume that each agent $i$ assigns a numeric valuation to every possible coalition in $\mathcal{N}_i$.

% Other systems of preference representation, such as \emph{anonymous preferences}~\cite{HOCSC16}, are very simple. Using anonymous preferences, each agent provides a preference list over possible coalition sizes, of length $|N|$. Given two coalitions of different sizes, an agent strictly prefers the coalition whose size appears earlier on their preference list and is otherwise indifferent.

For some applications LCs or IRLCs may not be practical since they require agents to list up to $2^{|N|-1}$ possible coalitions. For example, consider a hedonic game model used by 50 cooperative robots~\cite{IJangHedonicGamesRobots2018}. Using LCs might involve each robot listing ${\sim}\num{5.6e+14}$ coalitions, which is likely to be challenging in practice. In such an application, it might be better to assume that the agents' preferences have a specific structure. For example, we could assume that each agent has a preference list over the other (individual) agents. We say that this representation is more succinct since each preference list has length $|N| - 1$. This representation might also reflect a natural structure in the preferences of real people~\cite{CH02}.

To formalise such a hedonic game, in which agents have preference lists over the other agents, we must be able to infer from each agent's preference list a preference over coalitions. One way to do this systematically is to use a \emph{set extension rule}. Set extension rules have been well studied outside the context of coalition formation, both in general~\cite{BBP04} and in other areas of computational social choice, such as multi-winner voting~\cite{AzizLang2016}. A number of set extension rules have been considered in the hedonic games literature, which we shall introduce shortly.

We first introduce some new notation. For an agent $i$ let $P_i$ be a preference list over other (individual) agents. Thus, $P_i$ describes a strict weak order $\succsim_i$ over $N\setminus \{ i \}$. For any two agents $j$ and $k$ we say that $i$ \textit{prefers} $j$ to $k$, denoted by $j \succ_i k$, if $j$ precedes $k$ in $P_{i}$. It may be that the preference list $P_i$ contains \emph{ties}. A tie is a set of agents $T \subseteq N \setminus \{ i \}$ where $j \sim_i k$ for any pair of agents $j, k \in T$ and, for any $l \in T$ and $m \in N \setminus (T \cup \{ i \})$, either $l \succ_i m$ or $m \succ_i l$. If a preference list contains no ties then we say it is \emph{strict}. Strict preference lists therefore represent a \emph{total order}. Given a set of agents $S$ and an agent $i$ where $i \notin S$, let $\mathscr{W}_i(S)$ be the least-preferred agent in $S$ according to $P_i$ (or an arbitrary least-preferred agent if $P_i$ contains ties). We define $\mathscr{B}_i(S)$ analogously, as the most-preferred agent in $S$ according to $P_i$ (or an arbitrary most-preferred agent if $P_i$ contains ties).

We now introduce two well-known set extension rules, in which an agent compares two coalitions based on either the most-preferred or least-preferred agent in each coalition. Using \emph{$\mathscr{W}$-preferences}~\cite{CH04}, each agent $i$ prefers some coalition $S$ to another coalition $T$ if $\mathscr{W}_i(S) \succ_i \mathscr{W}_i(T)$, and is otherwise indifferent. A related set extension rule is \emph{$\mathscr{B}$-preferences}~\cite{CR01}, in which $i$ prefers $S$ to $T$ if $\mathscr{B}_i(S) \succ_i \mathscr{B}_i(T)$. Usually an additional rule is added, so that $i$ prefers $S$ to $T$ if either $\mathscr{B}_i(S) \succ_i \mathscr{B}_i(T)$, or $\mathscr{B}_i(S) \sim_i \mathscr{B}_i(T)$ and $|S| < |T|$. Hajdukov\'a~\cite{Haj06} noted the motivation for this additional rule: if there is no condition placed on the cardinalities of $S$ and $T$ then, given an arbitrary instance of a hedonic game using $\mathscr{B}$-preferences, the \emph{grand coalition partition} $\{ N \}$ is a perfect partition.

Cechl\'arov\'a and Hajdukov\'a~\cite{CH02} reason that a strong motivation for studying $\mathscr{W}$- and $\mathscr{B}$-preferences in hedonic games comes from more general work of Kannai and Peleg~\cite{KANNAI1984172} in social choice. Motivated by voting theory, Kannai and Peleg demonstrate that any set extension rule that satisfies two natural axioms makes any agent $i$ indifferent between any set $S$ and $\{\mathscr{B}_i(S),\mathscr{W}_i(S) \}$. Barber\`a, Bossert, and Pattanaik discuss this result and some related work in 
\emph{Ranking Sets of Objects}, Chapter 17 of the \emph{Handbook of Utility Theory}~\cite{BBP04}. It remains open whether Kannai and Peleg's axiomatic result can be further applied to the setting of hedonic games.

Another popular system is \emph{additively separable preferences}, which are the defining characteristic of an \emph{Additively Separable Hedonic Game} (\mysymbolfirstusedefinition{symboldef:ashg}{ASHG}) \cite{BKS01,Olsen2007}. Additively separable preferences are a form of cardinal preferences and can be defined in terms of \emph{valuation} functions. Each agent $i$ has a valuation function $v_i(j) : N \setminus \{ i \} \mapsto \mathbb{R}$ (where $\mathbb{R}$ is the set of real numbers). Given an agent $i$ and a set of agents $S\subseteq N$, we say that agent $i$ has \emph{utility} $u_i(S)=\sum_{j\in S\setminus \{ i \}} v_i(j)$ in $S$. For two coalitions $S$ and $T$, we define $S\succ_i T$ if $u_i(S) > u_i(T)$ and $S \sim_i T$ if $u_i(S) = u_i(T)$. Let $u_{i}(\pi)$ be short for $u_{i}(\pi(i))$. We also define the \emph{utilitarian welfare} of a partition $\pi$ as $u(\pi) = \sum_{i \in N} u_i(\pi)$. Various other measures related to ``welfare'' have been defined in ASHGs~\cite{AZIZ2013316}. Additively separable preferences generalise \emph{separable preferences}~\cite{Haj06}, which can be defined as a restriction of additively separable preferences in which $v_i(j)\in\{-1,0,1\}$ for any two agents $i$ and $j$. We remark that a similar model to an ASHG has been studied as a \emph{Weighted Graph Game}~\cite{Deng94}.  It is also possible to define other restrictions of additively separable preferences. For example, we say that valuations are \emph{symmetric} if $v_i(j) = v_j(i)$ for any two agents $i$ and $j$. We say that valuations are \emph{binary} (also termed \emph{simple}~\cite{ABBHOP19}) if $v_i(j) \in \{ 0, 1 \}$. Similarly, we say that valuations are \emph{ternary} if $v_i(j) \in \{ 0, 1, 2 \}$. 

In a seminal 2002 article, Bogomolnaia and Jackson \cite{BJ02} focused on solution concepts in AHSGs that involve the movement of individual agents away from their assigned coalitions (such as individual rationality). They observed that, for some of these solution concepts, if preferences are symmetric then a satisfactory partition must exist. The proof of this observation follows from the fact that any agent's movement (or \emph{deviation}) produces a partition with strictly higher utilitarian welfare. This type of proof has since been termed a ``potential function'' argument \cite{HedonicGamesHOCSC}. Since 2002, similar arguments have since been used to show new results in some other types of hedonic games \cite{BrandtBullingerWilczynski2021}.

In 2007, Huang~\cite{Huang07conference} proposed a restricted variant of additively separable preferences called \emph{Precedence by Ordinal Number} (PON), which is related to Borda scores~\cite{HedonicGamesHOCSC}. In PON, each agent has an ordinal preference list, and $v_i(j)$ is defined to be the rank, beginning from one, of $j$ in the preference list of $i$.

Another variant of additively separable preferences is \emph{fractional preferences}, in which the utility of a coalition is based on the average valuation over all agents in that coalition. Aziz et al.~\cite{ABBHOP19} presented an extensive survey of so-called \emph{Fractional Hedonic Games} (FHGs) in 2019. We remark that if we require any feasible coalition to have a fixed size $k$, then the definitions of an ASHG and an FHG are effectively equivalent.

From an algorithmic perspective, there are three main computational problems associated with a given model of a hedonic game. The first is a decision problem known as the \emph{existence problem}. For some fixed solution concept, it asks if a given instance of a hedonic game contains a coalition partition meeting that concept. The second is the \emph{construction problem}, which is the search problem of either finding a coalition partition that meets a given solution concept or reporting that no such partition exists. The third is the \emph{verification problem}. For a fixed solution concept and fixed instance of a hedonic game, it asks if a given partition in that instance meets that concept. These three problems are closely related. For example, if the existence problem is $\NP$-complete, for some model of a hedonic game, then it follows immediately that the verification problem is solvable in polynomial time and the construction problem is $\NP$-hard.

The system of preference representation is particularly meaningful when considering these computational problems. Consider the core-stablility existence problem for a model of a hedonic game in which preferences are represented using IRLCs. Encoding the agents' preferences requires $O(2^{|N|})$ space, but since they form part of the problem input it is possible to scan every preference list in linear time with respect to the size of the input. This observation means that the corresponding verification problem belongs to the complexity class $\P$. Ballester~\cite{Bal04} showed that the corresponding existence problem is $\NP$-hard and thus $\NP$-complete. In some cases, more succinct systems of preference representation lead to $\coNP$-complete verification problems~\cite{CSEH2019} and $\Sigma_2^\P$-complete ($\NP^\NP$-complete) existence problems~\cite{WOEGINGER2013101,OBISY17}.

In research on the algorithmics of hedonic games, the goal is often to show that, for a given preference representation and solution concept, the verification, existence, or construction problems are either solvable in polynomial time or are computationally hard. For example, in 2001 Cechl\'arov\'a and Romero-Medina~\cite{CR01} considered the core-stability existence problem in hedonic game models (referred to as the \emph{Stable Partition problem}) using $\mathscr{B}$- and $\mathscr{W}$-preferences. They showed that for $\mathscr{B}$-preferences, a core stable partition must exist, and can be found in polynomial time. Later, in 2004, Cechl\'arov\'a and Hajdukov\'a considered the analogous model using $\mathscr{W}$-preferences. They showed that in that model, a core stable partition may not exist, but that a polynomial-time algorithm exists that can either find a core stable partition or report that none exist. Interestingly, even though the definitions of $\mathscr{B}$- and $\mathscr{W}$-preferences are similar, the algorithms used in both settings are significantly different. 

In 2004, Ballester~\cite{Bal04} showed that the existence problem is $\NP$-complete for a number of hedonic games involving systems of preferences in which agents have preference lists over all possible coalitions. In particular, he showed that in a hedonic game using LCs, the core-stability existence problem is $\NP$-complete.

In 2010, Sung and Dimitrov~\cite{SUNG2010635} studied the existence of core stable partitions in ASHGs. They showed that an ASHG may not contain a core stable partition and that the existence problem is strongly $\NP$-hard, but left open the question of whether it belongs to $\NP$. They also presented hardness results relating to other solution concepts in ASHGs. In their 2011 paper, Aziz et al.~\cite{AZIZ2013316} strengthened Sung and Dimitrov's result, showing that the core stability existence problem is strongly $\NP$-hard even when valuations are symmetric. In 2013, Woeginger~\cite{WOEGINGER2013101} resolved this open question and showed the existence problem is $\Sigma_2^\P$-complete.

In 2019, Gairing and Savani \cite{GairingSavani19} considered certain solution concepts in ASHGs with symmetric preferences, where for each solution concept a potential function argument can be used to show that a satisfactory partition always exists. They observed that all of the associated construction problems can be modelled as local search problems in the class $\PLS$. Notably, they showed that many are also $\PLS$-complete. These results are interesting because it is unlikely that any problem in $\PLS$ is $\NP$-hard (which would imply $\NP=\coNP$), but it is also believed that no $\PLS$-complete problem can be solved in polynomial time \cite{JohnsonPapaYanna88}.

In their 2019 article on FHGs, Aziz et al.~\cite{ABBHOP19} showed that a core stable partition may not exist in a given FHG, even when valuations are binary and symmetric, and that the corresponding existence problem is $\Sigma_2^\P$-complete. They also presented some positive results for FHGs with binary and symmetric preferences.

In their 2013 paper, Aziz et al.~\cite{AZIZ2013316} considered the existence of envy-free partitions in ASHGs. They noted that the \emph{singleton coalition partition} $\{ \{ \alpha_1 \}, \{ \alpha_2 \}, \dots, \{ \alpha_{|N|} \} \}$ is trivially envy-free and thus considered the existence of partitions that simultaneously satisfy envy-freeness as well as other solution concepts. 
In 2018, Ueda~\cite{Ued18} considered envy-freeness and justified envy-freeness in a hedonic game model using LCs. He observed in this model that both the singleton partition and the grand partition $\{ N \}$ are also envy-free, although there exist instances in which no ``non-trivial'' coalition partition is envy-free, and additional instances in which no non-trivial partition is justified envy-free. He also observed that core stability implies justified envy-freeness.
In 2019, Barrot and Yokoo~\cite{BY19} noted Ueda's observation and continued exploring the existence of coalition partitions that satisfy a combination of solution concepts, including envy-freeness, weakly justified envy-freeness, and justified envy-freeness. As well as some non-existence results for ASHGs, they also considered other models involving more general systems of preference representation. Notably, they presented results relating to the existence of such coalition partitions in a setting in which preferences either satisfy \emph{top responsiveness} or \emph{bottom responsiveness}, two restrictions already well-established in the hedonic games literature.

\section{Coalitions of restricted size}
\label{sec:lit_review_matchingunderpreferences}

\subsection{Two-dimensional matching and roommates}

Historically, most of the research involving fixed-size coalitions relates to models in which agents are to be paired into coalitions of size two. We call such models \emph{two-dimensional}. The study of such models is closely related to the area of \emph{matching under preferences}. In his 2013 book, Manlove~\cite{AMUP} presents a broad survey of the literature of matching under preferences. This area is also related to the concept of matching in graph theory~\cite{combinatorialwest}.

A seminal model of two-dimensional matching under preferences was introduced in 1962 by Gale and Shapley~\cite{GS62}. A set of applicants are applying individually to a set of colleges. Each applicant produces a strict preference list of colleges from most-preferred to least-preferred. Similarly, each college produces a strict preference list of students. Each college may offer multiple places but each student must apply to exactly one college. The authors considered how best to match applicants to college places. They asked if there exists an assignment of students to college places such that no two applicants $\alpha$ and $\beta$, assigned to colleges $A$ and $B$, constitute a \emph{blocking pair}, meaning that $\beta$ prefers $A$ to $B$, and $A$ prefers $\beta$ to $\alpha$. They called such an assignment a \emph{stable matching}, and provided an efficient algorithm that can construct a stable matching. Interestingly, they also showed that this algorithm is, in a sense, optimal for the applicants: ``every applicant is at least as well off under the assignment given \dots as he would be under any other stable assignment''. Although this problem is stated in terms of colleges and applicants, it is commonly referred to as the \emph{Hospitals-Residents problem} (HR)~\cite{AMUP}.

In the same paper, Gale and Shapley also considered the restriction of HR when hospitals admit exactly one student. They proposed a heterosexual marriage metaphor involving a set of $n$ men and $n$ women, who are to be matched into $n$ pairs. For this reason this problem is known as the \emph{Stable Marriage problem} (\mysymbolfirstusedefinition{symboldef:sm}{SM})~\cite{AMUP}. They also described a third problem, known as the \emph{Stable Roommates problem} (\mysymbolfirstusedefinition{symboldef:sr}{SR}), which can be defined as a generalisation of SM. In SR, there exists a single set of agents, who have strict preference lists over all other agents. The goal, as for SM, is to construct a matching in which no two agents prefer each other to their respective assigned partners~\cite{Irv85}. SR can thus be equivalently viewed as a hedonic game in which agents provide LCs (see Section~\ref{sec:lit_review_hedonicgames}) and any feasible coalition has size two. Gale and Shapley showed that, in contrast to HR (and SM), there exist instances of SR that contain no stable matching. In 1976, Knuth~\cite[Problem~12]{Knu97english} asked if a stable matching can be found in polynomial time in a given instance of SR. This question was finally resolved by Irving~\cite{Irv85} in 1985 who presented a polynomial-time algorithm that can decide if a given instance of SR contains a stable matching, and constructs one if so.

Since 1962, a multitude of research in economics and computing science has been influenced by concepts and theory from Gale and Shapley. The 2012 Nobel prize in Economic Sciences was awarded to Shapley and Roth~\cite{RothShapleyNobelPrize} for their work on the theory and application of matching under preferences. The citation describes Shapley's theoretical research, including the Gale-Shapley algorithm, and the practical work of Roth, who successfully applied this theory to the assignment of doctors to hospital positions in the USA.

Much research in the area of matching under preferences involves variants and generalisations of HR, SM and SR \cite{AMUP}. For problems that involve preference lists, one natural generalisation is to allow \emph{incomplete lists}. In general terms this means that for each agent certain alternatives in their preference list are unacceptable, meaning that in no feasible matching is any agent assigned an unacceptable alternative. The variant of SM with incomplete lists is known as SMI~\cite{AMUP}. In SMI, incomplete preference lists characterise unacceptable pairs of men and women. Thus, it may be impossible to produce a stable matching involving all men and all women, so we generalise the definition of a stable marriage and allow agents to be \emph{unmatched}, meaning they have no partner in a given matching. A blocking pair now consists of a man $m$ and a woman $w$ that find each other acceptable, where: (1) either $m$ is unmatched or $m$ prefers $w$ to his partner, and (2) either $w$ is unmatched or $w$ prefers $m$ to her partner. The definition of SRI is analogous. The introduction of incomplete preference lists in SMI and SRI, and in other problems of matching under preferences, has generated significant interest~\cite{AMUP}.

\subsection{Multidimensional matching of a multipartite agent set}
\label{sec:lit_review_multidimensionalmatching}

In this section we focus on problem models that involve matching a multipartite set of agents. An early model of this type was proposed in 1976 by Knuth~\cite[Problem~11]{Knu97english}, who asked if SM can be extended to three sets, for example men, women and dogs. It is not immediately clear how this should be done, and various different models have since been proposed. In all of them, a feasible coalition must contain exactly one man, one woman, and one dog, and thus have size three. We classify such models as \emph{three-dimensional}. A close connection also exists to the \emph{Three-Dimensional Matching} and \emph{Partition into Triangles} problems, which can both be stated in terms of graph theory and do not involve agents with preferences.  For general information on Three-Dimensional Matching and Partition into Triangles we recommend Garey and Johnson's classic textbook~\cite{GJ79}. 

An early formalism of Knuth's idea was proposed by Alkan~\cite{Alk88} in 1988. He proposed the following model (we update the terminology and notation). There exists a set $N$ of $3n$ agents and a preference list $P_{\alpha_i}$ for each agent $\alpha_i \in N$. Let $P$ be the collection of preference lists $P_{\alpha_i}$ for each agent $\alpha_i$. Each agent has one of three types, which are called \emph{man}, \emph{woman}, and \emph{dog}. There are $n$ agents of each type, and the agents of each type are labelled $U = \{ u_1, u_2, \dots, u_n \}$, $W = \{ w_1, w_2, \dots, w_n \}$, and $D =  \{ d_1, d_2, \dots, d_n \}$ respectively. A \textit{family} is a $3$-tuple $( u_i, w_j, d_k ) \in U\times W\times D$.  A \textit{matching} is a set of families where each agent in $N$ is contained in exactly one family. Given an agent $\alpha_i$ and a matching $M$, we denote by $M(\alpha_i)$ the family in $M$ that contains $\alpha_i$. Each agent's preference list $P_{\alpha_i}$ describes a strict order over every pair of agents containing one agent of each of the other two types. In other words, each agent has a strict preference over all possible coalitions that they may belong to. We say that each agent $u_i \in U$ \textit{prefers} a family $( u_i, w_j, d_k )$ to a family $( u_i, w_{j'}, d_{k'} )$ if $( w_j, d_k )$ precedes $( w_{j'}, d_{k'} )$ in $P_{u_i}$. Analogous statements are true for each $w_j \in W$ and each $d_k \in D$. Given a matching $M$, we say that a family $f$ is \textit{blocking} if each agent $\alpha_i$ in $f$ prefers their pair of partners in $f$ to their pair of partners in $M(\alpha_i)$. A matching is \textit{stable} if it does not contain a blocking family. Alkan presented an example instance of this model that contains no stable matching.

Seemingly independently of Alkan, Ng and Hirschberg~\cite{NH91} defined the same model in 1991 as the \emph{Three-Gender Stable Marriage problem} (\mysymbolfirstusedefinition{symboldef:threegsm}{3GSM}). They showed that, in contrast with (two-dimensional) SM, it is $\NP$-complete to decide if a given instance $(N, P)$ of 3GSM contains a stable matching. Subramanian \cite{Sub94} provided an alternative proof of this result in 1994, in a paper exploring an interesting relationship between certain stable matching problems and the so-called \emph{Network Stability} problem. As defined by Subramanian, a network is similar to a circuit (defined in the standard way) except its underlying graph need not be acyclic. The Network Stability problem asks if it possible to assign boolean values to the arcs in a given network such that all gates are simultaneously satisfied. In his book, Manlove \cite{AMUP} reviews Subramanian's results, and also discusses some related subsequent results on the relationship between stable matching and Network Stability.

In his 2007 paper, Huang~\cite{Huang07conference} proposed a variant of 3GSM in which each agent $u_i$ has a strict preference list over $W$ and a strict preference list over $D$, and compares two pairs $( w_j, d_k )$ and $( w_{j'}, d_{k'} )$ based on the sum of the ranks of $w_j$, $w_{j'}$, $d_k$, and $d_{k'}$ in these lists. Huang called this system \emph{Precedence by Ordinal Number} (PON, which we also discussed in Section~\ref{sec:lit_review_hedonicgames}). Huang also considered the restriction of 3GSM in which preferences are \emph{consistent}. In this system, each man $u_i$ has underlying strict preference lists, $P_{u_i}^W$ and $P_{u_i}^D$, over the agents in $W$ and $D$ respectively. For any man $u_i$, the strict preference list $P_{u_i}$, which is a total order over pairs, must be a linear extension of the product order over $W \times D$ with respect to $P_{u_i}^W$ and $P_{u_i}^D$. In other words, $(w_j, d_k) \succ_i (w_{j'}, d_{k'})$ if and only if either (1) $w_j$ precedes $w_{j'}$ in $P_{u_i}^W$ and $d_k = d_{k'}$, (2) $d_k$ precedes $d_{k'}$ in $P_{u_i}^D$ and $w_j = w_{j'}$, or (3) $w_j$ precedes $w_{j'}$ in $P_{u_i}^W$ and $d_k$ precedes $d_{k'}$ in $P_{u_i}^D$ \cite{AMUP}. Similar statements are true for each woman, who has underlying lists over the agents in $U$ and $D$, and each dog, who has two underlying lists over the agents in $U$ and $W$. Huang showed that, in a given instance of 3GSM in which preferences are either consistent, or are additively separable and follow the PON restriction, a stable matching may not exist and the associated decision problem is $\NP$-complete. Noting that in the PON variant agents may be indifferent between pairs of partners, he also proposed and studied a hierarchy of solution concepts related to stability. In each of the solution concepts, some number of agents in any blocking family $f$ need only be indifferent between $f$ and their assigned family. He showed that the existence problems relating to so-called \emph{strongly-stable}, \emph{super-stable}, and \emph{ultra-stable} matchings in the PON system are all also $\NP$-complete.

We remark that the idea of consistency is conceptually similar to the principle of \emph{monotonic preferences} (or synonymously, \emph{independence} \cite{BBP04}), one of the axioms studied by Kannai and Peleg \cite{KANNAI1984172} in 1984. It is also unclear whether Kannai and Peleg's result (discussed in Section~\ref{sec:lit_review_hedonicgames}) can be related to the specific setting of 3GSM.

Another special case of 3GSM was proposed by Danilov~\cite{Dan02}, in 2003. In this model, each man $u_i$ has an underlying preference list over the set of women $W$ and, in the preference list $P_{u_i}$ over $W \times D$, the pair $( w_j, d_k )$ precedes $( w_{j'}, d_{k'} )$ only if $w_{j}$ precedes $w_{j'}$ in the underlying list of $u_i$. Similarly, each woman $w_j$ has an underlying preference list over the set of men $U$, and in the preference list $P_{w_j}$ over $U \times D$, the pair $( u_i, d_k )$ precedes $( u_{i'}, d_{k'} )$ only if $u_{i}$ precedes $u_{i'}$ in the underlying list of $w_j$. No assumption is made about the preferences of the agents in $D$. Danilov showed that, in this case, Gale and Shapley's algorithm for SM can be used to find a stable matching.

In 2004, Boros et al.~\cite{BGJK04} considered another model that can also be defined as a special case of 3GSM. The authors noted that in Danilov's model the preferences are \emph{acyclic}, and proposed a model with \emph{lexicographically cyclic} preferences, in which the types have a cyclic order in which $U$ precedes $W$, $W$ precedes $D$, and $D$ precedes $U$. In this model, each man $u_i$ has an underlying preference list over $W\times D$, where $(w_j, d_k)$ precedes $(w_{j'}, d_{k'})$ in $P_{u_i}$ only if $w_j$ precedes $w_{j'}$ in the underlying list of $u_i$. Similarly, each woman $w_j$ has an underlying preference list over the set of dogs $D$, and in the preference list $P_{w_j}$ over $U \times D$, the pair $( u_i, d_k )$ precedes $( u_{i'}, d_{k'} )$ only if $d_k$ precedes $d_{k'}$ in the underlying list of $w_j$. Similarly, each dog $d_k$ has an underlying preference list over the set of men $U$, and in the preference list $P_{d_k}$ over $U \times W$, the pair $( u_i, w_j )$ precedes $( u_{i'}, w_{j'} )$ only if $u_i$ precedes $u_{i'}$ in the underlying list of $d_k$. Boros et al.\ showed that in this model a stable matching must exist if $n \leq 2$ but need not exist if $n \geq 3$. They also proposed another model of \emph{purely cyclic} preferences, which is very closely related but not, strictly speaking, a special case of 3GSM (since agents may be indifferent between families). This model has since been termed \emph{Three-Dimensional Stable Matching with Cyclic preferences} (3-DSM-CYC, sometimes 3DSM \cite{BM10} or c3DSM \cite{Pashkovich20}).

3-DSM-CYC can be defined identically to 3GSM except the preference list $P_{u_i}$ of each agent $u_i$ is over the individual agents in $W$ and is strict, the preference list $P_{w_j}$ of each agent $w_j$ is over the individual agents in $D$ and is strict, and the preference list  $P_{d_k}$ of each agent $d_k$ is over the individual agents in $U$ and is strict. As in the model of Boros et al.~\cite{BGJK04}, the types have a cyclic order in which $U$ precedes $W$, $W$ precedes $D$, and $D$ precedes $U$. In 3-DSM-CYC, each man $u_i \in U$ prefers any pair $( w_j, d_k )$ to any pair $( w_{j'}, d_{k'} )$ if $w_j$ precedes $w_{j'}$ in $P_{u_i}$, each woman $w_j \in W$ prefers any pair $( u_i, d_k )$ to any pair $( u_{i'}, d_{k'} )$ if $d_k$ precedes $d_{k'}$ in $P_{w_j}$, and each dog $d_k \in D$ prefers any pair $( u_i, w_j )$ to any pair $( u_{i'}, w_{j'} )$ if $u_i$ precedes $u_{i'}$ in $P_{d_k}$. A family $f$ is \emph{blocking} if each agent $\alpha_i$ in $f$ prefers the agent of the next type in $f$ to the agent of the next type in $M(\alpha_i)$ (with respect to the cyclic order).

Since 2004, 3-DSM-CYC has generated a great deal of interest, particularly in the area of matching under preferences. In their 2004 paper, Boros et al.~\cite{BGJK04} showed that if $n=3$ then a stable matching must exist, but left open the case for $n \geq 4$.

In 2006, Eriksson et al.~\cite{ESS06} extended this result to $n = 4$. They conjectured that, based on evidence from computer search, any instance $(N, P)$ of 3-DSM-CYC contains a stable matching. Moreover, they conjectured that the minimum number of stable matchings over all instances of size $n$ increases with $n$. In 2009, Bir\'o and McDermid~\cite{BM10} studied two variants of 3-DSM-CYC. One involved incomplete preference lists, which is known as 3-DSMI-CYC, and the other involved complete preference lists with ties. They showed that for both variants a stable matching may not exist and the associated decision problem is $\NP$-complete. In 2018, Escamocher and O'Sullivan~\cite{Escamocher2018} considered a restricted set of instances of 3-DSM-CYC in which all agents of one type have the same \emph{master} preference list. They showed that the number of stable matchings in such an instance is exponential in $n$. They combined this result with an empirical study, which indicated that such instances contain the fewest stable matchings among all instances of the same size. They therefore conjectured that the number of stable matchings in an arbitrary instance is in fact exponential in $n$. In 2019, Pashkovich and Poirrier~\cite{Pashkovich20} extended the result of Eriksson et al.\ and showed that if $n=5$ then a stable matching must exist. Pashkovich and Poirrier formulated instances of 3-DSM-CYC as instances of the \emph{Satisfiability} problem, and solved them using a SAT solver. 

Generalisations of 3-DSM-CYC that involve more than three types have also been studied. In 2016, Hofbauer~\cite{HOFBAUER201672} extended the result of Boros et al.~\cite{BGJK04} to show, for any $k \geq 3$, that any instance of $k$-DSM-CYC in which there are at most $k+1$ agents of each type contains a stable matching, which can be found in polynomial time.

Building on previous results involving incomplete preferences, Lam and Plaxton showed in 2019 that, contrary to all previous conjectures, for any $k \geq 3$ there exist a family of instances of $k$-DSM-CYC that contain no stable matching, and the associated existence problem is $\NP$-complete~\cite{Plaxton3DSMCYCJournal}. We remark that for $k=3$, Lam and Plaxton identified an instance with $n=90$ that contains no stable matching.

Nevertheless, $k$-DSM-CYC, and 3-DSM-CYC in particular, have continued to attract attention. In 2020, Pittel \cite{Pittel20} presented a probabilistic analysis of $k$-DSM-CYC, and showed that the expected number of stable matchings in a random instance increases with $n$ as $(n \log n)^{k - 1}$. In 2022, Lerner~\cite{Lerner22} made some interesting further discoveries. His main result was an example instance of 3-DSMI-CYC in which $n=3$ that contains no stable matching, and showed that this instance is minimal since any instance in which $n < 3$ must contain a stable matching. He also presented an instance of 3-DSM-CYC in which $n=20$ that contains no stable matching, leaving the existence of a smaller instance with no stable matching as an intriguing open problem.

Recently in 2022, Cseh and Peters~\cite{CsehPeters22} considered further solution concepts related to popularity in the models of 3-DSMI-CYC and 3-DSM-CYC. They presented a number of $\NP$-hardness results related to 3-DSMI-CYC and, notably, a polynomial-time algorithm for a special case of 3-DSM-CYC that can find a so-called \emph{$A\cup B$-popular matching} in a given instance of 3-DSM-CYC. Also in 2022, Cseh et al.~\cite{Cseh2022} presented a paper in which they develop, and then analyse, a collection of Constraint Programming models for 3-DSM-CYC. They also integrated models for problems involving fair matchings, for some common definitions of fairness.

\subsection{Multidimensional roommates}

Just as 3GSM and 3-DSM-CYC generalise (two-dimensional) SM to three dimensions, other models have been proposed that generalise SR to three (or more) dimensions. We classify such models, in which the set of agents is homogenous, as models of \emph{Three-Dimensional Roommates} (3DR). Other models involve coalitions of a fixed size $k \geq 3$ and have been referred to as \emph{multidimensional roommates}.

As defined originally, (two-dimensional) SR could be viewed as a hedonic game in which any feasible coalition has size two. The solution concept corresponds exactly to core stability (although research on SR predates much of the research on hedonic games and core stability). SR has since been generalised to higher dimensions which, in a similar way, correspond to hedonic games.

The earliest known model of this type is the \emph{Three-Person Stable Assignment problem} (\mysymbolfirstusedefinition{symboldef:threepsa}{3PSA}), which was proposed by Ng and Hirschberg~\cite{NH91} in 1991, as a counterpart to 3GSM. An instance of 3PSA comprises a set $N$ of $3n$ agents and a strict preference list $P_{\alpha_i}$ of each agent $\alpha_i$ over all pairs of agents in $N \setminus \{ \alpha_i \}$. Note that in terms of hedonic games, this system of preferences is equivalent to LCs (see Section~\ref{sec:lit_review_hedonicgames}). Let $P$ be the collection of preference lists  $P_{\alpha_i}$ for each agent $\alpha_i$. A \emph{triple} is an unordered set of three agents and that a \emph{matching} is a partition of $N$ into $n$ triples. Given an agent $\alpha_i$ and a matching $M$, we denote by $M(\alpha_i)$ the triple in $M$ that contains $\alpha_i$. For any agent $\alpha_i$ and two triples $r$ and $s$, we say that $\alpha_i$ prefers $r$ to $s$, denoted $r \succ_{\alpha_i} s$, if $r \setminus \{ \alpha_i \}$ precedes $s \setminus \{ \alpha_i \}$ in $P_{\alpha_i}$. Given a matching $M$, we say that a triple $t$ is \emph{blocking} if each agent $\alpha_i$ in $t$ prefers $t$ to $M(\alpha_i)$. A matching is \emph{stable} if it does not admit a blocking triple. Ng and Hirschberg proved that, as in the case of 3GSM, a given instance $(N, P)$ of 3PSA may not contain a stable matching and the associated existence problem is $\NP$-complete. We remark that, unlike a general hedonic game, a blocking triple has size three, so the verification problem for both variants is solvable in $O(|N|^3)$ time so this existence problem belong to $\NP$.

As well as the aforementioned work on 3GSM, in his 2007 paper Huang \cite{Huang07conference} considered two variants of 3PSA. In the first variant, he supposed that agents provide preference lists over pairs of agents (i.e.\ agents have LCs) that are \emph{consistent}. The definition of consistent preferences in 3PSA is analogous to the definition of consistent in 3GSM (discussed in Section~\ref{sec:lit_review_multidimensionalmatching}). In 3PSA, we say that preferences are consistent in an instance of 3PSA if each agent $\alpha_i$ has an underlying strict preference list $P'_{\alpha_i}$ over the agents in $N\setminus \{ \alpha_i \}$. For any agent $\alpha_i$, the strict preference list $P_{\alpha_i}$, which is a total order over pairs, must be a linear extension of the product order over $\{ \{ \alpha_j, \alpha_k \} : (\alpha_j, \alpha_k) \in (N \setminus \{ \alpha_i \})^2 \}$ with respect to $P'_{\alpha_i}$. In the second variant, he considers additively separable preferences under a PON restriction (see Section~\ref{sec:lit_review_hedonicgames}). He showed that, in a given instance of 3PSA in which the preferences are either consistent, or are additively separable and follow the PON restriction, a stable matching may not exist and the associated decision problem is $\NP$-complete. As in the case of 3GSM, Huang also showed that various existence problems relating to so-called strongly-stable, super-stable, and ultra-stable matchings are $\NP$-complete.

Later in 2007, Iwama et al.~\cite{IMO07} considered another closely-related three-dimensional generalisation of SR. They also considered stable matchings characterised by the absence of a blocking triple. They supposed that each agent $\alpha_i$ has an strict preference list $P_{\alpha_i}$ over all agents in $N \setminus \{ \alpha_i \}$, and defined implicitly a set extension rule that is similar to consistent preferences (as defined by Huang). In the model of Iwama et al., any agent $\alpha_i$ prefers some triple $r$ where $\alpha_i \in r$ to another triple $s$ where $\alpha_i \in s$ and $s \neq r$ if $r \setminus \{ \alpha_i \}$ precedes $s \setminus \{ \alpha_i \}$ in the product order $\{ \{ \alpha_j, \alpha_k \} : (\alpha_j, \alpha_k) \in (N \setminus \{ \alpha_i \})^2 \}$ with respect to $P_{\alpha_i}$. We remark that, like $\mathscr{B}$ and $\mathscr{W}$-preferences, this rule defines a partial order over possible coalitions even though each agent has a strict preference list over possible partners. Interestingly, this set extensmion rule is almost identical to the construction of a so-called power-ordered set from a partially ordered set as defined by Bossong and Schweigert~\cite{Bossong2006} the previous year, in the context of formalised decision making. In 2011, Delort, Spanjaard and Weng~\cite{committeeselectiondelort} applied the aforementioned \emph{Bossong-Schweigert extension} rule to the problem of committee forming, which is closely related to hedonic games. In 2015, Lang et al.~\cite{LRRSS15} applied a generalised version of the Bossong-Schweigert rule to a model of a hedonic game.

In 2008, Iwama et al.~\cite{IMO08} considered an optimisation problem using the same model as in their 2007 paper, and showed that it was $\APX$-hard.

In 2009, Arkin et al.~\cite{ABEOMP09} presented another variant of 3PSA called \emph{Geometric 3D-SR}. In this model, agents have additively separable preferences. The agents' valuations are derived from their relative positions in a metric space. The authors also defined a more general version of stability, specific to additively separable preferences, called $\alpha$-stability, in which an agent prefers one pair to another pair if the sum of relative distances to the agents in the first pair is at least $\alpha$ times smaller than that of the second pair. The authors showed that a $2$-stable matching always exists and can be found in polynomial time. They also presented an instance in which no ($1$\nobreakdash-)stable matching exists but left open the complexity of the corresponding existence problem. In 2013, Deineko and Woeginger~\cite{DEINEKO20131837} resolved this open question by showing that this existence problem is $\NP$-complete. In fact, Chen and Roy~\cite{chen2022euclidean} later strengthened this result (in 2022) to show that \emph{Geometric $k$D-SR} is $\NP$-complete even when the metric space is the Euclidean plane.

In 2014, Ostrovsky and Rosenbaum~\cite{ostrovskyrosenbaum2014} revisited both 3GSM and 3PSA. For each model they considered two related optimisation problems. The first, called \textit{Maximally Stable Matching} (MSM), involves finding a matching with the maximum number of non-blocking triples. The second, called \textit{Maximum Stable Sub-matching} (MSS), involves finding a \emph{sub-matching} that involves only a subset of the agents. The problem is to find a submatching of maximum size, i.e.\ involving as many agents as possible, which is stable when considering only blocking triples involving agents in the submatching. They showed, in the context of both 3GSM and 3PSA, that MSM and MSS are $\NP$-hard to approximate within some fixed constant factor (i.e.\ that they are $\APX$-hard~\cite{ACGKMP99}). Positively, they also described a simple greedy algorithm that returns constant factor approximations for both MSM and MSS for both 3GSM and 3PSA.

In 2020, Boehmer and Elkind~\cite{Boe20} considered a number of different models of \emph{multidimensional roommates}. In each model they supposed that the agents have \emph{types} and an agent's preference between two coalitions depends only on the proportion of agents of each type in each coalition. They showed that, for a number of different solution concepts, the related existence problems are $\NP$-hard, although also that many are solvable in linear time if the room size is a fixed constant. Notably, they presented an integer programming-based algorithm that can solve the stability existence problem in polynomial time.

Also in 2020, Bredereck et al.~\cite{Bre20} considered two variations of multidimensional roommates that involve either a \emph{master list} or \emph{master poset}, from which all the agents' preference lists are derived. In the case of a master list, each agent obtains their preference list by deleting from the master list any pair that contains themselves. Surprisingly, the authors discovered that a stable matching may not exist in this model even if all agents' preference lists are derived from a single master list. They presented two positive results relating to restrictions of the problem involving a master poset, although they also showed, for either a master list or master poset, that in general the stability existence problem is either $\NP$-hard or $\W[1]$-hard, in terms of three well-motivated parameters.

\subsection{Other models}
\label{sec:lit_review_othermodels}

Various other models have been proposed in the literature that involve coalitions of restricted, but not necessarily fixed, size. For example, some problem models only require the size of any feasible coalition to be within some lower and upper bound. In this section we review a selection of such models.

In 2011, Aziz et al.~\cite{ABH11} considered Pareto optimal coalition partitions, in a variety of hedonic game variants. In one variant, which has since been termed a \emph{flatmate game}, any coalition must have size at most three. They showed that the Pareto optimality existence problem is $\NP$-hard in this particular model. The authors also presented polynomial-time solvability results in a model involving $\mathscr{W}$-preferences and for a model in which the maximum size of any coalition is two. In 2020, Brandt and Bullinger~\cite{Brandt2020FindingAR} presented a paper studying the existence of popular partitions in a number of hedonic game variants, including flatmate games. They applied a variety of preference structures and presenting results relating to partitions that are popular, \emph{strongly popular}, and \emph{mixed popular} (a \emph{mixed popular partition} is in fact a distribution over coalition partitions).

In 2015, Wright and Vorobeychik~\cite{WrightV15} studied the \emph{Team formation problem}, an instance of which resembles an ASHG. Taking a game-theoretic approach, they assumed that agents act competitively and strategically. They compared four different algorithms in terms of so-called \emph{strategy-proofness}, welfare, and fairness, focussing on experimental performance rather than algorithmic complexity. Notably, their model includes constraints on the sizes of coalitions with a lower and upper bound.
Other research works, mostly in the fields of economics and operations research, have since studied the Team formation problem~\cite{yekta2018finding}.

In 2018, Sless et al.~\cite{Sless18} proposed a model that can be viewed as a type of ASHG with symmetric preferences. They argued that a strong practical motivation exists for considering coalitions of restricted size, and thus focused on the existence of coalition partitions that contain exactly $k$ coalitions, for some fixed $k \geq 1$. They presented a number of theoretical and empirical results relating to this model. They showed, using a connection to the so-called Min-$k$-Cut problem, that the problem of finding a partition with maximum utilitarian welfare can be solved in polynomial time in the restricted case in which $k$ is fixed and there are, in a precise sense, relatively few negative edges. Otherwise, they showed that this construction problem is $\NP$-hard. They also presented a polynomial-time solvability result for a problem in which a central organiser can add edges to the instance. Notably, they proposed a heuristic-based procedure for the problem of finding a core stable partition that maximises utilitarian welfare. They showed that this procedure performs well on instances derived from real-world data from social networks.

In 2019, Cseh et al.~\cite{CSEH2019} considered Pareto optimal matchings in a general model that is comparable to a hedonic game. In this model there is a set of rooms with integer sizes, and any coalition must be allocated to exactly one room where the size of the room is exactly the size of the coalition. They studied two specific variants of this model, applying $\mathscr{B}$- and $\mathscr{W}$-preferences. They showed if $\mathscr{B}$-preferences are used and the agents' preference lists are strict then a polynomial-time algorithm based on serial dictatorship can be used to construct Pareto optimal matchings in polynomial time. They also showed that, in a number of other circumstances, a Pareto optimal matching may not exist and that in many cases the associated existence problems are either $\NP$-hard or $\NP$-complete.

In 2022, Li et al.~\cite{Li2022}, considered a model that can be viewed as a generalisation of multidimensional roommates with binary and symmetric additively separable preferences. Rather than a fixing the size of a coalition, a partition of the set of $n$ agents must be \emph{balanced}, meaning the number of coalitions $k\leq n$ is fixed and $\lfloor n/k \rfloor \leq |S| \leq \lceil n/k \rceil$ for any coalition $S$ in a feasible partition. They studied an approximation of envy-freeness, termed EF-$r$, in which the utility gained by any envious agent may be up to $r$ in a feasible partition, for some fixed $r \geq 0$ (our definition of envy-freeness is thus EF-$0$). Interestingly, they applied results from discrepancy theory to show that an approximate-envy free partition with a particular fixed asymptotic bound must exist, and can be found in polynomial time. They also considered restricted sets of instances, such those in which the underlying structure is a tree. They showed that in such an instance, an EF-$1$ partition must exist and can be found in polynomial time. A coalition partition of $3n$ agents in which every coalition has size three is by definition balanced, so some of the algorithmic results relating to approximate envy-free partitions~\cite[Theorems~9 and~10]{Li2022} also apply in a more restricted model of 3DR with additively separable preferences (such as the model that we define in Chapter~\ref{c:threed_sr_as}).

Also in 2022, Bil\`o et al.~\cite{Bilo22} proposed another model, which can be viewed as a type of ASHG. They analogized their model to a dinner party situation in which $n$ agents are assigned to exactly $k$ tables. They argued that Nash stability and core stability might not make sense in such a setting, since there is no free table. Instead, they considered three successively weaker solution concepts involving two agents that swap places. In a partition that is \emph{strictly swap stable}, no two agents $\alpha_i, \alpha_j$ exist where if $\alpha_j$ and $\alpha_j$ swap places then the utilities of both agents strictly increase. In a partition that is \emph{swap stable}, no two agents $\alpha_i, \alpha_j$ exist where if $\alpha_j$ and $\alpha_j$ swap places then the utility of $\alpha_i$ strictly increases and the utility of $\alpha_j$ does not decrease. In a partition $\pi$ that is \emph{swap stable under transferable utilities}, no two agents $\alpha_i, \alpha_j$ exist where if $\alpha_i$ and $\alpha_j$ swap places, in a new partition $\pi'$, then $u_{\alpha_i}(\pi') + u_{\alpha_j}(\pi') > u_{\alpha_i}(\pi) + u_{\alpha_j}(\pi)$. They remarked that a variation of these solution concepts had been previously studied in the context of the Stable Marriage problem as \emph{exchange stability}~\cite{Alc94}. We remark here that envy-freeness implies strict swap stability. In their paper, Bil\`o et al.\ show that for any of the three concepts, a feasible partition must exist and can be found by iteratively executing improving swaps from an arbitrary starting partition. Notably, they showed that the relevant existence problem is $\PLS$-complete in general but the iterative process converges in polynomial time if preferences are binary. They also analysed the problem of maximising utilitarian welfare, and the so-called \emph{price of anarchy}, and \emph{price of stability}, which are defined as follows. For any of the three stability concepts, the price of anarchy (stability) is the worst- (best-)case ratio between the utilitarian welfare of an arbitrary stable matching and the maximum possible utilitarian welfare over all possible partitions.

% \section{Integer Programming}

% \emph{Linear Programming} (LP) refers to a general-purpose technique to model and solve combinatorial optimisation problems~\cite{linearprogrammingvanderbei2020}. It involves modelling the problem using a set of variables, a set of linear inequalities (constraints), and a linear objective function. The model is referred to as a \emph{linear program}. Solving an arbitrary linear program means finding a matching of values to the variables that satisfies the constraints and maximises or minimises the objective function. There are practical algorithms to solve linear programs and in general solving an arbitrary linear program can be done efficiently~\cite{linearprogrammingvanderbei2020}. Some linear programming algorithms run in polynomial time.

% For many combinatorial problems a variation on LP is used in which the values of the variables must be integral in any feasible solution. This restriction is called \emph{Integer Linear Programming} (ILP, or simply IP). Unlike LP, solving a general integer program is $\NP$-hard~\cite{GJ79}. Nevertheless, a number of IP solvers exist, which are often efficient in practice and can be a practical way to find solutions to $\NP$-hard optimisation problems~\cite{Bixby12}. There is no general method to formulate an integer program to represent an arbitrary optimisation problem.  
\chapter{Approximability of Three-Dimensional Stable Matching with Cyclic Preferences}
\label{c:three_dsm_cyc}
\chaptermark{Approximability of 3-DSM-CYC}

\section{Introduction}
\label{sec:three_dsm_cyc_intro}
In this chapter we study the approximability of the Three-Dimensional Stable Matching problem with Cyclic Preferences (\mysymbolfirstusedefinition{symboldef:threedsmcyc}{3-DSM-CYC}, also known as 3DSM \cite{BM10} or c3DSM \cite{Pashkovich20}).

As we saw in Chapter~\ref{c:lit_review}, the question of whether every instance of 3-DSM-CYC contains a stable matching was open for several decades. It was only in 2019 that Lam and Plaxton~\cite{Plaxton3DSMCYCJournal} showed that a given instance of 3-DSM-CYC need not contain a stable matching and that the associated decision problem is $\NP$-complete. A natural next step is to consider approximately stable matchings, as we do here. To our knowledge, we present the first theoretical results on the approximability of 3-DSM-CYC. 

In this chapter we consider the optimisation problem of finding a matching with the maximum number of non-blocking families, which we call the {3-DSM-CYC Maximally Stable Matching problem} (\mysymbolfirstusedefinition{symboldef:threedsmcyc_msm}{3-DSM-CYC-MSM}).

We begin, in Section~\ref{sec:three_dsm_cyc_unrestricted_preferences}, by showing that an existing approximation algorithm for 3GSM-MSM, which is a closely related problem, can be used to devise a $9/4$-approximation algorithm for 3-DSM-CYC-MSM (Theorem~\ref{thm:three_dsm_cyc_ninefour}). We then show that a simple algorithm based on serial dictatorship gives an improved approximation ratio of $6/5$ (Theorem~\ref{thm:three_dsm_cyc_unrestricted}).

Next, in Section~\ref{sec:three_dsm_cyc_masterlist}, we consider a situation in which there exists a master preference list over all agents of one type, and quantify the similarity between the preference list of any agent of the previous type and that master list, in terms of a specific distance metric. We extend the approximation algorithm for 3-DSM-CYC-MSM and show that if the maximum distance is sufficiently small then as it is further reduced, the approximation ratio of the algorithm decreases from $6/5$ to $1$ (Theorem~\ref{thm:three_dsm_cyc_ml}). As a corollary, we show that if every agent of one type has the same preference list then the algorithm returns a matching that is stable (Corollary~\ref{cor:three_dsm_cyc_ml_stab}, which is also implied by a result of Escamocher and O'Sullivan \cite{Escamocher2018}).

Finally, in Section~\ref{sec:three_dsm_cyc_conclusion}, we recap on our results and discuss some directions for future work.

We proceed with some formal definitions and notation. An instance of 3-DSM-CYC comprises a set $N$ of $3n$ agents and a strict preference list for each agent $\alpha_i$, labelled $P_{\alpha_i}$. Each agent in $N$ has one of three \emph{types}, which we call \emph{man}, \emph{woman}, and \emph{dog}. There are $n$ agents of each type, and the agents of each type are labelled $U = \{ u_1, u_2, \dots, u_n \}$, $W = \{ w_1, w_2, \dots, w_n \}$, and $D =  \{ d_1, d_2, \dots, d_n \}$. The types have a cyclic order in which $W$ follows $U$, $D$ follows $W$, and $U$ follows $D$. Each agent's preference list $P_{\alpha_i}$ describes a strict order all agents in the next type. We say that an agent $\alpha_i$ \textit{prefers} $\beta_j$ to $\beta_k$ if $\beta_j$ precedes $\beta_k$ in the preference list $P_{\alpha_i}$ of $\alpha_i$. A \textit{family} is a $3$-tuple $( u_i, w_j, d_k ) \in U\times W\times D$. A \textit{matching} is a set of families where each agent in $N$ is contained in exactly one family. Given an agent $\alpha_i$ and a matching $M$, we denote by $M(\alpha_i)$ the family in $M$ that contains $\alpha_i$. Given a matching $M$, we say that a family $f$ is \textit{blocking} if each agent $\alpha_i$ in $f$ prefers the agent of the next type in $f$ to the agent of the next type in $M(\alpha_i)$. A matching is \textit{stable} if it does not contain a blocking family. Let $P$ be the collection of preference lists  $P_{\alpha_i}$ for each agent $\alpha_i$. For any instance $(N, P)$ of 3-DSM-CYC and any matching $M$, we denote by $\textrm{bf}(M, (N, P)) \subseteq U \times W \times D$ the set of families that block $M$ in $(N, P)$. Conversely, we denote by $\textrm{nbf}(M, (N, P)) = (U \times W\times D) \setminus \textrm{bf}(M, (N, P))$ the set of families that do not block $M$ in $(N, P)$. When the instance in question is clear from context, we simply write $\textrm{bf}(M)$ or $\textrm{nbf}(M)$. Formally, 3-DSM-CYC-MSM is the optimisation variant of 3-DSM-CYC in which the objective is to maximise $|\textrm{nbf}(M, (N, P))|$.

% The decision problem associated with 3-DSM-CYC is to test if a given instance contains a stable matching. We define the \emph{3-DSM-CYC Maximally Stable Matching problem} (3-DSM-CYC-MSM) as follows: given an instance of 3-DSM-CYC, construct a matching with the maximum number of non-blocking families. 




\section{Unrestricted preferences}
\label{sec:three_dsm_cyc_unrestricted_preferences}
In this section we consider the approximability of 3-DSM-CYC-MSM in the general case. We first apply an existing polynomial-time approximation algorithm for a related problem and construct a polynomial-time $9/4$-approximation algorithm. We then improve on this result and show that a polynomial-time algorithm based on serial dictatorship constitutes a $6/5$-approximation algorithm.

There is a close relationship between 3-DSM-CYC and the \emph{Three-Gender Stable Marriage Problem} (3GSM, introduced in Chapter~\ref{c:lit_review}). Recall that in 3-DSM-CYC, each agent has a strict preference list over all agents of the next type, and compares two families based on the relative rank of the two agents of the next type in each family. In 3GSM, each agent has a strict preference list over all possible families that they belong to, and compares any two families using on this preference list. Although neither problem is a strict generalisation of the other, we show that, given an instance $(N, P)$ of 3-DSM-CYC, it is possible to construct an instance $(N, P')$ of 3GSM with the same set of agents such that for any matching $M$, if a family blocks $M$ in $(N, P)$ then it also blocks $M$ in $(N, P')$. From this result it follows that an existing $9/4$-approximation algorithm for 3GSM-MSM, which is defined analogously to 3-DSM-CYC-MSM, can be applied to construct a $9/4$-approximation algorithm for 3-DSM-CYC-MSM. The existing $9/4$-approximation algorithm for 3GSM-MSM was presented by Rosenbaum in 2016 \cite{rosenbaum16} and is called Algorithm~\algorithmfont{AMSM}. It is an iterative greedy algorithm which involves, in each iteration, selecting a family that once added to the matching, intersects the maximum number of non-blocking families.

\begin{thm}
\label{thm:three_dsm_cyc_ninefour}
There exists a polynomial-time $9/4$-approximation algorithm for 3-DSM-CYC-MSM.
\end{thm}
\begin{proof}
The approximation algorithm for 3-DSM-CYC-MSM involves constructing a corresponding instance of 3GSM-MSM, which has the same set of agents, and running Algorithm~\algorithmfont{AMSM} \cite{rosenbaum16}.

We first describe how to construct, in polynomial time, a corresponding instance $(N, P')$ of 3GSM such that $|\textrm{nbf}(M, (N, P))| \geq |\textrm{nbf}(M, (N, P'))|$ for any matching $M$. For each agent $u_i$ in $U$ let $P'_{u_i}$ be the ordered list of tuples in $W \times D$ such that if $w_j$ precedes $w_i$ in $P_{u_i}$ then every family containing $w_j$ appears before every family containing $w_i$. The relative order of the families in $P'_{u_i}$ that contain the same agent in $W$ is arbitrary. Construct the preferences of each agent in $W$ and $D$ symmetrically. Now, suppose $M$ is an arbitrary matching in $(N, P)$ and $(u_i, w_j, d_k)$ is a family that blocks $M$ in $(N, P)$. We claim that this family also blocks $M$ in $(N, P')$. First, assume without loss of generality that $M(u_i) = (u_i, w_i, d_i)$. By definition, it must be that $w_j$ appears before $w_i$ in $P_{u_i}$. It follows by the construction of $P'$ that $(u_i, w_j, d_k)$ appears before $(u_i, w_i, d_i)$ in $P'_{u_i}$ and thus that $(u_i, w_j, d_k)$ also blocks $M$ in $(N, P')$. It follows immediately that $|\textrm{bf}(M, (N, P'))| \geq |\textrm{bf}(M, (N, P))|$ and thus that $|\textrm{nbf}(M, (N, P))| \geq |\textrm{nbf}(M, (N, P'))|$, as required.

Since Rosenbaum's~\cite{rosenbaum16} analysis of  Algorithm~\algorithmfont{AMSM} shows that $|\textrm{nbf}(M, (N, P'))| \geq 4n^3/9$, we can conclude that $|\textrm{nbf}(M, (N, P))| \geq 4n^3/9$.
\end{proof}

We now consider an algorithm for 3-DSM-CYC-MSM based on serial dictatorship, which we call Algorithm~\algorithmfont{cyclicSerialDictatorship}, shown in Algorithm~\ref{alg:threed_sr_b_threedsm_cyc_dictatorshipcyc}. In the context of computational social choice, serial dictatorship refers to a type of iterative algorithm used to assign resources to agents. Typically, in each iteration a dictator is selected who is then assigned their most-preferred resource that is still available. The classical application of serial dictatorship is for the problem of \emph{House allocation} \cite{HedonicGamesHOCSC} but it has also been applied to hedonic games \cite{ABS11} and problems involving coalitions of fixed size \cite{CSEH2019}.

The algorithm we present here for 3-DSM-CYC-MSM follows the pattern of serial dictatorship. First, an arbitrary yet-unmatched agent agent in $U$ is selected and labelled $u_i$. Next, $u_i$ selects their most-preferred yet-unmatched agent in $W$ which is labelled $w_j$. Finally, $w_j$ selects their most-preferred yet-unmatched agent in $D$ which is labelled $d_k$. The family $( u_i, w_j, d_k )$ is then added to the matching. A similar algorithm was in fact used by Hofbauer \cite{HOFBAUER201672} in order to construct stable matchings in the setting of $k$-DSM-CYC where $n \leq k + 1$.

\begin{algorithm}
\textbf{Input:} an instance $(N, P)$ of 3-DSM-CYC\\
\textbf{Output:} a matching $M$ in $(N, P)$
\smallskip
\begin{algorithmic}
\caption{Algorithm~\algorithmfont{cyclicSerialDictatorship} \label{alg:threed_sr_b_threedsm_cyc_dictatorshipcyc}} 
\State $V \gets N$
\State $M \gets \varnothing$

\While{$|V| > 0$}
    \State $u_i \gets$ any agent in $U \cap V$
    \State $w_j \gets$ the most-preferred agent in $W \cap V$ according to $P_{u_i}$
    \State $d_k \gets$ the most-preferred agent in $D \cap V$ according to $P_{w_j}$
    
    \State $M \gets M \cup \{ ( u_i, w_j, d_k ) \}$
    \State $V \gets V \setminus \{ u_i, w_j, d_k \}$
\EndWhile
\State \textbf{end while}
\smallskip

\State \Return $M$
\end{algorithmic}
\end{algorithm}

It is straightforward to show that Algorithm~\algorithmfont{cyclicSerialDictatorship} returns a matching $M$ in polynomial time. We now analyse its approximation ratio and show that our analysis is tight. This involves placing an upper bound on the number of blocking families in a matching $M$ returned by the algorithm. To do this, we consider each family in $M$ in the order that they were added to $M$ in the algorithm and count only the blocking families that intersect that family and do not intersect any previous family. To simplify the analysis, without loss of generality suppose the $( u_1, w_1, d_1 )$ was the first family added to $M$, $( u_2, w_2, d_2 )$ was the second family added to $M$, and so on.

\begin{thm}
\label{thm:three_dsm_cyc_unrestricted}
There exists a polynomial-time $6/5$-approximation algorithm for 3-DSM-CYC-MSM.
\end{thm}
\begin{proof}
There are exactly $n$ iterations of the while loop so it is straightforward to show that the algorithm runs in polynomial time. For each $i$ where $1\leq i \leq n$, let $F_i = ( u_i, w_i, d_i )$ be the family added to $M$ in the $i\textsuperscript{th}$ iteration. Let $U_i$, $W_i$, and $D_i$ be the set of agents in of $U \cap V$, $W \cap V$, and $D \cap V$ respectively at the start of the $i\textsuperscript{th}$ iteration. It follows that $U_i = \{ u_i, u_{i+1}, \dots, u_n \}$, $W_i = \{ w_i, w_{i+1}, \dots, w_n \}$, and $D_i = \{ d_i, d_{i+1}, \dots, d_n \}$.

For each $i$ where $1\leq i \leq n$, let $S_i$ be the set of families that block $M$, have a non-empty intersection with $F_i$, and have an empty intersection with $F_j$ for every $1 \leq j < i$. It follows that $S_n = \varnothing$ since any family that blocks $M$ and intersects the final family $F_n$ must contain some agent not in $F_n$, which must belong to some previous family $F_j$ where $1 \leq j < n$. 
We can now define $\text{bf}(M)$ in terms of $S_i$:
\begin{align*}
    \text{bf}(M) = \bigcup\limits_{i=1}^{n-1} S_i\enspace.
\end{align*}
By definition, the sets $S_i$ are pairwise disjoint so it follows that
\begin{align}
    |\text{bf}(M)| = \sum\limits_{i=1}^{n-1} |S_i|\enspace. \label{eqn:threedsm_cyc_si}
\end{align}
We now place an upper bound on $|S_i|$ for any $i$ where $1\leq i \leq n - 1$. For any such $i$, consider the $i\textsuperscript{th}$ iteration of the while loop. By the algorithm, $w_i$ is the most-preferred agent in $W_i$ according to $P_{u_i}$. It follows that any family that blocks $M$ and contains $u_i$ must also contain some agent in $W$ that has already been added to a family $F_j$ in $M$ where $j < i$. By the definition of $S_i$, it follows that no family in $S_i$ contains $u_i$. Similarly, since $d_i$ is the most-preferred agent in $D_i$ according to $P_{w_i}$, any family in $S_i$ that contains $w_i$ must also contain some agent in $D$ that has already been added to a family $F_j$ in $M$, where $j < i$. It follows similarly that no family in $S_i$ contains $w_i$.

It remains that every family in $S_i$ contains $d_i$. Consider an arbitrary family $( u_j, w_k, d_i )$ in $S_i$. By the definition of $S_i$ it must be that $j > i$ and $k > i$. Note that for any choice of $u_j$, by assumption $u_j$ prefers $w_k$ to $w_j$. It follows by the algorithm that $w_k \notin W_j$ and thus $k < j$. 

In conclusion, we have shown that for any family $( u_j, w_k, d_i ) \in S_i$ it must be that $i < j \leq n$ and $i < k < j$. We can now count $|S_i|$ by considering each possible value of $j$, from $i+1$ to $n$ inclusive and each possible value of $k$, of which there are $j - i - 1$. It follows that
\begin{align}
    |S_i| &\leq \sum\limits_{j=i+1}^{n} (j - i - 1)\nonumber\\
    &= \frac{(n-i)(n-i-1)}{2} \label{eqn:three_dsm_cyc_sizeofsi}\enspace.
\end{align}
Now
\begin{align}
    |\text{bf}(M)| &= \sum\limits_{i=1}^{n-1} |S_i| && \mbox{Equation~\ref{eqn:threedsm_cyc_si}}\nonumber\\
    &\leq \sum\limits_{i=1}^{n-1} \frac{(n - i)(n - i - 1)}{2} && \mbox{by Inequality~\ref{eqn:three_dsm_cyc_sizeofsi}}\nonumber\\
    &= \frac{n^3}{6} - \frac{n^2}{2} + \frac{n}{3}\label{eqn:threedsm_cyc_bf_basic}\enspace.
\end{align}
Suppose $M^*$ is a matching in $(N, P)$ with the maximum number of non-blocking families. The approximation ratio of the algorithm is thus
\begin{align*}
    \frac{|\textrm{nbf}(M^*)|}{|\textrm{nbf}(M)|} &\leq \frac{n^3}{|\textrm{nbf}(M)|}\\
    &= \frac{n^3}{n^3 - |\textrm{bf}(M)|} && \mbox{by the definition of $\textrm{nbf}$}\\
    &\leq \frac{6n^2}{5n^2 + 3n - 2} && \mbox{by Inequality~\ref{eqn:threedsm_cyc_bf_basic}}\\
    &\leq \frac{6}{5} && \mbox{since $n \geq 1$.} 
\end{align*}
\end{proof}
It is desirable to show that the analysis is tight, for example by constructing an instance $(N, P)$ such that there exists some execution of the algorithm that returns a matching $M$ for which $|\textrm{nbf}(M^*)|/|\textrm{nbf}(M)| = 6/5$, where $M^*$ is some matching in $(N, P)$ with the maximum number of non-blocking families. We show that this analysis is tight asymptotically, by constructing an instance $\mathcal{I}_n$ for some fixed $n \geq 1$, where the approximation ratio obtained by Algorithm~\algorithmfont{cyclicDicatorship} on $\mathcal{I}_n$ in the worst case is $6/5 - o(1)$.

The structure of the preferences of the agents in $\mathcal{I}_n$ corresponds directly to the counting argument used in the proof of Theorem~\ref{thm:three_dsm_cyc_unrestricted}. For any fixed $n$, construct $\mathcal{I}_n$ as follows. Let $U = \{ u_1, u_2, \dots, u_n \}$, $W =  \{ w_1, w_2, \dots, w_n \}$, and $D = \{ d_1, d_2, \dots, d_n \}$, and for each $i$ where $1\leq i \leq n$ let
\begin{flalign*}
\setlength\arraycolsep{2pt}
\begin{array}{r c c c c}
P_{u_i} :& w_1 & w_2 & \dots & w_{n}\\
P_{w_i} :& d_1 & d_2 & \dots & d_{n}\\
P_{d_i} :& u_n & u_{n-1} & \dots & u_1
\end{array}
\end{flalign*}
% \begin{flalign*}
% \setlength\arraycolsep{2pt}
% \begin{array}{r l l r}
% P_{u_i} :& [& W \text{ in ascending order} &]\\
% P_{w_i} :& [& D \text{ in ascending order} &]\\
% P_{d_i} :& [& U \text{ in descending order} &]\\
% \end{array}
% \end{flalign*}
In the algorithm the selection of each agent in $U \cap V$ is arbitrary, so suppose the algorithm selects $u_1$ in the first iteration, $u_2$ in the second iteration, and so on. It follows that the first family added to $M$ is $( u_1, w_1, d_1 )$ and $V = N \setminus \{ u_1, w_1, d_1 \}$ at the start of the second iteration. It then follows that the next family added to $M$ is $( u_2, w_2, d_2 )$. In general, in the $i\textsuperscript{th}$ iteration $w_i$ must be the most-preferred agent in $W \cap V$ according to $P_{u_i}$ and $d_i$ must be the most-preferred agent in $D \cap V$ according to $P_{w_i}$. It follows that the algorithm returns $M = \{ ( u_1, w_1, d_1 ), ( u_2, w_2, d_2 ), \dots, ( u_n, w_n, d_n ) \}$. As in the proof of Theorem~\ref{thm:three_dsm_cyc_unrestricted}, for each $i$ where $1\leq i \leq n$, let $F_i$ be the family $( u_i, w_i, d_i )$ selected in the $i\textsuperscript{th}$ iteration and let $S_i$ be the set of families that block $M$, have a non-empty intersection with $F_i$ and an empty intersection with $F_j$ for every $1\leq j < i$. Note that as before, $S_n = \varnothing$. As before, it can be shown that for any $i$ where $1\leq i \leq n$, no family in $S_i$ contains $u_i$ and no family in $S_i$ contains $w_i$. It remains that each family in $S_i$ contains $d_i$. Note that by the construction of $\mathcal{I}_n$, $d_i$ prefers to $u_i$ each agent $u_j$ where $j > i$. Moreover, any such $u_j$ prefers to $w_j$ each agent $w_k$ where $k < j$, and any such $w_k$ prefers to $d_k$ each agent $d_i$ where $i < k$. It follows that $S_i$ contains the family $( u_j, w_k, d_i )$ for each $j$ where $i < j \leq n$ and each $k$ where $i < k < j$. As in the proof of Theorem~\ref{thm:three_dsm_cyc_unrestricted}, it is straightforward to show that no other families are contained in $S_i$. It follows that since
\begin{align*}
    S_i = \{ ( u_j, w_k, d_i ) : i < j \leq n \text{ and } i < k < j \}
\end{align*}
it must be that
\begin{align*}
    |S_i| &= \sum\limits_{j = i + 1}^{n} (j - i - 1)
\end{align*}
which shows that the upper bound on $|S_i|$ shown in Inequality~\ref{eqn:three_dsm_cyc_sizeofsi} in Theorem~\ref{thm:three_dsm_cyc_unrestricted} is tight. The same argument used in the proof of Theorem~\ref{thm:three_dsm_cyc_unrestricted} then shows that
\begin{align}
    |\textrm{nbf}(M)| = \frac{5n^3}{6} + \frac{n^2}{2} - \frac{n}{3} \label{eqn:three_dsm_cyc_sizeofsiintight}\enspace.
\end{align}
We now show that a stable matching exists in $\mathcal{I}_n$. Let $M^* = \{ ( u_i, w_{n - i + 1}, d_{n - i + 1} ) : 1 \leq i \leq n \}$. Suppose for a contradiction that $M^*$ is not stable and thus some family $( u_j, w_k, d_i )$ blocks $M^*$ in $(N, P)$. By the definition of a blocking family, $u_j$ prefers $w_k$ to $w_{n - j + 1}$. By the preference list of $u_j$ it follows that $k < n - j + 1$. Similarly, since $w_k$ prefers $d_i$ to $d_k$ by the preference list of $w_k$ it must be that $i < k$. Since $k < n - j + 1$ and $i < k$ it follows that $j < n - i + 1$. By the construction of $P_{d_i}$, it follows that $u_j$ appears after $u_{n - i + 1}$ in $P_{d_i}$. This is a contradiction since by the definition of a blocking family $d_i$ must prefer $u_j$ to its assigned partner of the next type, $u_{n - i + 1}$. It follows that $M^*$ is stable. Now
\begin{align*}
    \frac{|\textrm{nbf}(M^*)|}{|\textrm{nbf}(M)|} &= \frac{n^3}{|\textrm{nbf}(M)|} && \mbox{since $M^*$ is stable}\\
    &= \frac{6n^2}{5n^2 + 3n - 2} && \mbox{by Equation~\ref{eqn:three_dsm_cyc_sizeofsiintight}}\\
    % &= \frac{6}{5} - \left( \frac{6}{7(n + 1)} + \frac{24}{35(5n - 2)} \right)\\
    &= \frac{6}{5} - \frac{18n - 12}{25n^2 + 15n - 10}\\
    &= \frac{6}{5} - o(1)
\end{align*}
which shows that the analysis is asymptotically tight.

Since all agents of the same type in $\mathcal{I}_n$ had the same preference list, it was straightforward to construct a stable matching. In the next section, we consider instances in which the preference lists of all agents of at least one type are derived from a master list, and consider the problem of finding a matching with the maximum number of non-blocking families.


\section{Preferences derived from a master list}
\label{sec:three_dsm_cyc_masterlist}
In this section, we consider a situation in which the preference lists of all agents in at least one of the sets $U, W, D$ are in some way similar. Specifically, we consider a situation in which there exists a master preference list over all agents of one type, and quantify the similarity between the preference list of any agent of the previous type and that master list, in terms of a specific distance metric. We extend Algorithm~\algorithmfont{cyclicSerialDictatorship} for this problem and show that if this distance is sufficiently small then as the distance is reduced the approximation ratio of the algorithm decreases from $6/5$ to $1$.

The assumption of a master list has been made in a number of problems involving matching under preferences \cite{AMUP}, including the multidimensional roommates problem \cite{Bre20}. In the setting of 3-DSM-CYC, Escamocher and O'Sullivan \cite{Escamocher2018} showed in 2018 that if all agents of one type have the same preference list then the number of stable matchings in that instance is exponential in $n$. In a paper on Constraint Programming models for 3-DSM-CYC, Cseh et al.\ \cite{Cseh2022} showed that the serial dictatorship-style algorithm that we present here can also be used to construct \emph{strongly stable} matchings in 3-DSM-CYC.

% The existence of a master list is a situation might arise in a real-world problem of matching under preferences.  

Suppose hereafter that all agents of at least one type have preferences derived from a master list. Without loss of generality, suppose the preference lists of all agents in $D$ are close to, relative to a measure that will be defined, some master list $\hat{P}$ over all agents in $U$.

We quantify the similarity between the preference list of each agent in $D$ to the master list $\hat{P}$. Here we use the \emph{Kendall tau distance} \cite{KendallTauCitation} (also known as \emph{bubblesort distance} or \emph{Kemeny distance} \cite{HedonicGamesHOCSC}) two linear orders over some set $S$. For any two elements $s_i, s_j$ in $S$, we say that $(s_i, s_j)$ is a \emph{discordant pair} between $L_1$ and $L_2$ if $s_i \succ_{L_1} s_j$ and $s_j \succ_{L_2} s_i$. The Kendall tau distance $\tau(L_1, L_2)$ between $L_1$ and $L_2$ is the number of discordant pairs between $L_1$ and $L_2$. The Kendall tau distance can be equivalently defined as the number of swaps made by the bubblesort algorithm when sorting $L_1$ according to the order of the elements in $L_2$ (or vice-versa) \cite{KendallTauCitation}. By definition, $0 \leq \tau(L_1, L_2) \leq \binom{n}{2}$ for any $L_1$ and $L_2$.

Suppose the maximum Kendall tau distance between $\hat{P}$ and the preference list $P_{d_i}$ of any agent $d_i \in D$ is $c$.  

\begin{thm}
\label{thm:three_dsm_cyc_ml}
If $c \leq n$ then there exists a polynomial-time algorithm for 3-DSM-CYC-MSM with approximation ratio ${6}/{(6 - (3d^2 - 2d^3))}$, where $d = c/n$.
\end{thm}
\begin{proof}
% If $n \leq 3$ then a stable matching must exist, and can be found in polynomial time \cite{BGJK04}. Suppose then that $n \geq 4$. 
The algorithm is a variation of Algorithm~\algorithmfont{cyclicSerialDictatorship}. The only difference is that in each round, instead of selecting a dictator agent in $U$ arbitrarily, the most-preferred yet-unmatched agent in $U$ is selected according to the master list $\hat{P}$. Intuitively, it follows that $u_i$ is among the most-preferred agents in the preference list of $d_k$ and thus the number of blocking families that contain $d_k$ is minimised. 

It is straightforward to show that this algorithm returns a matching $M$ in polynomial time. Without loss of generality, assume that $\hat{P} : u_1\ u_2\ \dots\ u_n$ and that $( u_1, w_1, d_1 )$ was the first family added to $M$, $( u_2, w_2, d_2 )$ was the second family added to $M$, and so on. 

For an arbitrary $i$ where $1 \leq i \leq n - 1$, consider the $i\textsuperscript{th}$ iteration of the while loop. As before in the proof of Theorem~\ref{thm:three_dsm_cyc_unrestricted}, it must be that each family in $S_i$ contains $d_i$. Also as before, for any family $( u_j, w_k, d_i )$ in $S_i$ it must be that $i < j \leq n$ and $i < k < j$ so
\begin{align}
    |S_i| &\leq \sum\limits_{j=i+1}^{n} (j - i - 1)\nonumber\\
    &= \frac{(n-i)(n-i-1)}{2}\label{eqn:three_dsm_cyc_sizeofsiml}\enspace.
\end{align}
We now show that because of the master list, if $i < n - c$ then the upper bound on $|S_{i}|$ shown in Inequality~\ref{eqn:three_dsm_cyc_sizeofsiml} can be improved. Consider an arbitrary $i$ where $1 \leq i < n - c$.

Consider some family $( u_j, w_k, d_{i} )$ in $S_{i}$. As before in the proof of Theorem~\ref{thm:three_dsm_cyc_unrestricted} it must be that $i < j$ and $i < k < j$.

Suppose for a contradiction that $j > i + c$ and thus that $j \geq i + c + 1$. We shall identify a number of discordant pairs between $P_{d_{i}}$ and the master list $\hat{P}$. By assumption, the position of $u_{i}$ in the master list $\hat{P}$ is exactly $i$. Suppose $\hat{i}$ is the position of $u_{i}$ in $P_{d_{i}}$. It follows that the number of discordant pairs between $P_{d_{i}}$ and $\hat{P}$ that contain $u_{i}$ is at least $|\hat{i} - i|$. Similarly, the number of discordant pairs between $P_{d_{i}}$ and $\hat{P}$ that contain $u_j$ is at least $|\hat{j} - j|$ where $\hat{j}$ is the position of $u_j$ in $P_{d_{i}}$. At most one discordant pair contains both $u_j$ and $u_{i}$ so the total number of discordant pairs containing either $u_j$ or $u_i$ is at least
\begin{align*}
    |\hat{i} - i| + |\hat{j} - j| - 1
    &= |\hat{i} - i| + |j - \hat{j}| - 1\\
    &\geq \hat{i} - i + j - \hat{j} - 1\\
    &\geq \hat{i} + c - \hat{j} && \mbox{since $j \geq i + c + 1$}\\
    &\geq c + 1 && \mbox{since $\hat{j} < \hat{i}$}
\end{align*}
which contradicts the definition of $c$ as the maximum number of discordant pairs between the preference list of any agent in $D$ and the master list $\hat{P}$. It follows that $j \leq i + c$. In conclusion, we have shown that $i < j \leq i + c$ and $i < k < j$.
It follows that, for any $i$ where $1\leq i < n - c$,
\begin{align}
    |S_{i}| &\leq \sum\limits_{j={i}+1}^{i + c} (j - i - 1)\nonumber\\
    &= \frac{c(c-1)}{2}\label{eqn:threedsm_cyc_si_ml}\enspace.
\end{align}
By Inequality~\ref{eqn:threedsm_cyc_si},
\begin{align}
    |\text{bf}(M)| &= \sum\limits_{i=1}^{n-1} |S_i| && \mbox{as before in the proof of Theorem~\ref{thm:three_dsm_cyc_unrestricted}}\nonumber\\
    &\leq \sum\limits_{i=1}^{n-c-1} \frac{c(c - 1)}{2} + \sum\limits_{i = n - c}^{n - 1} \frac{(n - i)(n - i - 1)}{2} && \mbox{by Inequalities~\ref{eqn:three_dsm_cyc_sizeofsiml} and~\ref{eqn:threedsm_cyc_si_ml}}\nonumber\\
    &= \frac{1}{2}c(n - c - 1)(c - 1) + \frac{1}{6}(c^3 - c) \label{eqn:threedsm_cyc_bf_ml}\enspace.
\end{align}
Suppose $M^*$ is a matching in $(N, P)$ with the maximum number of non-blocking families. By the definition of $\text{nbf}$,
\begin{align}
    |\text{nbf}(M)| &= n^3 - |\text{bf}(M)|\nonumber\\
    &\geq n^3 - \frac{1}{2}c(n - c - 1)(c - 1) - \frac{1}{6}(c^3 - c) && \label{eqn:nbfsize}\mbox{by Inequality~\ref{eqn:threedsm_cyc_bf_ml}}
\end{align}
so the approximation ratio is
\begin{align*}
    \frac{|\textrm{nbf}(M^*)|}{|\textrm{nbf}(M)|} &\leq \frac{n^3}{|\textrm{nbf}(M)|}\\
    &\leq \frac{6n^3}{6n^3 - 3nc^2  + 2c^3 + 3nc - 2c} && \mbox{by Inequality~\ref{eqn:nbfsize}}\\
    &\leq \frac{6n^3}{6n^3- 3nc^2 + 2c^3} && \mbox{since $n \geq 1$}\\
    &= \frac{6n^3}{6n^3 - c^2(3n - 2c)}\\
    &= \frac{6}{6 - (3d^2 - 2d^3)} && \mbox{where $d = c/n$.}
\end{align*}
\end{proof}

A consequence of Inequality~\ref{eqn:threedsm_cyc_bf_ml} in the proof of Theorem~\ref{thm:three_dsm_cyc_ml} is the following corollary. This corollary is also implied by Escamocher and O'Sullivan's \cite{Escamocher2018} result that the number of stable matchings in such an instance of 3-DSM-CYC is exponential in $n$.

\begin{cor}
\label{cor:three_dsm_cyc_ml_stab}
Any instance of 3-DSM-CYC contains a stable matching, which can be found in polynomial time, if all agents of at least one type have the same preference list.
\end{cor}
\begin{proof}
Assume without loss of generality that all agents in $D$ have the same preference list. Suppose $M$ is a matching returned by the variant of Algorithm~\algorithmfont{cyclicSerialDictatorship} described in Theorem~\ref{thm:three_dsm_cyc_ml}. By Inequality~\ref{eqn:threedsm_cyc_bf_ml} in the proof of Theorem~\ref{thm:three_dsm_cyc_ml}, since $c = 0$ it must be that $|\textrm{bf}(M)| = 0$ and thus that $M$ is stable.
\end{proof}

As before in the case of unrestricted preferences, we show that this analysis is tight asymptotically by constructing an instance $\mathcal{I}_n$ for some fixed $n \geq 1$ where for any fixed $c \leq n$ the approximation ratio obtained by the algorithm is $6 / (6 - (3d^2 - 2d^3)) - o(1)$, where $d = c/n$.

For any fixed $n$, construct $\mathcal{I}_n$ as follows. Let $U = \{ u_1, u_2, \dots, u_n \}$, $W =  \{ w_1, w_2, \dots, w_n \}$, and $D = \{ d_1, d_2, \dots, d_n \}$, and for each $i$ where $1\leq i \leq n$ let
\begin{align*}
\setlength\arraycolsep{2pt}
\begin{array}{r c c c c}
% P_{u_i} :& w_1 & w_2 & \dots & w_{n}\\
% P_{w_i} :& d_1 & d_2 & \dots & d_{n}\\
\hat{P} :& u_1 & u_2 & \dots & u_n
\end{array}
\end{align*}
and
\begin{align*}
\setlength\arraycolsep{2pt}
\begin{array}{r c c c c}
P_{u_i} :& w_1 & w_2 & \dots & w_{n}\\
P_{w_i} :& d_1 & d_2 & \dots & d_{n}
\end{array}
\end{align*}
To construct $P_{d_i}$ for each $d_i \in D$, first construct $P_{d_i} : u_1\hspace{2pt} u_2\hspace{2pt} \dots\hspace{2pt} u_n$. If $i \leq n - c$ then shift $u_i$ in $P_{d_i}$ to the right by $c$ places so that it appears just after $u_{i+c}$. If $i > n - c$ then shift $u_i$ in $P_{d_i}$ to the right by $n - i$ places so that it appears last. Note that now $d_i$ prefers to $u_i$ any agent $u_j$ where either $j < i$ or $i + 1 \leq j \leq i + c$.

% P_{d_i} :& u_1 & u_2 & \dots & u_{i - 1} & u_{i + 1} & u_{i + 2} & \dots & u_{i + c} & u_i & u_{i + c + 1} & u_{i + c + 2} & \dots & u_n
% \end{array}
% \end{flalign*}
Notice that for any $d_i \in D$ each discordant pair between $\hat{P}$ and $P_{d_i}$ comprises $(u_i, u_{i + j})$ where $j \leq c$. It follows that there are exactly $c$ discordant pairs between $\hat{P}$ and $P_{d_i}$ for each $d_i$.

It is straightforward to show that the algorithm returns a matching $M = \{ (u_1, w_1, d_1), (u_2, w_2, d_2), \dots, (u_n, w_n, d_n) \}$. Let $F_i$ be the family $( u_i, w_i, d_i )$ selected in the $i\textsuperscript{th}$ iteration and let $S_i$ be the set of families that block $M$, have a non-empty intersection with $F_i$ and an empty intersection with $F_j$ for every $1\leq j < i$. Note that $S_n = \varnothing$. 

As before in the proof of Theorem~\ref{thm:three_dsm_cyc_ml}, we consider two cases. First, suppose $i < n - c$. As before, it can be shown that no family in $S_i$ contains $u_i$ and no family in $S_i$ contains $w_i$. It follows that each family in $S_i$ contains $d_i$. As we noted, by the construction of $\mathcal{I}_n$ it must be that $d_i$ prefers to $u_i$ each agent $u_j$ where either $j < i$ or $i + 1 \leq j \leq i + c$. By the definition of $S_i$ no family in $S_i$ contains any agent $u_j$ where $j < i$ so it follows that any family in $S_i$ contains $d_i$ as well as some $u_j$ where $i + 1 \leq j \leq i + c$. Moreover, any such $u_j$ prefers to $w_j$ each agent $w_k$ where $k < j$, and any such $w_k$ prefers to $d_k$ each agent $d_i$ where $i < k$. It follows that $S_i$ contains the family $( u_j, w_k, d_i )$ for each $j$ where $i + 1 \leq j \leq i + c$ and each $k$ where $i < k < j$. By the proof of Theorem~\ref{thm:three_dsm_cyc_ml}, it is straightforward to show that no other families are contained in $S_i$. It follows that 
\begin{align*}
    S_i = \{ ( u_j, w_k, d_i ) : i + 1 \leq j \leq i + c, i < k < j \}
\end{align*}
and thus that
\begin{align*}
    |S_i| &= \sum\limits_{j = i + 1}^{i + c} (j - i - 1)\enspace.
\end{align*}
Second, suppose $i \geq n - c$. It follows in this case that
\begin{align*}
    |S_i| &= \sum\limits_{j = i + 1}^{n} (j - i - 1)\enspace.
\end{align*}
It follows, as in the proof of Theorem~\ref{thm:three_dsm_cyc_ml}, that
\begin{align*}
    |\textrm{bf}(M)| &= \sum\limits_{i=1}^{n - c - 1} \frac{c(c - 1)}{2} + \sum\limits_{i = n - c}^{n - 1} \frac{(n - i)(n - i - 1)}{2}\\
    &= \frac{1}{2}c(n - c - 1)(c - 1) + \frac{1}{6}(c^3 - c)
\end{align*}
which shows that the upper bound on $|S_i|$ shown in Inequality~\ref{eqn:three_dsm_cyc_sizeofsi} in Theorem~\ref{thm:three_dsm_cyc_unrestricted} is tight. It then follows that
\begin{align}
    |\textrm{nbf}(M)| = n^3 - \frac{1}{2}c(n - c - 1)(c - 1) - \frac{1}{6}(c^3 - c) \label{eqn:three_dsm_cyc_ml_sizeofsiintight}\enspace.
\end{align}
To see that at least one stable matching $M^*$ exists in $\mathcal{I}_n$, consider a new instance $\mathcal{I}'_n$ with sets $U'$, $W'$, and $D'$ constructed as for $\mathcal{I}_n$ except permuting the types of the agents so that every agent in $D'$ has the same preference list. It follows by Corollary~\ref{cor:three_dsm_cyc_ml_stab} that $\mathcal{I}'_n$ contains a stable matching, which can be relabelled to reveal a stable matching $M^*$ in $\mathcal{I}_n$.

Now
\begingroup
\allowdisplaybreaks
\begin{align*}
% \begin{split}
\frac{|\textrm{nbf}(M^*)|}{|\textrm{nbf}(M)|} &= \frac{n^3}{|\textrm{nbf}(M)|} && \mbox{since $M^*$ is stable}\\
    &= \frac{6n^3}{6n^3 - 3nc^2  + 2c^3 + 3nc - 2c} && \mbox{by Equation~\ref{eqn:three_dsm_cyc_ml_sizeofsiintight}}\\
    % &= \frac{6n^3}{6n^3 - 3nc^2 + 2c^3} - \frac{18cn^4 - 12cn^3}{\splitfrac{36n^6 + (18c - 36c^2)n^4}{\splitfrac{+ (24c^3 - 12c)n^3}{\splitfrac{+ (9c^4 - 9c^3)n^2}{\splitfrac{+ (6c^4 - 12c^5 + 6c^3)n}{+ 4c^6 - 4c^4}}}}}\\
    %  &= \frac{6n^3}{6n^3 - 3nc^2 + 2c^3} - \frac{18cn^4 - 12cn^3}{\begin{matrix*}[l]36n^6 + (18c - 36c^2)n^4\\\quad +\ (24c^3 - 12c)n^3\\\quad+\ (9c^4 - 9c^3)n^2\\\quad+\ (6c^3 - 12c^5 + 6c^4)n\\\quad+\ 4c^6 - 4c^4\end{matrix*}}\\
    &= \frac{6n^3}{6n^3 - 3nc^2 + 2c^3} - \frac{18cn^4 - 12cn^3}{\begin{matrix*}[l]36n^6 + (18c - 36c^2)n^4\\\quad +\ (24c^3 - 12c)n^3\\\quad+\ (9c^4 - 9c^3)n^2\\\quad+\ (6c^3 - 12c^5 + 6c^4)n\\\quad+\ 4c^6 - 4c^4\end{matrix*}}\\
    &= \frac{6n^3}{6n^3 - 3nc^2 + 2c^3} - o(1)\\
    % &\leq \frac{6n^3}{6n^3 - 3nc^2 + 2c^3} && \mbox{since $c \geq 0$}\\
    % &= \frac{6n^3}{6n^3 - c^2(3n - 2c)}\\
    &= \frac{6}{6 - (3d^2 - 2d^3)} - o(1) && \mbox{where $d = c/n$}
% \end{split}
\end{align*}
\endgroup
which shows that the analysis is tight asymptotically.

% Suppose then that $n$ is odd and thus that $n = 2l + 1$ where $l \geq 0$. In this case let $M^* = \{ ( u_n, w_n, d_n ) \} \cup \{ ( u_{2i}, w_{2i}, d_{2i - 1} ), ( u_{2i - 1}, w_{2i - 1}, d_{2i} ) : 1 \leq i \leq l \}$. Suppose for a contradiction that some family $( u_j, w_k, d_i )$ blocks $M^*$ in $(N, P)$. If $j < n$, $k < n$ and $i < n$ then, like the previous case in which we supposed $n$ is even, it must be that the agent in the agent in $M(w_k)$ in $D$ is either $d_{k + 1}$ or $d_{k - 1}$ and the agent in $M(d_k)$ in $U$ is either $u_{k + 1}$ or $u_{k - 1}$. It follows by the construction of $\mathcal{I}_n$ that $i \leq k$ and $j \leq i$ and thus that $j \leq k$, which is a contradiction since by the definition of a blocking family $u_j$ must prefer $w_k$ to its assigned partner of the next type, $w_j$. It remains that either $j = n$, $k = n$, or $i = n$. By the construction of the preference lists of the agents in $U$ and $W$ it must be that $w_n$ is the last agent in $P_{u_i}$ for any $u_i \in U$ and $d_n$ is the last agent in $P_{w_i}$ for any $w_i \in W$. It follows that neither $w_n$ nor $d_n$ belong to a family that blocks $M^*$ and thus the only possibility is that $j = n$. 

\section{Summary and open problems}
\label{sec:three_dsm_cyc_conclusion}
In this chapter we considered the approximability of the 3-DSM-CYC Maximally Stable Matching problem (3-DSM-CYC-MSM). We first presented a $9/4$-approximation algorithm based on an existing algorithm for 3GSM-MSM, which is a closely related problem \cite{rosenbaum16}. We then presented a $6/5$-approximation algorithm based on serial dictatorship, and showed that our analysis is tight asymptotically. Finally, we considered a situation in which the preference lists of all agents of at least one type are derived from some master list, and modified the aforementioned serial dictatorship algorithm for this setting. We considered the maximum Kendall tau distance $c$ between any such agent's list and the master list, and showed that if $c \leq n$ then the modified algorithm has an approximation ratio of $6 / (6 - (3d^2 - 2d^3))$, where $d = c/n$, which is tight asymptotically.

As we saw in Chapter~\ref{c:lit_review}, the history of 3-DSM-CYC spans several decades and it continues to generate interesting research. Given the recent result of Lam and Plaxton \cite{Plaxton3DSMCYCJournal} (who showed that stable matchings need not exist in general, and the associated decision problem is $\NP$-complete), it seems most natural to consider either the approximability of 3-DSM-CYC, as we do here, or parameterised complexity. 

In the optimisation version of 3-DSM-CYC that we studied here, the objective is to maximise the number of non-blocking families. Of course, one could also define a complementary problem in which the objective is to minimise the number of blocking families, which is arguably more natural. In fact, similar optimisation problems, in which the objective involves minimising the number of blocking coalitions, have been studied in relation to other problems of matching under preferences \cite{BMM10, ABM06, Hamada16}. Another possibility is to construct a sub-matching of maximum cardinality such that no three agents in families in the sub-matching form a blocking family in the sub-matching. Rosenbaum \cite{rosenbaum16} refers to the analogous problem for 3GSM as the \emph{3G Maximum Stable Sub-matching problem} (3G-MSS). 

Although the definitions of the maximisation and minimisation variants of 3-DSM-CYC are complementary, it seems as if the difficulty of characterising instances with no stable matching makes tackling the latter variant more challenging.

% It is of course possible that the approximation algorithms we presented here for 3-DSM-CYC-MSM can be improved upon. At a high level, our approximation algorithm in the case of a master list demonstrates one possible strategy when the preference lists of all agents of at least one type are sufficiently similar. One  be possible to augment this algorithm to handle a scenario  


\chapter{Three-Dimensional Stable Roommates with \texorpdfstring{$\mathscr{B}$}{B}-preferences}
\label{c:threed_sr_b}
\chaptermark{Stability in 3DR-B}

\section{Introduction}
\label{sec:threed_sr_b_intro}
In this chapter we study a new model of fixed-size coalition formation, which we call the \emph{Three-Dimensional Roommates with $\mathscr{B}$-preferences} (\mysymbolfirstusedefinition{symboldef:threedr_b}{3DR-B}). This model is closely related to some of the existing models of Three-Dimensional Roommates (3DR) discussed in Chapter~\ref{c:lit_review}, and in particular the \emph{Three-Person Stable Assignment problem} (3PSA) \cite{NH91} and the model of Iwama et al.~\cite{IMO07}. As in the model of Iwama et al., in 3DR-B each agent has a strict preference list over all other agents. A specific set extension rule is then used to infer from each agent's preference list that agent's preferences over coalitions. In 3DR-B, that set extension rule is known as \emph{$\mathscr{B}$-preferences} \cite{CH02}. Using $\mathscr{B}$-preferences, any agent prefers some triple $S$ to another triple $T$ if the most-preferred agent in $S$ is preferred to the most-preferred agent in $T$. In this chapter we consider in 3DR-B the existence of, and complexity of finding, matchings that are stable.

% There is an interesting motivation underlying this model: in 

We first show, in Section~\ref{sec:threed_sr_b_hardness}, that a given instance of 3DR-B may not contain a stable matching and that the associated decision problem is $\NP$-complete (Theorem~\ref{thm:threed_sr_b_existence}). This contrasts with an analogous model in which coalitions need not have a fixed size, in which a stable matching must always exist and can be found in polynomial time \cite{CR01}. It may seem intuitive that the additional requirement of fixed-size coalitions makes this particular problem harder to solve, and this result gives an example of a model in which that intuition holds. It also leads to interesting directions for subsequent work, for example involving approximately stable matchings, or alternative constraints on the size of feasible coalitions.

We then consider, in Section~\ref{sec:threed_sr_b_approximation}, a closely related optimisation problem in which the objective is to construct a matching that is, in terms of a specific measure, as stable as possible. We begin by showing that an existing approximation algorithm for a different model of 3DR can be used to devise a $9/4$-approximation algorithm for the 3DR-B problem (Theorem~\ref{thm:threed_sr_b_approxalgofournine}). We then show that a simple algorithm based on serial dictatorship gives an improved approximation ratio of $3/2$ (Theorem~\ref{thm:threed_sr_b_approxalgo}). Interestingly, this algorithm can be viewed as an adaptation of the algorithm developed by Cechl\'{a}rov\'{a} and Romero-Medina \cite{CR01} that can be used to construct a stable matching in the analogous model in which coalitions need not have a fixed size.

Next, in Section~\ref{sec:threed_sr_b_structural}, we consider the problem of identifying the smallest instance of 3DR-B that contains no stable matching. We show that such an instance contains at least $9$ (Theorem~\ref{thm:threed_sr_b_ifnis6thensmexists}) and at most $15$ agents (Theorem~\ref{thm:threed_sr_b_ifnis15thensmdoesnotexist}) but leave determining the precise number of agents as an open problem.

%By the reduction that we present in Section~\ref{sec:threed_sr_b_hardness}, such an instance must contain at most $63$ agents. We provide a lower bound and show that such an instance must contain at least $9$ agents (Theorem~\ref{thm:threed_sr_b_ifnis6thensmexists}).

Finally, in Section~\ref{sec:threed_sr_b_conclusion}, we recap on our results and discuss some directions for future work.

We proceed with some formal definitions and notation. An instance of 3DR-B comprises a set $N$ of $3n$ agents and a preference list of each agent $\alpha_i$, labelled $P_{\alpha_i}$, that describes a strict order over all agents in $N \setminus \{ \alpha_i \}$. We say that an agent $\alpha_i$ \emph{prefers} $\alpha_j$ to $\alpha_k$, denoted $\alpha_j \succ_{\alpha_i} \alpha_k$, if $\alpha_j$ precedes $\alpha_k$ in $P_{\alpha_i}$. A \emph{triple} is an unordered set of three agents. In order to compare triples, agents in an instance of 3DR-B use \emph{$\mathscr{B}$-preferences}, which are defined as follows. For any agent $\alpha_i$ and set of agents $S \subseteq N$ we denote by $\mathscr{B}_{\alpha_i}(S)$ the most-preferred agent in $S \setminus \{ \alpha_i \}$ according to $\alpha_i$. For any agent $\alpha_i$ and any two triples $r$ and $s$, we say that $\alpha_i$ \emph{prefers} $r$ to $s$, denoted $r \succ_{\alpha_i} s$, if $\mathscr{B}_{\alpha_i}(r) \succ_{\alpha_i} \mathscr{B}_{\alpha_i}(s)$. A \emph{matching} is a partition of $N$ into $n$ triples. Given an agent $\alpha_i$ and a matching $M$, we denote by $M(\alpha_i)$ the triple in $M$ that contains $\alpha_i$. Given a matching $M$, we say that a triple $t$ is \emph{blocking} if each agent $\alpha_i$ in $t$ prefers $t$ to $M(\alpha_i)$. A matching is \emph{stable} if it does not contain a blocking triple. Let $\mathscr{B}_{\alpha_i}(M)$ be short for $\mathscr{B}_{\alpha_i}(M(\alpha_i))$. Let $P$ be the collection of preference lists  $P_{\alpha_i}$ for each agent $\alpha_i$. For any instance $(N, P)$ of 3DR-B and any matching $M$, we denote by $\textrm{bt}(M, (N, P)) \subseteq \binom{N}{3}$ the set of triples that block $M$ in $(N, P)$. Conversely, we denote by $\textrm{nbt}(M, (N, P)) = \binom{N}{3} \setminus \textrm{bt}(M, (N, P))$ the set of triples that do not block $M$ in $(N, P)$. When the instance in question is clear from context, we simply write $\textrm{bt}(M)$ or $\textrm{nbt}(M)$. 

\section{Deciding existence}
\label{sec:threed_sr_b_hardness}
In this section we show that deciding if a given instance of 3DR-B contains a stable matching is $\NP$-complete. The reduction presented here is from the Three-Dimensional Stable Matching problem with Cyclic preferences (3-DSM-CYC, defined in Chapter~\ref{c:three_dsm_cyc}), which is $\NP$-complete~\cite{Plaxton3DSMCYCJournal}. 

Our reduction is similar to the reduction from 3GSM to the Three-Person Stable Assignment problem (3PSA, defined in Chapter~\ref{c:lit_review}) used by Ng and Hirschberg \cite{NH91}. Our reduction, like Ng and Hirschberg's, is from a problem involving a multipartite set of agents (in their case, 3GSM) to a problem involving a homogenous set of agents (in their case, 3PSA). Informally, the idea is to use the same set of agents in both problem instances but to design the preference lists of the agents in the constructed instance in such a way that each triple in any stable matching in the constructed instance must contain exactly one agent of each type. It is then straightforward to show that a stable matching exists in the instance of the latter problem if and only if a stable matching exists in the instance of the former problem. In fact, a similar technique was also used to reduce from (two-dimensional) SM to (two-dimensional) SR by Gusfield and Irving in 1989 \cite[Lemma~4.1.1]{GI89}.

% In Ng and Hirschberg's reduction, the set of agents in the constructed 3PSA instance is $A = U \cup W \cup D$. The new preference list of each agent in $U$ is the preference list of that agent in the 3GSM instance, which contains every pair of agents in $W \times D$, followed by every other possible pair of agents in $A$ in arbitrary order. The preference lists of each agent in $W$ and $D$ are constructed analogously. It is then straightforward to show that each triple in any stable matching in the 3PSA instance must contain one agent of each type and thus a stable matching exists in the 3PSA instance if and only if a stable matching exists in the 3GSM instance. In fact, a similar technique was also used to reduce from (two-dimensional) SM to (two-dimensional) SR by Gusfield and Irving in 1989 \cite[Lemma~4.1.1]{GI89}. % Here we add three additional ``sentinel'' agents to ensure that any stable matching in the 3D-SR-B instance either contains one agent of each type, or does after a small modification

% To simplify the proof we introduce some new notation. From now on, in an instance $(A, Q)$ of 3-DSM-CYC, we will refer to $I\times\{0\}$ as $U_A$, $I\times\{1\}$ as $W_A$, and $I\times\{2\}$ as $D_A$. A family in $(A,Q$) is then a 3-tuple $(u_i, w_j, d_k)$ where $u_i\in U_A$, $w_j\in W_A$, $d_k\in D_A$.

The reduction from 3-DSM-CYC to 3DR-B is as follows. Suppose $(N', P')$ is an arbitrary instance of 3-DSM-CYC, in which the sets of agents of each type are labelled $U$, $W$, and $D$. We shall construct an instance $(N, P)$ of 3DR-B. First construct three new `sentinel' agents $u_0, w_0, d_0$ in $N$ where:
\begin{flalign*}
\setlength\arraycolsep{2pt}
\begin{array}{r l c l c l c l c l c l}
P_{u_0} :& [& W\text{ in arbitrary order} &]\ &w_0\ &[& D\text{ in arbitrary order} &]\ &d_0\ &[& U\text{ in arbitrary order} &]\\
P_{w_0} :& [& D\text{ in arbitrary order} &]\ &d_0\ &[& U\text{ in arbitrary order} &]\ &u_0\ &[& W\text{ in arbitrary order} &]\\
P_{d_0} :& [& U\text{ in arbitrary order} &]\ &u_0\ &[& W\text{ in arbitrary order} &]\ &w_0\ &[& D\text{ in arbitrary order} &]
\end{array}
\end{flalign*}

Next, add each agent in $N' = U\cup W\cup D$ to $N$ and for each $i$ where $1\leq i \leq n$ let:
\begin{flalign*}
\setlength\arraycolsep{2pt}
\begin{array}{l l c l c l c l c l c l c}
P_{u_i} :& [& P'_{u_i} &]\ &w_0\ &[& D\text{ in arbitrary order} &]\ &d_0\ &[& U\setminus \{ u_i \}\text{ in arbitrary order} &]\ &u_0\\
P_{w_i} :& [& P'_{w_i} &]\ &d_0\ &[& U\text{ in arbitrary order} &]\ &u_0\ &[& W\setminus \{ w_i \}\text{ in arbitrary order} &]\ &w_0\\
P_{d_i} :& [& P'_{d_i} &]\ &u_0\ &[& W\text{ in arbitrary order} &]\ &w_0\ &[& D\setminus \{ d_i \}\text{ in arbitrary order} &]\ &d_0
\end{array}
\end{flalign*}

This completes the construction of $(N, P)$. Partition $N$ into three sets $U', W', D'$ where $U' = U \cup \{ u_0 \}$, $W' = W \cup \{ w_0 \}$, and $D' = D \cup \{ d_0 \}$. Note that in the constructed instance $(N, P)$ of 3DR-B:
\begin{itemize}
    \item Any agent in $U'$ prefers any agent in $U'$ to any agent in $W'$ and also prefers any agent in $W'$ to any agent in $D'$ (and similarly for any agent in $W'$ or $D'$).
    \item For any agent in $N$, $u_0$ is the least-preferred agent in $U'$, $w_0$ is the least-preferred agent in $W'$ and $d_0$ is the least-preferred agent in $D'$.
\end{itemize}

It is straightforward to show that this reduction can be performed in polynomial time. To prove that the reduction is correct we show that the 3DR-B instance $(N, P)$ contains a stable matching if and only if the 3-DSM-CYC instance $(N', P')$ contains a stable matching.

We first show that if the 3-DSM-CYC instance $(N', P')$ then the 3DR-B instance $(N, P)$ contains a stable matching.

\begin{lem}
\label{lem:threed_sr_b_stablematchinginaqimpliesstablematchinginnp}
If $(N', P')$ contains a stable matching then $(N, P)$ contains a stable matching.
\end{lem}
\begin{proof}
Suppose $M'$ is a stable matching in $(N', P')$. Let $M = \{ u_0, w_0, d_0 \} \cup \{\{ u_i, w_j, d_k \}\ : ( u_i, w_j, d_k )\in M'\}$.
 
Towards a contradiction, suppose $M$ is not stable in $(N, P)$ and that $\{ \alpha_i, \alpha_j, \alpha_k \}$ blocks $M$ in $(N, P)$. It must be that either $\alpha_i \in U'$, $\alpha_i \in W'$, or $\alpha_i \in D'$. Assume without loss of generality that $\alpha_i \in U'$. It follows that $\mathscr{B}_{\alpha_i}(M)\in W'$ and thus either $\alpha_j \in W'$ or $\alpha_k \in W'$. Suppose without loss of generality that $\alpha_j \in W'$. A similar argument then shows that $\alpha_k \in D'$ so we relabel $\{ \alpha_i, \alpha_j, \alpha_k \}$ as $\{ u_{i'}, w_{j'}, d_{k'} \}$ where $u_{i'} \in U'$, $w_{j'} \in W'$, and $d_{k'} \in D'$. Since $w_{j'} \succ_{u_{i'}} \mathscr{B}_{u_{i'}}(M)$ and, by the construction of $M'$, $\mathscr{B}_{u_{i'}}(M) \in W'$, it must be that $j' \neq 0$. A similar argument shows that $k' \neq 0$ and $i' \neq 0$. It then follows that the family $( u_{i'}, w_{j'}, d_{k'} )$ blocks $M$ in $(N', P')$, which is a contradiction.
\end{proof}

We now show, using a sequence of lemmas, that if the 3DR-B instance $(N, P)$ contains a stable matching then the 3-DSM-CYC instance $(N', P')$ contains a stable matching. The complication here is that a triple in a stable matching in $(N, P)$ need not contain exactly one agent in each of $U'$, $W'$, and $D'$. Nevertheless, we show that an arbitrary stable matching in $(N, P)$ has a relatively constrained structure and can thus be modified such that each triple contains exactly one agent from each of $U$, $W$, $D$. It is then straightforward to construct a stable matching in the 3-DSM-CYC instance $(N', P')$.

We say that a triple of three agents in $N$ is \emph{mixed} if it does not contain exactly one agent in each of $U'$, $W'$, and $D'$. Without loss of generality assume that the number of mixed triples in $M$ is minimal.

Suppose $X$ is an arbitrary element of $\{ U', W', D' \}$. Note that by definition, the number of agents in $X$ in non-mixed triples in $M$ is $n - |X|$. It follows that the number of agents in $X$ in mixed triples in $M$ is $|X|$ and thus the average number of agents in $X$ in each mixed triple in $M$ is $|X|/|X| = 1$.

\begin{lem}
\label{lem:threed_sr_b_uuulemma}
If $(N, P)$ contains a stable matching $M$ then no triple in $M$ contains three agents in exactly one of $U'$, $W'$, and $D'$.
\end{lem}
\begin{proof}
Assume without loss of generality that $M$ contains some mixed triple $t_1 = \{ u_{i_1}, u_{i_2}, u_{i_3} \}$ where $u_{i_1}, u_{i_2}, u_{i_3} \in U'$. Since $t_1$ contains more agents in $U'$ than the average number of agents in $U'$ in each mixed triple in $M$, it follows that there exists some mixed triple $t_2 \in M$ that contains fewer than the average number of agents in $U'$ in each mixed triple in $M$. Since the average number of agents in $U'$ in each mixed triple in $M$ is $1$ it must be that $t_2$ contains no agent in $U'$. Consider the possible contents of $t_2$. If either $t_2 = \{ w_{j_1}, d_{k_1}, d_{k_2} \}$ where $w_{j_1} \in W'$ and $d_{k_1}, d_{k_2} \in D'$, $t_2 = \{ w_{j_1}, w_{j_2}, d_{k_1} \}$ where $w_{j_1}, w_{j_2} \in W'$ and $d_{k_1} \in D'$, or $t_2 = \{ d_{k_1}, d_{k_2}, d_{k_3} \}$ where $d_{k_1}, d_{k_2}, d_{k_3} \in D'$, then $\{ u_{i_1}, u_{i_2}, d_{k_1} \}$ blocks $M'$, which is a contradiction. If $t_2 = \{ w_{j_1}, w_{j_2}, w_{j_3} \}$ where $w_{j_1}, w_{j_2}, w_{j_3} \in W'$ then $\{ w_{j_1}, w_{j_2}, u_{i_1} \}$ blocks $M$, which is a contradiction.
\end{proof}

\begin{lem}
\label{lem:threed_sr_b_uuw_implies_wdd}
Arbitrarily label $U', W', D'$ as $X, Y, Z$. If $(N, P)$ contains a stable matching $M$ that contains a mixed triple $\{ x_{i_1}, x_{i_2}, y_{j_1} \}$ where $x_{i_1}, x_{i_2} \in X$ and $y_{j_1} \in Y$ then $M$ also contains some mixed triple $\{ y_{j_2}, z_{k_1}, z_{k_2} \}$ where $y_{j_2} \in Y$ and $z_{i_1}, z_{i_2} \in Z$.
\end{lem}
\begin{proof}
Without loss of generality assume that $X = U'$, $Y = W'$, and $Z = D'$. Suppose for a contradiction that $M$ contains some mixed triple $\{ u_{i_1}, u_{i_2}, w_{j_1} \}$ where $u_{i_1}, u_{i_2} \in U'$ and $w_{j_1} \in W'$ and does not contain any triple $\{ w_{j_2}, d_{k_1}, d_{k_2} \}$ where $w_{j_2} \in W'$ and $d_{k_1}, d_{k_2} \in D'$. Observe that $\{ u_{i_1}, u_{i_2}, w_{j_1} \}$ contains more agents in $U'$ than the average number of agents in $U'$ in each mixed triple in $M$ and fewer agents in $D'$ than the average number of agents in $D'$ in each mixed triple in $M$. It follows that there exists some mixed triple $t_1 \in M$ in which the number of agents in $U'$ is $0$ and some mixed triple $t_2 \in M$ in which the number of agents in $D'$ is at least $2$. By Lemma~\ref{lem:threed_sr_b_uuulemma}, the number of agents in $t_2$ in $D'$ is exactly $2$. By assumption, no triple $\{ w_{j_2}, d_{k_1}, d_{k_2} \}$ exists in $M$ so the only possibility is that $t_1 \neq t_2$ and $t_1 = \{ w_{j_3}, w_{j_4}, d_{k_3} \}$ where $w_{j_3}, w_{j_4} \in W'$ and $d_{k_3} \in D'$ and $t_2 = \{ d_{k_4}, d_{k_5}, u_{i_3} \}$ where $d_{k_4}, d_{k_5} \in D'$ and $u_{i_3} \in U'$. Now $\{ u_{i_3}, w_{j_1}, d_{k_3} \}$ blocks $M'$, which is a contradiction.
\end{proof}

\begin{lem}
\label{lem:threed_sr_b_ifmisstablemprimeprimeexistswhichisalltypet0}
If $(N, P)$ contains a stable matching $M$ then no triple in $M$ is mixed.
\end{lem}
\begin{proof}
Assume for a contradiction that $M$ contains at least one mixed triple. By Lemma~\ref{lem:threed_sr_b_uuulemma}, no triple contains three agents in any one of $U'$, $W'$, and $D'$. Assume then without loss of generality that $M$ contains some mixed triple $t_1 = \{ u_{i_1}, u_{i_2}, w_{j_1} \}$ where $u_{i_1}, u_{i_2} \in U'$ and $w_{j_1} \in W'$. By Lemma~\ref{lem:threed_sr_b_uuw_implies_wdd} it follows that there exists some other mixed triple $\{ w_{j_2}, d_{k_1}, d_{k_2} \}$ in $M$ where $w_{j_2} \in W'$ and $d_{k_1}, d_{k_2} \in D'$.

We first claim that every mixed triple in $M$ either contains two agents in $U'$ and one agent in $W'$ or contains two agents in $D'$ and one agent in $W'$. If not, by Lemma~\ref{lem:threed_sr_b_uuw_implies_wdd} there are two possible cases: either $M$ contains two triples $\{ u_{i_3}, w_{j_3}, w_{j_4} \}, \{ u_{i_4}, d_{k_3}, d_{k_4} \}$ or $M$ contains two triples $\{ d_{k_3}, w_{j_3}, w_{j_4} \}, \{ u_{i_3}, u_{i_4}, d_{k_4} \}$, where in either case $u_{i_3}, u_{i_4} \in U'$, $w_{j_3}, w_{j_4} \in W'$, and $d_{k_3}, d_{k_4} \in D'$. In the former case, $\{ u_{i_4}, w_{j_3}, d_{k_1} \}$ blocks $M$. In the latter case, $\{ u_{i_4}, w_{j_1}, d_{k_3} \}$ blocks $M$.

Now consider $w_0$. If $M(w_0)$ is not mixed then $M(w_0) = \{ u_{i_3}, w_0, d_{k_3} \}$ where $u_{i_3} \in U'$ and $d_{k_3} \in D'$ and thus $\{ u_{i_3}, w_{j_1}, d_{k_1} \}$ blocks $M$, which is a contradiction. It remains that $M(w_0)$ is mixed. Since every mixed triple either contains two agents in $U'$ and one agent in $W'$ or contains two agents in $D'$ and one agent in $W'$, without loss of generality assume that either $j_1 = 0$ or $j_2 = 0$. Suppose firstly that $j_2 = 0$. It follows that $\{ d_{k_1}, d_{k_2}, w_{j_1} \}$ blocks $M'$, which is a contradiction. It remains that $j_1 = 0$. To show a contradiction, we now construct a new matching $\hat{M}$ in which the number of mixed triples in $\hat{M}$ is strictly fewer than the number of mixed triples in $M$. Consider $P_{w_{j_2}}$. It must be that either $d_{k_1} \succ_{w_{j_2}} d_{k_2}$ or $d_{k_2} \succ_{w_{j_2}} d_{k_1}$. Suppose without loss of generality that $d_{k_1} \succ_{w_{j_2}} d_{k_2}$. Now consider
\begin{align*}
    \hat{M} = (M \setminus \{ \{ u_{i_1}, u_{i_2}, w_{j_1} \}, \{ d_{k_1}, d_{k_2}, w_{j_2} \} \}) \cup \{ \{ u_{i_1}, w_0, d_{k_1}\}, \{ u_{i_2}, w_{j_2}, d_{k_2} \} \}
\end{align*}
in which:
\begin{itemize}
    \item $\mathscr{B}_{u_{i_1}}(\hat{M}) = \mathscr{B}_{u_{i_1}}(M) = w_0$
    \item $\mathscr{B}_{w_0}(\hat{M}) = d_{k_1}$ and $\mathscr{B}_{w_0}(M) \in U'$ and  so $\hat{M} \succ_{w_0} M$
    \item $\mathscr{B}_{d_{k_1}}(\hat{M}) = u_{i_1}$ and $\mathscr{B}_{d_{k_1}}(M) \in W'$ and so $\hat{M} \succ_{d_{k_1}} M$
    \item $\mathscr{B}_{u_{i_2}}(\hat{M}) = w_{j_2} \succ_{u_{i_2}} w_0 = \mathscr{B}_{u_{i_2}}(M)$
    \item $\mathscr{B}_{w_{j_2}}(\hat{M}) = \mathscr{B}_{w_{j_2}}(M) = d_{k_1}$
    \item $\mathscr{B}_{d_{k_2}}(\hat{M}) = \mathscr{B}_{d_{k_2}}(M) = w_{j_2}$.
\end{itemize}
It follows that any triple that blocks $\hat{M}$ in $(N, P)$ also blocks $M$ in $(N, P)$. Thus, since $M$ is stable, $\hat{M}$ is also stable. The number of mixed triples in $\hat{M}$ is exactly one fewer than in $M$, which contradicts our assumption that $M$ is a stable matching in $(N, P)$ with the minimal number of mixed triples.
\end{proof}
% \begin{lem}
% Label $U', W', D'$ arbitrarily as $X, Y, Z$. If $M'$ contains some triple $\{ x \}$ \end{lem}

\begin{lem}
\label{lem:threed_sr_b_misstableexistsmprimeprimeprime}
If $(N, P)$ contains a stable matching then $(N, P)$ contains a stable matching $\hat{M}$ such that $\hat{M}$ contains $\{ u_0, w_0, d_0 \}$ and no triple in $\hat{M}$ is mixed.
\end{lem}
\begin{proof}
Let $S$ be the set of triples in $M$ that each contain at least one agent in $\{ u_0, w_0, d_0 \}$. By definition, $1 \leq |S| \leq 3$. If $|S| = 1$, then $\{ u_0, w_0, d_0 \} \in M$ so $\hat{M} = M$. If $|S| = 3$ then by Lemma~\ref{lem:threed_sr_b_ifmisstablemprimeprimeexistswhichisalltypet0} it must be that $S = \{ \{ u_0, w_{j_1}, d_{k_1} \}, \{ u_{i_2}, w_0, d_{k_2} \}, \{ u_{i_3}, w_{j_3}, d_0 \} \}$, where $u_{i_2}, u_{i_3} \in U'$, $w_{j_1}, w_{j_2} \in W'$ and $d_{k_1}, d_{k_2} \in D'$. Now $\{ u_{i_2}, w_{j_3}, d_{k_1}\}$ blocks $M$, which is a contradiction. It remains that $|S| = 2$. By Lemma~\ref{lem:threed_sr_b_ifmisstablemprimeprimeexistswhichisalltypet0}, no triple in $M$ is mixed, so there are three possible cases: either $S = \{ \{ u_0, w_0, d_k \}, \{ u_i, w_j, d_0 \} \}$, $S = \{ \{ u_0, w_j, d_0\}, \{ u_i, w_0, d_k\} \}$, or $S = \{ \{ u_0, w_j, d_k\}, \{ u_i, w_0, d_0\} \}$, where in any case $u_i \in U'$, $w_j\in W'$, and $d_j\in D'$. In any case, let:
\begin{align*}
    \hat{M} = (M \setminus S) \cup \{ \{ u_i, w_j, d_k \}, \{ u_0, w_0, d_0 \} \}\enspace.
\end{align*}
Now:
\begin{itemize}
    \item either $\mathscr{B}_{u_i}(\hat{M}) \succ_{u_i} \mathscr{B}_{u_i}(M)$ or $\mathscr{B}_{u_i}(\hat{M}) = \mathscr{B}_{u_i}(M)$
    \item either $\mathscr{B}_{w_j}(\hat{M}) \succ_{w_j} \mathscr{B}_{w_j}(M)$ or $\mathscr{B}_{w_j}(\hat{M}) = \mathscr{B}_{w_j}(M)$
    \item either $\mathscr{B}_{d_k}(\hat{M}) \succ_{d_k} \mathscr{B}_{d_k}(M)$ or $\mathscr{B}_{d_k}(\hat{M}) = \mathscr{B}_{d_k}(M)$.
\end{itemize}
It follows that any triple that blocks $\hat{M}$ in $(N, P)$ also blocks $M$ in $(N, P)$. Thus, since $M$ is stable, $\hat{M}$ is also stable.
\end{proof}

\begin{lem}
\label{lem:threed_sr_b_stablematchinginnpimpliesstablematchinginaq}
If $(N, P)$ contains a stable matching then $(N', P')$ contains a stable matching.
\end{lem}
\begin{proof}

By Lemma~\ref{lem:threed_sr_b_misstableexistsmprimeprimeprime}, there exists a stable matching $\hat{M}$ that contains $\{ u_0, w_0, d_0 \}$ in which no triple is mixed. We claim that
\begin{align*}
    M' = \{ ( u_i, w_j, d_k ) : \{ u_i, w_j, d_k \} \in \hat{M} \}
\end{align*}
is a stable matching in $(N', P')$. Suppose for a contradiction that the family $( u_i, w_j, d_k )$ blocks $M'$ in $(N', P')$. It follows that the triple $\{ u_i, w_j, d_k \}$ blocks $\hat{M}$ in $(N, P)$, which is a contradiction.
\end{proof}

% \subsection{Correctness of the reduction: conclusion}
% \label{sec:3dsrbconclusion}

We have now shown that the 3DR-B instance $(N, P)$ contains a stable matching if and only if the 3-DSM-CYC instance $(N', P')$ contains a stable matching. This shows that the reduction is correct.

\begin{thm}
\label{thm:threed_sr_b_existence}
Deciding if a given instance of 3DR-B contains a stable matching is $\NP$-complete.
\end{thm}
\begin{proof}
This decision problem belongs to $\NP$ since the stability of a given matching $M$ can be verified in polynomial time, as follows. For each triple $r \in \binom{N}{3}$ consider each agent $\alpha_i$ in $r$ compare $\mathscr{B}_{\alpha_i}(r)$ and $\mathscr{B}_{\alpha_i}(M)$ using $P_{\alpha_i}$. If at least one agent $\alpha_i$ in each triple $r$ does not prefer $\mathscr{B}_{\alpha_i}(r)$ to  $\mathscr{B}_{\alpha_i}(M)$ then $M$ is stable.

We have presented a polynomial-time reduction from 3-DSM-CYC, which is $\NP$-complete \cite{Plaxton3DSMCYCJournal}. Given an arbitrary instance $(N', P')$ of 3-DSM-CYC, the reduction constructs an instance $(N, P)$ of 3DR-B. By use the reduction described to construct $(N, P)$. In Lemmas~\ref{lem:threed_sr_b_stablematchinginaqimpliesstablematchinginnp} and~\ref{lem:threed_sr_b_stablematchinginnpimpliesstablematchinginaq} we showed that $(N, P)$ contains a stable matching if and only if $(N', P')$ contains a stable matching and thus that this decision problem is $\NP$-hard.
\end{proof}

\section{Approximation}
\label{sec:threed_sr_b_approximation}
% Formally, 3DR-B-MSM is the optimisation variant of the 3DR-B stability existence problem in which the objective is to maximise $\textrm{nbt}(M, (N, P))$.
The \emph{3DR-B Maximally Stable Matching problem} (\mysymbolfirstusedefinition{symboldef:threedr_b_msm}{3DR-B-MSM}) is the following optimisation problem: given an instance of 3DR-B, find a matching with the maximum number of non-blocking triples. Formally, 3DR-B-MSM is a maximisation problem in which any instance $(N, P)$ of 3DR-B-MSM is also instance of 3DR-B, a solution is a matching in $(N, P)$, and the measure is $|\textrm{nbt}(M, (N, P))|$. We showed in Theorem~\ref{thm:threed_sr_b_existence} that deciding if an instance of 3DR-B contains a matching with $\binom{3n}{3}$ non-blocking triples is $\NP$-complete, so it follows that 3DR-B-MSM is $\NP$-hard. In this section we present two approximation algorithms for 3DR-B-MSM. The first is a direct application of an existing result relating to 3PSA. The second is a novel serial dictatorship-style algorithm.

There is a close relationship between 3DR-B and the Three-Person Stable Assignment problem (3PSA, introduced in Chapter~\ref{c:lit_review}). Recall that in 3DR-B, each agent has a strict preference list over the other $3n - 1$ agents. $\mathscr{B}$-preferences are then used to infer each agent's preferences over triples. In 3PSA, each agent instead has a strict preference list over the $\binom{3n - 1}{2}$ triples that they may belong to. We show that, given an instance $(N, P)$ of 3DR-B, it is possible to construct an instance $(N, P')$ of 3PSA with the same set of agents such that for any matching $M$, if a triple blocks $M$ in $(N, P)$ then it also blocks $M$ in $(N, P')$. From this result it follows that an existing $9/4$-approximation algorithm for 3PSA-MSM \cite{rosenbaum16}, which is defined analogously to 3DR-B-MSM, can be applied to construct a $9/4$-approximation algorithm for 3DR-B-MSM. The existing $9/4$-approximation algorithm for 3PSA-MSM was presented by Rosenbaum in 2016 \cite{rosenbaum16} and is called Algorithm~\algorithmfont{ASA}. It is an iterative greedy algorithm which involves, in each iteration, selecting a triple that once added to the matching, intersects the maximum number of non-blocking triples. 

The design of our $9/4$-approximation algorithm for 3DR-B-MSM, which makes use of the relationship between 3PSA and 3DR-B, is essentially the same as the $9/4$-approximation algorithm that we described in Chapter~\ref{c:three_dsm_cyc} for 3-DSM-CYC-MSM. As we saw in Chapter~\ref{c:three_dsm_cyc}, the latter algorithm makes use of the analogous relationship between 3GSM and 3-DSM-CYC (in fact, Algorithm~\algorithmfont{AMSM} is also essentially the same as Algorithm~\algorithmfont{ASA} \cite{rosenbaum16}).

\begin{thm}
\label{thm:threed_sr_b_approxalgofournine}
There exists a polynomial-time $9/4$-approximation algorithm for 3DR-B-MSM.
\end{thm}
\begin{proof}
The approximation algorithm for 3DR-B-MSM involves constructing a corresponding instance of 3PSA-MSM, which has the same set of agents, and running Algorithm~\algorithmfont{ASA} \cite{rosenbaum16}.

We first describe how to construct, in polynomial time, a corresponding instance $(N, P')$ of 3PSA such that $|\textrm{bt}(M, (N, P))| \geq |\textrm{bt}(M, (N, P'))|$ for any matching $M$. For each agent $\alpha_i \in N$, let $P'_{\alpha_i}$ be the list of all $2$-agent subsets of $N \setminus \{ \alpha_i \}$ in lexicographic order with respect to $P_{\alpha_i}$. Now, suppose $M$ is an arbitrary matching in $(N, P)$ and $r$ is a triple that blocks $M$ in $(N, P)$. We will show that $r$ also blocks $M$ in $(N, P')$. For any $\alpha_k$ in $r$ it must be that $\mathscr{B}_{\alpha_k}(r)$ precedes $\mathscr{B}_{\alpha_k}(M)$ in  $P_{\alpha_k}$. By the construction of $P'_{\alpha_k}$ as the lexicographic order of $P_{\alpha_k}$, it must be that $r$ precedes $M(\alpha_k)$ in $P'_{\alpha_k}$ and thus that $r$ also blocks $M$ in $(N, P')$. It follows that $|\textrm{bt}(M, (N, P'))| \geq |\textrm{bt}(M, (N, P))|$ and thus that $|\textrm{nbt}(M, (N, P))| \geq |\textrm{nbt}(M, (N, P'))|$, as required.

Since Rosenbaum's~\cite{rosenbaum16} analysis of  Algorithm~\algorithmfont{ASA} shows that $|\textrm{nbt}(M, (N, P'))| \geq 4\binom{3n}{3}/9$, we can conclude that $|\textrm{nbt}(M, (N, P))| \geq 4\binom{3n}{3}/9$.
% The algorithm involves first constructing a corresponding instance of 3PSA, involving the same set of agents, and running Algorithm~\algorithmfont{ASA} \cite{rosenbaum16}.
% Construct $(N, P')$ as described in Lemma~\ref{lem:threed_sr_b_approxlex} and run Algorithm~\algorithmfont{ASA} \cite{rosenbaum16} to construct a matching $M$. By the statement of Lemma~\ref{lem:threed_sr_b_approxlex}, the number of non-blocking triples in $M$ in $(N, P')$ is greater than or equal to the number of non-blocking triples in $(N, P)$. It follows that the approximation ratio of this algorithm is also $9/4$.
\end{proof}

We now present an algorithm with an improved approximation ratio. This algorithm, called Algorithm~\algorithmfont{serialDictatorship}, is based on serial dictatorship. In is a variation of Algorithm~\algorithmfont{cyclicSerialDictatorship} that we presented for 3-DSM-CYC in Chapter~\ref{c:three_dsm_cyc}. The accompanying analysis is loosely based on the analysis of Algorithm~\algorithmfont{ASA} given by Rosenbaum \cite{rosenbaum16}.

\begin{algorithm}
\textbf{Input:} an instance $(N, P)$ of 3DR-B\\
\textbf{Output:} a matching $M$ in $(N, P)$
\smallskip
\begin{algorithmic}
\caption{Algorithm~\algorithmfont{serialDictatorship} \label{alg:threed_sr_b_dictatorship}} 

\State $U \gets N$
\State $M \gets \varnothing$

\While{$|U| > 0$}
    \State $d_1 \gets$ an arbitrary agent in $U$
    \State $d_2 \gets \mathscr{B}_{d_1}(U)$
    \State $d_3 \gets \mathscr{B}_{d_2}(U \setminus \{ d_1 \})$
    
    \State $M \gets M \cup \{ \{ d_1, d_2, d_3 \} \}$
    \State $U \gets U \setminus \{ d_1, d_2, d_3 \}$
\EndWhile
\State \textbf{end while}
\smallskip

\State \Return $M$
\end{algorithmic}
\end{algorithm}

It is straightforward to show that Algorithm~\algorithmfont{serialDictatorship} returns a matching $M$ in polynomial time. We now analyse its approximation ratio in the same way we analysed the approximation ratio of Algorithm~\algorithmfont{cyclicSerialDictatorship} for Theorem~\ref{thm:three_dsm_cyc_unrestricted} in Chapter~\ref{c:three_dsm_cyc}, and show that our analysis is tight. We consider each triple in $M$ in the order that they were added to $M$ in the algorithm and count only the blocking triples that intersect that triple and do not intersect any previous triple.

\begin{thm}
\label{thm:threed_sr_b_approxalgo}
There exists a polynomial-time $3/2$-approximation algorithm for 3DR-B-MSM.
\end{thm}
\begin{proof}
Namely Algorithm~\algorithmfont{serialDictatorship}. By the pseudocode, there are $n$ iterations of the while loop so it is straightforward to show that the algorithm runs in polynomial time. For each $i$ where $1\leq i \leq n$, let $d_1^i, d_2^i, d_3^i$ be the agents labelled $d_1, d_2, d_3$ respectively in that iteration, $D_i = \{ d_1^i, d_2^i, d_3^i \}$, and $U_i$ be the set of agents in $U$ at the start of that iteration. Note that by the algorithm, $|U_i \setminus D_i| = 3n - 3i$.

For each $i$ where $1\leq i \leq n$, let $S_i$ be the set of triples that block $M$, have a non-empty intersection with $D_i$, and have an empty intersection with $D_j$ for every $j < i$. It follows that $S_n = \varnothing$ since any triple that blocks $M$ and intersects the final triple $D_n$ must contain some agent not in $D_n$, which must belong to some previous triple $D_j$ where $1 \leq j < n$. We can now define $\textrm{bt}(M)$ in terms of $S_i$:
\begin{align*}
    \text{bt}(M) = \bigcup\limits_{i=1}^{n - 1} S_i\enspace.
\end{align*}
By definition, the sets $S_i$ are pairwise disjoint so it follows that
\begin{align}
    |\text{bt}(M)| = \sum\limits_{i=1}^{n - 1} |S_i|\enspace. \label{eqn:threed_sr_b_instab_as_si}
\end{align}
We now place an upper bound on $|S_i|$ for any $i$ where $1\leq i \leq n - 1$. For any such $i$, consider the $i\textsuperscript{th}$ iteration of the while loop. By the algorithm, $\mathscr{B}_{d_1^i}(U_i) = d_2^i$. It follows that any triple that blocks $M$ and contains $d_1^i$ contains some agent in $N \setminus (U_i \cup D_i)$ that has already been added to some triple $D_j$ in $M$ where $j < i$. Thus, no triple in $S_i$ contains $d_1^i$. Similarly, by the algorithm it must be that either $\mathscr{B}_{d_2^i}(U_i) = d_3^i$ or $\mathscr{B}_{d_2^i}(U_i) = d_1^i$, so any triple that blocks $M$ and contains $d_2^i$ contains some agent in $N \setminus (U_i \cup D_i)$, so likewise no triple in $S_i$ contains $d_2^i$. It follows that any triple in $S_i$ contains $d_3^i$ and two other agents in $N$. By the definition of $S_i$, any triple in $S_i$ has an empty intersection with $D_j$ for any $j < i$ so it must be that any triple in $S_i$ contains $d_3^i$ as well as two agents in $U_i \setminus D_i$. Since $|U_i \setminus D_i| = 3n - 3i$ it follows that
\begin{align}
    |S_i| \leq \binom{3n - 3i}{2}\label{eqn:threed_sr_b_instab_sizeofsi}\enspace.
\end{align}
By definition,
\begingroup
\allowdisplaybreaks
\begin{align}
    |\text{nbt}(M)| &= \binom{3n}{3} - |\text{bt}(M)|\nonumber\\
    &= \binom{3n}{3} - \sum\limits_{i=1}^{n - 1} |S_i|  && \mbox{by Equation~\ref{eqn:threed_sr_b_instab_as_si}} \nonumber\\
    &\geq \binom{3n}{3} - \sum\limits_{i=1}^{n - 1} \binom{3n-3i}{2} && \mbox{by Inequality~\ref{eqn:threed_sr_b_instab_sizeofsi}}\nonumber\\
    &= 3n^3 - \frac{3n^2}{2} - \frac{n}{2}\enspace.\label{eqn:threed_sr_b_stabsize}
\end{align}
\endgroup
Suppose $M^*$ is a matching in $(N, P)$ with the maximum number of non-blocking triples. The approximation ratio of the algorithm is thus
\begin{align*}
    \frac{|\textrm{nbt}(M^*)|}{|\textrm{nbt}(M)|} &\leq 
    \binom{3n}{3} \frac{1}{|\textrm{nbt}(M)|} && \mbox{since $|\textrm{nbt}(M^*)| \leq \binom{3n}{3}$}\\[0.2em]
    &\leq \frac{9n^2 - 9n + 2}{6n^2 - 3n - 1} && \mbox{by Inequality~\ref{eqn:threed_sr_b_stabsize}}\\[0.2em]
    &\leq \frac{3}{2} && \mbox{since $n\geq 1$.}
\end{align*}
\end{proof}

It is desirable to show that this analysis is tight, by constructing an instance $(N, P)$ of 3DR-B and showing that there exists some execution of Algorithm~\algorithmfont{serialDictatorship} that returns a matching $M$ for which $|\textrm{nbt}(M^*)|/|\textrm{nbt}(M)| = 3/2$, where $M^*$ is some matching in $(N, P)$ with the maximum number of non-blocking triples. We show that the analysis is tight asymptotically, by constructing an instance $\mathcal{I}_n$ of 3DR-B for some fixed $n \geq 1$, where the approximation ratio obtained by Algorithm~\algorithmfont{serialDictatorship} on $\mathcal{I}_n$ is $3/2 - o(1)$ in the worst case. The proof follows the same pattern as the analysis of Algorithm~\algorithmfont{cyclicSerialDictatorship} in Chapter~\ref{c:three_dsm_cyc}.

The structure of the preferences of the agents in $\mathcal{I}_n$ corresponds directly to the counting argument used in the proof of Theorem~\ref{thm:threed_sr_b_approxalgo}. For any fixed $n$, construct $\mathcal{I}_n$ as follows. First, let $N = \{ \alpha_1, \alpha_2, \dots, \alpha_{3n} \}$. Next, for each agent $\alpha_j \in N$, construct $P_{\alpha_j}$ so that it lists every agent in $N \setminus \{ \alpha_j \}$ in ascending order of subscript. Finally, for each $i$ where $1\leq i \leq n$, modify $P_{\alpha_3i}$ by shifting $\alpha_{3i - 2}$ and $\alpha_{3i - 1}$ to the right so that $\alpha_{3i - 2}$ is the second-to-last agent in $P_{\alpha_i}$ and $\alpha_{3i - 1}$ is the last agent in $P_{\alpha_i}$.

% let
% \begin{flalign*}
% \setlength\arraycolsep{2pt}
% \begin{array}{r c c c c c c c c c c c}
% P_{\alpha_{3i - 2}} :& \alpha_1 & \alpha_2 & \dots & \alpha_{3i - 3} & \alpha_{3i - 1} & \alpha_{3i} & \alpha_{3i + 1 } & \dots & \alpha_{n} & &\\
% P_{\alpha_{3i - 1}} :& \alpha_1 & \alpha_2 & \dots & \alpha_{3i - 3} & \alpha_{3i - 2} & \alpha_{3i} & \alpha_{3i + 1} & \dots & \alpha_{n} & &\\
% P_{\alpha_{3i}} :& \alpha_1 & \alpha_2 & \dots & \alpha_{3i - 3} & \alpha_{3i + 1} & \alpha_{3i + 2} & \alpha_{3i + 3} & \dots & \alpha_{n} & \alpha_{3i - 2} & \alpha_{3i - 1}\\
% % P_{\alpha_{3i - 1}} :& \alpha_1 & \alpha_2 & \dots & \alpha_{n}\\
% % P_{\alpha_{3i}} :& \alpha_n & \alpha_{n-1} & \dots & \alpha_1
% \end{array}
% \end{flalign*}
As in the proof of Theorem~\ref{thm:threed_sr_b_approxalgo}, for each $i$ where $1\leq i \leq n$, let $d_1^i, d_2^i, d_3^i$ be the agents labelled $d_1, d_2, d_3$ respectively in that iteration, $D_i = \{ d_1^i, d_2^i, d_3^i \}$, and $U_i$ be the set of agents in $U$ at the start of that iteration. Since the selection of agents in $U$ is arbitrary, suppose $d_1^i = \alpha_1$. It follows that $d_2^i = \alpha_2$ and $d_2^i = \alpha_3$ and thus $U_2 = N \setminus \{ \alpha_1, \alpha_2, \alpha_3 \}$. Similarly, the second triple chosen is $D_2 = \{ \alpha_4, \alpha_5, \alpha_6 \}$. In general, it follows that $M = \{ \{ \alpha_1, \alpha_2, \alpha_3 \}, \{ \alpha_4, \alpha_5, \alpha_6 \}, \dots, \{ \alpha_{3n - 2}, \alpha_{3n - 1}, \alpha_{3n} \} \}$ and $U_i = \bigcup_{j = i}^{n} \{ \alpha_{3j - 2}, \alpha_{3j - 1}, \alpha_{3j} \}$. Note that now, for any $\alpha_k \in N$, $\alpha_k$ prefers to $\mathscr{B}_{\alpha_k}(M)$ any agent $\alpha_{3i}$ where $3i < k$.

As in the proof of Theorem~\ref{thm:threed_sr_b_approxalgo}, for each $i$ where $1\leq i \leq n$, let $S_i$ be the set of triples that block $M$, have a non-empty intersection with $D_i$, and have an empty intersection with $D_j$ for every $j < i$. By definition it follows that $S_i \subseteq \binom{U_i}{3}$ and $S_n = \varnothing$. As in the proof of Theorem~\ref{thm:threed_sr_b_approxalgo}, it can be shown that for any $i$ where $1\leq i \leq n$, no triple in $S_i$ contains $\alpha_{3i - 2}$ and no triple in $S_i$ contains $\alpha_{3i - 1}$. It remains that each triple in $S_i$ contains $\alpha_{3i}$ as well as two agents in $U_i \setminus D_i$, which we label $\alpha_k$ and $\alpha_l$. Since $U_i \setminus D_i = \bigcup_{j = i + 1}^{n} \{ \alpha_{3j - 2}, \alpha_{3j - 1}, \alpha_{3j} \}$, it must be that $k \geq 3i + 1$ and $l \geq 3i + 1$. Since $k \neq l$ assume without loss of generality that $k \geq 3i + 2$. It follows that $\mathscr{B}_{\alpha_{3i}}(\{ \alpha_k, \alpha_l \}) = \alpha_k$ precedes both $\alpha_{3i - 1}$ and $\alpha_{3i - 2}$ in $P_{\alpha_{3i}}$. As we noted, since $3i < k$ it must be that $\alpha_{3i}$ precedes $\mathscr{B}_{\alpha_k}(M)$ in $P_{\alpha_k}$. Similarly, since $3i < l$ it must be that $\alpha_{3i}$ precedes $\mathscr{B}_{\alpha_l}(M)$ in $P_{\alpha_l}$. It follows that the triple $\{ \alpha_{3i}, \alpha_k, \alpha_l \}$ blocks $M$ and hence also belongs to $S_i$. Since the selection of $\alpha_k, \alpha_l$ as two agents in $U_i \setminus D_i$ was arbitrary it follows that
\begin{align*}
    S_i = \{ \{ \alpha_{3i}, \alpha_k, \alpha_l \} : \alpha_k, \alpha_l \in U_i \setminus D_i \}\enspace.
\end{align*}
It follows immediately that
\begin{align*}
    |S_i| = \binom{3n - 3i}{2}
\end{align*}
which shows that the upper bound on $|S_i|$ in Inequality~\ref{eqn:threed_sr_b_instab_sizeofsi} in Theorem~\ref{thm:threed_sr_b_approxalgo} is tight. The same argument used in the proof of Theorem~\ref{thm:threed_sr_b_approxalgo} then shows that
\begin{align}
    |\textrm{nbt}(M)| = 3n^3 - \frac{3n^2}{2} - \frac{n}{2} \label{eqn:threed_sr_b_bsizeofnbttight}\enspace.
\end{align}
We now show that a stable matching exists in $\mathcal{I}_n$. Let
\begin{align*}
    M^* = \{ \alpha_1, \alpha_2, \alpha_{3n} \} \cup \bigcup\limits_{i = 1}^{n - 1} \{ \{ \alpha_{3i}, \alpha_{3i + 1}, \alpha_{3i + 2} \} \}\enspace.
\end{align*}
Note first that by the construction of $M^*$, $\mathscr{B}_{\alpha_2}(M^*) = \mathscr{B}_{\alpha_{3n}}(M^*) = \alpha_1$, $\mathscr{B}_{\alpha_1}(M^*) = \alpha_2$. Note also that for any $i$ where $1 \leq i \leq n - 1$, $\mathscr{B}_{\alpha_{3i}}(M^*) = \alpha_{3i + 1}$, $\mathscr{B}_{\alpha_{3i + 1}}(M^*) = \alpha_{3i + 2}$, and $\mathscr{B}_{\alpha_{3i + 2}}(M^*) = \alpha_{3i + 1}$. It follows in general, by the construction of $\mathcal{I}_n$, that for any two agents $\alpha_j$ and $\alpha_k$ if $\alpha_k$ precedes $\mathscr{B}_{\alpha_j}(M^*)$ in $P_{\alpha_j}$ then $k < j$. Consider an arbitrary triple $\{ \alpha_i, \alpha_j, \alpha_k \} \in \binom{N}{3}$ where $i < j < k$. It follows that $\mathscr{B}_{\alpha_i}(\{ \alpha_j, \alpha_k \})$ succeeds $\mathscr{B}_{\alpha_i}(M)$ in $P_{\alpha_i}$ and thus that this triple does not block $M^*$. It follows that $M^*$ is stable. Now
\begingroup
\allowdisplaybreaks
\begin{align*}
    \frac{|\textrm{nbt}(M^*)|}{|\textrm{nbt}(M)|} &= \binom{3n}{3}\frac{1}{|\textrm{nbt}(M)|} && \mbox{since $M^*$ is stable}\\[0.2em]
    &= \frac{9n^2 - 9n + 2}{6n^2 - 3n - 1} && \mbox{by Equation~\ref{eqn:threed_sr_b_bsizeofnbttight}}\\[0.2em]
    &= \frac{3}{2} - o(1)
\end{align*}
\endgroup
which shows that the analysis is tight asymptotically.



% $M(\alpha_{3i}) = \{ \alpha_{3i}, \alpha_{3i - 1}, \alpha_{3i - 2}$ each triple containing two agents in $U_i$

\section{Towards a minimal ``no'' instance}
\label{sec:threed_sr_b_structural}
As in the case of 3-DSM-CYC, it seems difficult to characterise instances of 3DR-B that do not contain a stable matching. With respect to the respective existence problems, which are decision problems, we call such instances ``no'' instances. A significant open question for both 3-DSM-CYC and 3DR-B involves the minimal number of agents required to construct such an instance. The smallest such instance of 3-DSM-CYC uses $60$ agents ($n = 20$) \cite{Lerner22}. In this section, we show that the smallest such instance of 3DR-B contains at least $9$ and at most $15$ agents, but leave determining the precise number as an open problem.

We first show, in Theorem~\ref{thm:threed_sr_b_ifnis6thensmexists}, that any instance of 3DR-B with at most $6$ agents must contain a stable matching. It follows that at least $9$ agents are required to construct an instance of 3DR-B that contains no stable matching.

\begin{thm}
\label{thm:threed_sr_b_ifnis6thensmexists}
In any instance $(N, P)$ of 3DR-B, if $|N| \leq 6$ then $(N, P)$ contains a stable matching.
\end{thm}
\begin{proof}
If $|N| = 3$ then any matching is stable so suppose $|N| = 6$. Consider the directed graph $G = (N, A)$ where $N$ is the set of agents and $( \alpha_i, \alpha_j ) \in A$ if $\mathscr{B}_{\alpha_i}(N) = \alpha_j$. We shall analyse the structure of $G$ in a case analysis, and in each case identify a stable matching $M$. Since the out-degree of each agent in $G$ is $1$, and by definition $G$ contains no self-loops, it must be that each weakly connected component of $G$ contains at least two agents and at least one directed cycle. If $G$ contains:
\begin{itemize}
    \item Three components $\{ \alpha_1, \alpha_2 \}$, $\{ \alpha_3, \alpha_4 \}$, $\{ \alpha_5, \alpha_6 \}$, each of size two, then $M = \{ \{ \alpha_1, \alpha_2, \alpha_3 \}, \{ \alpha_4, \alpha_5, \alpha_6 \} \}$ is stable since $\mathscr{B}_{\alpha_i}(M) = \mathscr{B}_{\alpha_i}(N)$ for each $i$ where $i \in \{ 1, 2, 4, 5 \}$.
    
    \item Two components $\{ \alpha_1, \alpha_2, \alpha_3 \}, \{ \alpha_4, \alpha_5, \alpha_6 \}$ each of size three, then $M = \{ \{ \alpha_1, \alpha_2, \alpha_3 \}, \{ \alpha_4, \alpha_5, \alpha_6 \} \}$ is stable since $\mathscr{B}_{\alpha_i}(M) = \mathscr{B}_{\alpha_i}(N)$ for each $\alpha_i \in N$.
    
    \item Two components $\{ \alpha_1, \alpha_2, \alpha_3, \alpha_4 \},\{ \alpha_5, \alpha_6 \}$ then there are two possible cases. In the first, the longest directed cycle among $\{ \alpha_1, \alpha_2, \alpha_3, \alpha_4 \}$ contains two agents. In this case, assume without loss of generality that this cycle is $\{ \alpha_1, \alpha_2 \}$ and that $( \alpha_3, \alpha_1 ) \in A$. Now $M = \{ \{ \alpha_1, \alpha_2, \alpha_3 \}, \{ \alpha_4, \alpha_5, \alpha_6 \} \}$ is stable since $\mathscr{B}_{\alpha_i}(M) = \mathscr{B}_{\alpha_i}(N)$ for each $i$ where $i \in \{ 1, 2, 5, 6 \}$. In the second, the longest directed cycle among $\{ \alpha_1, \alpha_2, \alpha_3, \alpha_4 \}$ contains three or more agents. In this case, we may assume without loss of generality that $( \alpha_1, \alpha_2 ) \in A$ and $( \alpha_2, \alpha_3 ) \in A$. Now $M = \{ \{ \alpha_1, \alpha_2, \alpha_3 \}, \{ \alpha_4, \alpha_5, \alpha_6 \} \}$ is stable since $\mathscr{B}_{\alpha_i}(M) = \mathscr{B}_{\alpha_i}(N)$ for each $i$ where $i \in \{ 1, 2, 5, 6 \}$.
    
    \item One component $\{ \alpha_1, \alpha_2, \alpha_3, \alpha_4, \alpha_5, \alpha_6 \}$. If the longest directed cycle contains:
    \begin{itemize}
        \item Six agents then $M = \{ \{ \alpha_1, \alpha_2, \alpha_3 \}, \{ \alpha_4, \alpha_5, \alpha_6 \} \}$ is stable since $\mathscr{B}_{\alpha_i}(M) = \mathscr{B}_{\alpha_i}(N)$ for each $i$ where $i \in \{ 1, 2, 4, 5 \}$.
        
        \item Five agents then assume without loss of generality that this cycle is $\{ \alpha_1, \alpha_2, \dots, \alpha_5 \}$. Now $M = \{ \{ \alpha_1, \alpha_2, \alpha_3 \}, \{ \alpha_4, \alpha_5, \alpha_6 \} \}$ is stable since $\mathscr{B}_{\alpha_i}(M) = \mathscr{B}_{\alpha_i}(N)$ for each $i$ where $i \in \{ 1, 2, 4, 5\}$.
        
        \item Four agents then assume without loss of generality that this cycle is $\{ \alpha_1, \alpha_2, \alpha_3, \alpha_4 \}$ and that $( \alpha_5, \alpha_1 ) \in A$. Consider $\mathscr{B}_{\alpha_6}(N)$. If $( \alpha_6, \alpha_1 ) \in A$ then $M = \{ \{ \alpha_1, \alpha_5, \alpha_6 \}, \{ \alpha_2, \alpha_3, \alpha_4 \} \}$ is stable since $\mathscr{B}_{\alpha_i}(M) = \mathscr{B}_{\alpha_i}(N)$ for each $i$ where $i \in \{ 2, 3, 5, 6 \}$. If either $( \alpha_6, \alpha_2 ) \in A$ or $( \alpha_6, \alpha_3 ) \in A$ then $M = \{ \{ \alpha_1, \alpha_4, \alpha_5 \}, \{ \alpha_2, \alpha_3, \alpha_6 \} \}$ is stable since $\mathscr{B}_{\alpha_i}(M) = \mathscr{B}_{\alpha_i}(N)$ for each $i$ where $i \in \{ 2, 4, 5, 6 \}$. If $( \alpha_6, \alpha_4 ) \in A$ then $M = \{ \{ \alpha_1, \alpha_2, \alpha_5 \}, \{ \alpha_3, \alpha_4, \alpha_6 \} \}$ is stable since $\mathscr{B}_{\alpha_i}(M) = \mathscr{B}_{\alpha_i}(N)$ for each $i$ where $i \in \{ 1, 3, 5, 6 \}$.
        
        \item Three agents then assume without loss of generality that this cycle is $\{ \alpha_1, \alpha_2, \alpha_3 \}$. Now $M = \{ \{ \alpha_1, \alpha_2, \alpha_3 \}, \{ \alpha_4, \alpha_5, \alpha_6 \} \}$ is stable since $\mathscr{B}_{\alpha_i}(M) = \mathscr{B}_{\alpha_i}(N)$ for each $i$ where $i \in \{ 1, 2, 3 \}$ and $\{ \alpha_1, \alpha_2, \alpha_3 \} \in M$.
        
        \item Two agents then assume without loss of generality that this cycle is $\{ \alpha_1, \alpha_2 \}$ and also that $( \alpha_3, \alpha_1 ) \in A$. Now $M = \{ \{ \alpha_1, \alpha_2, \alpha_3 \}, \{ \alpha_4, \alpha_5, \alpha_6 \} \}$ is stable since $\mathscr{B}_{\alpha_i}(M) = \mathscr{B}_{\alpha_i}(N)$ for each $i$ where $i \in \{ 1, 2, 3 \}$ and $\{ \alpha_1, \alpha_2, \alpha_3 \} \in M$.
        % \item Length two. Assume that the cycle is $\{ \alpha_1, \alpha_2 \}$. There must exist $\alpha_i$ where $\mathit{top}(\alpha_i) \in \{ \alpha_1, \alpha_2 \}$ otherwise a four-cycle must exist. Assume that $\alpha_i=\alpha_3$. If $M = \{ \{ \alpha_1, \alpha_2, \alpha_3 \}, \{ \alpha_4, \alpha_5, \alpha_6 \} \}$ then $\mathit{top}(\alpha_i) \in M(\alpha_i)$ for $1 \leq i \leq 3$ so $M$ is stable. 
    \end{itemize}
\end{itemize}
\end{proof}

We now show, in Theorem~\ref{thm:threed_sr_b_ifnis15thensmdoesnotexist}, that there exists an instance of 3DR-B with $15$ agents that contains no stable matching. That instance, which we denote by $(N', P')$, is illustrated in Figure~\ref{fig:threed_sr_b_ifnis15example_no_instance} and can be constructed as follows. First, for each $r$ where $1 \leq r \leq 3$ construct a set of five agents in $N'$ labelled $Q_r = \{ q_r^1, q_r^2, \dots, q_r^5 \}$, which we refer to as a \emph{pentagadget}. To simplify the description of the valuations in each pentagadget, in what follows we write $i \oplus y$ to denote $((i + y - 1) \bmod 5) + 1$. For each $r$ where $1 \leq r \leq 3$ and each $i$ where $1 \leq i \leq 5$ let
\begin{flalign*}
\setlength\arraycolsep{2pt}
\begin{array}{r l l l l l c r}
P_{q_r^i} :& q_r^{i \oplus 1} & q_r^{i \oplus 2} & q_r^{i \oplus 3} & q_r^{i \oplus 4} &[& N' \setminus Q_r\text{ in arbitrary order} &]
% [& W\text{ in arbitrary order} &]\ &w_0\ &[& D\text{ in arbitrary order} &]\ &d_0\ &[& U\text{ in arbitrary order} &]
\end{array}
\end{flalign*}

\begin{figure}
\centering\begin{tikzpicture}
% \node[draw=none] (casenumber) at (-1.5, 3.0) {\emph{Case 7}};
% \draw[help lines,step=0.5] (0,0) grid (14,4);
\def\scalefactorq{1.6}
\begin{scope}[every node/.style={circle,draw, minimum size=2.4mm}, scale=1.0]
    \begin{scope}
        \begin{scope}[shift={(-5.2, 0.0)}]
            \node[thick, circle, label={[label distance=0.4cm]90:$q_1^2$}] (q12) at ({90:\scalefactorq}) {};
            \node[thick, circle, label={[label distance=0.4cm]162:$q_1^1$}] (q11) at ({162:\scalefactorq}) {};
            \node[thick, circle, label={[label distance=0.4cm]234:$q_1^5$}] (q15) at ({234:\scalefactorq}) {};
            \node[thick, circle, label={[label distance=0.4cm]306:$q_1^4$}] (q14) at ({306:\scalefactorq}) {};
            \node[thick, circle, label={[label distance=0.4cm]378:$q_1^3$}] (q13) at ({378:\scalefactorq}) {};
            % \node[thick, circle, label={[label distance=0.4cm]0:$w_i^1$}] (v1) at (0,{1.0*1.4}) {};
            % \node[thick, circle, label={[label distance=0.4cm]270:$w_i^2$}] (v2) at ({-0.866*1.4},{-0.5*1.4}) {};
            % \node[thick, circle, label={[label distance=0.4cm]270:$w_i^3$}] (v3) at ({0.866*1.4},{-0.5*1.4}) {};]
            
            % \node[draw=none] (v1a) at (0,{1.0*2.5}) {};
            % \node[draw=none] (v2a) at ({-0.866*2.5},{-0.5*2.5}) {};
            % \node[draw=none] (v3a) at ({0.866*2.5},{-0.5*2.5}) {};
            
            % \begin{scope}[scale=2, shift={(0.0, 0.433)}]
            % \draw [rounded corners=6.5mm, densely dashed] (0.0, 0.0)--(-0.75, -1.3)--(0.75, -1.3)--cycle;
            % \end{scope}
        \end{scope}
        
        \begin{scope}[shift={(0.0, 0.0)}]
            \node[thick, circle, label={[label distance=0.4cm]90:$q_2^2$}] (q22) at ({90:\scalefactorq}) {};
            \node[thick, circle, label={[label distance=0.4cm]162:$q_2^1$}] (q21) at ({162:\scalefactorq}) {};
            \node[thick, circle, label={[label distance=0.4cm]234:$q_2^5$}] (q25) at ({234:\scalefactorq}) {};
            \node[thick, circle, label={[label distance=0.4cm]306:$q_2^4$}] (q24) at ({306:\scalefactorq}) {};
            \node[thick, circle, label={[label distance=0.4cm]378:$q_2^3$}] (q23) at ({378:\scalefactorq}) {};
        \end{scope}
        
        \begin{scope}[shift={(5.2, 0.0)}]
            \node[thick, circle, label={[label distance=0.4cm]90:$q_3^2$}] (q32) at ({90:\scalefactorq}) {};
            \node[thick, circle, label={[label distance=0.4cm]162:$q_3^1$}] (q31) at ({162:\scalefactorq}) {};
            \node[thick, circle, label={[label distance=0.4cm]234:$q_3^5$}] (q35) at ({234:\scalefactorq}) {};
            \node[thick, circle, label={[label distance=0.4cm]306:$q_3^4$}] (q34) at ({306:\scalefactorq}) {};
            \node[thick, circle, label={[label distance=0.4cm]378:$q_3^3$}] (q33) at ({378:\scalefactorq}) {};
        \end{scope}
        
    \end{scope}

\end{scope}

\begin{scope}
    % \foreach \from/\to in {v1/v2, v2/v3, v3/v1, v1/v1a, v2/v2a, v3/v3a}
    %     \draw [thick] (\from) -- (\to);

    \foreach \qr in {q1,q2,q3}
        \foreach \from/\to in {\qr1/\qr2, \qr2/\qr3, \qr3/\qr4, \qr4/\qr5, \qr5/\qr1}
            \draw [thick, firstchoicearrow] (\from) -- (\to);
        
    %     \foreach \from/\to in {gi1/gi4, gi2/gi4, gi3/gi4}
    %     \draw [thick] (\from) -- (\to);
        
    % \draw[thick] (dr4) to[out=225, in=70] (dr6);
    % \draw[thick] (dr4) to[out=315, in=110] (dr7);
    
    % \draw[thick] (dr5) to[out=70, in=225] (dr6);
    % \draw[thick] (dr5) to[out=20, in=225] (dr7);
    
    % \draw[thick] (dr8) to[out=160, in=315] (dr6);
    % \draw[thick] (dr8) to[out=110, in=315] (dr7);

\end{scope}
\end{tikzpicture}\caption[A representation of the instance $(N', P')$ of 3DR-B that contains no stable matching]{A representation of the instance $(N', P')$ of 3DR-B that contains no stable matching. An arrow exists from $q_r^i$ to $q_r^j$ if $\mathscr{B}_{q_r^i}(N') = q_r^j$.} 
    \label{fig:threed_sr_b_ifnis15example_no_instance}
\end{figure}

Suppose $M$ is an arbitrary matching in $(N', P')$. We shall eventually show that $M$ is not stable. For any agent $q_r^i \in N'$, we say that $q_r^i$ is \emph{external} if $\mathscr{B}_{q_r^i}(M) \notin Q_r$.
% Since $|Q_r| = 5$ it follows trivially that each set in $\{ Q_1, Q_2, Q_3 \}$ contains at least one external agent. We show in Lemma~\ref{lem:threed_sr_b_ifnis15example_lemma} that at least one such set contains at least two external agents.

\begin{lem}
\label{lem:threed_sr_b_ifnis15example_lemma}
At least one pentagadget contains at least two external agents.
\end{lem}
\begin{proof}
Since each pentagadget contains five agents it must be that $Q_1$ contains at least one external agent. Without loss of generality assume that $q_1^1$ is external. If $Q_1$ contains two external agents, including $q_1^1$, then the lemma statement holds, so suppose $q_1^1$ is the only external agent in $Q_1$. It follows that $\{ q_1^{i_1}, q_1^{i_2}, q_r^{j_1} \}, \{ q_1^{i_3}, q_1^{i_4}, q_s^{k_1} \} \in M$ where $\{ i_1, i_2, i_3, i_4 \} = \{ 2, 3, 4, 5 \}$, $r, s \in \{ 2, 3 \}$, and $1\leq {j_1}, {k_1}\leq 5$. If $r = s$ then $Q_r$ contains two external agents, namely $q_r^{j_1}$ and $q_r^{k_1}$, and thus the lemma statement holds, so suppose $r \neq s$. Assume without loss of generality that $q_r^{j_1} = q_2^1$ and $q_s^{k_1} = q_3^1$.

By definition, $q_2^1$ is external. If $Q_2$ contains two external agents, including $q_2^1$, then the lemma statement holds, so suppose $q_2^1$ is the only external agent in $Q_2$. 
It follows that $\{ q_2^{l_2}, q_2^{l_3}, q_t^{j_2} \}, \{ q_2^{l_4}, q_2^{l_5}, q_u^{k_2} \} \in M$, where $\{ l_2, l_3, l_4, l_5 \} = \{ 2, 3, 4, 5 \}$, $t, u \in \{ 1, 3 \}$, and $1\leq {j_2}, {k_2} \leq 5$. If $t = u$ then $Q_t$ contains two external agents, namely $q_t^{j_2}$ and $q_t^{k_2}$, and thus the lemma statement holds, so it must be that $t \neq u$. Assume without loss of generality that $t=1$ and $u=3$. Now $Q_3$ contains two external agents, namely $q_3^1$ and $q_3^{k_2}$, and thus the lemma statement holds.
% where $r, s \in \{ 1, 3 \}$ and $1\leq i, j \leq 5$.
% Since each set in $\{ Q_1, Q_2, Q_3 \}$ contains five agents, it must be that each such set contains at least one external agent and thus the total number of external agents is at least three. If the total number of external agents is four or more then by the pigeonhole principle at least one set in $\{ Q_1, Q_2, Q_3 \}$ contains at least two external agents, and thus the lemma statement holds. The only remaining possibility is that each set in $\{ Q_1, Q_2, Q_3 \}$ contains exactly one external agent. Consider $Q_1$. 
% Since $|N'|=15$ it must be that the total number of internal agents is exactly $12$. 
% Since each set in $\{ Q_1, Q_2, Q_3 \}$ contains five agents it must be that each such set contains at least one external agent, and thus the total number of internal agents is at most $12$. 
% Each triple in $M$ contains either three internal agents, two internal agents and one external agent, or three external agents. Let $T(y)$ be the triples in $M$ that contain exactly $y$ internal agents. It follows that the total number of internal agents is $3|T(3)| + 2|T(2)| = 12$, and thus that 
% By definition, the number of internal agents in any triple in $M$ is either $0$, $2$, or $3$. Suppose $k$ triples in $M$ contain exactly $3$ internal agents and $l$ triples in $M$ contain exactly $2$ internal agents. It follows that the total number of internal agents is $3k + 2l$. If the 
\end{proof}

% We now show, in Lemma~\ref{lem:threed_sr_b_ifnis15example_lemma2}, that there must exist some triple that blocks $M$. 

% \begin{lem}
% \label{lem:threed_sr_b_ifnis15example_lemma2}
% $M$ is not stable.
% \end{lem}
% \begin{proof}
% By Lemma~\ref{lem:threed_sr_b_ifnis15example_lemma}, assume without loss of generality that $Q_1$ contains at least two external agents, which we label $q_1^{i_1}$ and $q_1^{i_2}$. By the symmetry of each pentagadget, without loss of generality assume that $i_1 = 1$ and either $i_2 = 2$ or $i_2 = 3$. In the former case, $\{ q_1^1, q_1^2, q_1^5 \}$ blocks $M$, since $q_1^2 \succ_{q_1^1} \mathscr{B}_{q_1^1}(M)$, $\mathscr{B}_{q_1^2}(\{ q_1^1, q_1^5 \}) \succ_{q_1^2} \mathscr{B}_{q_1^2}(M)$, and $q_1^1 \succ_{q_1^5} \mathscr{B}_{q_1^5}(M)$. In the latter case, $\{ q_1^1, q_1^2, q_1^3 \}$ blocks $M$, since $q_1^2 \succ_{q_1^1} \mathscr{B}_{q_1^1}(M)$, $q_1^3 \succ_{q_1^2} \mathscr{B}_{q_1^2}(M)$, and $\mathscr{B}_{q_1^3}(\{ q_1^1, q_1^2 \}) \succ_{q_1^3} \mathscr{B}_{q_1^3}(M)$.
% \end{proof}

\begin{thm}
\label{thm:threed_sr_b_ifnis15thensmdoesnotexist}
There exists an instance of 3DR-B with $15$ agents that contains no stable matching.
\end{thm}
\begin{proof}
Namely the instance $(N', P')$. Recall that $M$ is an arbitrary matching in $(N', P')$. By Lemma~\ref{lem:threed_sr_b_ifnis15example_lemma}, assume without loss of generality that $Q_1$ contains at least two external agents, which we label $q_1^{i_1}$ and $q_1^{i_2}$. By the symmetry of each pentagadget, without loss of generality assume that $i_1 = 1$ and either $i_2 = 2$ or $i_2 = 3$. In the former case, $\{ q_1^1, q_1^2, q_1^5 \}$ blocks $M$, since $q_1^2 \succ_{q_1^1} \mathscr{B}_{q_1^1}(M)$, $\mathscr{B}_{q_1^2}(\{ q_1^1, q_1^5 \}) \succ_{q_1^2} \mathscr{B}_{q_1^2}(M)$, and $q_1^1 \succ_{q_1^5} \mathscr{B}_{q_1^5}(M)$. In the latter case, $\{ q_1^1, q_1^2, q_1^3 \}$ blocks $M$, since $q_1^2 \succ_{q_1^1} \mathscr{B}_{q_1^1}(M)$, $q_1^3 \succ_{q_1^2} \mathscr{B}_{q_1^2}(M)$, and $\mathscr{B}_{q_1^3}(\{ q_1^1, q_1^2 \}) \succ_{q_1^3} \mathscr{B}_{q_1^3}(M)$.
% We considered an arbitrary matching $M$ in the example instance $(\hat{N}, P')$. By Lemma~\ref{lem:threed_sr_b_ifnis15example_lemma2}, $M$ is not stable in $(\hat{N}, P')$. It follows that $|N'| \leq |\hat{N}| = 15$.
\end{proof}


\section{Summary and open problems}
\label{sec:threed_sr_b_conclusion}
In this chapter we studied a new formalism of 3DR, involving $\mathscr{B}$-preferences, which we called 3DR-B. We considered in 3DR-B the existence of, and complexity of finding, matchings that are stable. 

Our first result was that a given instance of 3DR-B may not contain a stable matching and that the associated decision problem is $\NP$-complete. We then considered a closely related optimisation problem, which we called 3DR-B-MSM, in which the objective is to construct, in a given instance of 3DR-B, a matching with the maximum number of non-blocking triples. We first devised a $9/4$-approximation algorithm for 3DR-B-MSM based on an existing algorithm for 3PSA-MSM, which is a closely related problem \cite{rosenbaum16}. Improving upon this approximation, we then presented a $3/2$-approximation algorithm based on serial dictatorship, and showed that our analysis is tight asymptotically. Finally, we considered the problem of identifying the smallest instance of 3DR-B that contains no stable matching. We showed that such an instance must have between $9$ and $15$ agents, inclusive.

% As we noted in Chapter~\ref{c:lit_review}, there are a wide variety of other systems of preference representation and solution concepts that could be used to formalise alternative model of 3DR, as well as $\mathscr{B}$, $\mathscr{W}$, and additively separable preferences. 

We now present some open problems specifically involving stability in 3DR-B. More general problems, involving solution concepts other than stability and other models of fixed-size coalition formation, are discussed in Chapter~\ref{c:conclusion}. 

An immediate open problem is to improve the bounds on the the smallest instance of 3DR-B that contains no stable matching. We have shown in Section~\ref{sec:threed_sr_b_structural} that such an instance contains at least $9$ and at most $15$ agents, but the precise number of agents remains open. To fully resolve this open question, it will be necessary to either prove that every some fixed size strictly greater than $6$ contains a stable matching (as in the proof of Theorem~\ref{thm:threed_sr_b_ifnis6thensmexists}), demonstrate that some instance with between $9$ and $12$ agents contains no stable matching (in a similar way to the proof of Theorem~\ref{thm:threed_sr_b_ifnis15thensmdoesnotexist}), or both.

In Chapter~\ref{c:three_dsm_cyc}, we presented an approximation algorithm for a restriction of 3-DSM-CYC-MSM in which the preferences of some agents were derived from a master list. We conjecture that a similar algorithm also exists for a restriction of 3DR-B-MSM in which the preferences of all agents are derived from a master list. In 2020, Bredereck et al.\ \cite{Bre20} considered a similar situation for a multidimensional generalisation of 3GSM \cite{NH91}, and it may be that some of their results or techniques can also be applied to 3DR-B.

A closely related objective is to estimate the probability that a random instance of 3DR-B contains a stable matching. Pittel's \cite{Pittel20} probabilistic analysis of $k$-DSM-CYC and Pittel and Irving's \cite{PI94} analysis of two-dimensional Stable Roommates (SR) are two possible starting points. A hybrid of theoretical and empirical techniques might also be informative, as it was in Escamocher and O'Sullivan's \cite{Escamocher2018} paper, which considered the same question in the setting to 3-DSM-CYC.

As well as investigating the existence of an improved approximation algorithm for 3DR-B-MSM, it might be possible to prove an inapproximability result for this problem. For example, it would be very informative to prove that the approximation ratio of Algorithm~\algorithmfont{serialDictatorship} is tight, by showing that no $(3/2 - \varepsilon)$-approximation algorithm exists for 3DR-B-MSM unless $\P = \NP$. Alternatively, it might be easier to prove the weaker result that 3DR-B-MSM is $\APX$-hard, meaning that there exists some constant factor $\varepsilon$ such that no $(1 + \varepsilon)$-approximation algorithm exists for 3DR-B-MSM, unless $\P = \NP$ \cite{ACGKMP99}. A starting point towards the latter result could be to modify the reduction of Iwama et al.\ \cite{IMO08} from a variant of \emph{Maximum 3D Matching} (Max 3DM) to an optimisation problem defined in a related model of 3DR (which is described in Chapter~\ref{c:lit_review}). Another possibility is to adapt the reduction of Rosenbaum \cite{rosenbaum16} from Max 3DM to 3PSA-MSM.

Various alternative optimisation problems and measures can also be defined that relate to stable matchings and 3DR-B.
One possibility is to construct a sub-matching of maximum cardinality such that no three agents in triples in the sub-matching form a blocking triple in the sub-matching. Rosenbaum \cite{rosenbaum16} refers to the analogous problem for 3PSA as the \emph{3PSA Maximum Stable Sub-matching problem} (3PSA-MSS). 
A second possibility is to consider $\alpha$-stability \cite{ABEOMP09} (discussed in Chapter~\ref{c:lit_review}) in the setting of 3DR-B. For example, for some fixed $\alpha \geq 1$, we could say that a matching $M$ is $\alpha$-stable if for any agent $\alpha_i$ and any triple $t$ where $\alpha_i \in t$ the increase in rank in $P_{\alpha_i}$ from $\mathscr{B}(M)$ to $\mathscr{B}(t)$ is at most $\alpha$. We could also then define an optimisation problem in which the objective is to find an $\alpha$-stable matching for a minimum such $\alpha$.
A third possibility is to define a complementary problem in which the objective is to minimise the number of blocking triples, which is arguably more natural. Similar optimisation problems, in which the objective is to minimise the number of blocking pairs, have been studied in the context of (two-dimensional) Stable Roommates \cite{ABM06}. % Unlike 3DR-B, deciding if a given instance of SR contains a stable matching is solvable in polynomial time, and it is unclear if the techniques relating to SR can be also applied to the minimisation problem of 3DR-B-MSM.

% In terms of parameterised complexity, a natural starting point is a recent work of Bredereck et al.\ \cite{Bre20}, which explores the parameterised complexity of a generalisation of 3GSM in which each agent's preference list, over sets of agents, is derived from a central master list or poset. It may be possible to consider an analogous situation in 3DR-B in which each agent's preference list, of individual agents, is derived from a central master list or poset.

\chapter{Three-Dimensional Stable Roommates with \texorpdfstring{$\mathscr{W}$}{W}-preferences}
\label{c:threed_sr_w}
\chaptermark{Stability in 3DR-W}

\section{Introduction}
\label{sec:threed_sr_w_intro}
In this chapter we study another new model of fixed-size coalition formation, which we call the \emph{Three-Dimensional Roommates with $\mathscr{W}$-preferences} (\mysymbolfirstusedefinition{symboldef:threedr_w}{3DR-W}). This model is in a sense dual to the \emph{Three-Dimensional Roommates with $\mathscr{B}$-preferences} (3DR-B), which we considered in Chapter~\ref{c:threed_sr_b}. As in 3DR-B, in 3DR-W each agent supplies a strict preference list over all other agents and uses a set extension rule to compare coalitions. In 3DR-W, the set extension rule used is known as \emph{$\mathscr{W}$-preferences} \cite{CH04} and means that any agent prefers some triple $S$ to another triple $T$ if the least-preferred agent in $S$ is more preferred than the least-preferred agent in $T$. In this chapter we consider in 3DR-W the existence of, and complexity of finding, matchings that are stable.

We first show, in Section~\ref{sec:threed_sr_w_hardness}, that a given instance of 3DR-W may not contain a stable matching and that the associated decision problem is $\NP$-complete (Theorem~\ref{thm:threed_sr_w_existence}). As in the case of 3DR-B, this also contrasts with an analogous model involving $\mathscr{W}$-preferences in which coalitions need not have a fixed size, in which a stable matching must exist and can be found in polynomial time \cite{CH04}. It may seem intuitive that the additional requirement of fixed-size coalitions makes this particular problem harder to solve, and this result gives another example of a model in which that intuition holds.

We then consider, in Section~\ref{sec:threed_sr_w_approximation}, a closely related optimisation problem in which the objective is to construct a matching that is, in terms of a specific measure, as stable as possible. We show that an existing approximation algorithm for a different model of 3DR can be used to devise a $9/4$-approximation algorithm for this problem (Theorem~\ref{thm:threed_sr_w_approxalgofournine}).

Finally, in Section~\ref{sec:threed_sr_w_conclusion}, we recap on our results and discuss some directions for future work.

An instance of 3DR-W comprises a set $N$ of $3n$ agents and a preference list of each agent $\alpha_i$, labelled $P_{\alpha_i}$, that describes a strict order over all agents in $N \setminus \{ \alpha_i \}$. We say that an agent $\alpha_i$ \emph{prefers} $\alpha_j$ to $\alpha_k$, denoted $\alpha_j \succ_{\alpha_i} \alpha_k$, if $\alpha_j$ precedes $\alpha_k$ in $P_{\alpha_i}$.  A \emph{triple} is an unordered set of three agents. In order to compare triples, agents in an instance of 3DR-W use \emph{$\mathscr{W}$-preferences}, which are defined as follows. For any agent $\alpha_i$ and set of agents $S \subseteq N$ we denote by $\mathscr{W}_{\alpha_i}(S)$ the least-preferred agent in $S \setminus \{ \alpha_i \}$ according to $\alpha_i$. For any agent $\alpha_i$ and any two triples $r$ and $s$, we say that $\alpha_i$ \emph{prefers} $r$ to $s$, denoted $r \succ_{\alpha_i} s$, if $\mathscr{W}_{\alpha_i}(r) \succ_{\alpha_i} \mathscr{W}_{\alpha_i}(s)$. A \emph{matching} is a partition of $N$ into $n$ triples. Given an agent $\alpha_i$ and a matching $M$, we denote by $M(\alpha_i)$ the triple in $M$ that contains $\alpha_i$. Given a matching $M$, we say that a triple $t$ is \emph{blocking} if each agent $\alpha_i$ in $t$ prefers $t$ to $M(\alpha_i)$. A matching is \emph{stable} if it does not contain a blocking triple. Let $\mathscr{W}_{\alpha_i}(M)$ be short for $\mathscr{W}_{\alpha_i}(M(\alpha_i))$. Let $P$ be the collection of preference lists  $P_{\alpha_i}$ for each agent $\alpha_i$. For any instance $(N, P)$ of 3DR-W and any matching $M$, we denote by $\textrm{bt}(M, (N, P)) \subseteq \binom{N}{3}$ the set of triples that block $M$ in $(N, P)$. Conversely, we denote by $\textrm{nbt}(M, (N, P)) = \binom{N}{3} \setminus \textrm{bt}(M, (N, P))$ the set of triples that do not block $M$ in $(N, P)$. When the instance in question is clear from context, we simply write $\textrm{bt}(M)$ or $\textrm{nbt}(M)$. 

% The following definition of 3DR-W is identical to the definition of 3DR-B 
% except the agents in an instance of 3DR-W compare triples using $\mathscr{W}$-preferences instead of $\mathscr{B}$-preferences, which can be formally defined as follows.
% For any agent $\alpha_i$ and set of agents $S \subseteq N$ we denote by $\mathscr{W}_{\alpha_i}(S)$ the least-preferred agent in $S \setminus \{ \alpha_i \}$ according to $\alpha_i$. For any agent $\alpha_i$ and any two triples $r$ and $s$, we say that $\alpha_i$ \emph{prefers} $r$ to $s$, denoted $r \succ_{\alpha_i} s$, if $\mathscr{W}_{\alpha_i}(r) \succ_{\alpha_i} \mathscr{W}_{\alpha_i}(s)$. All of the other notation and terminology used in this chapter is defined as for 3DR-B in Chapter~\ref{c:threed_sr_b}.
% In this chapter we formalise a model of 3DR that involves $\mathcal{W}$-preferences and consider in this formalism the existence of, and complexity of finding, stable matchings.
% Here, instead of $\mathscr{B}$-preferences, we use $\mathscr{W}$-preferences, which are defined as follows. Each agent $\alpha_i$ to supply an ordinal preference list $P_{\alpha_i}$. $P_{\alpha_i}$ describes a total order on $N\setminus \{ \alpha_i \}$. 
% % In the (two-dimensional) Stable Roommates Problem \cite{GS62}\cite{Irv85} such a preference list exactly describes an agent's preference between any two possible pairs that they may belong to. Here we extend agents' preference lists to a preference relation between possible triples. This extension from preferences over individuals to preferences over sets of individuals has been well studied in the context of computational social choice \cite{BBP04}\cite{Pattanaik1984} and has been applied in the related field of Hedonic Games (also referred to as Stable Partition problems) \cite{HOCSC16} \cite{CH04} \cite{CH04a}. In this context we apply \textit{$\mathscr{W}$-preferences}, which we define as follows. 
% Given an agent $\alpha_i$ and a set $S\subseteq N$ let $\mathscr{W}_{\alpha_i}(S)$ be the least-preferred agent in $S \setminus \{ \alpha_i \}$ according to $P_{\alpha_i}$.  For agents $\alpha_i,\alpha_j,\alpha_k,\alpha_l,\alpha_m$, we say that $\alpha_i \in N$ \textit{prefers} one triple $r_1$ to another $r_2$, written $r_1 \succ_{\alpha} r_2$, if $\mathscr{W}_{\alpha_i}(r_1) \succ_{\alpha_i} \mathscr{W}_{\alpha_i}(r_2)$. an agent's preference between matchings is inferred. agent $\alpha_i$ \textit{prefers} matching $M$ to $M'$, written $M \succ_{\alpha_i} M'$, if $M(\alpha_i) \succ_{\alpha_i} M'(\alpha_i)$. Note that for any $\alpha_i \in N$, whilst $P_{\alpha_i}$ is a total order on $N\setminus \{ \alpha_i \}$, in this model using $\mathscr{W}$-preferences do not produce a total order of all triples that contain $\alpha_i$. For example, suppose $\alpha_i, \alpha_j, \alpha_k, \alpha_l \in N$ where $P_{\alpha_i} = \alpha_j, \alpha_k, \alpha_l, \dots$. Neither $\{\alpha_i, \alpha_j, \alpha_l \} \succ_{\alpha_i} \{\alpha_i, \alpha_k, \alpha_l \}$ nor $\{\alpha_i, \alpha_k, \alpha_l \} \prec_{\alpha_i} \{\alpha_i, \alpha_j, \alpha_l \}$ holds.

% % We say that a triple $r=\{\alpha_i, \alpha_j, \alpha_k\}$ \textit{blocks} matching $M$ if $r \succ_{\alpha_i} M(\alpha_i)$, $r \succ_{\alpha_j} M(\alpha_j)$, and $r \succ_{\alpha_k} M(\alpha_k)$. We say that a triple is \emph{stable} if it does not block $M$. Given a matching $M$, if no three agents in $N$ form a triple that blocks $M$ then we say that $M$ is a \textit{stable matching}. We say that an instance $(N,P)$ of 3D-SR-W \emph{contains} or \emph{admits} a stable matching if there exists a stable matching in $(N,P)$.

% % \begin{model}The Stable Roommates problem With Triple Rooms and $\mathscr{W}$-preferences (3D-SR-W)\\
% % \label{prob:threed_sr_w}\hspace*{8pt}\inp $(N,P)$ where:\\
% % \hspace*{16pt} $N=\{\alpha_1, \dots, \alpha_{3n}\}$ is the set of agents, for $n\geq 1$\\
% % \hspace*{16pt} $P=\bigcup\limits_{\alpha_i\in N} P_{\alpha_i}$ where $P_{\alpha_i}$ is the the strict preference list of $\alpha_i$ over $N \setminus \{ \alpha_i \}$\\
% % \hspace*{8pt}\verificationproblem Is a given matching in $(N,P)$ stable using $\mathscr{W}$-preferences?\\
% % \hspace*{8pt}\existenceproblem Does a stable matching exist in $(N,P)$ using $\mathscr{W}$-preferences?
% % \begin{adjustwidth}{8pt}{}
% % \constructionproblem Construct a matching in $(N,P)$ that is stable using $\mathscr{W}$-preferences, or report that no such matching exists.
% % \end{adjustwidth}
% % \end{model}

% % It is straightforward to show that the problem of deciding if an instance of 3D-SR-W contains a stable matching belongs to $\NP$. Our main result is that that this problem, which we refer to as the 3D-SR-W decision problem, is $\NP$-complete.



\section{Deciding existence}
\label{sec:threed_sr_w_hardness}
In this section we show that deciding if a given instance of 3DR-W contains a stable matching is $\NP$-complete. The reduction is from a restricted case of \emph{Partition Into Triangles} (PIT, Problem~\ref{prob:pit}) \cite{GJ79}. 

\begin{myproblem}[Partition Into Triangles (PIT)]
\label{prob:pit}\mysymbolfirstusedefinition{symboldef:pit}{}
\begin{samepage}
\begin{adjustwidth}{8pt}{8pt}
\inp a simple undirected graph $G=(W, E)$ where $|W|=3q$ for some integer $q$\\
\ques Can the vertices of $G$ be partitioned into $q$ disjoint sets $X=\{ X_1, X_2, \dots, X_q \}$, each set containing exactly three vertices, such that each $X_p=\{ w_i, w_j, w_k \}\in W$ where $1\leq p\leq q$ is a triangle?
\end{adjustwidth}
\end{samepage}
\end{myproblem}

Given a graph simple undirected graph $G = (W, E)$, we say that a set of three vertices $\{ w_i, w_j, w_k \}$ is a \emph{triangle} if $\{ w_i, w_j \} \in E$, $\{ w_i, w_k \} \in E$, and $\{ w_j, w_k \} \in E$. Let $\mathcal{T} = \{ T_1, T_2, \dots, T_m \}$ be the set of triangles in $G$. We reduce from the special case of PIT in which $\mathcal{T}$ has a \emph{system of distinct representatives} (\mysymbolfirstusedefinition{symboldef:sdr}{SDR}), that is, a set $Z = \{ z_1, z_2, \dots, z_m \}$ of $m$ distinct vertices where $z_i \in T_i$ for each $i$ where $1\leq i \leq m$. We refer to this restricted problem as \emph{PIT-SDR}. It can be verified that in the reduction shown by Garey and Johnson \cite[Theorem~3.7]{GJ79} from \emph{Exact Cover by Three-sets} to PIT that the constructed graph admits an SDR\footnote{In the reduction of Garey and Johnson, every triangle in $G$ is contained in exactly one subset gadget. For each subset gadget $c_i \in C$, the SDR contains the vertices labelled ``$a_i[1]$'', ``$a_i[2]$'', ``$a_i[4]$``, ``$a_i[5]$'', ``$a_i[7]$'', ``$a_i[8]$'', and ``$a_i[6]$''.}. It follows that PIT-SDR is $\NP$-complete.

The reduction from PIT-SDR is as follows. Suppose $G = (W, E)$ is an arbitrary graph. We shall construct an instance $(N, P$) of 3DR-W. For each vertex $i$ where $1 \leq i \leq 3q$ construct a set of seven agents $H_i = \{ h_i^1, h_i^2, \dots, h_i^7 \}$, which we refer to as the \emph{$i\textsuperscript{th}$ heptagadget}. Let $H^7$ be the set $\bigcup_{1 \leq i \leq 3q} h_i^7$. We shall now construct the preferences of the agents in each heptagadget. First, for each heptagadget $H_i$ construct the agents' preferences as follows. Note that in the following construction ``$\dots$'' denotes all remaining agents in an arbitrary order.

\begin{flalign*}
\begin{array}{rlcrclcr}
h_i^1 : & & & &\; \; h_i^3 \ \ h_i^7 \ \ h_i^4 \ \ h_i^6 \ \ h_i^2 \ \ h_i^5 && \dots &\\
h_i^2 : & & & &\;\; h_i^7 \ \ h_i^1 \ \ h_i^3 \ \ h_i^6 \ \ h_i^4 \ \ h_i^5 & &
\multicolumn{1}{c}{\dots} & \\
h_i^3 : & & & &\;\; h_i^4 \ \ h_i^1 \ \ h_i^7 \ \ h_i^6 \ \ h_i^2 \ \ h_i^5 & &
\multicolumn{1}{c}{\dots} & \\
h_i^4 : & & & &\;\; h_i^6 \ \ h_i^7 \ \ h_i^1 \ \ h_i^3 \ \ h_i^2 \ \ h_i^5 & &
\multicolumn{1}{c}{\dots} & \\
h_i^5 : & & & &\;\; h_i^7 \ \ h_i^6 \ \ h_i^1 \ \ h_i^4 \ \ h_i^2 \ \ h_i^3 & &
\multicolumn{1}{c}{\dots} & \\
h_i^6 : & & & & \;\; h_i^1 \ \ h_i^3 \ \ h_i^7 \ \ h_i^4 \ \ h_i^2 \ \ h_i^5 & &
\multicolumn{1}{c}{\dots} & \\
h_i^7 : & \undermat{\text{\small{proper part}}}{[& h_j^7 \in (H^7\setminus \{ h_i^7 \}) : \{ w_i, w_j \} \in E &]} & \;\; h_i^6 \ \ h_i^4 \ \ h_i^1 \ \ h_i^3 \ \ h_i^2\ \ h_i^5 & &
\multicolumn{1}{c}{\dots} &\\[5.2ex]
\end{array}
\end{flalign*}
We now reorder the proper part of the preference list of each agent in $H^7$. Our aim is to ensure that for any three agents $h_i^7, h_j^7, h_k^7 \in H^7$ if the three corresponding vertices $w_i, w_j, w_k$ form a triangle in $G$ then one of the three agents is the least-preferred agent in the proper part of the preference list of at least one of the other two agents. To do this, first identify the set of triangles $\mathcal{T} = \{ T_1, T_2, \dots, T_m \}$ and then construct an SDR of $\mathcal{T}$ labelled $Z = \{ z_1, z_2, \dots, z_m \}$. Note that $Z$ can be constructed in polynomial time as a maximum matching in a bipartite graph. Next, for each $i$ where $1\leq a \leq m$ consider the triangle $T_a = \{ w_i, w_j, w_k \}$ in $G$, labelling the representative vertex $z_a$ in $Z$ as $w_i$ and the other two vertices as $w_j$ and $w_k$ arbitrarily. Since $\{ w_i, w_j \} \in E$ it must be that $h_j^7$ appears in the proper part of $P_{h_i^7}$. Reorder the proper part of $P_{h_i^7}$ such that $h_j^7$ is now the least-preferred agent in the proper part. Note that by the definition of an SDR, no preference list is modified more than once and thus we have achieved our aim. This completes the construction of $(N, P)$.

It is straightforward to show that this reduction can be performed in polynomial time. To prove that the reduction is correct we show that a stable matching exists in the 3DR-W instance $(N, P)$ if and only if a partition into triangles exists in the PIT-SDR instance $G$.

We first show that if the PIT-SDR instance $G$ contains a partition into triangles then the 3DR-W instance $(N, P)$ contains a stable matching.

\begin{lem}
\label{lem:threed_sr_w_stablematchingifpartitionexists}
If $G$ contains a partition into triangles then $(N, P)$ contains a stable matching.
\end{lem}
\begin{proof}
Suppose $\mathcal{X} = \{ X_1, X_2, \dots, X_q \}$ is a partition into triangles in $G$. For each triangle $X_p = \{ w_i, w_j, w_k \} \in \mathcal{X}$ construct in $M$ the triples $\{h_i^7,h_j^7,h_k^7\}$, $\{h_i^1,h_i^3,h_i^4\}$, $\{h_i^2,h_i^5,h_i^6\}$, $\{h_j^1,h_j^3,h_j^4\}$, $\{h_j^2,h_j^5,h_j^6\}$, $\{h_k^1,h_k^3,h_k^4\}$ and $\{h_k^2,h_k^5,h_k^6\}$.

To show that $M$ is stable, we consider each agent in an arbitrary heptagadget $H_i$. 

First, consider $h_i^7$. Suppose for a contradiction that $h_i^7$ belongs to a triple that blocks $M$. By the construction of $M$, $\mathscr{W}_{h_i^7}(M)$ belongs to the proper part of $P_{h_i^7}$ so it must be that this blocking triple comprises three agents $\{ h_i^7, h_j^7, h_k^7 \}$ where $\{ w_i, w_j \} \in E$ and $\{ w_i, w_k \} \in E$. A symmetric argument shows that $h_i^7$ must appear in the proper part of $P_{h_j^7}$ and thus that $\{ w_i, w_j, w_k \}$ is a triangle in $G$. Suppose the triangle $\{ w_i, w_j, w_k \}$ is labelled $T_a$ where $1\leq a \leq m$ and assume without loss of generality that $w_i$ is the representative vertex $z_a$ in the SDR $Z$ of $\mathcal{T}$. It follows that $\{ h_i^7, h_j^7, h_k^7 \}$ contains the least-preferred agent in the proper part of $P_{h_i^7}$. Assume without loss of generality that $h_j^7$ is the least-preferred agent in the proper part of $P_{h_i^7}$. This contradicts the fact that $h_j^7$ must appear before $\mathscr{W}_{h_i^7}(M)$, which also belongs to the proper part of $P_{h_i^7}$. It remains that $h_i^7$ does not belong to a triple that blocks $M$.

Next, consider $h_i^3$. No triple is preferred by $h_i^3$ to $M(h_i^3)$ so $h_i^3$ does not belong to a blocking triple. 
Similarly, the only triple that $h_i^1$ prefers to $M(h_i^1)$ contains $h_i^7$, which we have shown does not belong to a blocking triple. It follows that $h_i^1$ also does not belong to a blocking triple. 
Any triple $h_i^4$ prefers to $M(h_i^4)$ and contains $h_i^1$ or $h_i^3$ is not blocking, and the only triple that $h_i^4$ prefers to $M(h_i^4)$ that contains neither $h_i^1$ nor $h_i^3$ is $\{ h_i^4, h_i^6, h_i^7 \}$. Since $h_i^7$ does not belong to a blocking triple it follows thus that $h_i^4$ does not belong to a blocking triple. 
Similarly, the only triples that $h_i^2$, $h_i^5$, and $h_i^6$ prefer to $M(h_i^2)$, $M(h_i^5)$, and $M(h_i^6)$ contain at least one of $h_i^1$, $h_i^3$, $h_i^4$, and $h_i^7$ so are not blocking. It follows that neither $h_i^2$, $h_i^5$, nor $h_i^6$ belong to blocking triples.
\end{proof}

We now show, using a sequence of lemmas, that if the 3DR-W instance $(N, P)$ contains a stable matching then the PIT-SDR instance $G$ contains a partition into triangles.

In the reduction, for some matching $M$ we say that some agent $h_i^r$ is \emph{internal in $M$} if $M(h_i^r) \subset H_i$, some agent $h_i^7 \in H^7$ is \emph{proper in $M$} if $\mathscr{W}_{h_i^7}(M)$ belongs to the proper part of $P_{h_i^7}$ and some agent $h_i^r \in N$ is \emph{external in $M$} if $h_i^r$ is neither proper nor internal. It follows that every agent in $H_i$ is either proper, internal, or external in $M$. We will eventually show that in $M$ no agent is external and every agent in $H^7$ is proper, from which the existence of a partition into triangles is straightforward to show.

\begin{lem}
\label{lem:threed_sr_w_exactlysixinternal}
If $(N, P)$ contains a stable matching $M$ then each heptagadget $H_i$ contains exactly six agents that are internal in $M$.
\end{lem}
\begin{proof}
By definition, the number of internal agents in $M$ in $H_i$ is divisible by three. It follows that if $H_i$ does not contain six agents that are internal in $M$ it contains at most three agents that are internal in $M$. Suppose for a contradiction that $H_i$ contains at most three agents that are internal in $M$. Since by definition $H_i$ contains at most one proper agent in $M$, namely $h_i^7$, it follows that $H_i$ contains at least three agents that are external in $M$, which we label $h_i^r$, $h_i^s$, and $h_i^t$. By the definition of an external agent it must be that $\mathscr{W}(\{ h_i^s, h_i^t \}) \succ_{h_i^r} \mathscr{W}_{h_i^r}(M)$, $\mathscr{W}(\{ h_i^r, h_i^t \}) \succ_{h_i^s} \mathscr{W}_{h_i^s}(M)$, and $\mathscr{W}(\{ h_i^r, h_i^s \}) \succ_{h_i^t} \mathscr{W}_{h_i^t}(M)$. It follows that $\{ h_i^r, h_i^s, h_i^t \}$ blocks $M$, which is a contradiction.
\end{proof}

\begin{lem}
If $(N, P)$ contains a stable matching $M$ then no agent in $N$ is external in $M$.
\label{lem:threed_sr_w_noneeexternal}
\end{lem}
\begin{proof}
Consider an arbitrary heptagadget $H_i$, which by Lemma~\ref{lem:threed_sr_w_exactlysixinternal} contains exactly six agents that are internal in $M$. The remaining agent is either external or proper in $M$. Suppose for a contradiction that the remaining agent $h_i^r$ is external in $M$. If $r \neq 5$ then it must be that $h_i^5$ is internal in $M$ and thus $M(h_i^5) = \{ h_i^5, h_i^s, h_i^t \}$ where $s, t \in \{ 1, 2, 3, 4, 6, 7 \}$. Notice that by the design of the constructed instance $(N, P)$, it must be that $h_i^r \succ_{h_i^s} h_i^5$ and $h_i^r \succ_{h_i^t} h_i^5$. Moreover, since $h_i^r$ is external in $M$ it follows that $\mathscr{W}_{h_i^r}(\{ h_i^s, h_i^t \}) \succ_{h_i^r} \mathscr{W}_{h_i^r}(M)$ and thus $\{ h_i^r, h_i^s, h_i^t \}$ blocks $M$, which is a contradiction. It remains that $r = 5$. In this case we consider the two triples in $M$ that contain the six agents in $H_i$ that are internal in $M$. We enumerate the $\binom{6}{2}/2$ possible such pairs of triples and show that in any case $M$ is not stable. Since all agents in both triples belong to $H_i$, in the following table we shorten $\{ h_i^r, h_i^s, h_i^t \}$ to $\{ r, s, t \}$.
\begin{center}
\begin{tabular}{ c c }
triples in $M$ & $M$ is blocked by\Tstrut\Bstrut\\
\hline
$\{1,2,3\}$, $\{4,6,7\}$ & $\{1,3,6\}$\Tstrut\\
$\{1,2,4\}$, $\{3,6,7\}$ & $\{1,3,4\}$ \\
$\{1,2,6\}$, $\{3,4,7\}$ & $\{1,4,6\}$ \\
$\{1,2,7\}$, $\{3,4,6\}$ & $\{1,3,7\}$ \\
$\{1,3,4\}$, $\{2,6,7\}$ & $\{4,6,7\}$ \\
$\{1,3,6\}$, $\{2,4,7\}$ & $\{1,3,4\}$ \\
$\{1,3,7\}$, $\{2,4,6\}$ & $\{4,6,7\}$ \\
$\{1,4,6\}$, $\{2,3,7\}$ & $\{1,3,7\}$ \\
$\{1,4,7\}$, $\{2,3,6\}$ & $\{4,6,7\}$ \\
$\{1,6,7\}$, $\{2,3,4\}$ & $\{1,3,4\}$ \\
\end{tabular}
\end{center}
\end{proof}

We remark that from the proof of Lemma~\ref{lem:threed_sr_w_noneeexternal} it is straightforward to identify a (minimal) instance of 3DR-W that contains no stable matching, by using six agents $h_i^1$, $h_i^2$, $h_i^3$, $h_i^4$, $h_i^6$, and $h_i^7$ from a single gadget $H_i$.

\begin{lem}
\label{lem:threed_sr_w_pitexistsifmisstable}
If $(N, P)$ contains a stable matching then $G$ contains a partition into triangles.
\end{lem}
\begin{proof}
Suppose $M$ is a stable matching in $(N, P)$. Recall that $|N|=21q$. By Lemma~\ref{lem:threed_sr_w_exactlysixinternal}, there are $18q$ internal agents that by definition belong to $6q$ triples in $M$. By Lemma~\ref{lem:threed_sr_w_noneeexternal}, none of the remaining $3q$ agents are external and thus there are exactly $3q$ proper agents. It follows that there are $q$ triples in $M$ that each contain three proper agents. By the construction of the proper part of the preference list of each agent in $H^7$, each of these $q$ triples corresponds to a triangle in $G$. It follows that this set of $q$ triples corresponds directly to a partition into triangles in $G$.
\end{proof}

We have now shown that the 3DR-W instance $(N, P)$ contains a stable matching if and only if a partition into triangles exists in $G$. This shows that the reduction is correct.

\begin{thm}
\label{thm:threed_sr_w_existence}
Deciding if a given instance of 3DR-W contains a stable matching is $\NP$-complete.
\end{thm}
\begin{proof}
It is straightforward to show that this decision problem belongs to $\NP$. We have presented a polynomial-time reduction from a restricted version of Partition Into Triangles, known as PIT-SDR, which we showed was $\NP$-complete. Given an arbitrary instance $G$ of PIT-SDR, the reduction constructs an instance $(N, P)$ of 3DR-W. Lemmas~\ref{lem:threed_sr_w_stablematchingifpartitionexists} and~\ref{lem:threed_sr_w_pitexistsifmisstable} shows that $(N, P)$ contains a stable matching if and only if $G$ contains a partition into triangles and thus that this decision problem is $\NP$-hard.
\end{proof}

% \section{Deciding existence}
% \input{chapters/threed_sr_w/hardness_new}

% \section{Hardness of approximation}
% \input{chapters/threed_sr_w/apx_hardness}

\section{Approximation}
\label{sec:threed_sr_w_approximation}
The \emph{3DR-W Maximally Stable Matching problem} (\mysymbolfirstusedefinition{symboldef:threedr_w_msm}{3DR-W-MSM}) is the following optimisation problem: given an instance of 3DR-W, find a matching with the largest possible number of non-blocking triples. Formally, 3DR-W-MSM is a maximisation problem in which any instance $(N, P)$ of 3DR-W-MSM is also instance of 3DR-W, a solution is a matching in $(N, P)$, and the measure is $|\textrm{nbt}(M, (N, P))|$. We showed in Theorem~\ref{thm:threed_sr_w_existence} that deciding if an instance of 3DR-W contains a matching with $\binom{3n}{3}$ non-blocking triples is $\NP$-complete, so it follows that 3DR-W-MSM is $\NP$-hard. In this section we present a polynomial-time algorithm that can approximate 3DR-W-MSM with an approximation ratio of $9/4$. 

The approximation algorithm is analogous to the corresponding algorithm we presented before for 3DR-B-MSM in Theorem~\ref{thm:threed_sr_b_approxalgofournine} in Chapter~\ref{c:threed_sr_b}. As before, we show that it is possible to construct an instance $(N, P')$ of 3PSA with the same set of agents such that for any matching $M$, if a triple blocks $M$ in $(N, P)$ then it also blocks $M$ in $(N, P')$. We can then show, as before, that the approximation ratio of the algorithm for 3DR-W-MSM is $9/4$.

\begin{thm}
\label{thm:threed_sr_w_approxalgofournine}
There exists a polynomial-time $9/4$-approximation algorithm for 3DR-W-MSM.
\end{thm}
\begin{proof}
The proof is analogous to Theorem~\ref{thm:threed_sr_b_approxalgofournine} in Chapter~\ref{c:threed_sr_b}, except $(N, P')$ is constructed such that for any agent $\alpha_i$, $P'_{\alpha_i}$ is the list of all $2$-agent subsets of $N \setminus \{ \alpha_i \}$ in colexicographic order \cite{combinatorialwest} with respect to $P_{\alpha_i}$. Suppose $M$ is an arbitrary matching in $(N, P)$ and $r$ is a triple that blocks $M$ in $(N, P)$. We will show that $r$ also blocks $M$ in $(N, P')$. For any $\alpha_k$ in $r$ it must be that $\mathscr{W}_{\alpha_k}(r) \succ_{P_{\alpha_k}} \mathscr{W}_{\alpha_k}(M)$. By the construction of $P'_{\alpha_k}$ as the colexicographic order of $P_{\alpha_k}$, it must be that $r \succ_{P'_{\alpha_k}} M(\alpha_k)$ and thus that $r$ also blocks $M$ in $(N, P')$. It follows that $|\textrm{bt}(M, (N, P'))| \geq |\textrm{bt}(M, (N, P))|$ and thus that $|\textrm{nbt}(M, (N, P))| \geq |\textrm{nbt}(M, (N, P'))|$, as required.
\end{proof}

\section{Summary and open problems}
\label{sec:threed_sr_w_conclusion}
In this chapter we formalised a new model of 3DR, involving $\mathscr{W}$-preferences, which we called 3DR-W. We considered in 3DR-W the existence of, and complexity of finding, matchings that are stable.

We first showed that, as in the case of 3DR-B (see Chapter~\ref{c:threed_sr_b}), a given instance of 3DR-W may not contain a stable matching and the associated decision problem is $\NP$-complete. Next, we considered a closely related optimisation problem, which we called 3DR-W-MSM, in which the objective is to construct, in a given instance of 3DR-B, a matching with the maximum number of non-blocking triples. Finally, we presented a $9/4$-approximation algorithm for 3DR-W-MSM built on an existing algorithm for 3PSA-MSM, which is a closely related problem \cite{rosenbaum16}. 

We now present some open problems specifically involving stability in 3DR-W. More general problems, involving solution concepts other than stability and other models of fixed-size coalition formation, are discussed in Chapter~\ref{c:conclusion}. 

% Although the definitions of 3DR-W and 3DR-B are similar, and it is $\NP$-complete to decide if a given instance of either model contains a stable matching, it appears that their commonality is otherwise limited. For example, it is straightforward to construct an instance of 3DR-W with $6$ agents that contains no stable matching (see Section~\ref{sec:threed_sr_w_hardness}) but, as we saw in Chapter~\ref{c:threed_sr_b}, the smallest such instance of 3DR-B has at at least $9$ (and at most $63$) agents. 

Although the definitions of 3DR-B and 3DR-W are similar, and the stability existence problem is $\NP$-complete in both models, it is unclear whether the two models are further related. For example, in Chapter~\ref{c:threed_sr_b} we presented a $3/2$-approximation algorithm for 3DR-B-MSM, but it seems difficult to design such an algorithm for 3DR-W-MSM. One possible starting point is Rosenbaum's $9/4$-approximation algorithm for 3PSA-MSM, Algorithm~\algorithmfont{ASA}. Another idea is to adapt Irving's \cite{Irv85} algorithm for (two-dimensional) Stable Roommates. In fact, Irving's algorithm is the basis of another polynomial-time algorithm for an analogous model involving  $\mathscr{W}$-preferences \cite{CH04}, in which coalitions need not have a fixed size.

Alternatively, it might be possible to prove an inapproximability result for 3DR-W-MSM. For example, if one could show that no $3/2$-approximation algorithm exists for 3DR-W-MSM, unless $\P=\NP$, then a striking difference would be revealed between 3DR-B and 3DR-W. To this end, it might be useful to consider the reductions of Iwama et al.\ \cite{IMO08} and Rosenbaum \cite{rosenbaum16}, which show that two optimisation problems that are defined in related models of 3DR are both $\APX$-hard (both results are discussed in Chapter~\ref{c:lit_review}).



It would be very informative to estimate the probability that a random instance of 3DR-W contains a stable matching. In particular, if this probability could be compared between the models of 3DR-B and 3DR-W. As we noted in Chapter~\ref{c:threed_sr_b}, both 3-DSM-CYC \cite{Pittel20} and two-dimensional Stable Roommates (SR) \cite{PI94} have been studied from a probabilistic perspective, which are possible starting points. Alternatively, an empirical approach might be informative, as it has been for 3-DSM-CYC \cite{Escamocher2018}. In this direction, it might be useful to formulate the problem of finding a stable matching in a given instance of 3DR-W as an integer programming model.

Various other optimisation problems and measures can be defined in relation to stability and 3DR-W. In Chapter~\ref{c:threed_sr_b} we proposed some alternatives in the setting of 3DR-B, all of which can be defined analogously for 3DR-W.
\chapter{Three-Dimensional Stable Roommates with Additively Separable Preferences}
\label{c:threed_sr_as}
\chaptermark{Stability in 3DR-AS}

\section{Introduction}
\label{sec:threed_sr_as_intro}
In this chapter we formalise a model of Three-Dimensional Roommates (3DR) involving additively separable preferences, which we call \mysymbolfirstusedefinition{symboldef:threedr_as}{3DR-AS}. We consider in 3DR-AS the existence of, and complexity of finding, matchings that are stable, under three possible restrictions of the agents' preferences.

A strong motivation exists for a model of 3DR with additively separable preferences. The first such model, of which 3DR-AS is a generalisation, was first proposed by Huang \cite{Huang07conference} in 2007, who noted the natural definition and relative practicality of additively separable preferences compared to other possible systems of preference representation. For example, such a model could be applied to a social network graph involving a symmetric ``friendship'' relation between users \cite{Sless18}. Another special case of 3DR-AS is \emph{Geometric 3D-SR} \cite{ABEOMP09} (see Chapter~\ref{c:lit_review}). In a sense, all of these models can also be considered as a special type of additively separable hedonic game \cite{AZIZ2013316,SUNG2010635}, which have received much attention in the literature (and are also discussed in Chapter~\ref{c:lit_review}).

We begin in Section~\ref{sec:threed_sr_as_model} with some preliminary definitions and results. 

We then show, in Section~\ref{sec:threed_sr_as_symmetricbinary}, that any instance of 3DR-AS with binary and symmetric preferences must contain a stable matching, and present a polynomial-time algorithm that can construct a stable matching in a given such instance (Theorem~\ref{thm:threed_sr_as_symmetric_binary_construction}). We then consider the problem of finding a stable matching with maximum utilitarian welfare, given an instance in which preferences are binary and symmetric. We show that this optimisation problem is $\NP$-hard (Theorem~\ref{thm:threed_sr_as_maxutilstable_hard}) but also that the algorithm for constructing a stable matching in this setting can be modified to yield a $2$-approximation algorithm (Theorem~\ref{thm:threed_sr_as_approxratio}).

Next, we complement the previous tractability results with two hardness results. The first, shown in Section~\ref{sec:threed_sr_as_generalbinary}, is that a stable matching need not exist in general, and the associated decision problem is $\NP$-complete even when preferences are binary and not necessarily symmetric (Theorem~\ref{thm:threed_sr_as_binary_reduction}). The second, shown in Section~\ref{sec:threed_sr_as_symmetricternary}, is that the same decision problem is $\NP$-complete even when preferences are ternary and symmetric (Theorem~\ref{thm:threed_sr_as_symmetric_ternary_reduction}).

Finally, in Section~\ref{sec:threed_sr_as_conclusion}, we recap on our contribution and discuss some directions for future work.

\section{Preliminaries}
\label{sec:threed_sr_as_model}
An instance of 3DR-AS comprises a set $N$ of $3n$ agents with additively separable preferences over triples, which we define as follows. Each agent $\alpha_i$ supplies a \textit{valuation function} $v_{\alpha_i} : N\setminus \{ {\alpha_i} \} \mapsto \mathbb{Z}$. Given agent $\alpha_i$, let the \emph{utility} of any set $S\subseteq N$ be $u_{\alpha_i}(S) = \sum_{{\alpha_j}\in S \setminus \{ \alpha_i \}} v_{\alpha_i}({\alpha_j})$. A \emph{triple} is an unordered set of three agents. We say that $\alpha_i \in N$ \emph{prefers} some triple $r$ to another triple $s$ if $u_{\alpha_i}(r) > u_{\alpha_i}(s)$. A \emph{matching} is a partition of $N$ into $n$ triples (note that we shall slightly modify this definition in Section~\ref{sec:threed_sr_as_findingastablematching}). Given an agent $\alpha_i$ and a matching $M$, we denote by $M(\alpha_i)$ the triple in $M$ that contains $\alpha_i$. Given a matching $M$, we say that a triple $t$ is \emph{blocking} if each agent $\alpha_i$ in $t$ prefers $t$ to $M(\alpha_i)$. A matching is \emph{stable} if it does not contain a blocking triple. An agent's preference between two matchings depends only on the partners of that agent in each matching, so given a matching $M$ we let $u_{\alpha_i}(M)$ be short for $u_{\alpha_i}(M(\alpha_i))$. Formally, we represent an instance of 3DR-AS as a pair $(N, V)$, where $V$ is the collection of all agents' valuation functions.

% The model that we present in this chapter is similar to those already discussed, and involves a set of agents $N=\{\alpha_1,\dots,\alpha_{3n}\}$ with preferences. It also involves the construction of a set of disjoint triples. For technical purposes, here we relax the condition that a matching is a partition of the agent set. We say that a \emph{matching} $M$ is a set of disjoint triples in which each agent belongs to at most one triple. For any agent $\alpha_i$, if some triple in $M$ contains $\alpha_i$ then we say that $\alpha_i$ is \emph{matched} and use $M(\alpha_i)$ to refer to that triple. If no triple in $M$ contains $\alpha_i$ then we say that $\alpha_i$ is \emph{unmatched} and write $M(\alpha_i) = \varnothing$. Given a matching $M$ and two distinct agents $\alpha_i, \alpha_j$, if $M(\alpha_i)=M(\alpha_j)$ then we say that $\alpha_j$ is a \textit{partner} of $\alpha_i$.

% We define \emph{additively separable preferences} as follows. 

% Suppose we have some pair $(N, V)$ and a matching $M$ involving the agents in $N$. We say that a triple $\{\alpha_{k_1}, \alpha_{k_2}, \alpha_{k_3} \}$ \textit{blocks} $M$ in $(N, V)$ if $u_{\alpha_{k_1}}(\{\alpha_{k_2}, \alpha_{k_3} \}) > u_{\alpha_{k_1}}(M), u_{\alpha_{k_2}}(\{\alpha_{k_1}, \alpha_{k_3} \}) > u_{\alpha_{k_2}}(M)$, and $u_{\alpha_{k_3}}(\{\alpha_{k_1}, \alpha_{k_2} \}) > u_{\alpha_{k_3}}(M)$. If no triple in $N$ blocks $M$ in $(N, V)$ then we say that $M$ is \textit{stable} in $(N, V)$. We say that $(N, V)$ \emph{contains} a stable matching if at least one matching exists in $(N, V)$ that is stable.

For any instance $(N, V)$ of 3DR-AS and any matching $M$ in $(N, V)$, for a set $S \subseteq N$ of agents we define the \emph{utilitarian welfare} of $S$ as $u_{S}(M) = \sum_{\alpha_i \in S} u_{\alpha_i}(M)$. Let $u(M)$ be short for $u_{N}(M)$.

We also define three possible restrictions of agents' preferences. We say that valuations are \emph{binary}, if $v_{\alpha_i}(\alpha_j) \in \{ 0, 1 \}$ for any $\alpha_i, \alpha_j \in N$, \emph{ternary} if $v_{\alpha_i}(\alpha_j) \in \{ 0, 1, 2 \}$ for any $\alpha_i, \alpha_j \in N$, and \emph{symmetric} if $v_{\alpha_i}(\alpha_j)=v_{\alpha_j}(\alpha_i)$ for any $\alpha_i, \alpha_j \in N$.

It is possible to consider an instance of 3DR-AS as a weighted directed graph, in which the set of vertices is the set of agents and the weight of any arc from one agent to another is the corresponding valuation. In particular, an instance of 3DR-AS with binary and symmetric preferences corresponds directly to an undirected graph, which we shall refer to as the \emph{underlying graph}.

% Lemma~\ref{lem:threed_sr_as_blockerimprovement} illustrates a fundamental property of stable matchings in instances of 3DR-AS. We shall use it extensively in the proofs.
By the definition of stability, if $M$ and $M'$ are matchings in some instance of 3DR-AS, $M$ is stable, and $u_{\alpha_i}(M') \geq u_{\alpha_i}(M)$ for any agent $\alpha_i$ then it must follow that $M'$ is also stable. We prove a related statement in Proposition~\ref{prop:threed_sr_as_blockerimprovement}.
\begin{prop}
\label{prop:threed_sr_as_blockerimprovement}
Given an instance $(N, V)$ of 3DR-AS, suppose that $M$ and $M'$ are matchings in $(N, V)$. Any triple that blocks $M'$ but does not block $M$ contains at least one agent $\alpha_i \in N$ where $u_{\alpha_i}(M') < u_{\alpha_i}(M)$.
\end{prop}
\begin{proof}
Suppose some triple $\{ \alpha_i, \alpha_j, \alpha_k \}$ blocks $M'$. By the definition of a blocking triple, it must be that $u_{\alpha_i}(\{\alpha_j, \alpha_k \}) > u_{\alpha_i}(M')$, $u_{\alpha_j}(\{\alpha_i, \alpha_k \}) > u_{\alpha_j}(M')$, and $u_{\alpha_k}(\{\alpha_i, \alpha_j \}) > u_{\alpha_k}(M')$. Suppose for a contradiction that there exists no $\alpha_p\in \{ \alpha_i, \alpha_j, \alpha_k \}$ exists where $u_{\alpha_p}(M') < u_{\alpha_p}(M)$ and hence $u_{\alpha_{k_r}}(M') \geq u_{\alpha_{k_r}}(M)$ for $1 \leq r \leq 3$. It follows that $u_{\alpha_i}(\{\alpha_j, \alpha_k \}) > u_{\alpha_i}(M)$, $u_{\alpha_j}(\{\alpha_i, \alpha_k \}) > u_{\alpha_j}(M)$, and $u_{\alpha_k}(\{\alpha_i, \alpha_j \}) > u_{\alpha_k}(M)$ and thus that $\{ \alpha_i, \alpha_j, \alpha_k \}$ also blocks $M$, which is a contradiction.
\end{proof}

We also introduce some other notation. We denote by $L = \langle l_1, l_2, \dots, l_{|L|} \rangle$ an ordered list. If $L$ and $L'$ are lists then we denote by $L \cdot L'$ the concatenation of $L'$ to the end of $L$. We also use standard set notation with lists, such as $e \in L$.

\section{Symmetric binary preferences}
\label{sec:threed_sr_as_symmetricbinary}
\subsection{Finding a stable matching}
\label{sec:threed_sr_as_findingastablematching}

\subsubsection{Preliminaries}

In this section we show that every such instance of 3DR-AS contains a stable matching, which can be found in $O(|N|^3)$ time. We give a step-by-step constructive proof of this result between Sections~\ref{sec:threed_sr_as_removingtriangles}--\ref{sec:threed_sr_as_symmetric_binary_finding_in_general}.

For technical purposes, in this section (Section~\ref{sec:threed_sr_as_findingastablematching}) only, we make two relaxations to the model described in Section~\ref{sec:threed_sr_as_model}. In fact, we prove a slightly more general result than required. Firstly, we suppose an instance $(N, V)$ of 3DR-AS may contain a number of agents that is not necessarily divisible by three. Secondly, we define a \emph{matching} only as a set of triples, which does not necessarily partition $N$. For any matching $M$ and any agent $\alpha_i$, if no triple in $M$ contains $\alpha_i$ then we say that $\alpha_i$ is \emph{unmatched} in $M$ and write $M(\alpha_i) = \varnothing$. All other definitions and notation remain the same. Thus, in this section we show that any (relaxed) instance of 3DR-AS contains a (relaxed) stable matching. Now, for any (relaxed) instance of 3DR-AS and (relaxed) stable matching $M$, if the number of agents is divisible by three (and thus the instance meets the original definition) then any unmatched agents may be arbitrarily matched into triples. It follows by Proposition~\ref{prop:threed_sr_as_blockerimprovement} that the resulting (non-relaxed) matching $M'$ is stable in the (non-relaxed) instance, as required.

We also introduce a restricted type of (relaxed) matching called a \emph{$P$\nobreakdash-matching}. Recall that by definition, $M(\alpha_p)=\varnothing$ implies that $u_{\alpha_p}(M)=0$ for any $\alpha_p \in N$ in an arbitrary (relaxed) matching $M$. We say that a matching $M$ in $(N, V)$ is a \emph{$P$\nobreakdash-matching} if $M(\alpha_p) \neq \varnothing$ implies $u_{\alpha_p}(M) > 0$. A $P$\nobreakdash-matching thus corresponds to a $\{ K_3, P_3 \}$-packing in the underlying graph \cite{KH83}. Note that any triple in a $P$\nobreakdash-matching $M$ must contain at least one agent with utility two. A \emph{stable $P$\nobreakdash-matching} is a $P$\nobreakdash-matching that is also stable. Our main result is that any instance of 3DR-AS with binary and symmetric preferences contains a stable $P$\nobreakdash-matching.

\subsubsection{Removing triangles}
\label{sec:threed_sr_as_removingtriangles}

In an instance $(N, V)$ of 3DR-AS with binary and symmetric preferences, a \emph{triangle} comprises three agents $\alpha_{m_1}, \alpha_{m_2},\allowbreak \alpha_{m_3}$ such that $v_{\alpha_{m_1}}(\alpha_{m_2}) = v_{\alpha_{m_2}}(\alpha_{m_3}) = v_{\alpha_{m_3}}(\alpha_{m_1}) = 1$. If $(N, V)$ contains no triangle then we say it is \emph{triangle-free}. If $(N, V)$ is not triangle-free then it can be reduced, by successively removing triangles, until it is triangle-free. This operation is our first step towards constructing a stable $P$\nobreakdash-matching in $(N, V)$. We formalise this result in Lemma~\ref{lem:threed_sr_as_symmetric_binary_trianglefree}.

\begin{lem}
\label{lem:threed_sr_as_symmetric_binary_trianglefree}
Given an instance $(N, V)$ of 3DR-AS with binary and symmetric preferences, we can, in $O(|N|^3)$ time, identify an instance $(N', V')$ of 3DR-AS with binary and symmetric preferences and a $P$\nobreakdash-matching $M_{\triangle}$ such that $(N', V')$ is triangle-free, $|N'|\leq |N|$, and if $M$ is a stable $P$\nobreakdash-matching in $(N', V')$ then $M' = M \cup M_{\triangle}$ is a stable $P$\nobreakdash-matching in $(N, V)$.
\end{lem}
\begin{proof}
Construct $M_{\triangle}$ as a \emph{maximal triangle packing} \cite{CHATAIGNER20091396} in the underlying graph, in $O(|N|^3)$ time. Let $N' = N \setminus \bigcup M_{\triangle}$. Construct $V'$ accordingly. By definition, $M_{\triangle}$ is a $P$\nobreakdash-matching, $(N', V')$ is triangle-free, and $|N'| \leq |N|$. 

Suppose $M$ is a stable $P$\nobreakdash-matching in $(N', V')$. Consider $M' = M \cup M_{\triangle}$. By definition, $M'$ is a $P$\nobreakdash-matching. Since each triple in $M_{\triangle}$ corresponds to a triangle, any agent in any triple in $M_{\triangle}$ must have utility two in $M_{\triangle}$ and thus does not belong to a triple that blocks $M'$ in $(N, V)$. If some triple blocks $M'$ in $(N, V)$ then it must contain three agents in $N'$, and thus it must also block $M$ in $(N', V')$, which is impossible. It follows that $M'$ is stable in $(N, V)$.
\end{proof}

\subsubsection{Repairing a \texorpdfstring{$P$}{P}-matching in a triangle-free instance}
\label{sec:threed_sr_as_symmetric_binary_repairing}

In this section (Section~\ref{sec:threed_sr_as_symmetric_binary_repairing}), we consider instances of 3DR-AS that are triangle-free, and in them define a special type of $P$\nobreakdash-matching that is \emph{repairable}. 

We present Subroutine~\algorithmfont{repair}, shown in Algorithms~\ref{alg:threed_sr_as_almostthere_algo_phase1} and~\ref{alg:threed_sr_as_almostthere_algo_phase2} which, given a triangle-free instance $(N, V)$ and a repairable $P$\nobreakdash-matching $M$, constructs a new $P$\nobreakdash-matching $M'$ that is stable in $(N, V)$. We shall see in the next section (Section~\ref{sec:threed_sr_as_symmetric_binary_finding_in_triangle_free}) how this subroutine is used in a more general subroutine that, given a triangle-free instance of 3DR-AS, returns a $P$\nobreakdash-matching that is stable.

Given a triangle-free instance $(N, V)$ and a $P$\nobreakdash-matching $M$, we say that $M$ is \emph{repairable} if it is not stable and there exists some $\alpha_i \in N$ such that $u_{\alpha_i}(M)=0$ and any triple that blocks $M$ comprises $\{ \alpha_i, \alpha_{j_1}, \alpha_{j_2}\}$ for some $\alpha_{j_1}, \alpha_{j_2}\in N$ where $u_{\alpha_{j_1}}(M)=1$, $u_{\alpha_{j_2}}(M)=0$, and $v_{\alpha_i}(\alpha_{j_1})=v_{\alpha_{j_1}}(\alpha_{j_2})=1$.

We now provide some intuition behind Subroutine~\algorithmfont{repair} and refer the reader to Figure~\ref{fig:threed_sr_as_symmetric_binary_repair_algorithm_7_cases_original_case}. Recall that the overall goal of the subroutine is to construct a $P$\nobreakdash-matching $M'$ that is stable. Since $M$ is repairable, our goal will be to modify $M$ in such a way that $u_{\alpha_i}(M')\geq 1$ and no triple blocks $M'$ that did not also block $M$. By the definition of repairable, and Proposition~\ref{prop:threed_sr_as_blockerimprovement}, it follows that the resulting $P$\nobreakdash-matching $M'$ is stable. For example, one way to accomplish this goal would be to construct $M'$ in such a way that $u_{\alpha_i}(M')\geq 1$ and $u_{\alpha_p}(M')\geq u_{\alpha_p}(M)$ for any $\alpha_p \in N\setminus \{ \alpha_i \}$, from which it would follow by Proposition~\ref{prop:threed_sr_as_blockerimprovement} that $M'$ is stable.

\begin{figure}
    \centering
    \input{figures/threed_sr_as/repair_algorithm_7_cases_original_case.tikz}
    \caption[Agents and triples in $M$ before a new iteration of the while loop]{Agents and triples in $M$ before a new iteration of the while loop. Each vertex represents an agent. An edge is present from agent $\alpha_i$ to agent $\alpha_j$ if $v_{\alpha_i}(\alpha_j) = 1$.} 
    \label{fig:threed_sr_as_symmetric_binary_repair_algorithm_7_cases_original_case}
\end{figure}

The subroutine begins by selecting some triple $\{ \alpha_i, \alpha_{j_1}, \alpha_{j_2} \}$ that blocks $M$. The two agents in $M(\alpha_{j_1}) \setminus \{ \alpha_{j_1} \}$ are labelled $\alpha_{j_3}$ and $\alpha_{j_4}$. In order to introduce the mechanism of Subroutine~\algorithmfont{repair} we consider two example cases in which it is possible to construct a stable $P$\nobreakdash-matching. 

First, suppose there exists some agent $\alpha_{z_1}$ where $v_{\alpha_{j_3}}(\alpha_{z_1})=1$ and $u_{\alpha_{z_1}}(M)=0$.  Construct $M'$ from $M$ by removing $\{ \alpha_{j_1}, \alpha_{j_2}, \alpha_{j_3} \}$ and adding $\{ \alpha_i, \alpha_{j_1}, \alpha_{j_2} \}$ and $\{ \alpha_{j_3}, \alpha_{j_4}, \alpha_{z_1} \}$. Now $u_{\alpha_i}(M')=1$ and $u_{\alpha_p}(M')\geq u_{\alpha_p}(M)$ for any $\alpha_p \in N\setminus \{ \alpha_i \}$. It follows by Proposition~\ref{prop:threed_sr_as_blockerimprovement} that $M'$ is stable. Second, suppose there exists no such $\alpha_{z_1}$ but there exists some agent $\alpha_{z_2}$ where $v_{\alpha_{j_4}}(\alpha_{z_2})=1$ and $u_{\alpha_{z_2}}(M)=0$. In this case construct $M'$ from $M$ by removing $\{ \alpha_{j_1}, \alpha_{j_2}, \alpha_{j_3} \}$ and adding $\{ \alpha_i, \alpha_{j_1}, \alpha_{j_2} \}$ and $\{ \alpha_{j_3}, \alpha_{j_4}, \alpha_{z_2} \}$. Now $u_{\alpha_i}(M')=1$ and $u_{\alpha_p}(M')\geq u_{\alpha_p}(M)$ for any $\alpha_p \in N\setminus \{ \alpha_i, \alpha_{j_3} \}$. It can be shown that $\alpha_{j_3}$ does not belong to a triple that blocks $M'$ since no $\alpha_{z_1}$ exists as described. It follows again by Proposition~\ref{prop:threed_sr_as_blockerimprovement} that $M'$ is stable.

Generalising these example cases, Subroutine~\algorithmfont{repair} repair has two phases. In Phase 1, shown in Algorithm~\ref{alg:threed_sr_as_almostthere_algo_phase1}, it identifies some set of agents in the instance with a specific structure. In Phase 2, shown in Algorithm~\ref{alg:threed_sr_as_almostthere_algo_phase1}, it modifies the triples of these agents in order to construct a stable $P$\nobreakdash-matching $M'$. Phase 1 involves the construction of a list of agents $S$, which initially comprises $\langle \alpha_{j_1}, \alpha_{j_3},\allowbreak \alpha_{j_4} \rangle$. At any point in time, the list $S$ has length $3c$ for some $c \geq 1$ where $\{ S_{3c-2}, S_{3c-1}, S_{3c} \} \in M$ and $v_{S_p}(S_{p+1})=1$ for any $p$ where $1 \leq p < 3c$. It follows that $S$ corresponds to a path in the underlying graph. In each iteration of the main loop, three agents belonging to some triple in $M$ are appended to the end of $S$. The loop continues until $S$ satisfies at least one of six specific stopping conditions (shown in the first if/else statement). We will show that eventually at least one of these stopping conditions must hold. After the loop terminates, the subroutine enters Phase 2 and constructs $M'$. The exact construction of $M'$ depends on which stopping condition(s) caused the main loop to terminate. Two of these conditions, and the corresponding constructions of $M'$, generalise the two example cases (involving $\alpha_{z_1}$ and $\alpha_{z_2}$). The other four conditions, and the corresponding constructions of $M'$, relate to alternative cases in which it is possible to construct a stable $P$\nobreakdash-matching $M'$.

\input{algorithms/threed_sr_as/repair_almost_stable}

The six stopping conditions correspond to seven possible constructions of $M'$, which are labelled Construction 1--7. Each of the six stopping conditions corresponds to a single construction except the first condition, which corresponds to two constructions (Construction~1 and Construction~3).
Constructions 1 and 3 generalise the first example case (involving $\alpha_{z_1}$). Construction~2 generalises the second example case (involving $\alpha_{z_2}$). Constructions 4--7 correspond to alternative cases. 
Like in the two example cases, in each of Constructions 1--6 no agent identified by the subroutine, including $\alpha_i$, becomes unmatched in $M'$. This simplifies the proof that $M'$ is stable in Constructions 1--6. The proof that $M'$ is stable in Construction~7 is more complicated. In Construction~7, the final agent in the list $S$, labelled $S_{3c}$, becomes unmatched in $M'$. To prove that $S_{3c}$ does not then become part of a triple that blocks $M'$ we must invoke on the fact that no stopping condition relating to previous constructions held in any previous iteration of the main loop. In this way, the six stopping conditions and seven corresponding constructions of $M'$ are somewhat hierarchical. For another example, the proof that $M'$ is stable in Construction~4 relies on the fact that in no previous iteration did the stopping condition relating to Constructions 1 and 3 hold. A similar reliance exists among the proofs for the other constructions. This hierarchy helps demonstrate why the six stopping conditions and seven constructions of $M'$ are both necessary and sufficient. 

In order to prove the correctness and time complexity of Subroutine~\algorithmfont{repair} we use a number of lemmas. The following lemma, Lemma~\ref{lem:threed_sr_as_symmetric_binary_algalwaysterminates}, shows that the while loop in Subroutine~\algorithmfont{repair} must terminate in $O(|N|)$ time.

\begin{lem}
\label{lem:threed_sr_as_symmetric_binary_algalwaysterminates}
The while loop in Subroutine~\algorithmfont{repair} terminates after at most $\lfloor (|N|-2) \mathbin{/} 3 \rfloor$ iterations.
\end{lem}
\begin{proof}
By the pseudocode, any three agents $\{ \alpha_{w_1}, \alpha_{w_2}, \alpha_{w_3} \}$ added to $S$ in a single iteration form a triple in $M$. Just before the addition of $\langle \alpha_{w_1}, \alpha_{w_2}, \alpha_{w_3} \rangle$ to $S$, it must be that $\alpha_{w_1} \notin S$. It follows that $\alpha_{w_2}, \alpha_{w_3} \notin S$, so in general $S$ contains any agent in $N$ at most once. Since $\alpha_i, \alpha_{j_2} \notin S$ it follows that $|S|\leq |N| - 2$ and thus the while loop terminates after at most $\lfloor (|N|-2) \mathbin{/} 3 \rfloor$ iterations.
\end{proof}

In Construction~3, the subroutine identifies some agent $\alpha_{z_4}$ in $N\setminus \{ \alpha_i \}$ such that $v_{S_{3c-1}}(\alpha_{z_4}) = 1$ and $u_{\alpha_{z_4}}(M) = 0$. Proposition~\ref{prop:threed_sr_as_symmetric_binary_az4exists} shows that such an agent must exist.

\begin{prop}
\label{prop:threed_sr_as_symmetric_binary_az4exists}
In Construction~3 of Subroutine~\algorithmfont{repair}, some agent $\alpha_{z_4}$ in $N\setminus \{ \alpha_i, \alpha_{j_2} \}$ exists where $v_{S_{3c-2}}(\alpha_{z_4})=1$ and $u_{\alpha_{z_4}}(M)=0$.
\end{prop}
\begin{proof}
Refer to Figure~\ref{fig:threed_sr_as_symmetric_binary_repair_algorithm_7_cases_case_3}. 
\begin{figure}
    \centering
    \input{figures/threed_sr_as/repair_algorithm_7_cases_case_3.tikz}
    \caption{The structure of $M'$ in Construction~3} 
    \label{fig:threed_sr_as_symmetric_binary_repair_algorithm_7_cases_case_3}
\end{figure}
We first claim that $c > 1$. Suppose for a contradiction that $c = 1$. By the pseudocode, in Construction~3 it must be that $\alpha_{z_1} = \alpha_{j_2}$ and $v_{\alpha_{z_1}}(S_{3c-1})=1$. Since $c = 1$ it must be that $S_{3c-1}=\alpha_{j_3}$ so the triple $\{ \alpha_{j_1}, \alpha_{j_2}, \alpha_{j_3} \}$ forms a triangle in $(N, V)$, which contradicts the assumption that $(N, V)$ is triangle-free.

Since $c>1$ it follows that $c'=c-1$ is the value of $c$ in the second last iteration of the while loop. Consider the second-to-last iteration of the while loop. In this iteration, the subroutine identified some $\alpha_{w_1} = S_{3c-2}$ where $v_{S_{3c'}}(\alpha_{w_1})=1$, $\alpha_{w_1}\notin S$ and there existed some $\alpha_{z_3}\in N\setminus \{ \alpha_i \}$ where $v_{\alpha_{w_1}}(\alpha_{z_3})=1$ and $u_{\alpha_{z_3}}(M) = 0$. We shall identify the agent labelled $\alpha_{z_3}$ in this iteration as $\alpha_{z_4}$. It follows that $\alpha_{z_4} \neq \alpha_i$, $v_{S_{3c - 2}}(\alpha_{z_4}) = 1$, and $u_{\alpha_{z_4}}(M)=0$.

We claim that in $\alpha_{z_4} \neq \alpha_{j_2}$ since otherwise the triple $\{ \alpha_{z_4}, S_{3c-1}, S_{3c-2} \}$ forms a triangle in $(N, V)$, which contradicts the fact that $(N, V)$ is triangle-free. It follows that $\alpha_{z_4} \in N\setminus \{ \alpha_i, \alpha_{j_2} \}$, which completes the proof.
\end{proof}

Likewise in Construction~6, the subroutine identifies some agent $\alpha_{z_5}$ in $N\setminus \{ \alpha_i, \alpha_{j_2} \}$ exists where $v_{S_{3b+1}}(\alpha_{z_5})=1$ and $u_{\alpha_{z_5}}(M)=0$. Proposition~\ref{prop:threed_sr_as_symmetric_binary_az5exists} shows that such an agent must exist.

\begin{prop}
\label{prop:threed_sr_as_symmetric_binary_az5exists}
In Construction~6 of Subroutine~\algorithmfont{repair}, some agent $\alpha_{z_5}$ in $N\setminus \{ \alpha_i, \alpha_{j_2} \}$ exists where $v_{S_{3b+1}}(\alpha_{z_5})=1$ and $u_{\alpha_{z_5}}(M)=0$.
\end{prop}
\begin{proof}
Refer to Figure~\ref{fig:threed_sr_as_symmetric_binary_repair_algorithm_7_cases_case_6}. Consider the final value of $b$ in the subroutine. By the definition of $b$ and the pseudocode relating to Construction~6 it must be that $b < c$.

\begin{figure}
    \centering
    \input{figures/threed_sr_as/repair_algorithm_7_cases_case_6.tikz}
    \caption{The structure of $M'$ in Construction~6} 
    \label{fig:threed_sr_as_symmetric_binary_repair_algorithm_7_cases_case_6}
\end{figure}

Consider the $b\textsuperscript{th}$ iteration of the while loop. Since $b < c$, it must be that this iteration was not the final iteration. It follows that at the end of this iteration the subroutine identified some agent $\alpha_{w_1}=S_{3b+1}$ and then appended $\langle S_{3b+1}, S_{3b+2}, S_{3b+3} \rangle$ to the end of $S$. It also follows that, in this iteration, it also identified some agent $\alpha_{w_1}$ where there exists some $\alpha_{z_3} \in N \setminus \{ \alpha_i \}$ such that $v_{\alpha_{w_1}}(\alpha_{z_3})=1$ and $u_{\alpha_{z_3}}(M)=0$. We shall identify the agent labelled $\alpha_{z_3}$ in this iteration as $\alpha_{z_5}$. It follows that $\alpha_{z_5} \neq \alpha_i$, $v_{S_{3b + 1}}(\alpha_{z_5}) = 1$, and $u_{\alpha_{z_5}}(M)=0$.

We claim that in $\alpha_{z_5} \neq \alpha_{j_2}$. By the definition of $b$, it must be that $v_{S_{3b}}(\alpha_{j_2}) = 1$. Thus, if $\alpha_{z_5} = \alpha_{j_2}$ then the triple $\{ \alpha_{z_5}, S_{3b+1}, S_{3b} \}$ forms a triangle in $(N, V)$, which contradicts the fact that $(N, V)$ is triangle-free. It follows that $\alpha_{z_5} \in N\setminus \{ \alpha_i, \alpha_{j_2} \}$, which completes the proof.
\end{proof}

\begin{lem}
\label{lem:threed_sr_as_symmetric_binary_algreturnspmatching}
Subroutine~\algorithmfont{repair} returns a $P$\nobreakdash-matching.
\end{lem}
\begin{proof}
By the construction of $M'$ in Constructions 1--7 of $M'$, shown in Figures~\ref{fig:threed_sr_as_symmetric_binary_repair_algorithm_7_cases_case_1}\nobreakdash--\ref{fig:threed_sr_as_symmetric_binary_repair_algorithm_7_cases_case_7}.
\end{proof}

In Lemmas~\ref{lem:threed_sr_as_symmetric_binary_algocases1and3noalphapexists}, \ref{lem:threed_sr_as_symmetric_binary_algocases245and6noalphapexists} and~\ref{lem:threed_sr_as_symmetric_binary_algocase7noalphapexists} we show that in no construction of $M'$ does there exist any agent $\alpha_g$ where where $u_{\alpha_{g}}(M') < u_{\alpha_{g}}(M)$ and $\alpha_g$ belongs to a triple that blocks $M'$. This fact will help us show that $M'$ must be stable.

\begin{lem}
\label{lem:threed_sr_as_symmetric_binary_algocases1and3noalphapexists}
In Constructions 1 and 3 of Subroutine~\algorithmfont{repair}, there exists no agent $\alpha_{g}$ where $u_{\alpha_{g}}(M') < u_{\alpha_{g}}(M)$ and $\alpha_g$ belongs to a triple that blocks $M'$.
\end{lem}
\begin{proof}
Refer to Figures~\ref{fig:threed_sr_as_symmetric_binary_repair_algorithm_7_cases_case_3} and~\ref{fig:threed_sr_as_symmetric_binary_repair_algorithm_7_cases_case_1}. Suppose for a contradiction that there exists some such $\alpha_g\in N$. By the construction of $M'$ in Constructions 1 and 3, $u_{\alpha_p}(M')\geq u_{\alpha_p}(M)$ for any $\alpha_p\in N \setminus S$. It follows that $\alpha_g\in S$ and thus by the construction of $M'$ in Constructions 1 and 3 that $u_{\alpha_g}(M')\geq 1$. Since $u_{\alpha_{g}}(M') < u_{\alpha_{g}}(M)$ by assumption it must be that $u_{\alpha_g}(M) = 2$. The only such agents in $S$ are labelled $S_{3d-1}$ for some $d$ where $1 \leq d \leq c$, so it must be that $\alpha_g = S_{3d-1}$ for some such $d$. 

\begin{figure}
    \centering
    \input{figures/threed_sr_as/repair_algorithm_7_cases_case_1.tikz}
    \caption{The structure of $M'$ in Construction~1} 
    \label{fig:threed_sr_as_symmetric_binary_repair_algorithm_7_cases_case_1}
\end{figure}

First consider $S_{3c-1}$. Since $u_{S_{3c-1}}(M')=2$ it follows that $S_{3c-1}$ does not belong to a triple that blocks $M'$ and hence $\alpha_g\neq S_{3c-1}$. It remains that $\alpha_g = S_{3d-1}$ where $1\leq d < c$. By assumption, it must be that some triple $\{ S_{3d-1}, \alpha_{k_1}, \alpha_{k_2} \}$ blocks $M'$, where $\alpha_{k_1}, \alpha_{k_2} \in N$. Since $u_{S_{3d-1}}(M')=1$ it follows that $u_{S_{3d-1}}(\{ \alpha_{k_1}, \alpha_{k_2} \})=2$ and thus that $v_{S_{3d-1}}(\alpha_{k_1})=v_{S_{3d-1}}(\alpha_{k_2})=1$. Consider $\alpha_{k_1}$ and $\alpha_{k_2}$. Since $(N, V)$ is triangle-free, it must be that $v_{\alpha_{k_1}}(\alpha_{k_2}) = 0$ and thus that $u_{\alpha_{k_1}}(\{ S_{3d-1}, \alpha_{k_2} \})=u_{\alpha_{k_2}}(\{ S_{3d-1}, \alpha_{k_1} \})=1$. Since the triple $\{ S_{3d-1}, \alpha_{k_1}, \alpha_{k_2} \}$ blocks $M'$ it follows that $u_{\alpha_{k_1}}(M')=u_{\alpha_{k_2}}(M')=0$. By the construction of $M'$, there exists no $\alpha_p \in N$ where $u_{\alpha_p}(M') = 0$ and $u_{\alpha_p}(M') < u_{\alpha_p}(M)$. It follows that $u_{\alpha_{k_1}}(M)=u_{\alpha_{k_2}}(M)=0$. Recall the $d\textsuperscript{th}$ iteration of the while loop. We have shown that two agents $\alpha_{k_1}, \alpha_{k_2}$ exist in that iteration such that $v_{S_{3d-1}}(\alpha_{k_1})=v_{S_{3d-1}}(\alpha_{k_2})=1$ and $u_{\alpha_{k_1}}(M)=u_{\alpha_{k_2}}(M)=0$. It follows that, in that iteration, there existed some $\alpha_{z_1} \in N\setminus \{\alpha_i\}$ where $v_{\alpha_{z_1}}(S_{3d-1})=1$ and $u_{\alpha_{z_1}}(M)=0$, since either $\alpha_{z_1}=\alpha_{k_1}$ or $\alpha_{i}=\alpha_{k_1}$ and $\alpha_{z_1}=\alpha_{k_2}$. In that iteration, since $\alpha_{z_1}\neq \bot$ the break condition held and the while loop terminated. This is a contradiction since $d < c$.
\end{proof}

\begin{lem}
\label{lem:threed_sr_as_symmetric_binary_algocases245and6noalphapexists}
In Constructions 2, 4, 5, and 6 of Subroutine~\algorithmfont{repair}, there exists no agent $\alpha_{g}$ where $u_{\alpha_{g}}(M') < u_{\alpha_{g}}(M)$ and $\alpha_g$ belongs to a triple that blocks $M'$.
\end{lem}
\begin{proof}
Refer to Figures~\ref{fig:threed_sr_as_symmetric_binary_repair_algorithm_7_cases_case_6}, \ref{fig:threed_sr_as_symmetric_binary_repair_algorithm_7_cases_case_2}, \ref{fig:threed_sr_as_symmetric_binary_repair_algorithm_7_cases_case_4}, and~\ref{fig:threed_sr_as_symmetric_binary_repair_algorithm_7_cases_case_5}. Suppose for a contradiction that there exists some such $\alpha_g\in N$. By the construction of $M'$ in Constructions 2, 4, 5, and 6, $u_{\alpha_p}(M')\geq u_{\alpha_p}(M)$ for any $\alpha_p\in N \setminus S$. It follows that $\alpha_g \in S$ and hence $u_{\alpha_g}(M')\geq 1$. Since $u_{\alpha_{g}}(M') < u_{\alpha_{g}}(M)$ it must be that $u_{\alpha_g}(M) = 2$. The only such agents in $S$ are labelled $S_{3d-1}$ for some $d$ where $1 \leq d \leq c$, so it must be that $\alpha_g = S_{3d-1}$ for some such $d$. 

\begin{figure}
    \centering
    \input{figures/threed_sr_as/repair_algorithm_7_cases_case_2.tikz}
    \caption{The structure of $M'$ in Construction~2} 
    \label{fig:threed_sr_as_symmetric_binary_repair_algorithm_7_cases_case_2}
\end{figure}

\begin{figure}
    \centering
    \input{figures/threed_sr_as/repair_algorithm_7_cases_case_4.tikz}
    \caption{The structure of $M'$ in Construction~4} 
    \label{fig:threed_sr_as_symmetric_binary_repair_algorithm_7_cases_case_4}
\end{figure}

\begin{figure}
    \centering
    \input{figures/threed_sr_as/repair_algorithm_7_cases_case_5.tikz}
    \caption{The structure of $M'$ in Construction~5} 
    \label{fig:threed_sr_as_symmetric_binary_repair_algorithm_7_cases_case_5}
\end{figure}

Consider $S_{3d-1}$ for $1\leq d \leq c$. Note that in Constructions 2, 4, 5, and 6 it must be that $u_{S_{3d-1}}(M)=2$ and $u_{S_{3d-1}}(M')=1$. By assumption, it must be that some triple $\{ S_{3d-1}, \alpha_{k_1}, \alpha_{k_2} \}$ blocks $M'$, where $\alpha_{k_1}, \alpha_{k_2} \in N$. Since $u_{S_{3d-1}}(M')=1$ it follows that $u_{S_{3d-1}}(\{ \alpha_{k_1}, \alpha_{k_2} \})=2$. Consider $\alpha_{k_1}$ and $\alpha_{k_2}$. Since $(N, V)$ is triangle-free, it must be that $v_{\alpha_{k_1}}(\alpha_{k_2})=0$ and thus that $u_{\alpha_{k_1}}(\{ S_{3d-1}, \alpha_{k_2} \}) = u_{\alpha_{k_2}}(\{ S_{3d-1}, \alpha_{k_1} \}) = 1$. It follows that $u_{\alpha_{k_1}}(M') = u_{\alpha_{k_2}}(M') = 0$. By the construction of $M'$, there exists no $\alpha_p \in N$ where $u_{\alpha_p}(M') = 0$ and $u_{\alpha_p}(M') < u_{\alpha_p}(M)$. It follows that $u_{\alpha_{k_1}}(M)=u_{\alpha_{k_2}}(M)=0$. Recall the $d\textsuperscript{th}$ iteration of the while loop. We have shown that two agents $\alpha_{k_1}, \alpha_{k_2}$ exist where $v_{S_{3d-1}}(\alpha_{k_1})=v_{S_{3d-1}}(\alpha_{k_2})=1$ and $u_{\alpha_{k_1}}(M)=u_{\alpha_{k_2}}(M)=0$. It follows that, in that iteration, there existed some $\alpha_{z_1} \in N\setminus \{\alpha_i\}$ where $v_{\alpha_{z_1}}(S_{3d-1})=1$ and $u_{\alpha_{z_1}}(M)=0$, since either $\alpha_{z_1}=\alpha_{k_1}$ or $\alpha_{i}=\alpha_{k_1}$ and $\alpha_{z_1}=\alpha_{k_2}$. In that iteration, since $\alpha_{z_1}\neq \bot$ the break condition must have held, the while loop terminated, and the condition for either Construction~1 or Construction~3 was true. This is a contradiction since by assumption the subroutine constructed $M'$ according to one of Constructions 2, 4, 5, or 6.
\end{proof}

\begin{lem}
\label{lem:threed_sr_as_symmetric_binary_algocase7noalphapexists}
In Construction~7 of Subroutine~\algorithmfont{repair}, there exists no agent $\alpha_{g}$ where $u_{\alpha_{g}}(M') < u_{\alpha_{g}}(M)$ and $\alpha_g$ belongs to a triple that blocks $M'$.
\end{lem}
\begin{proof}

Refer to Figure~\ref{fig:threed_sr_as_symmetric_binary_repair_algorithm_7_cases_case_7}. Suppose for a contradiction that there exists some such $\alpha_g$.

\begin{figure}
    \centering
    \input{figures/threed_sr_as/repair_algorithm_7_cases_case_7.tikz}
    \caption{The structure of $M'$ in Construction~7} 
    \label{fig:threed_sr_as_symmetric_binary_repair_algorithm_7_cases_case_7}
\end{figure}

First, consider any $\alpha_p \in N$ where $\alpha_p \notin S \cup \{ \alpha_{j_2}, \alpha_{i} \}$. By the construction of $M'$, it can be seen that $M(\alpha_p)=M'(\alpha_p)$ so $u_{\alpha_p}(M)=u_{\alpha_p}(M')$ and thus $\alpha_g \notin S \cup \{ \alpha_{j_2}, \alpha_{i} \}$.

Next, consider $\alpha_i$ and $\alpha_{j_2}$. Since $u_{\alpha_i}(M) = 0 < 1 = u_{\alpha_i}(M')$ it follows that $\alpha_g \neq \alpha_i$.  Similarly, since $u_{\alpha_{j_2}}(M) = u_{\alpha_{j_2}}(M') = 0$ it follows that $\alpha_g \neq \alpha_{j_2}$.

It remains that $\alpha_g \in S$.

Consider any $S_{3d-2}$ where $1\leq d\leq c$. By construction of $M'$ it follows that $u_{S_{3d-2}}(M')=2$ so $\alpha_{p} \neq S_{3d-2}$ for any $d$ where $1\leq d\leq c$.

Next, consider any $S_{3d-1}$ where $1\leq d \leq c$. Suppose for a contradiction that $\alpha_g = S_{3d-1}$ where $1\leq d \leq c$ and thus that some triple $\{ S_{3d-1}, \alpha_{k_1}, \alpha_{k_2} \}$ blocks $M'$ where $\alpha_{k_1}, \alpha_{k_2} \in N$. Since $u_{S_{3d-1}}(M')=1$ it follows that $u_{S_{3d-1}}(\{ \alpha_{k_1}, \alpha_{k_2} \})=2$. Consider $\alpha_{k_1}$ and $\alpha_{k_2}$. Since $(N, V)$ is triangle-free, it must be that $v_{\alpha_{k_1}}(\alpha_{k_2}) = 0$ and thus that $u_{\alpha_{k_1}}(\{ S_{3d-1}, \alpha_{k_2} \})=u_{\alpha_{k_2}}(\{ S_{3d-1}, \alpha_{k_1} \})=1$. It follows that $u_{\alpha_{k_1}}(M')=u_{\alpha_{k_2}}(M')=0$. By construction of $M'$ it can be seen that $S_{3c}$ is the only agent $\alpha_p$ in $N$ where $u_{\alpha_p}(M')=0$ and $u_{\alpha_p}(M') < u_{\alpha_p}(M)$. Since the triple $\{ S_{3d-1}, \alpha_{k_1}, \alpha_{k_2} \}$ blocks $M'$ and $u_{\alpha_{k_1}}(\{ S_{3d-1}, \alpha_{k_2} \})=u_{\alpha_{k_2}}(\{ S_{3d-1}, \alpha_{k_1} \})=1$ it follows that either $u_{\alpha_{k_1}}(M)=0$, $u_{\alpha_{k_2}}(M)=0$, or both. Assume without loss of generality that $u_{\alpha_{k_1}}(M)=0$. Since $u_{\alpha_{k_1}}(M')=0$ it follows that $\alpha_{k_1}\neq \alpha_i$. Recall the $d\textsuperscript{th}$ iteration of the while loop. Since $v_{S_{3d-1}}(\alpha_{k_1})=1$, $u_{\alpha_{k_1}}(M)=0$, and $\alpha_{k_1}\neq \alpha_i$, it follows that, in that iteration, there existed some $\alpha_{z_1} \in N\setminus \{\alpha_i\}$, namely $\alpha_{k_1}$, where $v_{\alpha_{z_1}}(S_{3d-1})=1$ and $u_{\alpha_{z_1}}(M)=0$. In that iteration, since $\alpha_{z_1} \neq \bot$ the break condition must have held, the while loop terminated, and either the condition for Construction~1 was true or the condition for Construction~3 was true. This is a contradiction since by assumption the subroutine constructed $M'$ according to Construction~7. It follows that $\alpha_g \neq S_{3d-1}$ for any $d$ where $1\leq d \leq c$. 

Next, consider any $S_{3d}$ where $1\leq d < c$. By construction of $M'$ it follows that $u_{S_{3d}}(M')=u_{S_{3d}}(M)=1$ so $\alpha_{g} \neq S_{3d}$ for any such $d$.

The only possibility is thus that $\alpha_g = S_{3c}$. By the definition of $\alpha_g$, there exists some triple $\{ S_{3c}, \alpha_{k_1}, \alpha_{k_2} \}$ that blocks $M'$, where $\alpha_{k_1}, \alpha_{k_2} \in N$. Since $u_{S_{3c}}(M')=0$ it must be that either $v_{S_{3c}}(\alpha_{k_1})=1$, $v_{S_{3c}}(\alpha_{k_2})=1$, or both.

Firstly, suppose that both $v_{S_{3c}}(\alpha_{k_1})=1$ and $v_{S_{3c}}(\alpha_{k_2})=1$ so $u_{S_{3c}}(\{ \alpha_{k_1}, \alpha_{k_2} \})=2$. Since $(N, V)$ is triangle-free, it must be that $v_{\alpha_{k_1}}(\alpha_{k_2}) = 0$ and thus that $u_{\alpha_{k_1}}(\{ S_{3c}, \alpha_{k_2} \}) = u_{\alpha_{k_2}}(\{ S_{3c}, \alpha_{k_1} \}) = 1$. Since $\{ S_{3c}, \alpha_{k_1}, \alpha_{k_2} \}$ blocks $M'$ it must be that $u_{\alpha_{k_1}}(M')=u_{\alpha_{k_2}}(M')=0$. By the construction of $M'$ it can be seen that $S_{3c}$ is the only agent $\alpha_p$ in $N$ where $u_{\alpha_p}(M')=0$ and $u_{\alpha_p}(M') < u_{\alpha_p}(M)$. It follows that $u_{\alpha_{k_1}}(M)=u_{\alpha_{k_2}}(M)=0$. Note that since $u_{\alpha_i}(M')=1$ it follows that $\alpha_{k_1}\neq \alpha_i$ and $\alpha_{k_2}\neq \alpha_i$. It follows that either $\alpha_{k_1}\in N \setminus \{ \alpha_i, \alpha_{j_2} \}$, $\alpha_{k_2}\in N \setminus \{ \alpha_i, \alpha_{j_2} \}$, or both. Without loss of generality assume that $\alpha_{k_1}\in N \setminus \{ \alpha_i, \alpha_{j_2} \}$. In summary, after the termination of the while loop there existed some $\alpha_{z_2} \in N\setminus \{\alpha_i, \alpha_{j_2} \}$, namely $\alpha_{k_1}$, where $v_{\alpha_{z_2}}(S_{3c})=1$ and $u_{\alpha_{z_2}}(M)=0$. Since $\alpha_{z_2}\neq \bot$ the condition of Construction~2 holds, which is a contradiction since, by assumption, the subroutine entered Construction~7. 

Secondly, suppose either $v_{S_{3c}}(\alpha_{k_1})=1$ or $v_{S_{3c}}(\alpha_{k_2})=1$ but not both. Assume without loss of generality that $v_{S_{3c}}(\alpha_{k_1})=1$ and $v_{S_{3c}}(\alpha_{k_2})=0$. It follows that $u_{\alpha_{k_2}}(\{ S_{3c}, \alpha_{k_1} \})=1$ and hence $u_{\alpha_{k_2}}(M')=0$. Since $S_{3c}$ is the only agent $\alpha_p$ in $N$ where $u_{\alpha_p}(M')=0$ and $u_{\alpha_p}(M') < u_{\alpha_p}(M)$, it follows that $u_{\alpha_{k_2}}(M)=0$. It must be that $v_{\alpha_{k_1}}(\alpha_{k_2})=1$ since $u_{\alpha_{k_2}}(\{ S_{3c}, \alpha_{k_1} \})=1$ and $v_{S_{3c}}(\alpha_{k_2})=0$. In summary, since $v_{S_{3c}}(\alpha_{k_1})=1$ and $v_{\alpha_{k_1}}(\alpha_{k_2})=1$ it follows that $u_{\alpha_{k_1}}(\{ S_{3c}, \alpha_{k_2} \})=2$. It follows that either $u_{\alpha_{k_1}}(M')=1$ or $u_{\alpha_{k_1}}(M')=0$. Suppose firstly that $u_{\alpha_{k_1}}(M')=0$. Since $S_{3c}$ is the only agent $\alpha_p$ in $N$ where $u_{\alpha_p}(M')=0$ and $u_{\alpha_p}(M') < u_{\alpha_p}(M)$ it follows that $u_{\alpha_{k_1}}(M)=0$. There are now two possibilities. Firstly, that $\alpha_{k_1}=\alpha_{j_2}$. Secondly, that $\alpha_{k_1} \neq \alpha_{j_2}$. In the first possibility, since $\alpha_{k_1}=\alpha_{j_2}$ then after the termination of the while loop there exists some $\alpha_{y_2}\in N$, namely $\alpha_{k_2}$, where $v_{\alpha_{S_{3c}}}(\alpha_{j_2})=v_{\alpha_{y_2}}(\alpha_{j_2})=1$ and $u_{\alpha_{y_2}}(M)=0$. In the algorithm, since $\alpha_{y_2}\neq \bot$ the condition of Construction~5 holds, which is a contradiction. In the second possibility, recall that $\alpha_{k_1} \neq \alpha_{j_2}$. Since $u_{\alpha_i}(M')=1$ it follows that $\alpha_i\neq \alpha_{k_1}$ and hence there exists some $\alpha_{z_2} \in N\setminus \{\alpha_i, \alpha_{j_2} \}$, namely $\alpha_{k_1}$, where $v_{\alpha_{z_2}}(S_{3c})=1$ and $u_{\alpha_{z_2}}(M)=0$. It follows that after the termination of the while loop $\alpha_{z_2} \neq \bot$ and thus the condition of Construction~2 holds, which is a contradiction. It remains that $u_{\alpha_{k_1}}(M')=1$. To recap, we have shown that $v_{S_{3c}}(\alpha_{k_1})=v_{\alpha_{k_1}}(\alpha_{k_2})=1$, $u_{\alpha_{k_2}}(M')=u_{\alpha_{k_2}}(M)=0$, and $u_{\alpha_{k_1}}(M')=1$. This situation is illustrated in Figure~\ref{fig:threed_sr_as_symmetric_binary_repair_algorithm_proof_case_7_explanation_1}.

\begin{figure}[ht]
    \centering
    \input{figures/threed_sr_as/repair_algorithm_proof_case_7_explanation_1.tikz}
    \caption[A hypothetical blocking triple in $M'$ in Construction~7]{A hypothetical blocking triple in $M'$ in Construction~7. In Lemma~\ref{lem:threed_sr_as_symmetric_binary_algocase7noalphapexists} we suppose for a contradiction that some triple $\{ S_{3c}, \alpha_{k_1}, \alpha_{k_2} \}$ blocks $M'$ where $\alpha_{k_1}, \alpha_{k_2}\in N$. We then show that $v_{S_{3c}}(\alpha_{k_1})=v_{\alpha_{k_1}}(\alpha_{k_2})=1$, $u_{\alpha_{k_2}}(M')=u_{\alpha_{k_2}}(M)=0$, and $u_{\alpha_{k_1}}(M')=1$. We then show that this is a contradiction, and conclude that no such $\alpha_{k_1}, \alpha_{k_2}$ exist. This shows that $S_{3c}$ does not belong to a triple that blocks $M'$.}
    \label{fig:threed_sr_as_symmetric_binary_repair_algorithm_proof_case_7_explanation_1}
\end{figure}

By the condition of Construction~7, after the termination of the while loop it must have been that $\alpha_{w_1}=\bot$. By the pseudocode, it follows that in the last iteration there existed no $\alpha_{w_1} \in N$ where $v_{S_{3c}}(\alpha_{w_1})=1$, $u_{\alpha_{w_1}}(M)=1$, $\alpha_{w_1}\notin S$, and that there existed some $\alpha_{z_3}\in N\setminus \{ \alpha_i \}$ where $v_{\alpha_{z_3}}(\alpha_{w_1})=1$ and $u_{\alpha_{z_3}}(M)=0$. If $u_{\alpha_{k_1}}(M)=1$ and $\alpha_{k_1}\notin S$ then in the last iteration there existed some such $\alpha_{w_1}$ and $\alpha_{z_3}$, namely $\alpha_{k_1}$ and $\alpha_{k_2}$, which is a contradiction. It follows that either $u_{\alpha_{k_1}}(M) \neq 1$, $\alpha_{k_1}\in S$, or both.

Firstly suppose that $u_{\alpha_{k_1}}(M)\neq 1$. Recall that $u_{\alpha_{k_1}}(M')=1$. By the construction of $M'$ in Construction~7, $u_{\alpha_p}(M') = u_{\alpha_p}(M)$ for any $\alpha_p \in N \setminus (S \cup \{ \alpha_i \})$. It follows that $\alpha_{k_1} \in S \cup \{ \alpha_i \}$. By assumption, $\alpha_{k_1} \notin S$ so it must be that $\alpha_{k_1} = \alpha_i$. In this case, in the last iteration of the while loop there existed some $\alpha_{y_1} \in N$, namely $\alpha_{k_2}$, where $v_{S_{3c}}(\alpha_i)=v_{\alpha_i}(\alpha_{y_1})=1$ and $u_{\alpha_{y_1}}(M)=0$. It follows that, in the last iteration, the subroutine enters Construction~4, which is a contradiction. 

It remains that $\alpha_{k_1}\in S$. Recall that $u_{\alpha_{k_1}}(M')=1$. Since $u_{S_{3d-2}}(M')=2$ for any $d$ where $1\leq d\leq c$ it follows that $\alpha_{k_1} \neq S_{3d-2}$ for any such $d$. It follows that either $\alpha_{k_1} = S_{3d-1}$ or $\alpha_{k_1} = S_{3d}$ for some $d$ where $1\leq d\leq c$.

Suppose that $\alpha_{k_1}=S_{3d-1}$ for some $d$ where $1\leq d\leq c$. Recall the $d\textsuperscript{th}$ iteration of the while loop. We have shown that in that iteration, there existed some $\alpha_{z_1} \in N\setminus \{ \alpha_i \}$, namely $\alpha_{k_2}$, where $v_{\alpha_{z_1}}(S_{3d-1})=1$ and $u_{\alpha_{z_1}}(M)=0$. It follows that after the $d\textsuperscript{th}$ iteration of the while loop, $\alpha_{z_1} \neq \bot$ and thus that $d = c$ and the subroutine entered either Construction~1 or Construction~3, which is a contradiction.

It now remains that $\alpha_{k_1}=S_{3d}$ for some $d$ where $1\leq d \leq c$. Recall that $S_{3c} \neq \alpha_{k_1}$ so $d < c$. Recall the $d\textsuperscript{th}$ iteration of the while loop. Since $v_{S_{3d}}(\alpha_{k_2})=1$, $u_{\alpha_{k_2}}(M)=0$, and $\alpha_{k_2}\neq \alpha_i$, and $u_{\alpha_i}(M')=1$, it follows that in that iteration there existed some $\alpha_{z_2} \in N\setminus \{ \alpha_i \}$, namely $\alpha_{k_2}$, where $v_{\alpha_{z_2}}(S_{3d})=1$ and $u_{\alpha_{z_2}}(M)=0$. There are two possibilities. The first is that $\alpha_{k_2} \neq \alpha_{j_2}$. The second is that $\alpha_{k_2} = \alpha_{j_2}$. Suppose first that $\alpha_{k_2}\neq \alpha_{j_2}$. In this case there existed some $\alpha_{z_2} \in N\setminus \{ \alpha_i, \alpha_{j_2} \}$, namely $\alpha_{k_2}$, where $v_{\alpha_{z_2}}(S_{3d})=1$ and $u_{\alpha_{z_2}}(M)=0$. It follows that, in that iteration, $\alpha_{z_2}\neq \bot$ so the break condition held and the while loop terminated after that iteration. This is a contradiction since $d < c$. It remains that that $\alpha_{k_2} = \alpha_{j_2}$. It follows that, in that iteration, there existed some index $b$, namely $d$, where $1\leq b < c$ and $v_{S_{3b}}(\alpha_{j_2})=v_{S_{3c}}(S_{3b})=1$. It follows that, after the final iteration of the while loop, the condition for Construction~6 was true, which is a contradiction.
\end{proof}

It is now relatively straightforward to show that $M'$ must be stable.

\begin{lem}
\label{lem:threed_sr_as_symmetric_binary_algoreturnsstablematching_notimecomplex}
Subroutine~\algorithmfont{repair} returns a stable $P$\nobreakdash-matching $M'$.
\end{lem}
\begin{proof}
By Lemma~\ref{lem:threed_sr_as_symmetric_binary_algalwaysterminates} the subroutine must eventually terminate. By Lemma~\ref{lem:threed_sr_as_symmetric_binary_algreturnspmatching} the subroutine returns a $P$\nobreakdash-matching.

Suppose $M'$ is a $P$\nobreakdash-matching returned by the algorithm. By Lemmas~\ref{lem:threed_sr_as_symmetric_binary_algocases1and3noalphapexists}, \ref{lem:threed_sr_as_symmetric_binary_algocases245and6noalphapexists}, and~\ref{lem:threed_sr_as_symmetric_binary_algocase7noalphapexists}, in Constructions 1--7, there exists no $\alpha_g \in N$ where $u_{\alpha_{g}}(M') < u_{\alpha_{g}}(M)$ and $\alpha_g$ belongs to a triple that blocks $M'$.

Suppose for a contradiction that $M'$ is not stable and some triple $\{ \alpha_{k_1}, \alpha_{k_2},\allowbreak \alpha_{k_3} \}$ blocks $M'$. It follows that $u_{\alpha_{k_r}}(M') \geq u_{\alpha_{k_r}}(M)$ for $1 \leq r \leq 3$, otherwise some such $\alpha_g$ exists. By Proposition~\ref{prop:threed_sr_as_blockerimprovement}, it follows that $\{ \alpha_{k_1}, \alpha_{k_2}, \alpha_{k_3} \}$ also blocks $M$. By the definition of repairable, any triple that blocks $M$ must contain $\alpha_i$ so assume without loss of generality that $\alpha_{k_1}=\alpha_i$.

In Construction~4, $u_{\alpha_i}(M')=2$ and thus $\alpha_i$ does not belong to a triple that blocks $M'$, which is a contradiction. It follows that $M'$ is stable in Construction~4.

In Constructions 1, 2, 3, 5, 6, and 7, it must be that $u_{\alpha_i}(M')=1$. It follows that $u_{\alpha_i}(\{ \alpha_{k_2}, \alpha_{k_3} \})=2$ so $v_{\alpha_i}(\alpha_{k_2})=v_{\alpha_i}(\alpha_{k_3})=1$. Since $(N, V)$ is triangle-free, it must be that $v_{\alpha_{k_2}}(\alpha_{k_3})=0$ and thus that $u_{\alpha_{k_2}}(\{ \alpha_{i}, \alpha_{k_3} \})=u_{\alpha_{k_3}}(\{ \alpha_{i}, \alpha_{k_2} \})=1$. Since $\{ \alpha_{i}, \alpha_{k_2}, \alpha_{k_3} \}$ blocks $M$, It follows that $u_{\alpha_{k_2}}(M)=u_{\alpha_{k_3}}(M)=0$ and thus that $\{ \alpha_i, \alpha_{k_2}, \alpha_{k_3} \}$ also blocks $M$. It follows that $M$ is not repairable, since there exists some triple $\{ \alpha_i, \alpha_{k_2}, \alpha_{k_3} \}$ that blocks $M$ where $\alpha_{k_2}, \alpha_{k_3}\in N$ and $u_{\alpha_{k_2}}(M)=0$, $u_{\alpha_{k_3}}(M)=0$. This is a contradiction, so it follows that $M'$ is stable in Constructions 1, 2, 3, 5, 6, and 7.
\end{proof}

\begin{lem}
\label{lem:threed_sr_as_symmetric_binary_almosttherealgo_runningtime}
Subroutine~\algorithmfont{repair} has running time $O(|N|^2)$.
\end{lem}
\begin{proof}
The pseudocode of Subroutine~\algorithmfont{repair}, shown in Algorithms~\ref{alg:threed_sr_as_almostthere_algo_phase1} and~\ref{alg:threed_sr_as_almostthere_algo_phase2}, describes the subroutine at a high level. To analyse the worst-case time complexity we describe a suitable system of data structures, which we set up in a preprocessing step.%  Relying on the unit cost of standard operations in these data structures, we analyse the worst case time complexity of Subroutine~\algorithmfont{repair} in terms of $|N|$.

Suppose that $(N, V)$ is stored such that, for a given $\alpha_{p}\in N$, the subroutine can iterate through the set $\{ \alpha_{q} \in N : v_{\alpha_{p}}(\alpha_{q})=1 \}$ in $O(|N|)$ time. Suppose that $M$ is stored such that the subroutine can iterate through each triple in $O(|N|)$ time. For example, $(N, V)$ could be stored as a graph using adjacency lists and $M$ could be stored as a linked list. It follows that, given three agents $\alpha_{h_1}, \alpha_{h_2}, \alpha_{h_3}\in N$ the subroutine can compute $u_{\alpha_{h_1}}(\{ \alpha_{h_2}, \alpha_{h_3} \}), u_{\alpha_{h_2}}(\{ \alpha_{h_1}, \alpha_{h_3} \}),$ and $u_{\alpha_{h_3}}(\{ \alpha_{h_1}, \alpha_{h_2} \})$ in $O(|N|)$ time.

The preprocessing step involves constructing two lookup tables. Each lookup table contains exactly $|N|$ entries and is indexed by some $\alpha_{p}\in N$. Each entry in each table contains some integer less than or equal to $|N|$. It follows that finding an entry given its index requires constant time.  
Each entry in $L_1$ will contain either zero, one, or two. For each agent $\alpha_{p}\in N$, the subroutine constructs $L_1$ so that the ${p}\textsuperscript{th}$ entry contains $u_{\alpha_p}(M)$. By assumption, the subroutine can compute $u_{\alpha_p}(M)$ for any $\alpha_p\in N$ in $O(|N|)$ time. It follows that $L_1$ can be constructed in $O(|N|^2)$ time by iterating through $M$ and computing $u_{\alpha_{h_1}}(M)$, $u_{\alpha_{h_2}}(M)$, and $u_{\alpha_{h_3}}(M)$ for each triple $\{ \alpha_{h_1}, \alpha_{h_2}, \alpha_{h_3} \} \in M$. Since $|M|=O(|N|)$ this step takes $O(|N|^2)$ time. It follows that we can use $L_1$ to look up $u_{\alpha_{p}}(M)$ for any $\alpha_{p}\in N$ in constant time. Each entry in $L_2$ contains either the label of some agent or $\bot$. Construct $L_2$ such that for any $\alpha_p\in N$, the ${p}\textsuperscript{th}$ entry either contains some $\alpha_{q}\in N \setminus \{ \alpha_i \}$ where $v_{\alpha_{p}}(\alpha_{q})=1$ and $u_{\alpha_{q}}(M)=0$ if it exists and otherwise $\bot$. The subroutine will use $L_2$ primarily in the body of the loop to identify $\alpha_{w_1}$, if it exists, using $S_{3c}$. The lookup table $L_2$ can be constructed in $O(|N|^2)$ time, as follows. For each $\alpha_{p}\in N$, look up $u_{\alpha_{p}}(M)$ in $L_1$. If $u_{\alpha_{p}}(M)=0$ then consider each $\alpha_{q}\in N$ where $v_{\alpha_{p}}(\alpha_{q})=1$ and $\alpha_{q}\neq \alpha_i$. If the $q\textsuperscript{th}$ entry of $L_2$ is currently $\bot$ then set that entry to $\alpha_{p}$.

The list $S$ can be stored using a linked list or any data structure in which a new element can be appended to the end of $S$ in constant time and iterating through $S$ takes $O(|N|)$ time. The list $S$ will be supplemented with a lookup table $L_S$. For any $\alpha_p \in N$, the table $L_S$ can be used to test membership in $S$ and look up the position of any agent in $S$ in constant time. This is possible because the only modification that the subroutine makes to $S$ is appending a single agent to the end of $S$ in each iteration. As noted in Lemma~\ref{lem:threed_sr_as_symmetric_binary_algalwaysterminates}, any agent is added to $S$ at most than once. Like the tables $L_1$ and $L_2$, the table $L_S$ contains exactly $|N|$ entries and is indexed by each $\alpha_{p}\in N$. Each entry in $L_S$ contains some integer position less than or equal to $|S|$. Before the subroutine appends an element $\alpha_p\in N$ to the end of $S$, it can maintain $L_S$ in constant time by setting the $p\textsuperscript{th}$ entry to $|S|$.

The first step in the subroutine involves identifying agents $\alpha_{j_1}, \alpha_{j_2}$ where $\{ \alpha_i, \alpha_{j_1}, \alpha_{j_2}\}$ blocks $M$ in $(N, V)$ and $u_{\alpha_{j_1}}(M)=1$ as follows. Given any $\alpha_{j_1}, \alpha_{j_2}\in N$ where $u_{\alpha_{j_1}}(M)=1$, $u_{\alpha_{j_2}}(M)=0$ and $v_{\alpha_i}(\alpha_{j_1})=v_{\alpha_i}(\alpha_{j_2})=1$, the triple $\{ \alpha_i, \alpha_{j_1}, \alpha_{j_2}\}$ blocks $M$ in $(N, V)$. It follows that some $\alpha_{j_1}, \alpha_{j_2}\in N$ can be found in $O(|N|)$ time, as follows. Consider each agent $\alpha_{p}$ for which $v_{\alpha_i}(\alpha_{p})=1$, and look up $u_{\alpha_{p}}(M)$ in $L_1$. If $u_{\alpha_{p}}(M)=1$ then look up the ${p}\textsuperscript{th}$ entry of $L_2$. By the construction of $L_2$, if this entry is not equal to $\bot$ then it contains some $\alpha_{q}\in N\setminus \{ \alpha_i \}$ where $v_{\alpha_{p}}(\alpha_{q})=1$ and $u_{\alpha_{q}}(M)=0$. In this case let $\alpha_{j_1}=\alpha_{p}$ and $\alpha_{j_2}=\alpha_{q}$. Since $M$ is not stable in $(N, V)$, by the condition of $M$ there must exist some such $\alpha_{j_1}, \alpha_{j_2}$.

 The second step in the subroutine involves identifying agents $\alpha_{j_3}, \alpha_{j_4}$ where $\alpha_{j_3}, \alpha_{j_4} \in M(\alpha_{j_1}) \setminus \{ \alpha_{j_1} \}$ and $u_{\alpha_{j_3}}(M)=2$. This can be done in $O(|N|)$ time, as follows. Consider each triple in $M$ until $M(\alpha_{j_1})$ is found. This takes $O(|N|)$ time. Use $L_1$ to identify $\alpha_{j_3}$ and $\alpha_{j_4}$.
 
 The initialisation of $S, c, \alpha_{z_1}, \alpha_{z_2}, \alpha_{y_1}, \alpha_{y_2}$ and $\alpha_{w_1}$ in the subroutine takes constant time.
 
 Consider the while loop. By Lemma~\ref{lem:threed_sr_as_symmetric_binary_algalwaysterminates}, there are at most $\lfloor (|N|-2) \mathbin{/} 3 \rfloor = O(|N|)$ iterations. Setting up the lookup tables allows us to ensure that each iteration takes $O(|N|)$ time. It follows that the loop terminates in $O(|N|^2)$ time.
 
 To identify $\alpha_{z_1}$ as described, first identify $S_{3c-1}$, in constant time. Consider each $\alpha_{p}\in N$ for which $v_{S_{3c-1}}(\alpha_{p})=1$. This takes $O(|N|)$ time. For each such $\alpha_{p}$, if $\alpha_{p}=\alpha_i$ then continue. If $\alpha_{p}\neq \alpha_i$ then look up $u_{\alpha_{p}}(M)$ in $L_1$. If $u_{\alpha_{p}}(M)=0$ then set $\alpha_{z_1}=\alpha_{p}$. If no such $\alpha_{p}$ is found then no such $\alpha_{z_1}$ exists so set $\alpha_{z_1}=\bot$.
 
 Similarly, to identify some $\alpha_{z_2}$ as described, first identify $S_{3c}$. Consider each $\alpha_{l_1}\in N$ for which $v_{S_{3c}}(\alpha_{l_1})=1$. This takes $O(|N|)$ time. For each such $\alpha_{l_1}$, if $\alpha_{l_1}=\alpha_i$ or $\alpha_{l_1}=\alpha_{j_2}$ then continue. If not, look up $u_{\alpha_{l_1}}(M)$ in $L_1$. If $u_{\alpha_{l_1}}(M)=0$ then set $\alpha_{z_2}=\alpha_{l_1}$. If no such $\alpha_{p}$ is found then no such $\alpha_{z_2}$ exists so set $\alpha_{z_2}=\bot$.
 
 To identify $\alpha_{y_1}$ as described, test if $v_{S_{3c}}(\alpha_{i})=1$. This takes $O(|N|)$ time. If $v_{S_{3c}}(\alpha_{i})=0$ then no such $\alpha_{y_1}$ exists. If $v_{S_{3c}}(\alpha_{i})=1$ then consider each $\alpha_{p}\in N$ for which $v_{\alpha_{i}}(\alpha_{p})=1$. Note that $\alpha_p\neq \alpha_{j_2}$ since otherwise $v_{\alpha_{j_2}}(\alpha_{i})=1$, from which it follows that $\{ \alpha_i, \alpha_{j_1}, \alpha_{j_2} \}$ is a triangle in $(N, V)$. Look up $u_{\alpha_{p}}(M)$ in $L_1$. If $u_{\alpha_{p}}(M)=0$ then set $\alpha_{y_1}=\alpha_{p}$. If no such $\alpha_{p}$ where $u_{\alpha_{l_1}}(M)=0$ is found then no such $\alpha_{y_1}$ exists so set $\alpha_{y_1}=\bot$. The identification of $\alpha_{y_2}$, if it exists, can be performed similarly in $O(|N|)$ time.
 
 To compute $1 \leq b < c$ as described, if there exists some such $S_{3b}$ where $v_{S_{3b}}(\alpha_{j_2})=v_{S_{3c}}(S_{3b})=1$, consider each $\alpha_{p}\in N$ for which $v_{S_{3c}}(\alpha_{p})=1$. This takes $O(|N|)$ time. For each such $\alpha_{p}$, determine its position $b'$ in $S$ if it belongs to $S$. If $\alpha_{p}$ belongs to $S$ and $b'$ is divisible by three and less than $c$ then set $b=b'$. Otherwise, it must be that no such $S_{3b}$ exists so set $b=0$.
 
To identify some $\alpha_{w_1}$ as described, first identify $S_{3c}$ in constant time. Consider each $\alpha_{p}\in N$ for which $v_{S_{3c}}(\alpha_{p})=1$. This takes $O(|N|)$ time. For each such $\alpha_{p}$, test if $\alpha_{p}$ belongs to $S$ using $L_S$. If so, then continue. If not, then look up the $p\textsuperscript{th}$ entry in $L_2$. If this entry is $\bot$ then continue. If not, then suppose this entry is $\alpha_{q}$. By the construction of $L_2$, it follows that $\alpha_{q}\in N \setminus \{ \alpha_i \}$, $v_{\alpha_{p}}(\alpha_{q})=1$ and $u_{\alpha_{q}}(M)=0$. Accordingly, set $\alpha_{w_1}$ to $\alpha_{p}$ since the subroutine has identified $\alpha_{z_3}=\alpha_{q}\in N \setminus \{ \alpha_i \}$ for which $v_{\alpha_{w_1}}(\alpha_{z_3})=1$ and $u_{\alpha_{z_3}}(M)=0$.

Evaluating the break condition in the loop can be performed in constant time. If the break condition is true then $\alpha_{w_1}$ exists. The identification of $\alpha_{w_2}$ and $\alpha_{w_3}$ can be accomplished in $O(|N|)$ time, using the same process as for $\alpha_{j_3}$ and $\alpha_{j_4}$. Adding three elements to $S$ requires constant time.

Now consider the final if/else statement and the seven possible constructions of $M'$. In each of the seven cases, $M'$ contains each triple in $\{ r \in M : S\cap r = \varnothing \}$. This set can be constructed in $O(|N|)$ time by considering each triple in $M$ and the three corresponding entries in $L_S$. In Constructions 3 and 6, the agents $\alpha_{z_4}$ and $\alpha_{z_5}$ can each be identified in $O(|N|)$ time, using a similar procedure as described for $\alpha_{z_1}$ in the loop body. The remaining triples in $M'$ can be constructed after one scan of $S$ in $O(|N|)$ time.
\end{proof}

\begin{lem}
\label{lem:threed_sr_as_symmetric_binary_algoreturnsstablematching}
Subroutine~\algorithmfont{repair} returns a stable $P$\nobreakdash-matching in $O(|N|^2)$ time.
\end{lem}
\begin{proof}
By Lemmas~\ref{lem:threed_sr_as_symmetric_binary_algoreturnsstablematching_notimecomplex} and~\ref{lem:threed_sr_as_symmetric_binary_almosttherealgo_runningtime}.
\end{proof}
\subsubsection{Finding a stable \texorpdfstring{$P$}{P}-matching in a triangle-free instance}
\label{sec:threed_sr_as_symmetric_binary_finding_in_triangle_free}

In the previous section, we presented Subroutine~\algorithmfont{repair}, which given a triangle-free instance and a repairable $P$\nobreakdash-matching in that instance, can construct a $P$\nobreakdash-matching in that instance that is stable. In this section we present Subroutine~\algorithmfont{findStableInTriangleFree}, which calls Subroutine~\algorithmfont{repair}. Given a triangle-free instance, Subroutine~\algorithmfont{findStableInTriangleFree} can find a $P$\nobreakdash-matching in that instance that is stable. 

Subroutine~\algorithmfont{findStableInTriangleFree}, shown in Algorithm~\ref{alg:threed_sr_as_find_stable_pmatching_in_triangle_free_instance}, is recursive. The subroutine first constructs a smaller instance $(N', V')$ from $(N, V)$ by removing an arbitrary agent $\alpha_i$. It then uses a recursive call to construct a $P$\nobreakdash-matching $M$ that is stable in the smaller instance $(N', V')$. By Proposition~\ref{prop:threed_sr_as_blockerimprovement}, any triple that blocks $M$ in the larger instance $(N, V)$ must either contain $\alpha_i$ or also block $M$ in the smaller instance $(N', V')$. There are three possible cases involving types of triple that block $M$ in $(N', V')$. In two out of three cases, the subroutine constructs $M'$ in a straightforward way by adding to $M$ a new triple that contains $\alpha_i$ and two agents that are unmatched in $M$. In the third case, $M$ must be repairable. It follows by Lemma~\ref{lem:threed_sr_as_symmetric_binary_algoreturnsstablematching} that Subroutine~\algorithmfont{repair} can be used to construct a $P$\nobreakdash-matching that is stable in $(N, V)$.

\input{algorithms/threed_sr_as/find_stable_pmatching_triangle_free}

In the following lemma we verify the correctness of Subroutine~\algorithmfont{findStableInTriangleFree}.

\begin{lem}
\label{lem:threed_sr_as_symmetric_binary_algfindsstablepmatching_notimecomplex}
Given a triangle-free instance $(N, V)$, Subroutine~\algorithmfont{findStableInTriangleFree} returns a stable $P$\nobreakdash-matching in $(N, V)$.
\end{lem}
\begin{proof}
By strong induction on $|N|$. In the base case, suppose $|N| \leq 2$. It follows by the pseudocode that the subroutine returns $\varnothing$, which is a stable $P$\nobreakdash-matching in $(N, V)$.

We now show the inductive step. Consider the execution of the subroutine given some an arbitrary instance $(N, V)$. By the inductive hypothesis it follows that the subroutine returns a stable $P$\nobreakdash-matching $M$ in the smaller instance $(N', V')$, since $|N'| < |N|$.

Consider the first branch of the if/else statement in Subroutine~\algorithmfont{findStableInTriangleFree}. By construction, $u_{\alpha_{i}}(M') = 2$ and $u_{\alpha_{l_1}}(M')=u_{\alpha_{l_2}}(M')=1$. Since $M$ is a $P$\nobreakdash-matching, it follows that the subroutine returns some $P$\nobreakdash-matching $M'$. Since $u_{\alpha_i}(M')=2$, no triple that contains $\alpha_i$ blocks $M'$ in $(N, V)$. By construction, $u_{\alpha_p}(M')\geq u_{\alpha_p}(M)$ for any $\alpha_p \in N$, so it follows that any triple that blocks $M'$ in $(N, V)$ also blocks $M$ in $(N', V')$. It follows that $M'$ is stable in $(N, V)$.

Consider the second branch of the if/else statement. By construction, $u_{\alpha_{l_3}}(M')=2$ and $u_{\alpha_{i}}(M')=u_{\alpha_{l_3}}(M')=1$. Since $M$ is a $P$\nobreakdash-matching, it follows that the subroutine returns some $P$\nobreakdash-matching $M'$. Suppose for a contradiction that some triple blocks $M'$ in $(N, V)$. By construction, $u_{\alpha_p}(M')\geq u_{\alpha_p}(M)$ for any $\alpha_p\in N$ so any triple that blocks $M'$ in $(N, V)$ must contain $\alpha_i$, for otherwise that triple blocks $M$ in $(N', V')$, which is a contradiction. Suppose then that some triple $\{ \alpha_i, \alpha_{k_1}, \alpha_{k_2} \}$ blocks $M'$ in $(N, V)$, where $\alpha_{k_1}, \alpha_{k_2} \in N'$. By construction, $u_{\alpha_i}(M')=1$ so it must be that $u_{\alpha_i}(\{ \alpha_{k_1}, \alpha_{k_2} \})=2$ and thus that $v_{\alpha_i}(\alpha_{k_1})=v_{\alpha_i}(\alpha_{k_2})=1$. Since $(N, V)$ is triangle-free, it follows that $v_{\alpha_{k_1}}(\alpha_{k_2}) = 0$ and thus that $u_{\alpha_{k_1}}(\{ \alpha_{i}, \alpha_{k_2} \})=u_{\alpha_{k_2}}(\{ \alpha_{i}, \alpha_{k_1} \})=1$. It then follows that $u_{\alpha_{k_1}}(M')=u_{\alpha_{k_2}}(M')=0$. By the construction of $M'$ it must be that $u_{\alpha_p}(M')\geq u_{\alpha_p}(M)$ for any $\alpha_p\in N$ so $u_{\alpha_{k_1}}(M)=u_{\alpha_{k_2}}(M)=0$. This contradicts the condition of the first branch of the if/else statement, since there exist two agents $\alpha_{l_1}$ and $\alpha_{l_2}$, namely $\alpha_{k_1}$, and $\alpha_{k_2}$, where $u_{\alpha_{l_1}}(M)=u_{\alpha_{l_2}}(M)=0$ and $v_{\alpha_i}(\alpha_{l_1})=v_{\alpha_i}(\alpha_{l_2})=1$.

Consider the third branch of the if/else statement. It must be that the conditional expressions in the first and second branches of the if/else statement do not hold. It follows from this that any triple that blocks $M$ in $(N', V')$ comprises $\{ \alpha_i, \alpha_{l_5}, \alpha_{l_6} \}$ where $\alpha_{l_5}, \alpha_{l_6}\in N$, $u_{\alpha_{l_5}}(M)=1$, $u_{\alpha_{l_6}}(M)=0$, and $v_{\alpha_i}(\alpha_{l_5})=v_{\alpha_{l_5}}(\alpha_{l_6})=1$. Note that $u_{\alpha_i}(M)=0$ and thus $M$ is repairable (defined in Section~\ref{sec:threed_sr_as_symmetric_binary_repairing}). By Lemma~\ref{lem:threed_sr_as_symmetric_binary_algoreturnsstablematching}, Subroutine~ \algorithmfont{repair} returns some $P$\nobreakdash-matching $M'$ that is stable in $(N, V)$.

Consider the fourth branch of the if/else statement. It must be that the conditional expressions in the first, second, and third branches of the if/else statement do not hold. By construction, $u_{\alpha_p}(M') = u_{\alpha_p}(M)$ for any $\alpha_p\in N$. It follows that any triple that blocks $M'$ in $(N, V)$ must contain $\alpha_i$, for otherwise that triple also blocks $M$ in $(N', V')$, which is a contradiction. Suppose for a contradiction that some triple $\{ \alpha_i, \alpha_{k_1}, \alpha_{k_2} \}$ blocks $M'$ in $(N, V)$.

Suppose firstly that $u_{\alpha_i}(\{ \alpha_{k_1}, \alpha_{k_2} \})=2$. Since $(N, V)$ is triangle-free, it follows that $u_{\alpha_{k_1}}(\{ \alpha_{i}, \alpha_{k_2} \})=u_{\alpha_{k_2}}(\{ \alpha_{i}, \alpha_{k_1} \})=1$. It follows that $u_{\alpha_{k_1}}(M')=u_{\alpha_{k_2}}(M')=0$. Since $u_{\alpha_p}(M')\geq u_{\alpha_p}(M)$ for any $\alpha_p\in N$, it must be that $u_{\alpha_{k_1}}(M)=u_{\alpha_{k_2}}(M)=0$. This contradicts the condition of the first branch of the if/else statement.

Suppose secondly that $u_{\alpha_i}(\{ \alpha_{k_1}, \alpha_{k_2} \})=1$. It must be that either $u_{\alpha_{k_1}}(\{ \alpha_{i}, \alpha_{k_2} \})=2$ or $u_{\alpha_{k_2}}(\{ \alpha_{i}, \alpha_{k_1} \})=2$. Suppose without loss of generality that $u_{\alpha_{k_1}}(\{ \alpha_{i}, \alpha_{k_2} \})=2$. It follows that $v_{\alpha_{k_1}}(\alpha_{i})=v_{\alpha_{k_1}}(\alpha_{k_2})=1$. There are two possibilities: either $u_{\alpha_{k_1}}(M)=1$ or $u_{\alpha_{k_1}}(M)=0$. The first possibility implies that the conditional expression of the second if/else branch holds, which is a contradiction. The second possibility implies that the conditional expression of the third if/else branch holds, also a contradiction.
\end{proof}

We now consider the worst-case time complexity of Subroutine~\algorithmfont{findStableInTriangleFree}.

\begin{lem}
\label{lem:threed_sr_as_symmetric_binary_algfindstablepmatchingrunningtime}
Subroutine~\algorithmfont{findStableInTriangleFree} has running time $O(|N|^3)$.
\end{lem}
\begin{proof}
% The pseudocode for Subroutine~\algorithmfont{findStableInTriangleFree} gives an outline of the subroutine at a high level. As before, to analyse the worst-case time complexity we provide a more detailed description of certain steps in terms of the unit cost of operations in standard data structures. This description suffices to show that the running time of the subroutine is $O(|N|^3)$. 
% Let $T(|N|)$ be the running time of the subroutine given an instance $(N, V)$. We will prove inductively that $T(|N|)=O(|N|^3)$.

Since Subroutine~\algorithmfont{findStableInTriangleFree} is recursive, and the recursive call involves an instance $(N', V')$ where $|N'| = |N| - 1$, it suffices to show that worst-case time complexity of the subroutine excluding the recursive call is $O(|N|^2)$.

Suppose that the input $(N, V)$ is given such that, for a given $\alpha_{p}\in N$, the subroutine can iterate through the set $\{ \alpha_{q} \in N : v_{\alpha_{p}}(\alpha_{q})=1 \}$ in $O(|N|)$ time. For example, $(N, V)$ could be stored as a graph using adjacency lists. It follows that, given three agents $\alpha_{h_1}, \alpha_{h_2}, \alpha_{h_3}\in N$ the subroutine can compute $u_{\alpha_{h_1}}(\{ \alpha_{h_2}, \alpha_{h_3} \})$, $u_{\alpha_{h_2}}(\{ \alpha_{h_1}, \alpha_{h_3} \})$ and $u_{\alpha_{h_3}}(\{ \alpha_{h_1}, \alpha_{h_2} \})$ in $O(|N|)$ time. The subroutine will return a $P$\nobreakdash-matching $M'$ stored as a linked list or similar data structure that allows a triple to be appended to the end of list in constant time.

% By inspection, when $|N|=2$ the subroutine returns immediately and hence $T(2)=O(1)$. In this case the subroutine will return an empty linked list or similar data structure.

The constructed instance $(N',V')$ can be stored using adjacency lists or an equivalent data structure. A straightforward procedure to identify $\alpha_i$ and construct $(N', V')$ takes $O(|N|)$ time.

After this call, the subroutine can construct a supplementary lookup table $L_1$, with exactly $|N|-1$ entries indexed by each $\alpha_p\in N'$. Each entry will contain either zero, one, or two. For each agent $\alpha_p\in N$, the subroutine constructs $L_1$ so that the $p\textsuperscript{th}$ entry contains $u_{\alpha_p}(M)$. By assumption, the subroutine can compute $u_{\alpha_p}(M)$ for any $\alpha_p\in N$ in $O(|N|)$ time. It follows that $L_1$ can be constructed in $O(|N|^2)$ time by iterating through $M$ and computing $u_{\alpha_{h_1}}(M)$, $u_{\alpha_{h_2}}(M)$, and $u_{\alpha_{h_3}}(M)$ for each triple $\{ \alpha_{h_1}, \alpha_{h_2}, \alpha_{h_3} \} \in M$. Since $|M|=O(|N|)$ this step takes $O(|N|^2)$ time. It follows that we can use $L_1$ to look up $u_{\alpha_{p}}(M)$ for any $\alpha_{p}\in N$ in constant time.

The construction of $L_1$ allows the subroutine to identify some $\alpha_{l_1}$ and $\alpha_{l_2}\in N$ where $u_{\alpha_{l_1}}(M)=u_{\alpha_{l_2}}(M)=0$ and $v_{\alpha_i}(\alpha_{l_1})=v_{\alpha_i}(\alpha_{l_2})=1$, if two such agents exist, in $O(|N|^2)$ time. One way to do this is to consider each pair $\{ \alpha_{l_1}, \alpha_{l_2} \} \in \binom{N}{2}$ and look up $u_{\alpha_{l_1}}(M)$ and $u_{\alpha_{l_2}}(M)$ in $L_1$. Since $M$ is stored using a linked list or similar data structure, if such $\alpha_{l_1}, \alpha_{l_2}\in N$ exist then $M'$ can be constructed by adding the triple $\{ \alpha_i, \alpha_{l_1}, \alpha_{l_2} \}$ to $M$, in constant time. Similarly, the identification of $\alpha_{l_3}, \alpha_{l_4}\in N$ where $u_{\alpha_{l_3}}(M)=u_{\alpha_{l_4}}(M)=0$ and $v_{\alpha_i}(\alpha_{l_3})=v_{\alpha_{l_3}}(\alpha_{l_4})=1$ can be performed in $O(|N|^2)$ time and the corresponding construction of $M'$ in constant time. In the third branch of the if/else statement, the identification of $\alpha_{l_5}, \alpha_{l_6}\in N$ where $u_{\alpha_{l_3}}(M)=1$, $u_{\alpha_{l_4}}(M)=0$ and $v_{\alpha_i}(\alpha_{l_3})=v_{\alpha_{l_3}}(\alpha_{l_4})=1$ can be similarly performed in $O(|N|^2)$ time. By Lemma~\ref{lem:threed_sr_as_symmetric_binary_almosttherealgo_runningtime}, the call to Subroutine~\algorithmfont{repair} also takes $O(|N|^2)$ time.
\end{proof}
\subsubsection{Finding a stable \texorpdfstring{$P$}{P}-matching in an arbitrary instance}
\label{sec:threed_sr_as_symmetric_binary_finding_in_general}

In the previous section, we presented Subroutine~\algorithmfont{findStableInTriangleFree}, which given a triangle-free instance can construct a $P$\nobreakdash-matching in that instance that is stable. In this section we present Algorithm~\algorithmfont{findStable}, which given an arbitrary instance of 3DR-AS can construct a $P$\nobreakdash-matching that is stable in that instance.

Algorithm~\algorithmfont{findStable} involves two steps, as follows. In the first step, it constructs a $P$\nobreakdash-matching $M_{\triangle}$ in $(N, V)$ and a triangle-free instance $(N', V')$ where $|N'| \leq |N|$ and if $M$ is a stable $P$\nobreakdash-matching in $(N', V')$ then $M_{\triangle} \cup M$ is a stable $P$\nobreakdash-matching in $(N, V)$. By Lemma~\ref{lem:threed_sr_as_symmetric_binary_trianglefree}, this can be done in $O(|N|^3)$ time. In the second step, it calls Subroutine~\algorithmfont{findStableInTriangleFree} on $(N', V')$ to construct a $P$\nobreakdash-matching $M'$ that is stable in $(N', V')$. It then returns $M' = M \cup M_{\triangle}$.

% Using the results in Sections~\ref{sec:threed_sr_as_removingtriangles}, \ref{sec:threed_sr_as_symmetric_binary_repairing}, and \ref{sec:threed_sr_as_symmetric_binary_finding_in_triangle_free}, we show in Theorem~\ref{thm:threed_sr_as_symmetric_binary_construction} that Algorithm~\algorithmfont{findStable} must return a stable $P$\nobreakdash-matching in $O(|N|^3)$ time.

% In the previous section we considered instances of 3DR-AS with binary and symmetric preferences that are triangle-free. We showed that, given such an instance, Subroutine~\algorithmfont{findStableInTriangleFree} can be used to find a stable $P$\nobreakdash-matching in $O(|N|^3)$ time (Lemma~\ref{lem:threed_sr_as_symmetric_binary_algfindsstablepmatching}). In Section~\ref{sec:threed_sr_as_symmetric_binary_prelims}, we showed that an arbitrary instance can be reduced in $O(|N|^3)$ time to construct a corresponding triangle-free instance (Lemma~\ref{lem:threed_sr_as_symmetric_binary_trianglefree}). We define a subroutine, \algorithmfont{eliminateTriangles}, which reduces an arbitrary instance in this way, and returns a pair containing the reduced instance and a set of triangles $M_{\triangle}$. Algorithm~\algorithmfont{findStable} therefore comprises two steps. First, the instance is reduced with a call to \algorithmfont{eliminateTriangles}. Then, Subroutine~\algorithmfont{findStableInTriangleFree} is called to construct a $P$\nobreakdash-matching $M'$ that is stable in the reduced, triangle-free instance.

% \input{algorithms/threed_sr_as/overall}

% \begin{lem}
% \label{lem:threed_sr_as_symmetric_binary_construction_norunningtime}
% Given an instance $(N, V)$ of 3DR-AS with binary and symmetric preferences, Algorithm~\algorithmfont{findStable} returns a stable $P$\nobreakdash-matching.
% \end{lem}
% \begin{proof} 
% A suitable implementation of Subroutine~\algorithmfont{eliminateTriangles} returns a pair $((N', V'), M_{\triangle})$ where $|N'|\leq |N|$ and if $M$ is a stable $P$\nobreakdash-matching in $(N', V')$ then $M' = M \cup M_{\triangle}$ is a stable $P$\nobreakdash-matching in $(N, V)$ (Lemma~\ref{lem:threed_sr_as_symmetric_binary_trianglefree}). By Lemma~\ref{lem:threed_sr_as_symmetric_binary_algfindsstablepmatching}, Subroutine~\algorithmfont{findStableInTriangleFree} returns $P$\nobreakdash-matching $M'$ that is stable in in $(N', V')$. It follows that $M' \cup M_{\triangle}$ is a $P$\nobreakdash-matching that is stable in $(N, V)$.
% \end{proof}

% \begin{lem}
% \label{lem:threed_sr_as_symmetric_binary_construction_runningtime}
% Algorithm~\algorithmfont{findStable} has running time $O(|N|^3)$.
% \end{lem}
% \begin{proof}
% By definition, Subroutine~\algorithmfont{eliminateTriangles} has running time $O(|N|^3)$ (Lemma~\ref{lem:threed_sr_as_symmetric_binary_trianglefree}). By Lemma~\ref{lem:threed_sr_as_symmetric_binary_algfindstablepmatchingrunningtime}, Subroutine~\algorithmfont{findStableInTriangleFree} also has running time $O(|N|^3)$. It follows that Algorithm~\algorithmfont{findStable} has total running time $O(|N|^3)$.
% \end{proof}

\begin{thm}
\label{thm:threed_sr_as_symmetric_binary_construction}
Given an instance $(N, V)$ of 3DR-AS with binary and symmetric preferences, a stable $P$\nobreakdash-matching, and hence a stable matching, must exist and can be found in $O(|N|^3)$~time.
\end{thm}
\begin{proof}
By Lemma~\ref{lem:threed_sr_as_symmetric_binary_trianglefree}, the first step of the algorithm takes $O(|N|^3)$ time and $(N', V')$ is a triangle-free instance where $|N'| \leq |N|$ and if $M$ is a stable $P$\nobreakdash-matching in $(N', V')$ then $M_{\triangle} \cup M$ is a stable $P$\nobreakdash-matching in $(N, V)$.

By Lemma~\ref{lem:threed_sr_as_symmetric_binary_algfindsstablepmatching_notimecomplex}, the $P$\nobreakdash-matching $M$ returned by the call to Subroutine~\algorithmfont{findStableInTriangleFree} is stable in $(N', V')$. By Lemma~\ref{lem:threed_sr_as_symmetric_binary_algfindstablepmatchingrunningtime}, this call, which constitutes the second step of Algorithm~\algorithmfont{findStable}, takes $O(|N|^3)$ time. It follows that $M' = M_{\triangle} \cup M$ is a stable $P$\nobreakdash-matching in $(N, V)$ and the worst-case time complexity of Algorithm~\algorithmfont{findStable} is $O(|N|^3)$.
% By Lemmas~\ref{lem:threed_sr_as_symmetric_binary_construction_norunningtime} and~\ref{lem:threed_sr_as_symmetric_binary_construction_runningtime}. If $|N|$ is a multiple of three, then if required the agents unmatched in $M' \cup M_{\triangle}$ can be arbitrarily matched into triples. By Lemma~\ref{lem:threed_sr_as_blockerimprovement}, the resulting matching is still stable in $(N, V)$. 
\end{proof}
\subsection{Maximising utilitarian welfare}
\label{sec:threed_sr_as_symmetric_binary_welfare}

We have shown that in an instance of 3DR-AS with binary and symmetric preferences, a stable matching must exist and can be found in polynomial time. In this section we consider a related optimisation problem, in which the goal is to find a stable matching with maximum utilitarian welfare given an instance of 3DR-AS with binary and symmetric preferences. We first show that this problem is $\NP$-hard and then extend Algorithm~\algorithmfont{findStable} to devise a $2$-approximation algorithm.

We formalise this optimisation problem as the \emph{3DR-AS Stable Maximum Utilitarian Welfare problem} (3DR-AS-SMUW). It is straightforward to show that 3DR-AS-SMUW is $\NP$-hard, as follows.

\begin{thm}
\label{thm:threed_sr_as_maxutilstable_hard}
3DR-AS-SMUW is $\NP$-hard.
\end{thm}
\begin{proof}
A direct reduction exists from PIT to the problem of deciding if a given instance of 3DR-AS-SMUW contains a stable matching $M$ with utilitarian welfare greater than or equal to a given bound, as follows. Suppose $G = (W, E)$ is an arbitrary undirected graph. First, for each vertex $w_i$ in $W$ construct one agent $\alpha_i$ in $N$. Next, for any two agents $\alpha_i, \alpha_j \in N$, let $v_{\alpha_i}(\alpha_j) = 1$ if $\{ w_i, w_j \} \in E$ and $0$ otherwise. It is straightforward to show that $(N, V)$ contains a stable matching with utilitarian welfare $2|W|$ if and only if $G$ contains a partition into triangles.
\end{proof}

Note that the reduction from PIT to 3DR-AS-SMUW also shows that the problem of finding a (not-necessarily stable) matching with maximum utilitarian welfare in a given instance of 3DR-AS is also $\NP$-hard, even when preferences are binary and symmetric.

% In Section~\ref{sec:threed_sr_as_symmetric_binary_finding_in_general} we showed that, given an arbitrary instance $(N, V)$ of 3DR-AS with binary and symmetric preferences, a stable $P$\nobreakdash-matching exists and can be found in $O(|N|^3)$ time. 

We now present an approximation algorithm for 3DR-AS-SMUW, which we call Algorithm~\algorithmfont{findStableUW}, shown in Algorithm~\ref{alg:threed_sr_as_approximationalgo}. We first provide some intuition regarding its design and then prove that it is correct and analyse its approximation ratio.

\input{algorithms/threed_sr_as/approx}

At a high level, Algorithm~\algorithmfont{findStableUW} involves two phases. In the first phase, it calls Algorithm~\algorithmfont{findStable} to construct a stable $P$\nobreakdash-matching $M_1$. In the second phase, it orders the unmatched agents $U$ in $M_1$ into triples such that utilitarian welfare of the agents in $U$ is maximised. In order to do this, it constructs a maximal matching in the subgraph induced by $U$ and then orders the agents in $U$ to triples such that the number of triples that contain an edge in the maximal matching is maximised.

It is straightforward to show that Algorithm~\algorithmfont{findStableUW} returns a matching $M$ in polynomial time. We now analyse its approximation ratio.

Suppose $(N, V)$ is an arbitrary instance of 3DR-AS with binary and symmetric preferences, $M$ is a matching returned by Algorithm~\algorithmfont{findStableUW} given $(N, V)$ and $M^*$ is a stable matching with maximum utilitarian welfare in $(N, V)$. % Recall that $|N|=3k + l$ for some $k \geq 0$ and $0 \leq l < 3$ and by Proposition~\ref{prop:threed_sr_as_completematching} we assume that at most $l$ agents are unmatched in $M^*$ and thus that $|M^*| = k$.

At a high level, the analysis involves placing a lower bound on the welfare in $M$ of the agents in each triple apportioned by the triples in $M^*$. To do this, let $T(y)$ be the triples in $M$ with utilitarian welfare $y$ and $T^*(y)$ be the triples in $M^*$ with utilitarian welfare $y$, for some $y \geq 0$. In fact, since preferences are binary and symmetric it must be that the utilitarian welfare of any triple in $(N, V)$ must be either $0$, $2$, $4$, or $6$. Thus, by definition
\begin{align}
M &= T(6) \cup T(4) \cup T(2) \cup T(0)\label{eqn:threed_sr_as_macomposition}
\end{align}
and
\begin{align}
M^* &= T^*(6) \cup T^*(4) \cup T^*(2) \cup T^*(0)\label{eqn:threed_sr_as_moptcomposition}\enspace.
\end{align}
It follows that
% \begin{align}
%     M^* &= T^*(6) \cup T^*(4) \cup T^*(2) \cup T^*(0) \label{eqn:threed_sr_as_moptcomposition}\\
%     M &= T(6) \cup T(4) \cup T(2) \cup T(0)\label{eqn:threed_sr_as_macomposition}
% \end{align}
% \begin{alignat}{7}
%     M^* &= \matheqbox{RARoptbox}{T^*(6)} &\,& \cup &\,& \matheqbox{RARoptbox}{T^*(4)} &\,& \cup &\,& \matheqbox{RARoptbox}{T^*(2)} &\,& \cup &\,& \matheqbox{RARoptbox}{T^*(0)} \label{eqn:threed_sr_as_moptcomposition}\\
%     M &= 
%     \matheqbox{RARoptbox}{T^*(6)} &&\cup&& \matheqbox{RARoptbox}{T^*(4)} &&\cup&& \matheqbox{RARoptbox}{T^*(2)} &&\cup&& \matheqbox{RARoptbox}{T^*(0)} \label{eqn:threed_sr_as_macomposition}
% \end{alignat}
% and thus
% \begin{align}
% u(M^*) &= 6|T^*(6)| + 4|T^*(4)| + 2|T^*(2)| \label{eqn:threed_sr_as_welfareofmopt} \\
% u(M) &= 6|T^*(6)| + 4|T^*(4)| + 2|T^*(2)|\enspace. \label{eqn:threed_sr_as_bwelfareofma}
% \end{align}
\begin{align}
    u(M) &= 6|T(6)| + 4|T(4)| + 2|T(2)|\label{eqn:threed_sr_as_bwelfareofma} && \mbox{by Equation~\ref{eqn:threed_sr_as_macomposition}} 
\end{align}
and
\begin{align}
    u(M^*) &= 6|T^*(6)| + 4|T^*(4)| + 2|T^*(2)| && \mbox{by Equation~\ref{eqn:threed_sr_as_moptcomposition}.} \label{eqn:threed_sr_as_welfareofmopt}
\end{align}

We first place a lower bound on $|T(6)|$ in terms of $|T^*(6)|$.
\begin{lem}
\label{lem:threed_sr_as_tau_a_geq_tau_opt_over_3}
$|T(6)| \geq |T^*(6)|/3$.
\end{lem}
\begin{proof}
By the pseudocode of Algorithm~\algorithmfont{findStable} and the definition of Subroutine~\algorithmfont{eliminateTriangles}, $T(6)$ contains a maximal triangle packing in the underlying graph $(N, E)$. Since by definition $T^*(6)$ is also a maximal triangle packing in the same graph, it must be that $|T(6)| \geq |T^*(6)|/3$.
% Consider an arbitrary triple $\{ \alpha_{h_1}, \alpha_{h_2}, \alpha_{h_3} \} \in T^*(6)$. It must be that $v_{\alpha_{h_1}}(\alpha_{h_2})=v_{\alpha_{h_2}}(\alpha_{h_3})=v_{\alpha_{h_3}}(\alpha_{h_1})=1$. Recall that the first step of Algorithm~\algorithmfont{findStable} involved selecting a maximal set of triangles. In the pseudocode description of Algorithm~\algorithmfont{findStable}, we described this operation using Subroutine~\algorithmfont{eliminateTriangles}, which we refer to here. Since  $\{ \alpha_{h_1}, \alpha_{h_2}, \alpha_{h_3} \}$ is a triangle in $(N, V)$, either Subroutine~\algorithmfont{eliminateTriangles} selected this triple, and $\{ \alpha_{h_1}, \alpha_{h_2}, \alpha_{h_3} \} \in T^*(6)$, or at least one of $\alpha_{h_1}, \alpha_{h_2}, \alpha_{h_3}$ was added to a different triple in $T^*(6)$. In either case, any triple in $T^*(6)$ contains at least one agent that belongs to some triple in $T^*(6)$. Triples in $T^*(6)$ are disjoint, hence the number of agents in triples in $T^*(6)$ is at least $|T^*(6)|$. It follows that $|T^*(6)| \geq |T^*(6)|/3$.
\end{proof}

We now show that if no triple in $M$ has utilitarian welfare $0$ then $2u(M) \geq u(M^*)$.

\begin{lem}
\label{lem:threed_sr_as_no000exists_lem}
If $T(0) = \varnothing$ then $2{u(M)} \geq u(M^*)$.
\end{lem}
\begin{proof}
First consider $M^*$. 
% By Equation~\ref{eqn:threed_sr_as_welfareofmopt},
% \begin{align}
%     u(M^*) &= 6|T^*(6)| + 4|T^*(4)| + 2|T^*(2)|\nonumber\\
%     &\leq 6|T^*(6)| + 4(|T^*(4)| + |T^*(2)|)\nonumber\\
%     &= 6|T^*(6)| + 4(|M^*| - |T^*(6)|) && \mbox{by Equation~\ref{eqn:threed_sr_as_moptcomposition}}\nonumber\\
%     &= 6|T^*(6)| + 4(n - |T^*(6)|) && \mbox{since by definition, $|M^*|=n$}\nonumber\\
%     &= 2|T^*(6)| + 4n\label{eqn:threed_sr_as_no000exists_lem_inequality1}\enspace.
% \end{align}
Now
\begingroup
\allowdisplaybreaks
\begin{align*}
    2u(M) &= 12|T(6)| + 8|T(4)| + 4|T(2)| && \mbox{by Equation~\ref{eqn:threed_sr_as_bwelfareofma}}\\
    &\geq 12|T(6)| + 4(|T(4)| + |T(2)|)\\
    &= 12|T(6)| + 4(|M| - |T(6)|) && \mbox{by Equation~\ref{eqn:threed_sr_as_macomposition}, since $T(0)=\varnothing$}\\
    &= 12|T(6)| + 4(n - |T(6)|) && \mbox{by definition, $|M|=n$}\\
    % hack because we have code in main.tex that automatically adjusts vertical spacing
    &= 8|T(6)| + 4n\\[0.7em]
    &\geq \smash{\frac{8|T^*(6)|}{3}} + 4n && \mbox{by Lemma~\ref{lem:threed_sr_as_tau_a_geq_tau_opt_over_3}}\\[0.4em]
    &\geq 2|T^*(6)| + 4n\\
    &= 6|T^*(6)| - 4|T^*(6)| + 4n\\
    &= 6|T^*(6)| + 4(n - |T^*(6)|)\\
    &= 6|T^*(6)| + 4(|M^*| - |T^*(6)|)\\
    &= 6|T^*(6)| + 4(|T^*(4)| + |T^*(2)| + |T^*(0)|) && \mbox{by Equation~
    \ref{eqn:threed_sr_as_moptcomposition}}\\
    &= 6|T^*(6)| + 4|T^*(4)| + 4|T^*(2)| && \mbox{since $u(T^*(0))=0$}\\
    &\geq 6|T^*(6)| + 4|T^*(4)| + 2|T^*(2)|\\
    &= u(M^*) && \mbox{by Equation~\ref{eqn:threed_sr_as_welfareofmopt}.}
\end{align*}
\endgroup
\end{proof}
We now consider the case when there exists at least one triple in $T(0)$. The existence of such a triple allows us to deduce that $|\mathcal{Q}| < |U|/3$ and thus that any two agents that form an edge in the maximal matching $\mathcal{Q}$ must be assigned to the same triple in $M$.

\begin{lem}
\label{lem:threed_sr_as_approx_if000thentisnotcomplete}
If $|T(0)| > 0$ then $|\mathcal{Q}| < |U|/3$.
\end{lem}
\begin{proof}
We prove the contrapositive. Suppose $|\mathcal{Q}| \geq |U|/3$. By the pseudocode of Algorithm~\algorithmfont{findStableUW}, $\mathcal{X}\subseteq \mathcal{Q}$ is a set of pairs where $v_{x_i}(x_j)=1$ for each pair $\{ x_i, x_j \} \in \mathcal{X}$. It follows that each triple in $M_2$ contains two agents $x_i, x_j$ for which $v_{x_i}(x_j)=1$. It follows that $u_{t}(M) \geq 2$ for any triple $t \in M_2$. Since $M_1$ is a $P$\nobreakdash-matching, by definition  $u_{t}(M) \geq 2$ for any $t \in M_1$ so it must be that $|T(0)| = \varnothing$.
\end{proof}

\begin{lem}
\label{lem:threed_sr_as_approx_if000theneveryagentintgets1}
If $|T(0)| > 0$ then $u_{\alpha_p}(M) \geq 1$ for any $\alpha_p \in \bigcup \mathcal{Q}$.
\end{lem}
\begin{proof}
Suppose $|T^*(0)|>0$. Consider an arbitrary $\alpha_p \in \bigcup \mathcal{Q}$. It follows that some $\alpha_q\in N$ exists where $\{ \alpha_p, \alpha_q \}\in \mathcal{Q}$ and hence $v_{\alpha_p}(\alpha_q)=1$, by the definition of $\mathcal{Q}$. 

By Lemma~\ref{lem:threed_sr_as_approx_if000thentisnotcomplete}, $|\mathcal{Q}| < |U|/3$. It follows that $\{ \alpha_p, \alpha_q \} \in \mathcal{X}$. It follows that there exists some $i$ where $1 \leq i \leq |U|/3$ such that $X_i = \{ \alpha_p, \alpha_q \}$ and hence, by construction of $M_2$, the triple $X_i \cup \{ y_i \}$ belongs to $M_2$. It follows that $\alpha_q\in M_2(\alpha_p)$ and hence $u_{\alpha_p}(M)\geq 1$.
\end{proof}

\begin{lem}
\label{lem:threed_sr_as_000exists_stlemma}
If $|T(0)|>0$ then for any $\alpha_r, \alpha_s \in N$ where $v_{\alpha_r}(\alpha_s) = 1$ it must be that $u_{\{ \alpha_r, \alpha_s\}}(M) \geq 1$.
\end{lem}
\begin{proof}
Suppose for a contradiction that $|T(0)| > 0$ and that there exists some $\alpha_r, \alpha_s \in N$ where $v_{\alpha_r}(\alpha_s)=1$ and $u_{\{ \alpha_r, \alpha_s\}}(M) = 0$. It follows that $u_{\alpha_r}(M) = u_{\alpha_s}(M) = 0$. It follows by Lemma~\ref{lem:threed_sr_as_approx_if000theneveryagentintgets1} that $\alpha_r \notin \bigcup \mathcal{Q}$ and $\alpha_s \notin \bigcup \mathcal{Q}$. It follows that $\mathcal{Q}' = \mathcal{Q} \cup \{ \alpha_r, \alpha_s \}$ is a disjoint set of pairs of agents in $U$ where $v_{\alpha_p}(\alpha_q)=1$ for each pair $\{ \alpha_p, \alpha_q \} \in \mathcal{Q}'$. Since $|\mathcal{Q}'|>|\mathcal{Q}|$, this contradicts the maximality of $\mathcal{Q}$ (which is computed in Subroutine~\algorithmfont{maximal2DMatching}).
\end{proof}

\begin{lem}
\label{lem:threed_sr_as_some000exists_r1_lem}
If $|T(0)| > 0$ then $u_t(M)\geq 3$ for any $t \in T^*(6)$.
\end{lem}
\begin{proof}
Suppose $|T(0)| > 0$. Consider an arbitrary $\{ \alpha_{h_1}, \alpha_{h_2}, \alpha_{h_3} \} \in T^*(6)$. By definition, $v_{\alpha_{h_1}}(\alpha_{h_2})=v_{\alpha_{h_2}}(\alpha_{h_3})=v_{\alpha_{h_3}}(\alpha_{h_1})=1$. Since $M$ is a stable matching, the triple $\{ \alpha_{h_1}, \alpha_{h_2}, \alpha_{h_3} \}$ does not block $M$. It follows that at least one of the following holds: $u_{\alpha_{h_1}}(M)=2$, $u_{\alpha_{h_2}}(M)=2$, or $u_{\alpha_{h_3}}(M)=2$. Suppose without loss of generality that $u_{\alpha_{h_1}}(M)=2$. By Lemma~\ref{lem:threed_sr_as_000exists_stlemma}, it must be that $u_{\{ \alpha_{h_2}, \alpha_{h_3}\}}(M)\geq 1$. In total, $u_{\{ \alpha_{h_1}, \alpha_{h_2},\allowbreak \alpha_{h_3} \}}(M) \geq 3$.
\end{proof}

\begin{lem}
\label{lem:threed_sr_as_some000exists_r2_lem}
If $|T(0)| > 0$ then $u_t(M)\geq 2$ for any $t \in T^*(4)$.
\end{lem}
\begin{proof}
Suppose $|T(0)| > 0$. Consider an arbitrary $\{ \alpha_{h_1}, \alpha_{h_2}, \alpha_{h_3} \} \in T^*(4)$ where $v_{\alpha_{h_1}}(\alpha_{h_2}) = v_{\alpha_{h_2}}(\alpha_{h_3}) = 1$ and $v_{\alpha_{h_1}}(\alpha_{h_3})=0$. Suppose for a contradiction that $u_{\{ \alpha_{h_1}, \alpha_{h_2}, \alpha_{h_3} \}}(M) < 2$.

If $u_{\{ \alpha_{h_1}, \alpha_{h_2}, \alpha_{h_3} \}}(M) = 0$, then $\{ \alpha_{h_1}, \alpha_{h_2}, \alpha_{h_3} \}$ blocks $M$ in $(N,V)$. It must be that $u_{\{ \alpha_{h_1}, \alpha_{h_2}, \alpha_{h_3} \}}(M)=1$. By Lemma~\ref{lem:threed_sr_as_000exists_stlemma}, it must be that $u_{\{ \alpha_{h_1}, \alpha_{h_2}\}}(M)\geq 1$ and also that $u_{\{ \alpha_{h_2}, \alpha_{h_3}\}}(M)\geq 1$. It follows that $u_{\alpha_{h_1}}(M) = u_{\alpha_{h_3}}(M) = 0$ and $u_{\alpha_{h_2}}(M) = 1$. In this case, $\{ \alpha_{h_1}, \alpha_{h_2}, \alpha_{h_3} \}$ blocks $M$ in $(N, V)$, which is a contradiction. It follows that $u_{\{ \alpha_{h_1}, \alpha_{h_2}, \alpha_{h_3} \}}(M) \geq 2$.
\end{proof}

\begin{lem}
\label{lem:threed_sr_as_some000exists_r3_lem}
If $|T(0)| > 0$ then $u_t(M)\geq 1$ for any $t \in T^*(2)$.
\end{lem}
\begin{proof}
Suppose $|T(0)| > 0$. Consider an arbitrary $\{ \alpha_{h_1}, \alpha_{h_2}, \alpha_{h_3} \} \in T^*(2)$ where $v_{\alpha_{h_1}}(\alpha_{h_2})=1$ and $v_{\alpha_{h_1}}(\alpha_{h_3})=v_{\alpha_{h_2}}(\alpha_{h_3})=0$. By Lemma~\ref{lem:threed_sr_as_000exists_stlemma}, it must be that $u_{\{ \alpha_{h_1}, \alpha_{h_2}\}}(M)\geq 1$ and hence $u_{\{ \alpha_{h_1}, \alpha_{h_2}, \alpha_{h_3} \}}(M)\geq 1$.
\end{proof}

We can now combine Lemmas~\ref{lem:threed_sr_as_some000exists_r1_lem}, \ref{lem:threed_sr_as_some000exists_r2_lem}, and~\ref{lem:threed_sr_as_some000exists_r3_lem} to prove the approximation ratio of Algorithm~\algorithmfont{findStableUW}.

\begin{lem}
\label{lem:threed_sr_as_some000exists_final_lem}
The approximation ratio of Algorithm~\algorithmfont{findStableUW} is $2$.
\end{lem}
\begin{proof}
If $|T(0)| = 0$ then by Lemma~\ref{lem:threed_sr_as_no000exists_lem} it must be that $2u(M) \geq u(M^*)$. It remains to consider the case in which $|T(0)| > 0$. In this case, 
\begin{align*}
    2u(M) &= 2 \sum_{\mathclap{t\in M^*}} u_{t}(M)\\[0.5em]
    &= 2 \left( \hspace*{0.3mm} \sum_{t \in T^*(6)} u_{t}(M) + \sum_{\mathclap{t \in T^*(4)}} u_{t}(M)  + 
    \sum_{\mathclap{t \in T^*(2)}} u_{t}(M) + \sum_{\mathclap{t \in T^*(0)}} u_{t}(M) \right) && \mbox{by the definition of $T^*$}\\[0.6em]
    &\geq 2 \sum_{\mathclap{t \in T^*(6)}} u_{t}(M) + 2\sum_{\mathclap{t \in T^*(4)}} u_{t}(M)  + 
    2 \sum_{\mathclap{t \in T^*(2)}} u_{t}(M)\\[0.2em]
    &\geq 6|T^*(6)| + 4|T^*(4)| + 2|T^*(2)| && \mbox{by Lemmas~\ref{lem:threed_sr_as_some000exists_r1_lem}--\ref{lem:threed_sr_as_some000exists_r3_lem}}\\
    &= u(M^*)\enspace.
\end{align*}
\end{proof}

\begin{thm}
\label{thm:threed_sr_as_approxratio}
There exists a polynomial-time $2$-approximation algorithm for 3DR-AS-SMUW.
\end{thm}
\begin{proof}
It is straightforward to show that Algorithm~\algorithmfont{findStableUW} returns a matching in polynomial time. In Lemma~\ref{lem:threed_sr_as_some000exists_final_lem} we show that the approximation ratio of this algorithm is~$2$.
\end{proof}

Recall that in Section~\ref{sec:threed_sr_as_symmetric_binary_finding_in_general}, we showed that a stable matching exists in an instance of 3DR-AS even if the number of agents is not divisible by three and the definition of a matching allows for at most two agents to be unmatched. We remark that it is straightforward to adapt the proof in this section to show that Algorithm~\algorithmfont{findStableUW} has the same approximation ratio even if the number of agents is not divisible by three and the definition of a matching allows for agents to be unmatched.

It is straightforward to show that the analysis of Algorithm~\algorithmfont{findStableUW} is tight, which we do as follows. Consider the instance of 3DR-AS shown in Figure~\ref{fig:threed_sr_as_max_welfare_example}, which has binary and symmetric preferences. Algorithm~\algorithmfont{findStableUW} is bound to return $M=\{ \{ \alpha_3, \alpha_5, \alpha_6 \} \}$ while $M^*=\{\{ \alpha_1, \alpha_2, \alpha_3 \},\allowbreak \{ \alpha_4, \alpha_5, \alpha_8 \}, \{ \alpha_6, \alpha_7, \alpha_9 \}\}$. Since $u(M)=6$ and $u(M^*)=12$ it follows that $u(M^*)=2u(M)$ and thus that our analysis of Algorithm~\algorithmfont{findStableUW} is tight. Interestingly, this particular instance also shows that any approximation algorithm with a better performance ratio than $2$ must not always begin, like Algorithm~\algorithmfont{findStableUW} does, by selecting a maximal set of triangles.
\begin{figure}
    \centering
    \input{figures/threed_sr_as/max_welfare_example.tikz}
    \vspace*{1mm}
    \caption[An instance of 3DR-AS-SMUW in which $u(M^*)=2u(M)$]{An instance of 3DR-AS-SMUW in which $u(M^*)=2u(M)$. The dashed enclosure depicts $M^*$.}
    \label{fig:threed_sr_as_max_welfare_example}
\end{figure}

\section{Binary preferences}
\label{sec:threed_sr_as_generalbinary}
In this section we show that deciding if a given instance of 3DR-AS contains a stable matching is $\NP$-complete, even when preferences are binary (and not necessarily symmetric). The reduction is from \emph{Partition Into Triangles} (PIT, Problem~\ref{prob:pit}).

The reduction, illustrated in Figure~\ref{fig:threed_sr_as_binary_reduction}, is as follows. Unless otherwise specified assume that $v_{\alpha_i}(\alpha_j)=0$ for any $\alpha_i, \alpha_j \in N$. For each $i$ where $1 \leq i \leq 3q$ construct three agents labelled $a_{2i}$, $a_{2i-1}$, and $b_i$. Let $v_{a_{2i}}(a_{2i-1})=v_{a_{2i}}(b_i)=1$, $v_{a_{2i-1}}(a_{2i})=v_{a_{2i-1}}(b_i)=1$, and $v_{b_i}(a_{2i})=v_{b_i}(a_{2i-1})=1$. For each $w_i, w_j \in W$ let  $v_{b_i}(b_j)=1$ if $\{ w_i, w_j \} \in E$ and $0$ otherwise. Next, for each $r$ where $1 \leq r \leq 6q$ construct a set of five agents $P_r = \{ p_r^1, p_r^2, \dots, p_r^5 \}$, which we refer to as the \emph{$r\textsuperscript{th}$ pentagadget}. To simplify the description of the valuations in each pentagadget, in this section we write $i \myoplus y$ to denote $((i + y - 1) \bmod 5) + 1$. For each $i$ where $1\leq i \leq 5$ let $v_{p_r^i}(p_r^{i \myoplus 1}) = v_{p_r^{i \myoplus 1}}(p_r^i) = 1$ and $v_{p_r^i}(p_r^{i \myoplus 2}) = 1$.
This completes the construction of $(N, V)$. Note that $|N|=39q$.

It is straightforward to show that this reduction can be performed in polynomial time. To prove that the reduction is correct we show that the 3DR-AS instance $(N, V)$ contains a stable matching if and only if the PIT instance $G$ contains a partition into triangles.
%  In Section~\ref{sec:threed_sr_as_binary_reduction_firstdirection} we consider the first direction and show that if a partition into triangles $X=\{X_1,X_2,\dots,X_q\}$ exists in $G$ then a stable matching $M$ exists in $(N, V)$. In Section~\ref{sec:threed_sr_as_binary_reduction_seconddirection} we consider the second direction and show that if a stable matching $M$ exists in $(N,V)$ then a partition into triangles $X = \{ X_1, X_2, \dots, X_q\}$ exists in $G$. 

\begin{figure}
  \centering
    \begin{tikzpicture}
\def\scalefactorp{2.1}
    % \draw[help lines] (0,0) grid (12,6);
    \begin{scope}[every node/.style={circle,thick,draw,minimum size=2.4mm}]
        \def\plabeldist{0.4cm}
        \begin{scope}[scale=1.0]
            % \node[draw=none, align=center] (Pc) at (3,0.0) {pentagadget $P_r$ for some $1\leq r \leq 6q$};
            % https://www.mathopenref.com/coordpolycalc.html
            \node[thick, circle, label={[label distance=0.4cm]90:$p_r^2$}] (pr2) at ({90:\scalefactorp}) {};
            \node[thick, circle, label={[label distance=0.4cm]162:$p_r^1$}] (pr1) at ({162:\scalefactorp}) {};
            \node[thick, circle, label={[label distance=0.4cm]234:$p_r^5$}] (pr5) at ({234:\scalefactorp}) {};
            \node[thick, circle, label={[label distance=0.4cm]306:$p_r^4$}] (pr4) at ({306:\scalefactorp}) {};
            \node[thick, circle, label={[label distance=0.4cm]378:$p_r^3$}] (pr3) at ({378:\scalefactorp}) {};
        \end{scope}
        
        \begin{scope}[shift={(5.0, 0.0)}]
            \node[thick, circle, label={[label distance=\plabeldist]90:$b_i$}] (bi) at (1.8, 0.0) {};
            \node[thick, circle, label={[align=right, label distance=0.4cm]180:$a_{2i}$}] (ai1) at (0.0,-1.5) {};
            \node[thick, circle, label={[align=right, label distance=0.6cm]180:$a_{2i-1}$}] (ai2) at (0.0,1.5) {};
            \node[draw=none] (bk1) at (4.0,-1.65) {};
            \node[draw=none] (bk2) at (4.0,-0.55) {};
            \node[draw=none] (bkdots) at (4.0,-1.1) {\hspace{2pt}$\dots$};
            \node[draw=none] (bj) at (4.0,0.55) {$b_k$};
            \node[draw=none] (bk) at (4.0,1.65) {$b_j$};
        \end{scope}
        
        % \node[draw=none, text width=5.5cm, align=center] (aic) at (10.2,0.0) {for each $1\leq i \leq 3q$ where $N(w_i)=\{ w_j, w_k, \dots \}$};
    \end{scope}
    \begin{scope}
        \foreach \from/\to in {pr1/pr2, pr2/pr3, pr3/pr4, pr4/pr5, pr5/pr1}
            \draw [thick, darrow] (\from) -- (\to);
        % \foreach \from/\to in {pr2/pr1, pr3/pr2, pr4/pr3, pr5/pr4, pr1/pr5}
        %     \path [thick, -<--] (\from) -- (\to);
        \foreach \from/\to in {pr1/pr3, pr3/pr5, pr5/pr2, pr2/pr4, pr4/pr1}
            \draw [thick, ->-] (\from) -- (\to);
        % TRIANGLE
         \foreach \from/\to in {bi/bj, bi/bk, bi/bk1, bi/bk2}
            \draw [thick, farrow] (\from) -- (\to);
        \foreach \from/\to in {bi/ai1, ai1/ai2, ai2/bi}
            \draw [thick, -->-] (\from) -- (\to);
        \foreach \from/\to in {ai1/bi, ai2/ai1, bi/ai2}
            \path [thick, -<--] (\from) -- (\to);
    \end{scope}
    \end{tikzpicture}
    \caption[The reduction from PIT to the problem of deciding if an instance of 3DR-AS with binary preferences contains a stable matching]{The reduction from PIT to the problem of deciding if an instance of 3DR-AS with binary preferences contains a stable matching. Each vertex represents an agent. An arc is present from agent $\alpha_i$ to agent $\alpha_j$ if $v_{\alpha_i}(\alpha_j) = 1$. Depicted is some pentagadget $P_r$ and some agents $b_i$, $a_{2i}$, and $a_{2i - 1}$ where $1\leq i \leq 3q$ and $N(w_i) = \{ w_j, w_k, \dots \}$.}
    \label{fig:threed_sr_as_binary_reduction}
\end{figure}

% \subsection{Correctness of the reduction: first direction}
% \label{sec:threed_sr_as_binary_reduction_firstdirection}

We first show that if the PIT instance $G$ contains a partition into triangles then the 3DR-AS instance $(N, V)$ contains a stable matching.

\begin{lem}
\label{lem:threed_sr_as_binary_reduction_firstdirection}
If $G$ contains a partition into triangles then $(N, V)$ contains a stable matching.
\end{lem}
\begin{proof}
Suppose $G$ contains a partition into triangles $X = \{ X_1, X_2, \dots, X_q \}$. We shall construct a matching $M$ that is stable in $(N, V)$. For each triangle $X_p=\{ w_i, w_j, w_k \}\in W$, add $\{b_i, b_j, b_k\}$ to $M$. For each $r$ where $1 \leq r \leq 6q$, add $\{ p_r^1, p_r^2, p_r^3 \}$ to $M$. This leaves agents $a_{2i}$ and $a_{2i-1}$ for each $1 \leq i \leq 3q$ and agents $p_r^4$ and $p_r^5$ for each $0 \leq r \leq 6q$. For each $1 \leq i \leq 3q$, add to $M$ the triples $\{a_{2i}, p_{2i}^4, p_{2i}^5\}, \{a_{2i-1}, p_{2i-1}^4, p_{2i-1}^5\}$. 

Since $u_{b_i}(M)=2$ for each $1\leq i\leq 3q$ it follows that $b_i$ does not belong to a triple that blocks $M$.

Suppose for a contradiction that some agent $a_{2i}$ where $1\leq i\leq 3q$ belongs to a triple $t$ that blocks $M$. We have shown that $b_i$ does not belong to a triple that blocks $M$, so it must be that $a_{2i-1}\in t$, otherwise $u_{a_{2i}}(t)=0$, which is impossible. Suppose then that $t=\{ a_{2i}, a_{2i-1}, \alpha_j \}$ where $\alpha_j \in N$ and $\alpha_j \neq b_i$. Considering the design of the instance, for any such $\alpha_j$ it must be that $u_{\alpha_j}(\{ a_{2i}, a_{2i-1} \})=u_{\alpha_j}(t)=0$, which is a contradiction. A symmetric argument shows that no $a_{2i-1}$ where $1\leq i \leq 3q$ belongs to a triple that blocks $M$.

The remaining possibility is that some triple $\{ p_r^{s_1}, p_r^{s_2}, p_r^{s_3} \}$ blocks $M$ where $1\leq r \leq 6q$. By the construction of $M$, $u_{p_r^1}(M) = u_{p_r^2}(M) = 2$, so neither $p_r^1$ nor $p_r^2$ blocks $M$. It follows that $\{ s_1, s_2, s_3 \} = \{ 3, 4, 5 \}$, which is a contradiction since $u_{p_r^5}(M) = 1 = u_{p_r^5}(\{ p_r^3, p_r^4 \})$.
\end{proof}

% \subsection{Correctness of the reduction: second direction}
% \label{sec:threed_sr_as_binary_reduction_seconddirection}

We now show, using a sequence of lemmas, that if the 3DR-AS instance $(N, V)$ contains a stable matching then $G$ contains a partition into triangles. 

We also introduce some new notation. For any set $S \subseteq N$ let $\sigma(S, N)$ be the number of triples in $N$ that each contain at least one agent in $S$.

\begin{lem}
\label{lem:threed_sr_as_pentagadgetagentsbelongtotwoagents}
If $(N, V)$ contains a stable matching $M$ then $\sigma(P_r, M) = 2$ for any $r$ where $1 \leq r \leq 6q$.
\end{lem}
\begin{proof}
By definition, $2 \leq \sigma(P_r, M) \leq 5$. Suppose for a contradiction that $\sigma(P_r) \geq 4$. It must be that at least three triples in $M$ contain exactly one agent in $P_r$. Label these three triples as $M(p_r^{s_1})$, $M(p_r^{s_2})$, and $M(p_r^{s_3})$. It follows that $u_{p_r^{s_1}}(M) = u_{p_r^{s_2}}(M) = u_{p_r^{s_3}}(M) = 0$. By the design of the reduction it must be that $u_{p_r^{s_1}}(\{p_r^{s_2}, p_r^{s_3}\}) \geq 1$, $u_{p_r^{s_2}}(\{p_r^{s_1}, p_r^{s_3}\}) \geq 1$, and $u_{p_r^{s_3}}(\{p_r^{s_1}, p_r^{s_2}\}) \geq 1$. Now $\{ p_r^{s_1}, p_r^{s_2},p_r^{s_3} \}$ blocks $M$, which is a contradiction.

Suppose then, for a contradiction, that $\sigma(P_r, M) = 3$. There are two possibilities. In the first, two triples in $M$ each contain exactly two agents in $P_r$ and one triple in $M$ contains exactly one agent in $P_r$. In the second, two triples in $M$ each contain exactly one agent in $P_r$ and one triple in $M$ contains exactly three agents in $P_r$.

Suppose firstly that two triples in $M$ each contain exactly two agents in $P_r$ and one triple in $M$ contains exactly one agent in $P_r$. Assume without loss of generality that $p_r^1$ is the agent in the latter triple. It follows that $u_{p_r^1}(M) = 0$. By assumption, $M(p_r^4)$ and $M(p_r^5)$ each contain exactly two agents in $P_r$ so it follows that $u_{p_r^4}(M) \leq 1$ and $u_{p_r^5}(M) \leq 1$. Now $\{ p_r^1, p_r^4, p_r^5 \}$ blocks $M$ since $u_{p_r^4}(\{p_r^5, p_r^1\}) = u_{p_r^5}(\{p_r^1, p_r^4\}) = 2$ and $u_{p_r^1}(\{p_r^4, p_r^5\}) = 1$, which is a contradiction.

It remains that two triples in $M$ each contain exactly one agent in $P_r$ and one triple in $M$ contains exactly three agents in $P_r$. Suppose $p_r^{s_1}$ and $p_r^{s_2}$ are the two agents in the former two triples. Excluding symmetries, there are two possible cases. In the first case, $s_1 = 1$ and $s_2 = 2$. It follows that $\{ p_r^3, p_r^4, p_r^5 \} \in M$. Now $\{p_r^5, p_r^1, p_r^2 \}$ blocks $M$ since $u_{p_r^5}(\{p_r^1, p_r^2\}) = u_{p_r^1}(\{p_r^2, p_r^5\}) = 2$ and $u_{p_r^2}(\{p_r^1, p_r^5\}) = 1$, which is a contradiction. In the second case, $s_1 = 1$ and $s_2 = 3$. It follows that $\{ p_r^2, p_r^4, p_r^5 \} \in M$. Now $\{p_r^1, p_r^2, p_r^3 \}$ blocks $M$ since $u_{p_r^1}(\{p_r^2, p_r^3\}) = u_{p_r^2}(\{p_r^1, p_r^3\}) = 2$ and $u_{p_r^3}(\{p_r^1, p_r^2\}) = 1$, which is also a contradiction.
\end{proof}

\begin{lem}
\label{lem:threed_sr_as_allairscores0}
If $(N, V)$ contains a stable matching $M$ then $u_{a_k}(M)=0$ for each $1\leq k\leq 6q$.
\end{lem}
\begin{proof}
Suppose $M$ is a stable matching in $(N, V)$. Consider an arbitrary pentagadget index $r_1$ where $1 \leq {r_1} \leq 6q$. By Lemma~\ref{lem:threed_sr_as_pentagadgetagentsbelongtotwoagents}, it must be that $\sigma(P_{r_1}, M) = 2$. It follows that one triple in $M$ contains exactly three agents in $P_{r_1}$ and another triple in $M$ contains exactly two agents in $P_r$ as well as some third agent $\alpha_h$. It follows that $u_{\alpha_h}(M) = 0$. We now show that $\alpha_h = a_{k}$ where $1\leq k \leq 6q$.

By the design of the reduction, it must be that either $\alpha_h \in P_{r_2}$ where $1\leq r_2\leq 6q$, $\alpha_h = b_j$ where $1\leq j\leq 3q$, or $\alpha_h = a_k$ where $1\leq k\leq 6q$. 

Suppose firstly that ${\alpha_h} \in P_r$ where $1\leq r_2\leq 6q$. Label $\alpha_h = p_{r_2}^{s}$ where $1\leq s\leq 5$. By the definition of $\alpha_h = p_{r_2}^{s}$, it must be that ${r_1} \neq {r_2}$. Since $M(p_{r_2}^s)$ contains $p_{r_2}^s$ and two agents in $P_{r_1}$, by Lemma~\ref{lem:threed_sr_as_pentagadgetagentsbelongtotwoagents} the four agents in $P_{r_2} \setminus \{ p_{r_2}^s \}$ must belong to exactly one triple in $M$, which is clearly a contradiction.

Suppose then that ${\alpha_h} = b_j$ where $1 \leq j \leq 3q$. Consider $a_{2j}$ and $a_{2j-1}$. Since $a_{2j} \notin M(b_j)$ and $a_{2j-1} \notin M(b_j)$ it must be that $u_{a_{2j}}(M) \leq 1$ and $u_{a_{2j-1}}(M) \leq 1$. Since $u_{\alpha_h}(M) = u_{b_j}(M) = 0$ it follows that $\{ b_j, a_{2j}, a_{2j-1} \}$ blocks $M$, since $u_{b_j}(\{ a_{2j}, a_{2j-1} \}) = u_{a_{2j}}(\{b_j, a_{2j-1}\}) = u_{a_{2j-1}}(\{b_j, a_{2j}\}) = 2$, which is a contradiction.

It remains that ${\alpha_h} = a_k$ where $1 \leq k \leq 6q$. Since the choice of ${r_1}$ where $1\leq r_1 \leq 6q$ was arbitrary, there are exactly $6q$ choices of $\alpha_h$. It follows that $u_{a_k}(M) = 0$ for every $1 \leq k \leq 6q$.
\end{proof}

\begin{lem}
\label{lem:threed_sr_as_allbiscores2}
If $(N, V)$ contains a stable matching $M$ then $u_{b_i}(M)=2$ for any $i$ where $1 \leq i \leq 3q$.
\end{lem}
\begin{proof}
Suppose for a contradiction that there exists some $1 \leq i \leq 3q$ where $u_{b_i}(M) < 2$. Lemma~\ref{lem:threed_sr_as_allairscores0} shows that $u_{2i}(M) = u_{a_{2i-1}}(M) = 0$. Considering the valuation functions of $a_{2i}$, $a_{2i-1}$, and $b_i$, we can see that $u_{b_i}(\{ a_{2i}, a_{2i-1} \}) = u_{a_{2i}}(\{ b_i, a_{2i-1} \}) = u_{a_{2i-1}}(\{ b_i, a_{2i} \}) = 2$. Now $\{ b_i, a_{2i}, a_{2i-1} \}$ blocks $M$, which is a contradiction.
\end{proof}

\begin{lem}
\label{lem:threed_sr_as_allbiintriplestogether}
If $(N, V)$ contains a stable matching $M$ then for any $b_i$ where $1 \leq i \leq 3q$, the triple $M(b_i)$ comprises $\{b_i, b_j, b_k\}$ where $1 \leq j,k \leq 3q$ and $\{w_i, w_j\}, \{w_i, w_k\} \in E$.
\end{lem}
\begin{proof}
Lemma~\ref{lem:threed_sr_as_allbiscores2} shows that $u_{b_i}(M)=2$. Suppose $M(b_i)=\{ b_i, \alpha_k, \alpha_l \}$ for some $\alpha_k, \alpha_l\in N$. Since $u_{b_i}(M)=2$, it must be that $v_{b_i}(\alpha_k)=1$ and hence either $\alpha_k = a_{2i}$, $\alpha_k = a_{2i-1}$, or $\alpha_k = b_j$ where $1\leq j \leq 3q$ where $\{w_i, w_j\}\in E$. Suppose first that $\alpha_k = a_{2i}$. Since $b_i \in M(a_{2i})$ it follows that $u_{a_{2i}}(M) \geq 1$ which contradicts Lemma~\ref{lem:threed_sr_as_allairscores0}.  A similar argument shows that $\alpha_k \neq a_{2i-1}$. It remains that $\alpha_k = b_j$ where $1\leq j \leq 3q$ such that $\{ w_i, w_j \} \in E$. The same argument shows that $\alpha_l = b_k$ where $1\leq k \leq 3q$ where $\{ w_i, w_k \} \in E$. We have shown that $M(b_i) = \{ b_i, b_j, b_k \}$ for some $j, k$ where $1 \leq j,k \leq 3q$ and $\{ w_i, w_j \}, \{ w_i, w_k \} \in E$.
\end{proof}

\begin{lem}
\label{lem:threed_sr_as_binary_reduction_second_direction_conclusion}
If $(N, V)$ contains a stable matching then $G$ contains a partition into triangles.
\end{lem}
\begin{proof}
Lemma~\ref{lem:threed_sr_as_allbiintriplestogether} shows that for an arbitrary $b_i$ where $1 \leq i \leq 3q$, $M(b_i)$ comprises $\{b_i, b_j, b_k\}$ where $1 \leq j,k \leq 3q$, $\{w_i, w_j\}\in E$, and $\{w_i, w_k\}\in E$. It follows that there are exactly $q$ triples in $M$ each containing three agents $\{b_i, b_j, b_k\}$, where the three corresponding vertices $w_i, w_j, w_k$ are pairwise adjacent in $G$. From these triples of pairwise adjacent vertices, a partition into triangles $X$ can be easily constructed.
\end{proof}

% \subsection{Conclusion}

We have now shown that the 3DR-AS instance $(N, V)$ contains a stable matching if and only if the PIT instance $G$ contains a partition into triangles. This shows that the reduction is correct.

\begin{thm}
\label{thm:threed_sr_as_binary_reduction}
Deciding if a given instance of 3DR-AS contains a stable matching is $\NP$-complete, even when preferences are binary.
\end{thm}
\begin{proof}
It is straightforward to show that this decision problem belongs to $\NP$. We have presented a polynomial-time reduction from Partition Into Triangles (PIT, Problem~\ref{prob:pit}), which is $\NP$-complete \cite{GJ79}. Given an arbitrary instance $G$ of PIT, the reduction constructs an instance $(N, V)$ of 3DR-AS with binary preferences. Lemmas~\ref{lem:threed_sr_as_binary_reduction_firstdirection} and~\ref{lem:threed_sr_as_binary_reduction_second_direction_conclusion} show that $(N, V)$ contains a stable matching if and only if $G$ contains a partition into triangles and thus that this decision problem is $\NP$-hard.
% to the problem of deciding if a given instance of 3DR-AS contains a stable matching. If a partition into triangles exists in the PIT instance $G=(W, E)$ then a stable matching $M$ exists in $(N, V)$ where $|M|=|N|/3$ (Lemma). If a stable matching $M$ exists in $(N, V)$ where $|M|=|N|/3$ then a partition into triangles exists in $G$ (Lemma).
\end{proof}

\section{Symmetric ternary preferences}
\label{sec:threed_sr_as_symmetricternary}
We saw in Section~\ref{sec:threed_sr_as_generalbinary} that an instance $(N, V)$ of 3DR-AS may not contain a stable matching, and the associated decision problem is $\NP$-complete, even when valuations are binary. It follows that the decision problem for ternary valuations, i.e.\ $v_{\alpha_i}(\alpha_j) \in \{ 0, 1, 2 \}$ for each $\alpha_i, \alpha_j \in N$ is also $\NP$-complete. In contrast we saw in Section~\ref{sec:threed_sr_as_symmetricbinary} that when valuations are binary and symmetric a stable matching always exists and can be found in polynomial time. It is natural to ask if this polynomial-time solvability also holds in the more general case of ternary and symmetric valuations. We answer this question in the negative (assuming $\P=\NP$), and show that deciding if a given instance of 3DR-AS contains a stable matching is $\NP$-complete, even when valuations are ternary and symmetric. We remark that a result of Deineko and Woeginger \cite{DEINEKO20131837} for \emph{Geometric 3D-SR} (see Chapter~\ref{c:lit_review}) implies a weaker version of our result, namely that if valuations are symmetric (but not necessarily ternary) then deciding if a given instance of 3DR-AS contains a stable matching is $\NP$-complete, even when valuations are ternary and symmetric.

% We define 3DR as the restriction of 3D-SR-AS in which valuations are ternary and symmetric. Like 3D-SR-AS, the decision version of 3D-SR-SAS-TER belongs to the class $\NP$. 

We present a polynomial-time reduction from Partition into Triangles (PIT, Problem~\ref{prob:pit}), which is similar to the reduction that we presented in Section~\ref{sec:threed_sr_as_generalbinary} for the analogous decision problem involving preferences that are binary but not necessarily symmetric. The main difference is in the design of the gadgets. Instead of ``pentagadgets'' we introduce a number of ``octogadgets''. Nevertheless, the purpose of the octogadgets is the same as the pentagadgets in the previous reduction and the proof follows the same structure as before.

The reduction, illustrated in Figure~\ref{fig:threed_sr_as_ternary_symmetric_reduction}, is as follows. Since valuations are symmetric in $(N, V)$, we shall usually specify valuations in one direction only. For example, instead of writing ``let $v_{\alpha_i}(\alpha_j)=v_{\alpha_j}(\alpha_i)=1$'' we write ``let $v_{\alpha_i}(\alpha_j)=1$''. Unless otherwise specified assume that $v_{\alpha_i}(\alpha_j)=0$ for any $\alpha_i, \alpha_j \in N$. To simplify the description of the valuations in the reduction, in this section we write $i \myoplus y$ to denote $((i + y - 1) \bmod 8) + 1$.

For each $i$ where $1\leq i \leq 3q$ construct three agents labelled $a_{2i}$, $a_{2i-1}$, and $b_i$. Let $v_{a_{2i}}(a_{2i-1}) = v_{b_i}(a_{2i}) = v_{b_i}(a_{2i-1}) = 1$. For each $w_i, w_j \in W$ let $v_{b_i}(b_j)=1$ if $\{ w_i, w_j \} \in E$ and $0$ otherwise. Next, for each $r$ where $1 \leq r \leq 6q$ construct a set of eight agents $H_r = \{ h_r^1, h_r^2, \dots, h_r^8 \}$, which we refer to as the \emph{$h\textsuperscript{th}$ octogadget}. For each $i$ and $j$ where $1\leq i, j \leq 8$ let $v_{h_r^i}(h_r^j) = 2$ if both $i$ is odd and $j = i \myoplus 1$ otherwise $1$.
This completes the construction of $(N, V)$. Note that $|N| = 57q$.

\begin{figure}
    \centering
    \begin{tikzpicture}
% \draw[help lines] (0,0) grid (12,6);
\begin{scope}[every node/.style={circle,thick,draw,minimum size=2.4mm}, scale=1.0]
    % \node[draw=none, align=center] (Pc) at (2.4,-0.5) {octogadget $H_r$ for some $1\leq r \leq 6q$};
    
    \def\hscale{0.5}
    
    % this one the angles are the same
    % \begin{scope}[shift={(3.0, 3.0)}]
    % \node[draw=none] (hr1i) at (\hscale*0 ,\hscale*-5) {};
    % \node[draw=none] (hr2i) at (\hscale*-3,\hscale*-4) {};
    % \node[draw=none] (hr3i) at (\hscale*-5,\hscale*-2) {};
    % \node[draw=none] (hr4i) at (\hscale*-5,\hscale*1) {};
    % \node[draw=none] (hr5i) at (\hscale*-4,\hscale*3) {};
    % \node[draw=none] (hr6i) at (\hscale*-1,\hscale*5) {};
    % \node[draw=none] (hr7i) at (\hscale*1,\hscale*5) {};
    % \node[draw=none] (hr8i) at (\hscale*4,\hscale*3) {};
    % \node[draw=none] (hr9i) at (\hscale*5,\hscale*1) {};
    % \node[draw=none] (hr10i) at (\hscale*5,\hscale*-2) {};
    % \node[draw=none] (hr11i) at (\hscale*3,\hscale*-4) {};
    % \end{scope}
    
    % this one is a 12-gon with the two at the bottom merged
    \begin{scope}[shift={(2.8, 3.0)}]
    % http://www.rotaryspin.com/markb/courses/projects/polygon.html
    \node[draw=none] (hr3i) at (\hscale*4.62,\hscale*1.91) {};
    \node[draw=none] (hr2i) at (\hscale*1.91,\hscale*4.62) {};
    \node[draw=none] (hr1i) at (\hscale*-1.91,\hscale*4.62) {};
    \node[draw=none] (hr8i) at (\hscale*-4.62,\hscale*1.91) {};
    \node[draw=none] (hr7i) at (\hscale*-4.62,\hscale*-1.91) {};
    \node[draw=none] (hr6i) at (\hscale*-1.91,\hscale*-4.62) {};
    \node[draw=none] (hr5i) at (\hscale*1.91,\hscale*-4.62) {};
    \node[draw=none] (hr4i) at (\hscale*4.62,\hscale*-1.91) {};
    \end{scope}
    
    \def\hlabeldist{0.4cm}
    
    \node[label={[label distance=\hlabeldist]90:$b_i$}] (bi) at (10,3) {};
    \node[label={[label distance=\hlabeldist]180:$a_{2i}$}] (ai1) at (8.2,1.5) {};
    \node[label={[label distance=\hlabeldist+0.19cm]180:$a_{2i-1}$}] (ai2) at (8.2,4.5) {};
    \node[draw=none] (bk1) at (12.2,1.35) {};
    \node[draw=none] (bk2) at (12.2,2.45) {};
    \node[draw=none] (bkdots) at (12.5,1.9) {\hspace{2pt}$\dots$};
    \node[draw=none] (bj) at (12.2,3.55) {$b_k$};
    \node[draw=none] (bk) at (12.2,4.65) {$b_j$};
    % \node[draw=none, text width=6cm, align=center] (aic) at (10.2,0.-0.4) {for each $1\leq i\leq 3q$ where $N(w_i)=\{ w_j, w_k, \dots \}$};
\end{scope}
\begin{scope}
    \foreach \from in {hr1i, hr2i, hr3i, hr4i, hr5i, hr6i, hr7i, hr8i}{
        \foreach \to in {hr1i, hr2i, hr3i, hr4i, hr5i, hr6i, hr7i, hr8i}
            \draw[line width=0.05mm,color=black!39] (\from.center) -- (\to.center);
    }
    
    % \fill [black!10] (hr1i.center) -- (hr2i.center) -- (hr3i.center) -- (hr4i.center) -- (hr5i.center) -- (hr6i.center) -- (hr7i.center) -- (hr8i.center) -- (hr9i.center) -- (hr10i.center) -- (hr11i.center) -- cycle;
    
    \foreach \from/\to in {hr1i/hr2i, hr3i/hr4i, hr5i/hr6i, hr7i/hr8i}
        \draw [black, double=white, line width = 1pt, double distance = 3pt ] (\from) -- (\to);
    
    
    % \foreach \from/\to in {pr2/pr1, pr3/pr2, pr4/pr3, pr5/pr4, pr1/pr5}
    %     \path [thick, -<--] (\from) -- (\to);
    % \foreach \from/\to in {pr1/pr3, pr3/pr5, pr5/pr2, pr2/pr4, pr4/pr1}
        % \draw [thick] (\from) -- (\to);
    % TRIANGLE
     \foreach \from/\to in {bi/bj, bi/bk, bi/bk1, bi/bk2}
        \draw [thick] (\from) -- (\to);
    \foreach \from/\to in {bi/ai1, ai1/ai2, ai2/bi}
        \draw [thick] (\from) -- (\to);
    \foreach \from/\to in {ai1/bi, ai2/ai1, bi/ai2}
        \path [thick] (\from) -- (\to);
\end{scope}
\begin{scope}[every node/.style={circle,thick,draw,minimum size=2.4mm, fill=white}, scale=1.0]
    \def\hlabeldist{0.4cm}
    
    \node[thick, circle, label={[label distance=\hlabeldist]90:$h_r^1$}] (hr1) at (hr1i) {};
    \node[thick, circle, label={[label distance=\hlabeldist]90:$h_r^2$}] (hr2) at (hr2i) {};
    \node[thick, circle, label={[label distance=\hlabeldist]0:$h_r^3$}] (hr3) at (hr3i) {};
    \node[thick, circle, label={[label distance=\hlabeldist]0:$h_r^4$}] (hr4) at (hr4i) {};
    \node[thick, circle, label={[label distance=\hlabeldist]270:$h_r^5$}] (hr5) at (hr5i) {};
    \node[thick, circle, label={[label distance=\hlabeldist]270:$h_r^6$}] (hr6) at (hr6i) {};
    \node[thick, circle, label={[label distance=\hlabeldist]180:$h_r^7$}] (hr7) at (hr7i) {};
    \node[thick, circle, label={[label distance=\hlabeldist]180:$h_r^8$}] (hr8) at (hr8i) {};
\end{scope}
\end{tikzpicture}

    
    
    
    
    
    
    
    

%%%%%%%%%%%%%%%%%%%%%%%%%%%%%%%%%%%%%%%%%%%%%%%%%%%%%%%%%%%%%%%%%%%%%%
%%%%%%%%%%%%%%%%%%%%%%%%%%%%%%%%%%%%%%%%%%%%%%%%%%%%%%%%%%%%%%%%%%%%%%
%%%%%%%%%%%%%%%%%%%%%%%%%%%%%%%%%%%%%%%%%%%%%%%%%%%%%%%%%%%%%%%%%%%%%%


% OLD VERSION with labels inside nodes


% \begin{tikzpicture}
% % \draw[help lines] (0,0) grid (12,6);
% \begin{scope}[every node/.style={circle,thick,draw,inner sep=0.8mm,minimum size=1.6mm}, scale=1.0]
%     % \node (A) at (6,0) {$.$};
%     % \node (B) at (6,6) {$.$};
%     \node[draw=none, align=center] (Pc) at (3,-0.6) {hendecagadget $H_r$ for some $1\leq r \leq 6q$};
    
%     \def\hscale{0.55}
    
%     % this one the angles are the same
%     % \begin{scope}[shift={(3.0, 3.0)}]
%     % % https://www.mathopenref.com/coordpolycalc.html
%     % \node[draw=none] (hr1i) at (\hscale*0 ,\hscale*-5) {};
%     % \node[draw=none] (hr2i) at (\hscale*-3,\hscale*-4) {};
%     % \node[draw=none] (hr3i) at (\hscale*-5,\hscale*-2) {};
%     % \node[draw=none] (hr4i) at (\hscale*-5,\hscale*1) {};
%     % \node[draw=none] (hr5i) at (\hscale*-4,\hscale*3) {};
%     % \node[draw=none] (hr6i) at (\hscale*-1,\hscale*5) {};
%     % \node[draw=none] (hr7i) at (\hscale*1,\hscale*5) {};
%     % \node[draw=none] (hr8i) at (\hscale*4,\hscale*3) {};
%     % \node[draw=none] (hr9i) at (\hscale*5,\hscale*1) {};
%     % \node[draw=none] (hr10i) at (\hscale*5,\hscale*-2) {};
%     % \node[draw=none] (hr11i) at (\hscale*3,\hscale*-4) {};
%     % \end{scope}
    
%     % this one is a 12-gon with the two at the bottom merged
%     \begin{scope}[shift={(3.0, 3.0)}]
%     % http://www.rotaryspin.com/markb/courses/projects/polygon.html
%     \node[draw=none] (hr9i) at (\hscale*5.0 ,\hscale*0.0) {};
%     % \node[draw=none] (hr2i) at (\hscale*-1,\hscale*-5) {};
%     \node[draw=none] (hr10i) at (\hscale*4.3,\hscale*2.5) {};
%     \node[draw=none] (hr11i) at (\hscale*2.5,\hscale*4.3) {};
%     \node[draw=none] (hr1i) at (\hscale*0.0,\hscale*5.0) {};
%     \node[draw=none] (hr2i) at (\hscale*-2.5,\hscale*4.3) {};
%     \node[draw=none] (hr3i) at (\hscale*-4.3,\hscale*2.5) {};
%     \node[draw=none] (hr4i) at (\hscale*-5.0,\hscale*0.0) {};
%     \node[draw=none] (hr5i) at (\hscale*-4.3,\hscale*-2.5) {};
%     \node[draw=none] (hr6i) at (\hscale*-1.6,\hscale*-4.3) {};
%     % \node[draw=none] (hr7i) at (\hscale*0.0,\hscale*-5.0) {};
%     \node[draw=none] (hr7i) at (\hscale*1.6,\hscale*-4.3) {};
%     \node[draw=none] (hr8i) at (\hscale*4.3,\hscale*-2.5) {};
%     \end{scope}
    
    

%     \node (bi) at (10,3) {$b_i$};
%     \node[minimum size=10mm] (ai1) at (8.2,1.5) {$a_{2i}$};
%     \node[minimum size=10mm] (ai2) at (8.2,4.5) {$a_{2i-1}$};
%     \node[draw=none] (bk1) at (12.2,1.35) {};
%     \node[draw=none] (bk2) at (12.2,2.45) {};
%     \node[draw=none] (bkdots) at (12.5,1.9) {\dots};
%     \node[draw=none, label={[shift={(0.2, -0.5)}]$b_k$}] (bj) at (12.2,3.55) {};
%     \node[draw=none, label={[shift={(0.2, -0.5)}]$b_j$}] (bk) at (12.2,4.65) {};
%     \node[draw=none, text width=6cm, align=center] (aic) at (10.2,0.-0.2) {for each $1\leq i\leq 3q$ where $N(w_i)=\{ w_j, w_k, \dots \}$};
% \end{scope}
% \begin{scope}
%     % \foreach \from in {hr1i, hr2i, hr3i, hr4i, hr5i, hr6i, hr7i, hr8i, hr9i, hr10i, hr11i}{
%     %     \foreach \to in {hr1i, hr2i, hr3i, hr4i, hr5i, hr6i, hr7i, hr8i, hr9i, hr10i, hr11i}
%     %         \draw[line width=0.05mm] (\from) -- (\to);
%     % }
    
%     \fill [black!10] (hr1i.center) -- (hr2i.center) -- (hr3i.center) -- (hr4i.center) -- (hr5i.center) -- (hr6i.center) -- (hr7i.center) -- (hr8i.center) -- (hr9i.center) -- (hr10i.center) -- (hr11i.center) -- cycle;
    
%     \foreach \from/\to in {hr2i/hr3i, hr4i/hr5i, hr6i/hr7i, hr8i/hr9i, hr10i/hr11i}
%         \draw [black, double=white, line width = 1pt, double distance = 3pt ] (\from) -- (\to);
    
    
%     % \foreach \from/\to in {pr2/pr1, pr3/pr2, pr4/pr3, pr5/pr4, pr1/pr5}
%     %     \path [thick, -<--] (\from) -- (\to);
%     % \foreach \from/\to in {pr1/pr3, pr3/pr5, pr5/pr2, pr2/pr4, pr4/pr1}
%         % \draw [thick] (\from) -- (\to);
%     % TRIANGLE
%      \foreach \from/\to in {bi/bj, bi/bk, bi/bk1, bi/bk2}
%         \draw [thick] (\from) -- (\to);
%     \foreach \from/\to in {bi/ai1, ai1/ai2, ai2/bi}
%         \draw [thick] (\from) -- (\to);
%     \foreach \from/\to in {ai1/bi, ai2/ai1, bi/ai2}
%         \path [thick] (\from) -- (\to);
% \end{scope}
% \begin{scope}[every node/.style={circle,thick,draw,inner sep=0.8mm,minimum size=1.6mm, fill=white}, scale=1.0]

%     \node (hr1) at (hr1i) {$h_r^{1}$};
%     \node (hr2) at (hr2i) {$h_r^{2}$};
%     \node (hr3) at (hr3i) {$h_r^{3}$};
%     \node (hr4) at (hr4i) {$h_r^{4}$};
%     \node (hr5) at (hr5i) {$h_r^{5}$};
%     \node (hr6) at (hr6i) {$h_r^{6}$};
%     \node (hr7) at (hr7i) {$h_r^{7}$};
%     \node (hr8) at (hr8i) {$h_r^{8}$};
%     \node (hr9) at (hr9i) {$h_r^{9}$};
%     \node (hr10) at (hr10i) {$h_r^{10}$};
%     \node (hr11) at (hr11i) {$h_r^{11}$};
    
% \end{scope}
% \end{tikzpicture}
    \vspace*{1mm}
    \caption[The reduction from PIT to the problem of deciding if an instance of 3DR-AS with ternary preferences contains a stable matching]{The reduction from PIT to the problem of deciding if an instance of 3DR-AS with ternary preferences contains a stable matching. Each vertex represents an agent. A single edge is present from agent $\alpha_i$ to agent $\alpha_j$ if $v_{\alpha_i}(\alpha_j) = 1$. A double edge is present from $\alpha_i$ to $\alpha_j$ if $v_{\alpha_i}(\alpha_j) = 2$. Depicted is some octogadget $H_r$ and some agents $b_i$, $a_{2i}$, and $a_{2i - 1}$ where $1\leq i \leq 6q$ and $N(w_i) = \{ w_j, w_k, \dots \}$.}
    \label{fig:threed_sr_as_ternary_symmetric_reduction}
\end{figure}

It is straightforward to show that the reduction runs in polynomial time. To prove that the reduction is correct we show that the 3DR-AS instance $(N, V)$ contains a stable matching if and only if the PIT instance $G$ contains a partition into triangles.

We first show that if the PIT instance $G$ contains a partition into triangles then the 3DR-AS instance $(N, V)$ contains a stable matching.

\begin{lem}
\label{lem:threed_sr_as_ternary_symmetric_reduction_firstdirection}
If $G$ contains a partition into triangles then $(N, V)$ contains a stable matching.
\end{lem}
\begin{proof}
Suppose $G$ contains a partition into triangles $X = \{ X_1, X_2, \dots, X_q \}$. We shall construct a matching $M$ that is stable in $(N, V)$. For each triangle $X_p=\{ w_i, w_j, w_k \}\in W$, add $\{ b_i, b_j, b_k \}$ to $M$. For each index $r$ where $1 \leq r \leq 6q$ add the triples $\{ h_r^1, h_r^2, h_r^3 \}$, $\{ h_r^4, h_r^5, p_r^6 \}$, $\{ h_r^7, h_r^8, a_r \}$ to $M$. Note that $u_{h_r^p}(M)\geq 2$ for each $1\leq p \leq 8$ by the design of the octogadget $H_r$.

Since $u_{b_i}(M)=2$ for each $1\leq i\leq 3q$ it follows that $b_i$ does not belong to a triple that blocks $M$.

Suppose for a contradiction that some agent $a_{2i}$ where $1\leq i\leq 3q$ belongs to a triple $t$ that blocks $M$. We have shown that $b_i$ does not belong to a triple that blocks $M$, so it must be that $a_{2i-1}\in t$, otherwise $u_{a_{2i}}(t)=0$, which is impossible. Suppose then that $t=\{ a_{2i}, a_{2i-1}, \alpha_j \}$ where $\alpha_j \in N$ and $\alpha_j \neq b_i$. Considering the design of the instance, for any such $\alpha_j$ it must be that $u_{\alpha_j}(\{ a_{2i}, a_{2i-1} \})=u_{\alpha_j}(t)=0$, which is a contradiction. A symmetric argument shows that no $a_{2i-1}$ where $1\leq i \leq 3q$ belongs to a triple that blocks $M$.

The only remaining possibility is that some triple $\{ h_r^{s_1}, h_r^{s_2}, h_r^{s_3} \}$ blocks $M$ where $1\leq r\leq 6q$ and $1\leq s_1, s_2, s_3 \leq 8$. Suppose for a contradiction that some such triple exists. We noted earlier in this proof that $u_{h_r^p}(M)\geq 2$ for each $1\leq p \leq 11$ so it must be that $u_{h_r^{s_1}}(\{ h_r^{s_2}, h_r^{s_3} \}) \geq 3$, $u_{h_r^{s_2}}(\{ h_r^{s_1}, h_r^{s_3} \}) \geq 3$, and $u_{h_r^{s_3}}(\{ h_r^{s_1}, h_r^{s_2} \}) \geq 3$. Considering the valuations of the agents in $H_r$ we can see that no such $h_r^{s_1}, h_r^{s_2}, h_r^{s_3}$ exist, which is a contradiction.
\end{proof}

% \subsection{Correctness of the reduction: second direction}
% \label{sec:threed_sr_as_ternary_symmetric_reduction_seconddirection}

We now show, using a sequence of lemmas, that if the 3DR-AS instance $(N, V)$ contains a stable matching then $G$ contains a partition into triangles.

\begin{lem}
\label{lem:threed_sr_as_symmetric_ternary_no_hendecagadget_scores_0}
If $(N, V)$ contains a stable matching $M$ then $u_{h_r^s}(M)\geq 1$ for any $r$ and $s$ where $1\leq r\leq 6q$ and $1\leq s\leq 11$.
\end{lem}
\begin{proof}
Suppose for a contradiction that $u_{h_r^{s_1}}(M)=0$ for some $1\leq r\leq 6q$ and $1\leq {s_1} \leq 8$. Note that it must be that $M(h_r^{s_1})$ contains exactly one agent, namely $h_r^{s_1}$, in $H_r$.

We claim that no triple in $M$ contains exactly two agents in $H_r$. Suppose for a contradiction that some such triple $\{ h_r^{s_2}, h_r^{s_3}, \alpha_i \}$ exists where $1\leq s_2, s_3 \leq 8$ and $\alpha_i \notin H_r$. By the construction of $H_r$, it must be that $u_{h_r^{s_1}}(\{ h_r^{s_2}, h_r^{s_3} \})\geq 2$. Since $\alpha_i \notin H_r$ it must also be that $v_{h_r^{s_2}}(h_r^{s_1}) > v_{h_r^{s_2}}(\alpha_i)$ and $v_{h_r^{s_3}}(h_r^{s_1}) > v_{h_r^{s_3}}(\alpha_i)$ so $u_{h_r^{s_2}}(\{ h_r^{s_1}, h_r^{s_3} \}) > u_{h_r^{s_2}}(M)$ and $u_{h_r^{s_3}}(\{ h_r^{s_1}, h_r^{s_2} \}) > u_{h_r^{s_3}}(M)$. It follows that the triple $\{ h_r^{s_1}, h_r^{s_2}, h_r^{s_3} \}$ blocks $M$, which is a contradiction.

We now claim that at most two triples in $M$ contain exactly one agent in $H_r$. Suppose three or more triples in $M$ each contain exactly one agent in $H_r$. It follows that these three agents in $H_r$ each have utility zero in $M$. By the construction of $H_r$, these three agents block $M$.

We have now established that no triple in $M$ contains exactly two agents in $H_r$ and at most two triples in $M$ contain exactly one agent in $H_r$. Since $|H_r|=8$ the only possibility is that two triples in $M$ each contain exactly three agents in $H_r$ and two triples in $M$ each contain exactly one agent in $H_r$.

By the design of the octogadget $H_r$ there are four disjoint pairs of agents $\{ h_r^{p_1}, h_r^{p_2} \}$ for which $v_{h_r^{p_1}}(h_r^{p_2})=2$. Since there are exactly two triples that contain three agents in $H_r$ there exists at least two disjoint pairs of agents $\{ h_r^{p_1}, h_r^{p_2} \}$ and $\{ h_r^{p_3}, h_r^{p_4} \}$ where $v_{h_r^{p_1}}(h_r^{p_2})=v_{h_r^{p_3}}(h_r^{p_4})=2$ and $M(h_r^{p_1}) \neq M(h_r^{p_2})$ and $M(h_r^{p_3}) \neq M(h_r^{p_4})$. For example, if $\{ h_r^3, h_r^4, h_r^5 \} \in M$ and $\{ h_r^6, h_r^7, h_r^8 \} \in M$ then $\{ \{ h_r^{p_1}, h_r^{p_2} \}, \{ h_r^{p_3}, h_r^{p_4} \} \} = \{ \{ h_r^1, h_r^2 \}, \{ h_r^5, h_r^6 \} \}$. Suppose without loss of generality that $\{ h_r^{p_1}, h_r^{p_2} \}$ does not contain $h_r^{s_1}$. Since $h_r^{p_2} \notin M(h_r^{p_1})$, by the design of $H_r$ it must be that $u_{h_r^{p_1}}\leq 2$. Similarly, since $h_r^{p_1} \notin M(h_r^{p_2})$ it must be that $u_{h_r^{p_2}}\leq 2$. It follows that the triple $\{ h_r^{s_1}, h_r^{p_1}, h_r^{p_2} \}$ blocks $M$, since $u_{h_r^{s_1}}(M) = 0 < u_{h_r^{s_1}}(\{ h_r^{p_1}, h_r^{p_2} \}) = 2$, $u_{h_r^{p_1}}(M) \leq 2 < u_{h_r^{p_1}}(\{ h_r^{s_1}, h_r^{p_2} \}) = 3$, and $u_{h_r^{p_2}}(M) \leq 2 < u_{h_r^{p_2}}(\{ h_r^{s_1}, h_r^{p_3} \}) = 3$. This contradicts the supposition that $M$ is stable.
\end{proof}

\begin{lem}
\label{lem:threed_sr_as_symmetric_ternary_allairscores0}
If $(N, V)$ contains a stable matching $M$ then $u_{a_k}(M)=0$ for each $1\leq k\leq 6q$.
\end{lem}
\begin{proof}
Consider some arbitrary octogadget $H_{r_1}$ where $1\leq r_1 \leq 6q$. Since $|H_{r_1}|=8$ there exists at least one triple in $M$ that contains some agent $h_{r_1}^{s_1}$ where $1\leq s_1 \leq 8$ and some agent $\alpha_i \notin H_{r_1}$. By Lemma~\ref{lem:threed_sr_as_symmetric_ternary_no_hendecagadget_scores_0}, $u_{h_{r_1}^{s_1}}(M)\geq 1$, and it follows that $M(h_{r_1}^{s_1})=\{ h_{r_1}^{s_1}, h_{r_1}^{s_2}, \alpha_i \}$ where $1\leq s_2 \leq 8$ and $\alpha_i \in N$. It must be that $\alpha_i \notin H_{r_2}$ for any $r_2$ where $1\leq r_2 \leq 6q$, for otherwise $u_{\alpha_i}(M)=0$, which contradicts Lemma~\ref{lem:threed_sr_as_symmetric_ternary_no_hendecagadget_scores_0}. It remains that either $\alpha_i = b_j$ where $1\leq j \leq 3q$ or $\alpha_i = a_{k}$ where $1\leq k\leq 6q$. Note that by the design of the instance it follows that $u_{\alpha_i}(M)=0$.

Suppose for a contradiction that $\alpha_i = b_j$ where $1\leq j \leq 3q$. It must also be that $u_{a_{2j}}(M) \leq 1$ and $u_{a_{2j-1}}(M) \leq 1$, since $a_{2j} \notin M(b_j)$ and $a_{2j-1} \notin M(b_j)$. We can now see that $\{ b_j, a_{2j}, a_{2j-1} \}$ blocks $M$ since $u_{b_j}(M) = 0 < u_{b_j}(\{ a_{2j}, a_{2j-1} \}) = 2$, $u_{a_{2j}}(M) \leq 1 < 2 = u_{a_{2j}}(\{ b_{j}, a_{2j-1} \})$ and $u_{a_{2j-1}}(M) \leq 1 < 2 = u_{a_{2j-1}}(\{ b_{j}, a_{2j} \})$. This contradicts the supposition that $M$ is stable. It remains that $\alpha_i = a_{k}$ where $1\leq k\leq 6q$. Recall that $u_{\alpha_i}(M)=0$. Since there are $6q$ octogadgets and the choice of $r_1$ was arbitrary the only possibility is that $u_{a_k}(M)=0$ for every $1\leq k \leq 6q$.
\end{proof}

\begin{lem}
\label{lem:threed_sr_as_symmetric_ternary_allbiscores2}
If $(N, V)$ contains a stable matching $M$ then $u_{b_i}(M)=2$ for any $i$ where $1 \leq i \leq 3q$.
\end{lem}
\begin{proof}
Suppose for a contradiction that there exists some $1 \leq i \leq 3q$ where $u_{b_i}(M) < 2$. Lemma~\ref{lem:threed_sr_as_symmetric_ternary_allairscores0} shows that $u_{2i}(M) = u_{a_{2i-1}}(M) = 0$. Considering the valuation functions of $a_{2i}$, $a_{2i-1}$, and $b_i$, we can see that $u_{b_i}(\{ a_{2i}, a_{2i-1} \}) = u_{a_{2i}}(\{ b_i, a_{2i-1} \}) = u_{a_{2i-1}}(\{ b_i, a_{2i} \}) = 2$. The triple $\{ b_i, a_{2i}, a_{2i-1} \}$ therefore blocks $M$, which is a contradiction.
\end{proof}

\begin{lem}
\label{lem:threed_sr_as_symmetric_ternary_allbiintriplestogether}
If $(N, V)$ contains a stable matching $M$ then for any $b_i$ where $1 \leq i \leq 3q$, the triple $M(b_i)$ comprises $\{ b_i, b_j, b_k \}$ for some $1 \leq j, k \leq 3q$ where $\{ w_i, w_j \}, \{ w_i, w_k \} \in E$.
\end{lem}
\begin{proof}
Lemma~\ref{lem:threed_sr_as_symmetric_ternary_allbiscores2} shows that $u_{b_i}(M)=2$. Suppose $M(b_i)=\{ b_i, \alpha_k, \alpha_l \}$ for some $\alpha_k, \alpha_l \in N$. Since $u_{b_i}(M)=2$, it must be that $v_{b_i}(\alpha_k)=1$ and hence either $\alpha_k = a_{2i}$, $\alpha_k = a_{2i-1}$, or $\alpha_k = b_j$ where $1\leq j \leq 3q$ where $\{w_i, w_j\}\in E$. Suppose first that $\alpha_k = a_{2i}$. Since $b_i \in M(a_{2i})$ it follows that $u_{a_{2i}}(M) \geq 1$ which contradicts Lemma~\ref{lem:threed_sr_as_symmetric_ternary_allairscores0}. A similar argument shows that $\alpha_k \neq a_{2i-1}$. It remains that $\alpha_k = b_j$ where $1\leq j \leq 3q$ such that $\{ w_i, w_j \} \in E$. The same argument shows that $\alpha_l = b_k$ where $1\leq k \leq 3q$ and $\{ w_i, w_k \} \in E$. We have shown that $M(b_i) = \{ b_i, b_j, b_k \}$ where $1 \leq j,k \leq 3q$ and $\{w_i, w_j\}, \{w_i, w_k\} \in E$.
\end{proof}

\begin{lem}
\label{lem:threed_sr_as_symmetric_ternary_reduction_seconddirection}
If $(N, V)$ contains a stable matching then $G$ contains a partition into triangles.
\end{lem}
\begin{proof}
Lemma~\ref{lem:threed_sr_as_symmetric_ternary_allbiintriplestogether} shows that for an arbitrary $b_i$ where $1 \leq i \leq 3q$, $M(b_i)$ comprises $\{ b_i, b_j, b_k \}$ where $1 \leq j,k \leq 3q$, $\{w_i, w_j\}\in E$, and $\{w_i, w_k\}\in E$. It follows that there are exactly $q$ triples in $M$ each containing three agents $\{b_i, b_j, b_k\}$, where the three corresponding vertices $w_i, w_j, w_k$ are pairwise adjacent in $G$. From these triples of pairwise adjacent vertices, a partition into triangles $X$ can be easily constructed.
\end{proof}

% \subsection{Conclusion}

We have now shown that the 3DR-AS instance $(N, V)$ contains a stable matching if and only if the PIT instance $G$ contains a partition into triangles. This shows that the reduction is correct.

\begin{thm}
\label{thm:threed_sr_as_symmetric_ternary_reduction}
Deciding if a given instance of 3DR-AS contains a stable matching is $\NP$-complete, even when preferences are ternary and symmetric.
\end{thm}
\begin{proof}
It is straightforward to show that this decision problem belongs to $\NP$. We have presented a polynomial-time reduction from Partition Into Triangles (PIT), which is $\NP$-complete \cite{GJ79}. Given an arbitrary instance $G$ of PIT, the reduction constructs an instance $(N, V)$ of 3DR-AS with ternary and symmetric preferences. Lemmas~\ref{lem:threed_sr_as_ternary_symmetric_reduction_firstdirection} and~\ref{lem:threed_sr_as_symmetric_ternary_reduction_seconddirection} show that $(N, V)$ contains a stable matching if and only if $G$ contains a partition into triangles and thus that this decision problem is $\NP$-hard.
\end{proof}

\section{Summary and open problems}
\label{sec:threed_sr_as_conclusion}
In this chapter we formalised a model of 3DR involving additively separable preferences, which we called 3DR-AS. We considered in instances of 3DR-AS the existence of, and complexity of finding, matchings that are stable, under three possible restrictions of the agents' valuations. 

We first showed that any instance of 3DR-AS with binary and symmetric preferences contains a stable matching, and presented a polynomial-time algorithm that can construct a stable matching in such an instance. 
We then considered the problem of finding a stable matching with maximum utilitarian welfare, given an instance with binary and symmetric preferences. We called this problem the \emph{3DR-AS Stable Maximum Utilitarian Welfare problem} (\mysymbolfirstusedefinition{symboldef:threedr_as_smuw}{3DR-AS-SMUW}). We proved that 3DR-AS-SMUW is $\NP$-hard, but showed that the algorithm for constructing a stable matching in this setting can be modified to yield a $2$-approximation algorithm.  We then complemented the previous tractability results with two hardness results. The first is that a stable matching need not exist in general, and the associated decision problem is $\NP$-complete even when preferences are binary and not necessarily symmetric. The second is that the same decision problem is $\NP$-complete even when preferences are ternary and symmetric. 

Later, in Chapter~\ref{c:threed_efr_as}, we shall combine some of the results from this chapter with other results for 3DR-AS and present a comprehensive analysis of the existence and complexity of feasible matchings in 3DR-AS under restricted preferences, for four related solution concepts. This classification is shown in Table~\ref{tab:introduction_3dsras_mainresults} in Chapter~\ref{c:threed_efr_as}.

We now present some open problems specifically involving stability in 3DR-AS. More general problems, involving solution concepts other than stability and other models of fixed-size coalitions, are discussed in Chapter~\ref{c:conclusion}. 

An immediate open problem is whether our results for binary preferences hold in a more general setting in which $v_{\alpha_i}(\alpha_j) \in \{ a, b \}$ for any non-negative integers $a$ and $b$ where $a < b$, and whether our results for ternary preferences hold when $v_{\alpha_i}(\alpha_j) \in \{ a, b, c \}$ for any non-negative integers $a$, $b$, and $c$ where $a < b < c$. We conjecture that both statements are true, and that both the polynomial-time algorithm and $\NP$-hardness reductions can be modified accordingly.

As we noted in Section~\ref{sec:threed_sr_as_symmetric_binary_welfare}, any approximation algorithm for 3DR-AS-SMUW with approximation ratio strictly less than $2$ must not always begin, like Algorithm~\algorithmfont{findStableUW} does, by selecting a maximal set of triangles. While it would be very interesting to derive such an algorithm, it would also be informative to derive an inapproximability result for 3DR-AS-SMUW. We conjecture that 3DR-AS-SMUW is $\APX$-hard.

An open direction of work involves the \emph{price of anarchy} and \emph{price of stability} of 3DR-AS \cite{Bilo22}. In the setting of 3DR-AS, we could define the price of anarchy (stability) as the worst- (best-)case ratio between the utilitarian welfare of an arbitrary stable matching and the maximum possible utilitarian welfare over all matchings. Our example instance, shown in Figure~\ref{fig:threed_sr_as_max_welfare_example}, thus shows that the price of anarchy is at least $2$, even when preferences are binary and symmetric. 

It would be very interesting to identify other restrictions of 3DR-AS in which a stable matching can be found in polynomial time. For example, a crucial part of the reduction shown in Section~\ref{sec:threed_sr_as_generalbinary} was the pentagadget, which corresponds to a unique type of regular tournament graph \cite{McDonald2020}. It would be quite remarkable if, for example, every instance of 3DR-AS that does not contain a pentagadget contains a stable matching. As a starting point, we conjecture that any instance corresponding to a directed acyclic graph must contain a stable matching.

Continuing the connection to graph theory (discussed in Section~\ref{sec:threed_sr_as_intro}), it might also be possible to consider the problem of finding a stable matching in a given instance of 3DR-AS from the perspective of the parameterised complexity with respect to a graph parameter. For example, in the case of binary and symmetric preferences, one could consider the tree-width \cite{Robertson84} of the underlying graph.

As we noted for 3DR-B and 3DR-W (in Chapters~\ref{c:threed_sr_b} and~\ref{c:threed_sr_w}), it would be very interesting to estimate the probability that a random instance of 3DR-AS contains a stable matching, or to estimate the same probability in a random instance of 3DR-AS with binary or ternary preferences. It might be possible to use probabilistic techniques from graph theory, such as the Erd\H{o}s-R\'enyi model of a random graph. Alternatively an empirical approach might be informative, for example by formulating the problem as an integer program \cite{IPbook}.

% The connection between 3DR-AS and graph theory gives rise to other interesting research questions. For example, one could study the problem of finding a stable matching in a given instance of 3DR-AS with from the view of its parameterised complexity. A variety of parameters are possible, in particular in the setting of binary preferences, in which we can view an instance of 3DR-AS as a directed graph. For example, the maximum in- or out-degree. 


% \section{Open questions}
% \label{sec:conclusion}
% In this chapter we studied a new formalism of 3DR, involving $\mathscr{B}$-preferences, which we called 3DR-B. We considered in 3DR-B the existence of, and complexity of finding, matchings that are stable. 

Our first result was that a given instance of 3DR-B may not contain a stable matching and that the associated decision problem is $\NP$-complete. We then considered a closely related optimisation problem, which we called 3DR-B-MSM, in which the objective is to construct, in a given instance of 3DR-B, a matching with the maximum number of non-blocking triples. We first devised a $9/4$-approximation algorithm for 3DR-B-MSM based on an existing algorithm for 3PSA-MSM, which is a closely related problem \cite{rosenbaum16}. Improving upon this approximation, we then presented a $3/2$-approximation algorithm based on serial dictatorship, and showed that our analysis is tight asymptotically. Finally, we considered the problem of identifying the smallest instance of 3DR-B that contains no stable matching. We showed that such an instance must have between $9$ and $15$ agents, inclusive.

% As we noted in Chapter~\ref{c:lit_review}, there are a wide variety of other systems of preference representation and solution concepts that could be used to formalise alternative model of 3DR, as well as $\mathscr{B}$, $\mathscr{W}$, and additively separable preferences. 

We now present some open problems specifically involving stability in 3DR-B. More general problems, involving solution concepts other than stability and other models of fixed-size coalition formation, are discussed in Chapter~\ref{c:conclusion}. 

An immediate open problem is to improve the bounds on the the smallest instance of 3DR-B that contains no stable matching. We have shown in Section~\ref{sec:threed_sr_b_structural} that such an instance contains at least $9$ and at most $15$ agents, but the precise number of agents remains open. To fully resolve this open question, it will be necessary to either prove that every some fixed size strictly greater than $6$ contains a stable matching (as in the proof of Theorem~\ref{thm:threed_sr_b_ifnis6thensmexists}), demonstrate that some instance with between $9$ and $12$ agents contains no stable matching (in a similar way to the proof of Theorem~\ref{thm:threed_sr_b_ifnis15thensmdoesnotexist}), or both.

In Chapter~\ref{c:three_dsm_cyc}, we presented an approximation algorithm for a restriction of 3-DSM-CYC-MSM in which the preferences of some agents were derived from a master list. We conjecture that a similar algorithm also exists for a restriction of 3DR-B-MSM in which the preferences of all agents are derived from a master list. In 2020, Bredereck et al.\ \cite{Bre20} considered a similar situation for a multidimensional generalisation of 3GSM \cite{NH91}, and it may be that some of their results or techniques can also be applied to 3DR-B.

A closely related objective is to estimate the probability that a random instance of 3DR-B contains a stable matching. Pittel's \cite{Pittel20} probabilistic analysis of $k$-DSM-CYC and Pittel and Irving's \cite{PI94} analysis of two-dimensional Stable Roommates (SR) are two possible starting points. A hybrid of theoretical and empirical techniques might also be informative, as it was in Escamocher and O'Sullivan's \cite{Escamocher2018} paper, which considered the same question in the setting to 3-DSM-CYC.

As well as investigating the existence of an improved approximation algorithm for 3DR-B-MSM, it might be possible to prove an inapproximability result for this problem. For example, it would be very informative to prove that the approximation ratio of Algorithm~\algorithmfont{serialDictatorship} is tight, by showing that no $(3/2 - \varepsilon)$-approximation algorithm exists for 3DR-B-MSM unless $\P = \NP$. Alternatively, it might be easier to prove the weaker result that 3DR-B-MSM is $\APX$-hard, meaning that there exists some constant factor $\varepsilon$ such that no $(1 + \varepsilon)$-approximation algorithm exists for 3DR-B-MSM, unless $\P = \NP$ \cite{ACGKMP99}. A starting point towards the latter result could be to modify the reduction of Iwama et al.\ \cite{IMO08} from a variant of \emph{Maximum 3D Matching} (Max 3DM) to an optimisation problem defined in a related model of 3DR (which is described in Chapter~\ref{c:lit_review}). Another possibility is to adapt the reduction of Rosenbaum \cite{rosenbaum16} from Max 3DM to 3PSA-MSM.

Various alternative optimisation problems and measures can also be defined that relate to stable matchings and 3DR-B.
One possibility is to construct a sub-matching of maximum cardinality such that no three agents in triples in the sub-matching form a blocking triple in the sub-matching. Rosenbaum \cite{rosenbaum16} refers to the analogous problem for 3PSA as the \emph{3PSA Maximum Stable Sub-matching problem} (3PSA-MSS). 
A second possibility is to consider $\alpha$-stability \cite{ABEOMP09} (discussed in Chapter~\ref{c:lit_review}) in the setting of 3DR-B. For example, for some fixed $\alpha \geq 1$, we could say that a matching $M$ is $\alpha$-stable if for any agent $\alpha_i$ and any triple $t$ where $\alpha_i \in t$ the increase in rank in $P_{\alpha_i}$ from $\mathscr{B}(M)$ to $\mathscr{B}(t)$ is at most $\alpha$. We could also then define an optimisation problem in which the objective is to find an $\alpha$-stable matching for a minimum such $\alpha$.
A third possibility is to define a complementary problem in which the objective is to minimise the number of blocking triples, which is arguably more natural. Similar optimisation problems, in which the objective is to minimise the number of blocking pairs, have been studied in the context of (two-dimensional) Stable Roommates \cite{ABM06}. % Unlike 3DR-B, deciding if a given instance of SR contains a stable matching is solvable in polynomial time, and it is unclear if the techniques relating to SR can be also applied to the minimisation problem of 3DR-B-MSM.

% In terms of parameterised complexity, a natural starting point is a recent work of Bredereck et al.\ \cite{Bre20}, which explores the parameterised complexity of a generalisation of 3GSM in which each agent's preference list, over sets of agents, is derived from a central master list or poset. It may be possible to consider an analogous situation in 3DR-B in which each agent's preference list, of individual agents, is derived from a central master list or poset.
\chapter{Three-Dimensional Envy-Free Roommates with Additively Separable Preferences}
\label{c:threed_efr_as}
\chaptermark{Envy-freeness in 3DR-AS}

\section{Introduction}
\label{sec:threed_efr_as_intro}
In the previous chapter we considered the existence of stable matchings in a model of Three-Dimensional Roommates with additively separable preferences, called 3DR-AS. We showed that in general, a stable matching may not exist in a given instance of 3DR-AS, and the associated decision problem is $\NP$-complete. We also showed that if the agents' preferences are sufficiently restricted then a stable matching must always exist, which can be found in polynomial time. In this chapter we consider three alternative solution concepts, one of which is a strictly weaker concept than stability. For each concept, we study the existence of such matchings under different restrictions on the agents' preferences, and the computational complexity of the associated existence and construction problems (see Chapter~\ref{c:lit_review}). This culminates in a full complexity classification, which includes for each of the three solution concepts a dichotomy between polynomial-time solvability and $\NP$-hardness.

The three solution concepts that we study all relate to matchings in which there exist no two agents $\alpha_i$ and $\alpha_j$ where $\alpha_i$ would prefer to swap with $\alpha_j$. In such a case we say that $\alpha_i$ has \emph{envy} for $\alpha_j$. If a matching $M$ contains no agent with envy for another agent then we say that it is \emph{envy-free}. The other two solution concepts involve the other agents in the triple of $\alpha_j$, $M(\alpha_j)$. Informally, we say that $\alpha_i$ has \emph{justified envy} (abbreviated j-envy) for $\alpha_j$ if $\alpha_i$ has envy for $\alpha_j$ and each of the other agents in $M(\alpha_j)$ prefer $\alpha_i$ to $\alpha_j$. We say that $\alpha_i$ has \emph{weakly justified envy} (wj-envy) for $\alpha_j$ if $\alpha_i$ has envy for $\alpha_j$ and each of the other agents in $M(\alpha_j)$ either prefers $\alpha_i$ to $\alpha_j$ or is indifferent. Formally, an agent $\alpha_i$ has envy for another agent $\alpha_j$ in a matching $M$ if $u_{\alpha_i}(M(\alpha_j) \setminus \{ \alpha_j \}) > u_{\alpha_i}(M)$. If $\alpha_i$ has envy for $\alpha_j$ and $v_{\alpha_k}(\alpha_i) > v_{\alpha_k}(\alpha_j)$ for each $\alpha_k \in M(\alpha_j) \setminus \{ \alpha_j \}$ then we say that $\alpha_i$ has j-envy for $\alpha_j$. If $v_{\alpha_k}(\alpha_i) \geq v_{\alpha_k}(\alpha_j)$ for each $\alpha_k \in M(\alpha_j) \setminus \{ \alpha_j \}$ then we say that $\alpha_i$ has wj-envy for $\alpha_j$. The definitions of j-envy-free and wj-envy-free matchings are analogous.

These three solution concepts are hierarchical: by definition, any matching that is envy-free is also wj-envy-free, and any matching that is wj-envy-free must also be j-envy-free. In fact, they are also related to other solution concepts proposed in the literature. It is relatively straightforward to show that any envy-free matching is also \emph{strictly swap stable} \cite{Bilo22}. It is also straightforward to show that any stable matching is also j-envy-free. In fact, j-envy-freeness is a strictly weaker concept than stability. With the results of Chapter~\ref{c:threed_sr_as} in mind, this observation partly motivates our study of j-envy-freeness, since for example it would be interesting if a strictly weaker restriction on the agents' preferences than binary and symmetric valuations (which guarantees the existence of a stable matching) is sufficient to guarantee the existence of a j-envy-free matching.

% In this chapter, we study in 3DR-AS the existence of matchings that satisfy alternative solution concepts. 
% In this chapter we consider the existence of envy-free, weakly justified envy-free, and justified envy-free matchings in 3DR-AS (defined in Chapter~\ref{c:threed_sr_as}) and the complexity of the associated existence and construction problems. For each of the three solution concepts, we consider various restrictions on the agents' valuations.

A strong motivation also exists for combining these three solution concepts with a setting of fixed-size coalitions (and of size three). As we discussed in Chapter~\ref{c:lit_review}, in a model in which coalitions have arbitrary size, such as a hedonic game, it can be trivial to construct partitions that are envy-free by, for example, placing all agents into the grand coalition \cite{BY19,Ued18}.

We begin in Section~\ref{sec:threed_efr_as_ef} by showing that an arbitrary instance of 3DR-AS may not contain an envy-free matching, even when preferences are binary and symmetric and the maximum degree of the underlying graph is $2$. We describe a polynomial-time algorithm for this case that can either construct an envy-free matching or report that no such matching exists (Theorem~\ref{thm:threed_efr_as_ef_algorithm}). We then contrast this result by showing that the corresponding existence problem is $\NP$-complete even when the maximum degree of the underlying graph is $3$ (Theorem~\ref{thm:threed_efr_as_regularenvy_npcomplete}).

In Section~\ref{sec:threed_efr_as_wjef}, we identify a similar dichotomy for wj-envy-freeness. We first show that a wj-envy-free matching may not exist, even when preferences are binary and symmetric and the maximum degree of the underlying graph is $2$. Notably, the set of instances of 3DR-AS with maximum degree $2$ that do not contain a wj-envy-free matching is a strict subset of the set of instances with maximum degree $2$ that do not contain an envy-free matching. We describe a slightly more complex polynomial-time algorithm for this case, compared to the corresponding algorithm in Section~\ref{sec:threed_efr_as_ef}, that can either construct a wj-envy-free matching or report that no such matching exists (Theorem~\ref{thm:threed_efr_as_wjef_algowjpathscycles}). As for envy-freeness, we show that the corresponding existence problem is also $\NP$-complete even when the maximum degree of the underlying graph is $3$ (Theorem~\ref{thm:threed_efr_as_wjef_npcomplete}).

Next, in Section~\ref{sec:threed_efr_as_jef} we consider j-envy-freeness. We first remark that any stable matching is also j-envy-free. We then show that if preferences are binary but not necessarily symmetric, a j-envy-free matching must exist and can be found in polynomial time (Theorem~\ref{thm:threed_efr_as_jef_binary_algorithm}). Significantly, this tractability result holds in a larger set of instances of 3DR-AS than the corresponding result for stability, which we saw in Chapter~\ref{c:threed_sr_as}. Nevertheless, we complement this result with two hardness results. The first is that a given instance of 3DR-AS may not contain a j-envy-free matching even when preferences are ternary but not symmetric, and the associated existence problem is $\NP$-complete (Theorem~\ref{thm:threed_efr_as_jef_terasym_npcomplete}). The second is that a given instance of 3DR-AS may not contain a j-envy-free matching even when preferences are non-binary and symmetric, and the associated existence problem is $\NP$-complete (Theorem~\ref{thm:threed_efr_as_jef_symmetric_6_npcomplete}).

Finally, in Section~\ref{sec:threed_efr_as_conclusion}, we recap on our contribution and discuss some directions for future work.

\section{Envy-freeness}
\label{sec:threed_efr_as_ef}
\subsection{Symmetric binary preferences with maximum degree two}
\label{sec:threed_efr_as_envyfreeness_maxdeg2}


Our first result is a necessary and sufficient condition for the existence of an envy-free matching in an instance $(N, V)$ of 3DR-AS with binary and symmetric preferences and maximum degree $2$. Since preferences are binary and symmetric, in the following proof we refer to the \emph{underlying graph} $(N, E)$ of the instance $(N, V)$. The underlying graph $(N, E)$ is constructed such that $\{ \alpha_i, \alpha_j \} \in E$ if and only if $v_{\alpha_i}(\alpha_j) = 1$, for any two agents $\alpha_i, \alpha_j \in N$.

\begin{lem}
\label{lem:threed_efr_as_ef_if_and_only_if}
Consider an instance of 3DR-AS with binary and symmetric preferences and maximum degree $2$. Let $P$ be the set of isolated agents, $\mathcal{Q}$ be the set of connected components of size $3{k_1} - 2$ for any ${k_1} > 1$, and $\mathcal{R}$ be the set of connected components of size $3{k_2} - 1$ for any ${k_2} \geq 1$. An envy-free matching exists if and only if $2|\mathcal{Q}| + |\mathcal{R}| \leq |P|$.
\end{lem}
\begin{proof}
Consider an arbitrary instance of 3DR-AS with binary and symmetric preferences, represented by its underlying graph $(N, E)$. Let $\mathcal{S}$ be the set of connected components of size $3{k_3}$ for any ${k_3} \geq 1$.

To show the first direction, suppose $2|\mathcal{Q}| + |\mathcal{R}| \leq |P|$. We shall construct a matching $M$ and demonstrate that it is envy-free. First, observe that if any agent has utility $1$ or more then that agent is not envious, since the maximum degree of the underlying graph is $2$.

Construct $M$ as follows. First, consider each $S = ( s_1, s_2, \dots, s_{3{k_3}} )$ in $\mathcal{S}$. For each $i$ where $1\leq i\leq {k_3}$, add $\{ s_{3i-2}, s_{3i-1}, s_{3i} \}$ to $M$. It follows that each agent in $S$ has utility at least $1$ and thus is not envious. Now consider each $R = ( r_1, r_2, \dots, r_{3{k_2}-1} )$ in $\mathcal{R}$. For each $i$ where $1\leq i \leq {k_2}-1$, add $\{ r_{3i-2}, r_{3i-1} r_{3i} \}$ to $M$. Next, add $\{ r_{3{k_2}-2}, r_{3{k_2}-1}, p_{2|\mathcal{Q}|+i} \}$ to $M$ (recall that $|P| \geq 2|\mathcal{Q}| + |\mathcal{R}|$). It follows that each agent in $R$ is not envious. Now consider each $Q = ( q_1, q_2, \dots, q_{3{k_1}-2} )$ in $\mathcal{Q}$. For each $i$ where $1\leq i \leq {k_1}-2$, add $\{ q_{3i-2}, q_{3i-1}, q_{3i} \}$ to $M$. Next, add to $M$ the triples $\{ q_{3{k_1}-5}, q_{3{k_1}-4}, p_{i} \}$ and $\{ q_{3{k_1}-3}, q_{3{k_1}-2}, p_{2i} \}$. It follows that each agent in $Q$ has utility at least $1$  and thus is not envious.

Finally, arbitrarily add the remaining agents in $P$ to triples in $M$. Since these agents are isolated they are not envious.

To show the second direction, suppose for a contradiction that $(N, E)$ has an envy-free matching $M$ and $2|\mathcal{Q}| + |\mathcal{R}| > |P|$. Since the degree of any agent in $\bigcup \mathcal{Q}$ is at least $1$, it must be that the utility of each agent in $\bigcup \mathcal{Q}$ is at least $1$. Similarly, the utility of each agent in $\bigcup \mathcal{R}$ must also be at least $1$. It follows that any agent in $M$ that has utility $0$ belongs to $P$.

Now consider some $Q \in \mathcal{Q}$. By definition, $|Q| = 3{k_1} - 2$ for some ${k_1} > 1$. It follows that there exists two triples in $M$ that each contain exactly two agents in $Q$ and some agent with utility $0$ in $M$. Similarly, for each $R \in \mathcal{R}$ there must exist at least one triple in $M$ that contains exactly two agents in $R$ and some agent with utility $0$ in $M$. It follows that there are at least $2|\mathcal{Q}| + |\mathcal{R}|$ agents with utility $0$ in $M$. The only possibility is that there exists $2|\mathcal{Q}| + |\mathcal{R}|$ agents in $P$, which is a contradiction.
\end{proof}

\begin{thm}
\label{thm:threed_efr_as_ef_algorithm}
Consider an instance of 3DR-AS with binary and symmetric preferences and maximum degree $2$. There exists an $O(|N|)$-time algorithm that can either find an envy-free matching or report that no such matching exists.
\end{thm}
\begin{proof}
Lemma~\ref{lem:threed_efr_as_ef_if_and_only_if} gives a necessary and sufficient condition for the existence of an envy-free matching in $(N, E)$, based on the number of connected components of different sizes. Define $P$, $\mathcal{Q}$, and $\mathcal{R}$ as before in Lemma~\ref{lem:threed_efr_as_ef_if_and_only_if}. We outline a linear-time algorithm based on the constructive proof in Lemma~\ref{lem:threed_efr_as_ef_if_and_only_if}. The algorithm outputs either $\bot$, if no envy-free matching exists, or a labelling $\tau$ of each agent $\alpha_i$ with some index $1\leq \tau(\alpha_i) \leq n$ that represents the index of $M(\alpha_i)$ in some arbitrary ordering of the triples in an envy-free matching $M$. 

The algorithm has three phases. In the first phase, the algorithm constructs a stack $P$ that contains all isolated agents in $(N, E)$. It also constructs a stack $T$, which contains exactly one agent per connected component of size two or more, such that if a connected component is a path then $T$ contains one of its endpoints. The construction of $T$ can thus be performed in linear time.

The algorithm now enters the second phase, which is as follows. The algorithm maintains a counter $r$ to track the label of the agent last labelled. Initially, $r=1$. The algorithm pops an unmarked agent $m_i$ from the stack $T$ and marks $\tau(m_i)=1$. It sets a counter $c$ to $1$, which will track the size of the connected component that contains $m_i$. It then identifies successive adjacent agents and labels each one with $r$, incrementing $r$ by one every third agent, following the path or cycle in the underlying graph. The successive agents are therefore marked $1,1,1,2,2,2,3,3,3\dots$. The counter $c$ is updated to ensure that $c$ is the size of this connected component. Eventually, either some agent with degree $1$ or some previously labelled agent is discovered. In this case, there are three possibilities. 

The first possibility is that $c=3{k_3}$ for some ${k_3}\geq 1$. In this case the algorithm pops some yet unlabelled $m_i$ from the stack $T$ and repeats the above process. 

The second possibility is that $c=3{k_2}-1$ for some ${k_2}\geq 1$. In this case the algorithm pops some isolated agent $p_i$ from the stack $P$ and labels $\tau(p_i)=r$. If the stack $P$ is empty then it must be that $2|\mathcal{Q}| + |\mathcal{R}| > |P|$ and thus the algorithm returns $\bot$. The algorithm then pops some yet unlabelled $m_i$ from the stack $T$ and repeats the above process.

The third possibility is that $c=3{k_1}-2$ for some ${k_1}>1$. In this case it must be that exactly one agent $\alpha_j$ has been marked with $r$. The algorithm identifies the last agent $\alpha_l$ that was labelled with $r-1$, which must be adjacent to $\alpha_j$. It relabels $\tau(\alpha_l)=r$. It follows that exactly two adjacent agents are labelled with $r-1$ and exactly two adjacent agents are labelled with $r$. The algorithm then pops two isolated agents $p_g, p_h$ from the stack $P$ and labels $\tau(p_g)=r$ and $\tau(p_h)=r-1$. If $|P|<2$ then it must be that $2|\mathcal{Q}| + |\mathcal{R}|>|P|$ and thus the algorithm returns $\bot$. The algorithm then pops some yet-unlabelled $m_i$ from the stack $T$ and repeats the above process.

In the third phase, since the algorithm has not yet returned $\bot$, it must be that each agent with degree $1$ or more has been labelled and therefore assigned to some triple in $M$. The algorithm arbitrarily assigns the remaining agents in $P$ to triples in $M$ by popping successive agents $p_i$ from the stack $P$ and labelling $\tau(p_i)=r$, incrementing $r$ every third agent.
\end{proof}


\subsection{Symmetric binary preferences with maximum degree three}
\label{sec:threed_efr_as_envyfreeness_maxdeg3}

We now consider instances of 3DR-AS with binary and symmetric preferences in which the maximum degree of the underlying graph is $3$. In contrast with Theorem~\ref{thm:threed_efr_as_ef_algorithm}, we show that deciding if a given instance of 3DR-AS contains an envy-free matching is $\NP$-complete, even when preferences are binary and symmetric and the maximum degree is $3$.

We present a polynomial-time reduction from a variant of \emph{Exact Satisfiability} (XSAT) \cite{GJ79}. An instance of XSAT is a boolean formula in conjunctive normal form (CNF). We write $C=\{ c_1, c_2, \dots, c_{m} \}$ to represent such a formula as the set of its clauses. We represent each clause $c_r\in C$ as a set of literals. Each literal is either an occurrence of a single variable or its negation. Given a formula, we write $X$ to mean the set of variables contained in the formula. A \emph{truth assignment} $f : X \mapsto \{\text{true}, \text{false}\}$ is an assignment of values to the set of variables. We say that an \emph{exact model} is a truth assignment to the variables such that each clause contains exactly one true literal (and therefore exactly two false literals). Given an instance $C$, if an exact model exists then we say that $C$ is \emph{exactly satisfiable}. Deciding if an instance $C$ of XSAT is exactly satisfiable is $\NP$-complete \cite{SchaeferSatComplexity78} even when each clause contains exactly three literals. Porschen et al.\ \cite[Lemma~5]{PSSW14} show that this problem remains $\NP$-complete even for the restricted case every literal is positive and each variable occurs in exactly three clauses, which they denote by $3$-$\text{CNF}_{+}^3$-$\text{XSAT}$. We shall refer to this variant as \porschenxsatvariant/ (Problem~\ref{pr:xsatvariant}). Note that in this problem $|X|=m$.

\begin{myproblem}[\porschenxsatvariant/]
\label{pr:xsatvariant}\mysymbolfirstusedefinition{symboldef:porschenxsatvariant}{}
\begin{samepage}
\begin{adjustwidth}{8pt}{8pt}
\inp a boolean formula in conjunctive normal form, represented as a set of clauses $C = \{ c_1, c_2, \dots, c_{m} \}$, in which every literal is positive and each variable occurs in exactly three clauses\\
\ques Is $C$ exactly satisfiable?
\end{adjustwidth}
\end{samepage}
\end{myproblem}
The reduction, illustrated in Figure~\ref{fig:threed_efr_as_regular_envy_free_reduction}, is as follows. Suppose $C$ is an arbitrary instance of \porschenxsatvariant/. We shall construct an instance $(N, E)$ of 3DR-AS.

For each variable $x_i \in X$, there are three corresponding literals in three different clauses. For each such $x_i$, arbitrarily label each of these literals as the first, second, and third occurrences of $x_i$.
\begin{figure}
    \centering
    \input{figures/threed_efr_as/regular_envy_free_reduction.tikz}
    \caption[The reduction from \porschenxsatvariant/ to the problem of deciding if a given instance of 3DR-AS contains an envy-free matching]{The reduction from \porschenxsatvariant/ to the problem of deciding if a given instance of 3DR-AS contains an envy-free matching. A variable gadget $W_i$ and clause gadget $D_r$ are represented as undirected graphs.}
    \label{fig:threed_efr_as_regular_envy_free_reduction}
\end{figure}
For each variable $x_i \in X$ construct a set of three agents $W_i = \{ w_i^1, w_i^2, w_i^3 \}$, which we refer to as the \emph{$i\textsuperscript{th}$ variable gadget}. Add the edges $\{ w_i^1, w_i^2 \}$, $\{ w_i^2, w_i^3 \}$, and $\{ w_i^3, w_i^1 \}$ to $E$. Next, for each clause $c_r\in C$ construct a set of eight agents $D_r = \{ d_r^1, d_r^2, \dots, d_r^8 \}$, which we refer to as the \emph{$r\textsuperscript{th}$ clause gadget}. Add the edges $\{ d_r^1, d_r^4 \}$, $\{ d_r^2, d_r^5 \}$, $\{ d_r^3, d_r^8 \}$, $\{ d_r^4, d_r^6 \}$, $\{ d_r^4, d_r^7 \}$, $\{ d_r^5, d_r^6 \}$, $\{ d_r^5, d_r^7 \}$, $\{ d_r^8, d_r^6 \}$, and $\{ d_r^8, d_r^7 \}$. Next, we shall connect the variable and clause gadgets. Consider each clause $c_r = \{ x_i, x_j, x_k \}$. If $c_r$ contains the first occurrence of $x_i$ then add the edge $\{ d_r^1, w_i^1 \}$. Alternatively, if $c_r$ contains the second occurrence of $x_i$ then add the edge $\{ d_r^1, w_i^2 \}$. Alternatively, if $c_r$ contains the third occurrence of $x_i$ then add the edge $\{ d_r^1, w_i^3 \}$. Similarly, add an edge between $d_r^3$ and an agent in $W_j$ depending on the index of the occurrence of $x_j$ in the clause $c_r$. Finally, add an edge between $d_r^4$ and an agent in $W_k$ depending on the index of the occurrence of $x_k$ in the clause $c_r$. We say that the clause gadget $D_r$ is adjacent to the variable gadgets $W_i$, $W_j$, and $W_k$, and vice-versa.

This completes the construction of $(N, E)$. Note that each agent in a variable gadget has degree $3$, the agents $d_r^4, d_r^5, d_r^6, d_r^7$ and $d_r^8$ for each $1 \leq r \leq m$ have degree $3$, and the agents $d_r^1, d_r^2, d_r^3$ for each $1 \leq r \leq m$ have degree $2$. It follows that the maximum degree of $(N, E)$ is $3$. 

It is straightforward to show that this reduction can be performed in polynomial time. To prove that the reduction is correct we show that the 3DR-AS instance $(N, E)$ contains an envy-free matching if and only if the \porschenxsatvariant/ instance $C$ is exactly satisfiable.

We first prove some preliminary results. Recall that for any set of agents $S \subseteq N$, $\sigma(S, N)$ denotes the number of triples in $N$ that each contain at least one agent in $S$.
\ifdefined\vdkr
\else
\errmessage{if this is not in the thesis, change this so it does not say recall}
\fi

\begin{lem}
\label{lem:threed_efr_as_util1_neighbourhoodthreetriples}
Suppose $M$ is a matching in $(N, E)$. If $u_{\alpha_i}(M)=1$ and $\sigma(N(\alpha_i), M)=\deg(\alpha_i)$ then $\alpha_i$ is not envious in $M$.
\end{lem}
\begin{proof}
Suppose, to the contrary, that $\alpha_i$ envies some $\alpha_j\in N$. Then $u_{\alpha_i}(M(\alpha_j) \setminus \{ \alpha_j \})=2$. It follows that two agents in $N(\alpha_i)$ belong to the same triple, $M(\alpha_j)$, so $\sigma(N(\alpha_i), M) < \deg(\alpha_i)$, which is a contradiction. 
\end{proof}

\begin{lem}
\label{lem:threed_efr_as_util0}
Suppose $M$ is a matching in $(N, E)$. If $u_{\alpha_i}(M)=0$ then $\alpha_i$ is envious in $M$.
\end{lem}
\begin{proof}
Note that by construction of $(N, E)$, each agent has degree at least one. Suppose then, for a contradiction, that there exists some $\alpha_i\in N$ where $u_{\alpha_i}(M)=0$. There must exist some $\alpha_j$ where $\{ \alpha_i, \alpha_j \}\in E$, so it follows that $\alpha_i$ envies both agents in $M(\alpha_j)$, which is a contradiction.
\end{proof}

% t(S) is at most |S|
% t(S) is at least \lceil |S|/3 \rceil

% \begin{figure}[h]
%     \centering
%     \includegraphics[width=0.6\textwidth]{figures/efr_reduction_1.png}
%     \caption{A variable gadget and clause gadget represented as undirected graphs}
%     \label{fig:threed_efr_as_my_label}
% \end{figure}


We now show that if the \porschenxsatvariant/ instance $C$ is exactly satisfiable then the 3DR-AS instance $(N, E)$ contains an envy-free matching.

\begin{lem}
\label{lem:threed_efr_as_regularenvy_firstdirection}
If $C$ is exactly satisfiable then $(N, E)$ contains an envy-free matching.
\end{lem}
\begin{proof}
Suppose $f$ is an exact model of $C$. We shall construct a matching $M$ that is envy-free. For each $x_i$ in $X$ where $f(x_i)$ is false, add $\{ w_i^1, w_i^2, w_i^3 \}$ to $M$. Next, consider each clause $c_r = \{ x_i, x_j, x_k \}$ and the corresponding clause gadget $D_r$, labelling $i, j, k$ such that $W_i$ contains an agent adjacent to $d_r^1$, $W_j$ contains an agent adjacent to $d_r^2$, and $W_k$ contains an agent adjacent to $d_r^3$. There are three cases: $f(x_i)$ is true while both $f(x_j)$ and $f(x_k)$ are false, $f(x_j)$ is true while both $f(x_i)$ and $f(x_k)$ are false, and $f(x_k)$ is true while both $f(x_i)$ and $f(x_j)$ are false. In the first case, suppose $c_r$ contains the $u\textsuperscript{th}$ occurrence of $x_i$. Add the triples $\{ w_i^u, d_r^1, d_r^4 \}$, $\{ d_r^2, d_r^5, d_r^7 \}$, and $\{ d_r^3, d_r^6, d_r^8 \}$. The constructions in the second and third cases are symmetric: in the second case, suppose $c_r$ contains the $u\textsuperscript{th}$ occurrence of $x_j$. Add the triples $\{ w_j^u, d_r^2, d_r^5 \}$, $\{ d_r^1, d_r^4, d_r^6 \}$, and $\{ d_r^3, d_r^7, d_r^8 \}$. In the third case, suppose $c_r$ contains the $u\textsuperscript{th}$ occurrence of $x_k$. Add the triples $\{ w_k^u, d_r^3, d_r^8 \}$, $\{ d_r^1, d_r^4, d_r^6 \}$, and $\{ d_r^2, d_r^5, d_r^7 \}$.

Now for any variable gadget $W_i$ either $\sigma(W_i, M) = 1$ or $\sigma(W_i, M) = 3$. For each clause gadget $D_r$, there exist two triples in $M$ that each contain three agents in $D_r$ and one triple in $M$ that contains two agents in $D_r$ and one agent in some variable gadget. There are therefore three kinds of triple in $M$: those triples that contain three agents belonging to the same variable gadget; those triples that contain one agent in some variable gadget and two agents in a clause gadget; and those triples that contain three agents in the same clause gadget. We shall show that no triple of each kind contains an envious agent.

First, consider some triple $t\in M$ where $t=W_i$ for some variable gadget $W_i$. Since each agent in $t$ has utility $2$, no agent in $t$ is envious.

Second, consider some triple $t\in M$ that contains one agent $w_i^a$ in some variable gadget $W_i$ and two agents in some clause gadget $D_r$. By the construction of $M$, either the triple contains $\{ w_i^u, d_r^1, d_r^4 \}$, $\{ w_i^a, d_r^2, d_r^5 \}$, or $\{ w_i^a, d_r^3, d_r^8 \}$. Suppose the triple contains $\{ w_i^a, d_r^1, d_r^4 \}$. Since $u_{d_r^1}(M)=2$ clearly $d_r^1$ is not envious. By construction, $\sigma(W_i, M)=3$ so since $u_{w_i^a}(M)\geq 1$ it follows by Lemma~\ref{lem:threed_efr_as_util1_neighbourhoodthreetriples} that $w_i^u=a$ is also not envious. Similarly, since $M(d_r^6) \neq M(d_r^7)$ it follows that $\sigma(N(d_r^4), M)=3$ so, by Lemma~\ref{lem:threed_efr_as_util1_neighbourhoodthreetriples}, $d_r^4$ is also not envious. The proof in the two remaining cases, in which the triple comprises $\{ w_i^a, d_r^2, d_r^5 \}$ or $\{ w_i^a, d_r^3, d_r^8 \}$, is symmetric.

Third, consider some triple $t\in M$ where $t\subset D_r$ for some clause gadget $D_r$. There are four cases: either $t=\{ d_r^1, d_r^4, d_r^6 \}$, $t=\{ d_r^2, d_r^5, d_r^7 \}$, $t=\{ d_r^3, d_r^6, d_r^8 \}$, or $t=\{ d_r^3, d_r^7, d_r^8 \}$. Suppose $t=\{ d_r^1, d_r^4, d_r^6 \}$. Since $\sigma(N(d_r^1), M)=2$ and $u_{d_r^1}(M)=1$ by Lemma~\ref{lem:threed_efr_as_util1_neighbourhoodthreetriples} $d_r^1$ is not envious. Since $M(d_r^5) \neq M(d_r^8)$, it must be that $\sigma(N(d_r^6), M)=3$ so since $u_{d_r^6}(M)=1$, by Lemma~\ref{lem:threed_efr_as_util1_neighbourhoodthreetriples} $d_r^6$ is not envious. Since $u_{d_r^4}(M)=2$, $d_r^4$ is also not envious. The proof in the three remaining cases, in which $t=\{ d_r^2, d_r^5, d_r^7 \}$, $t=\{ d_r^3, d_r^6, d_r^8 \}$, or $t=\{ d_r^3, d_r^7, d_r^8 \}$, is symmetric. It follows that no agent in $t$ is envious. 
\end{proof}

We now show that if the 3DR-AS instance $(N, E)$ contains an envy-free matching then the \porschenxsatvariant/ instance $C$ is exactly satisfiable.

\begin{lem}
\label{lem:threed_efr_as_regularenvy_seconddirection_triangle_split_stay}
If $(N, E)$ contains an envy-free matching $M$ then for any variable gadget $W_i$, either $\sigma(W_i, M)=1$ or $\sigma(W_i, M)=3$.
\end{lem}
\begin{proof}
Since $|W_i|=3$, clearly $1 \leq \sigma(W_i, M) \leq 3$. Suppose for a contradiction that $\sigma(W_i, M)=2$. It must be that some triple in $M$ contains exactly two agents in $W_i$ and some third agent, which we label $\alpha_j$. Label the remaining agent in the variable gadget $w_i^a$. Since the maximum degree of the instance is three, it must be that $u_{w_i^a}(M)\leq 1$. It follows that $w_i^a$ envies $\alpha_j$ since $u_{w_i^a}(M(\alpha_j) \setminus \{ \alpha_j \}) = 2$.
\end{proof}

\begin{lem}
\label{lem:threed_efr_as_regularenvy_seconddirection}
If $(N, E)$ contains an envy-free matching then $C$ is exactly satisfiable.
\end{lem}
\begin{proof}
Suppose $M$ is an envy-free matching in $(N, E)$. By Lemma~\ref{lem:threed_efr_as_regularenvy_seconddirection_triangle_split_stay}, for any variable gadget $W_i$ either $\sigma(W_i, M)=1$ or $\sigma(W_i, M)=3$. Construct a truth assignment $f$ in $C$ by setting $f(x_i)$ to be true if $\sigma(W_i, M)=3$ and false otherwise. Each variable $x_i$ corresponds to exactly one variable gadget $W_i$ so it follows that $f$ is a valid truth assignment. By the construction of $(N, E)$, each clause $c_r$ corresponds to exactly one clause gadget $D_r$. Each clause gadget is adjacent to three variable gadgets that correspond to the three variables in that clause. To show that $f$ is an exact model of $C$, it is sufficient to show that for each clause gadget $D_r$ there exists exactly one variable gadget $W_i$ such that $D_r$ is adjacent to $W_i$ and $\sigma(W_i, M)=3$.
 
Consider an arbitrary clause gadget $D_r$ and the corresponding clause $c_r=\{ x_i, x_j, x_k\}$, labelling $i, j, k$ such that $d_r^1$ is adjacent to some agent in $W_i$, $d_r^2$ is adjacent to some agent in $W_j$ and $d_r^3$ is adjacent to some agent in $W_k$. We shall show that a contradiction exists if either $D_r$ is adjacent to two variable gadgets $W_i, W_j$ where $\sigma(W_i, M)=\sigma(W_j, M)=3$ or $D_r$ is adjacent to three variable gadgets $W_i, W_j, W_k$ where $\sigma(W_i, M)=\sigma(W_j, M)=\sigma(W_k, M)=1$. The remaining possibility is that $D_r$ is adjacent to exactly one variable gadget $W_i$ where $\sigma(W_i, M)=3$ and exactly two variable gadgets $W_j, W_k$ where $\sigma(W_j, M)=\sigma(W_k, M)=1$.

First, suppose that $D_r$ is adjacent to two variable gadgets $W_{i}$ and $W_{j}$ where $\sigma(W_i, M)=3$ and $\sigma(W_j, M)=3$. Suppose $c_r$ contains the $a\textsuperscript{th}$ occurrence of $x_i$ and the $b\textsuperscript{th}$ occurrence of $x_j$. Consider $M(w_i^a)$. By Lemma~\ref{lem:threed_efr_as_util0}, no agent has utility $0$ in $M$, so either $\{ d_r^1, d_r^4 \} \in M(w_i^a)$, $\{ d_r^2, d_r^5 \} \in M(w_i^a)$, or $\{ d_r^3, d_r^8 \} \in M(w_i^a)$. Similarly, either $\{ d_r^1, d_r^4 \} \in M(w_j^b)$, $\{ d_r^2, d_r^5 \} \in M(w_j^b)$, or $\{ d_r^3, d_r^8 \} \in M(w_j^b)$. By the symmetry of the clause gadget, assume without loss of generality that $\{ d_r^1, d_r^4 \} \in M(w_i^a)$ and $\{ d_r^2, d_r^5 \} \in M(w_j^b)$. Consider $d_r^6, d_r^7$ and $d_r^8$. Since no agent has utility $0$, the only possibility is that $\{ d_r^6, d_r^7, d_r^8 \} \in M$. It then follows that $d_r^4$ envies $d_r^8$, since $u_{d_r^4}(M)=1$ and $u_{d_r^4}(\{ d_r^6, d_r^7 \})=2$, which is a contradiction.

Second, suppose that $D_r$ is adjacent to three variable gadgets $W_i, W_j, W_k$ where $\sigma(W_i, M)=\sigma(W_j, M)=\sigma(W_k, M)=1$. Since $|D_r|=8$, there exists triple in $M$ that contains some agent $d_r^a \in D_r$ as well as some agent $\alpha_p \notin D_r$. It follows that either $u_{d_r^a}(M)=0$ or $u_{\alpha_p}(M)=0$, which contradicts Lemma~\ref{lem:threed_efr_as_util0}.
\end{proof}

We have now shown that the 3DR-AS instance $(N, E)$ contains an envy-free matching if and only if the \porschenxsatvariant/ instance $C$ is exactly satisfiable. This shows that the reduction is correct.

\begin{thm}
\label{thm:threed_efr_as_regularenvy_npcomplete}
Deciding if a given instance of 3DR-AS contains an envy-free matching is $\NP$-complete, even when preferences are binary and symmetric and the underlying graph has maximum degree $3$.
\end{thm}
\begin{proof}
It is straightforward to show that this decision problem belongs to $\NP$, since for any two agents $\alpha_i, \alpha_j \in N$ we can test if $\alpha_i$ envies $\alpha_j$ in constant time. 

We have presented a polynomial-time reduction from \porschenxsatvariant/, which is $\NP$-complete \cite{PSSW14}. Given an arbitrary instance $C$ of \porschenxsatvariant/, the reduction constructs an instance $(N, E)$ of 3DR-AS with binary and symmetric preferences and maximum degree $3$. Lemmas~\ref{lem:threed_efr_as_regularenvy_firstdirection} and~\ref{lem:threed_efr_as_regularenvy_seconddirection} show that $(N, E)$ contains an envy-free matching if and only if $C$ is exactly satisfiable and thus that this decision problem is $\NP$-hard.
\end{proof}


\section{Weakly justified envy-freeness}
\label{sec:threed_efr_as_wjef}
\subsection{Symmetric binary preferences with maximum degree two}

We begin by defining a class of instances of 3DR-AS, \iwjnomaxdegreetwofamily/, such that membership in the class is a necessary and sufficient condition for the non-existence of a wj-envy-free matching, among instances of 3DR-AS with binary and symmetric preferences and maximum degree $2$.

\begin{mydefinition}[\iwjnomaxdegreetwofamily/]
\label{def:threed_efr_as_wjef_max_deg_2_ino}
\begin{adjustwidth}{8pt}{8pt}
    An instance of 3DR-AS belongs to \iwjnomaxdegreetwofamily/ if and only if the underlying graph comprises a set of disjoint $4$-cycles and a single isolated agent.
    \end{adjustwidth}
\end{mydefinition}

We now show, using a sequence of lemmas, that any instance of 3DR-AS that belongs to \iwjnomaxdegreetwofamily/ does not contain a wj-envy-free matching. Consider an instance of 3DR-AS with binary and symmetric preferences that belongs to \iwjnomaxdegreetwofamily/ and label the underlying graph $(\hat{N}, \hat{E})$. Suppose $\hat{M}$ is an arbitrary matching in $(\hat{N}, \hat{E})$.

\begin{lem}
\label{lem:threed_efr_as_wjenvy_maxdeg2_no3plus1c4s}
For any $4$-cycle $R$ in $(\hat{N}, \hat{E})$, if some triple in $\hat{M}$ contains three agents in $R$ then $\hat{M}$ is not wj-envy-free.
\end{lem}
\begin{proof}
Suppose $R=(r_1, r_2, r_3, r_4)$. By definition, $\{ r_1, r_2 \}, \{ r_2, r_3 \}, \{ r_3, r_4 \}, \{ r_4, r_1 \} \in E$. Without loss of generality we need only consider the case when $\{ r_1, r_2, r_3 \} \in \hat{M}$. In this case it must be that $r_4$ has wj-envy for $r_2$ since $u_{r_4}(\hat{M}) = 0 < 2 = u_{r_4}(\{ r_1, r_3 \})$, $v_{r_1}(r_2) = 1 \leq 1 = v_{r_1}(r_4)$, and $v_{r_3}(r_2) = 1 \leq 1 = v_{r_3}(r_4)$.
\end{proof}

\begin{lem}
\label{lem:threed_efr_as_wjenvy_maxdeg2_notriplespreadacross3}
For any three distinct connected components $C_1, C_2, C_3$ in $(\hat{N}, \hat{E})$, if some triple in $\hat{M}$ contains one agent in each of $C_1$, $C_2$, and $C_3$ then $\hat{M}$ is not wj-envy-free.
\end{lem}
\begin{proof}
Suppose some such triple in $\hat{M}$ exists. By the construction of $(\hat{N}, \hat{E})$, it must be that at least two of $C_1$, $C_2$, and $C_3$ are $4$-cycles. Assume without loss of generality that $C_1$ and $C_2$ are $4$-cycles, so label $C_1$ as $R_1 = ( r_1^1, r_1^2, r_1^3, r_1^4 )$ and $C_2$ as $R_2 = ( r_2^1, r_2^2, r_2^3, r_3^4 )$. We may further assume that this triple is $\{ r_1^1, r_2^1, c_3^1 \}$ where $r_1^1 \in C_1$, $r_2^1 \in C_2$, and $c_3^1 \in C_3$. Note that $\{ r_1^1, c_3^1 \} \notin E$ and $\{ r_2^1, c_3^1 \} \notin E$. Now consider $R_1$. If $u_{r_1^2}(\hat{M}) = 0$ then $r_1^2$ has wj-envy for $c_3^1$ since $u_{r_1^2}(\hat{M}) = 0 < 1 = u_{r_1^2}(\{ r_1^1, r_2^1 \})$, $v_{r_1^1}(c_3^1) = 0 \leq 1 = v_{r_1^1}(r_1^2)$, and $v_{r_2^1}(c_3^1) = 0 \leq 0 = v_{r_2^1}(r_1^2)$. It follows that $u_{r_1^2}(\hat{M}) \geq 1$ so it must be that $\hat{M}(r_1^2)$ contains $r_1^3$. A symmetric argument shows that $u_{r_1^4}(\hat{M}) \geq 1$ and thus that $\hat{M}(r_1^2)$ must also contain $r_1^3$. Now $\hat{M}(r_1^2) = \{ r_1^2, r_1^3, r_1^4 \}$ so by Lemma~\ref{lem:threed_efr_as_wjenvy_maxdeg2_no3plus1c4s} it follows that $\hat{M}$ is not wj-envy-free. 
\end{proof}

\begin{lem}
\label{lem:threed_efr_as_wjenvy_maxdeg2_tripletouches2components}
For any triple $t \in \hat{M}$, if $\hat{M}$ is wj-envy-free then the agents in $t$ belong to exactly two connected components in $(\hat{N}, \hat{E})$.
\end{lem}
\begin{proof}
Since $t$ is a triple, the agents in $t$ belong to either 1, 2, or 3 connected components. If $\hat{M}$ is wj-envy-free, then by Lemmas~\ref{lem:threed_efr_as_wjenvy_maxdeg2_no3plus1c4s} and~\ref{lem:threed_efr_as_wjenvy_maxdeg2_notriplespreadacross3} the agents in $t$ do not belong to either 1 or 3 connected components in $(\hat{N}, \hat{E})$.
\end{proof}

Recall that for any set of agents $S \subseteq N$, $\sigma(S, N)$ denotes the number of triples in $N$ that each contain at least one agent in $S$.

\begin{lem}
\label{lem:threed_efr_as_wjenvy_maxdeg2_c4splits2or4}
For any $4$-cycle $R$ in $(\hat{N}, \hat{E})$, if $\hat{M}$ is wj-envy-free then $\sigma(R, \hat{M}) \in \{ 2, 4 \}$.
\end{lem}
\begin{proof}
By definition, $2 \leq \sigma(R, \hat{M}) \leq 4$ for any $4$-cycle $R$ in $(\hat{N}, \hat{E})$. It suffices to show that if $\hat{M}$ is wj-envy-free then $\sigma(R, \hat{M}) \neq 3$ for any such $R$. Suppose then, for a contradiction, that $\hat{M}$ is wj-envy-free and there exists some $4$-cycle $R$ in $(\hat{N}, \hat{E})$ where $\sigma(R, \hat{M}) = 3$. Label $R = ( r_1, r_2, r_3, r_4 )$. Since $\sigma(R, \hat{M}) = 3$ there must exist one triple in $\hat{M}$ that contains exactly two agents in $R$ and two triples in $\hat{M}$ that each contain exactly one agent in $R$. Label the former triple $\{ r_{i_1}, r_{i_2}, \alpha_{j_1} \}$ and the latter two triples $\{ r_{i_3}, \alpha_{j_2}, \alpha_{j_3} \}$ and $\{ r_{i_4}, \alpha_{j_4}, \alpha_{j_5} \}$, where $\alpha_{j_1}, \alpha_{j_2}, \dots, \alpha_{j_5} \in N \setminus R$. Since $R$ is a $4$-cycle it must be that $\{ r_{i_1}, \alpha_{j_1} \} \notin E$ and $\{ r_{i_2}, \alpha_{j_1} \} \notin E$. Similarly, $\{ r_{i_3}, \alpha_{j_2} \} \notin E$ and $\{ r_{i_3}, \alpha_{j_3} \} \notin E$. It must also be that either $\{ r_{i_3}, r_{i_1} \} \in E$ or $\{ r_{i_3}, r_{i_2} \} \in E$. We can now see that $r_{i_3}$ has wj-envy for $\alpha_{j_1}$ since $u_{r_{i_3}}(\hat{M}) = 0 < 1 \leq u_{r_{i_3}}(\{ r_{i_1}, r_{i_2} \})$, $v_{r_{i_1}}(\alpha_{j_1}) = 0 \leq v_{r_{i_1}}(r_{i_3})$, and $v_{r_{i_2}}(\alpha_{j_1}) = 0 \leq v_{r_{i_2}}(r_{i_3})$.
\end{proof}

\begin{lem}
\label{lem:threed_efr_as_wjef_maxdeg2_ino}
If an instance of 3DR-AS belongs to \iwjnomaxdegreetwofamily/ then it does not contain a wj-envy-free matching.
\end{lem}
\begin{proof}
We considered an arbitrary instance of 3DR-AS that belongs to \iwjnomaxdegreetwofamily/ in which the preferences are binary and symmetric and its underlying graph $(\hat{N}, \hat{E})$. We also supposed $\hat{M}$ is an arbitrary matching of $(\hat{N}, \hat{E})$.

Suppose for a contradiction that $\hat{M}$ is wj-envy-free. Construct a graph $(\mathcal{C}, \Gamma)$ where $\mathcal{C}$ is the set of connected components in $(\hat{N}, \hat{E})$ and $\Gamma$ is constructed as follows. For any triple $t \in \hat{M}$, it must be that the agents in $t$ belong to exactly two connected components in $(\hat{N}, \hat{E})$ (Lemma~\ref{lem:threed_efr_as_wjenvy_maxdeg2_tripletouches2components}). For each triple $t \in \hat{M}$ identify the two such connected components $C_i, C_j$ in $(\hat{N}, \hat{E})$ and add the edge $\{ C_i, C_j \}$ to $\Gamma$. By the design of $(\hat{N}, \hat{E})$, there exists exactly one connected component in $(\hat{N}, \hat{E})$ that is not a $4$-cycle, which contains exactly one agent. Label this component $C_1$. Label the remaining connected components in $(\hat{N}, \hat{E})$, which are all $4$-cycles, as $C_2, C_3, \dots, C_{|\mathcal{C}|}$. Since $|C_1|=1$ it must be that exactly one triple in $\hat{M}$ contains the agent in $C_1$ so the degree of vertex $C_1$ in the graph $(\mathcal{C}, \Gamma)$ is 1. Consider the $4$-cycles $C_2, C_3, \dots, C_{|\mathcal{C}|}$. By Lemma~\ref{lem:threed_efr_as_wjenvy_maxdeg2_c4splits2or4}, it must be that $\sigma(C_i, \hat{M}) \in \{ 2, 4 \}$ for each such $4$-cycle $C_i$ (where $2\leq i\leq |\mathcal{C}|$). It follows that the degree of each vertex $C_i$ in $\mathcal{C}$ where $2\leq i\leq |\mathcal{C}|$ is either 2 or 4. It follows that the sum of the degrees of all vertices in $\mathcal{C}$ is odd, which is impossible.
\end{proof}

Building on Lemma~\ref{lem:threed_efr_as_wjef_maxdeg2_ino}, we present Algorithm~\algorithmfont{wjPathsCycles}, shown in Algorithm~\ref{alg:3defr_wje_paths_cycles}. Given an instance of 3DR-AS with binary and symmetric preferences and maximum degree $2$, this algorithm either returns a wj-envy-free matching $M$ or reports that $(N, E)$ belongs to \iwjnomaxdegreetwofamily/. With Lemma~\ref{lem:threed_efr_as_wjef_maxdeg2_ino}, this establishes the fact that \iwjnomaxdegreetwofamily/ is a necessary and sufficient condition for the non-existence of a wj-envy-free matching in instances of 3DR-AS with with binary and symmetric preferences and maximum degree $2$.

In some respects the approach taken by Algorithm~\algorithmfont{wjPathsCycles} is straightforward. For example, paths or cycles that contain a number of agents divisible by $3$ are broken up into triples of three successively adjacent agents. Other paths and cycles, except $4$-cycles, are broken up in a similar fashion leaving one or two surplus agents per connected component. More care is required in the assignment of the agents in $4$-cycles to triples. The $12$ agents in three $4$-cycles can be assigned to four triples in a relatively straightforward way that ensures no agent is wj-envied in some resulting matching. The main complexity of Algorithm~\algorithmfont{wjPathsCycles} stems from the case when the number of $4$-cycles is not divisible by $3$.

Algorithm~\algorithmfont{wjPathsCycles} contains calls to five subroutines, which are presented separately in order to simplify the overall presentation. Four of the subroutines take as input some agents in $(N, E)$ and constructs a set of triples containing some or all of the agents in that set. The final subroutine is a helper function used to shorten the pseudocode of the main algorithm.

The first subroutine is Subroutine~\algorithmfont{nonC4Components}, shown in Algorithm~\ref{alg:3defr_wje_subroutine_nonC4s}. This subroutine takes as input a set of connected components $\mathcal{C}$ in $(N, E)$, none of which are $4$-cycles. It returns a pair $(T, S)$ where $T$ is a set of triples of agents and $S$ is a set of agents. For each component in $\mathcal{C}$, the corresponding set of triples in $T$ is constructed in a straightforward way by breaking up the component into triples of three successively adjacent agents. This procedure leaves remaining at most two agents from each component, which are then added to $S$. It follows that the maximum degree of the subgraph induced by $S$ in $(N, E)$ is $1$.

\input{algorithms/threed_efr_as/wj_max_degree_two_nonC4s}


\begin{lem}
\label{lem:threed_efr_as_max_degree_2_subgraph_nonC4s_part0}
Suppose $\mathcal{C}$ is some set of connected components in $(N, E)$ that are not $4$-cycles. Suppose $(T, S)$ is returned by a call $\algorithmfont{nonC4Components}(\mathcal{C})$ and $M$ is a matching in $(N, E)$. For any agent $c_i \in \bigcup T$, if $M(c_i) \in T$ then $c_i$ is not wj-envious in $M$.
\end{lem}
\begin{proof}
Suppose $c_i \in \bigcup T$ belongs to some connected component $C \in \mathcal{C}$, which must not be a $4$-cycle. By the construction of $T$ in Subroutine~\algorithmfont{nonC4Components}, it must be that the triple in $T$ that contains $c_i$ either contains $c_{i-1}$ or $c_{i+1}$. Since $M(c_i) \in T$ by assumption, it follows that $u_{c_i}(M) \geq 1$. If $c_i$ has wj-envy in $M$ then it must be that two agents not in $M(c_i)$ are adjacent to $c_i$ in $(N, E)$. Since $u_{c_i}(M) \geq 1$ it follows that $c_i$ has degree $3$ in $(N, E)$, which is a contradiction.
\end{proof}

\begin{lem}
\label{lem:threed_efr_as_max_degree_2_subgraph_nonC4s}
Suppose $\mathcal{C}$ is a set of connected components in $(N, E)$ that are not $4$-cycles. Suppose $(T, S)$ is returned by a call $\algorithmfont{nonC4Components}(\mathcal{C})$ and $M$ is a matching in $(N, E)$. For any agent $c_j \in \bigcup T$, if $M(c_j) \in T$ then $c_j$ is not wj-envied in $M$.
\end{lem}
\begin{proof}
Suppose for a contradiction that some $c_j \in \bigcup T$ where $M(c_j) \in T$ is wj-envied in $M$. Consider the pseudocode of Subroutine~\algorithmfont{nonC4Components}. Let $i = \lceil j/3 \rceil$. It must be that the triple $t = \{ c_{3i-2}, c_{3i-1}, c_{3i} \}$, which contains $c_j$, was added to $T$ in the $i\textsuperscript{th}$ iteration of the inner for loop, in the particular iteration of the outer for loop in which component $C$ was identified. Note that $\{ c_{3i-2}, c_{3i-1} \} \in E$ and $\{ c_{3i-1}, c_{3i} \} \in E$, by definition. There are now three possibilities: $j = 3i-2$, $j=3i-1$, and $j=3i$.

Suppose either $j = 3i-2$ or $j=3i$. Since $\alpha_k$ has wj-envy for $c_j$ it must be that $v_{c_{3i-1}}(\alpha_k) \geq v_{c_{3i-1}}(c_j) = 1$. It follows that $v_{c_{3i-1}}(\alpha_k) = v_{c_{3i-1}}(c_{3i-2}) = v_{c_{3i-1}}(c_{3i}) = 1$ and thus that $c_{3i-1}$ has degree $3$ in $(N, E)$, which is a contradiction.

Suppose then that $j=3i-1$. Since $\alpha_k$ has wj-envy for $c_j$ it must be that $v_{c_{3i-2}}(\alpha_k) \geq v_{c_{3i-2}}(c_{3i-1}) = 1$ and $v_{c_{3i}}(\alpha_k) \geq v_{c_{3i}}(c_{3i-1}) = 1$. It follows that $v_{c_{3i-2}}(\alpha_k) = v_{c_{3i}}(\alpha_k) = 1$. The only possibility is that $C$ is a $4$-cycle comprising $( c_{3i-2}, c_{3i-1}, c_{3i}, \alpha_k )$, which contradicts the statement of the lemma.
\end{proof}

\begin{lem}
\label{lem:threed_efr_as_wjenvy_max_degree_2_nonC4s}
Subroutine~\algorithmfont{nonC4Components} terminates in $O(|\bigcup \mathcal{C}|)$ time and returns a pair $(T, S)$ where $\bigcup \mathcal{C} = S \cup \bigcup T$.
\end{lem}
\begin{proof}
There are $|\mathcal{C}|$ iterations of the outer for loop. In each iteration of the outer loop, some connected component $C$ is identified, and the number of iterations of the inner for loop is $O(|C|)$. It follows that the total number of iterations of the inner for loop is $O(|\bigcup \mathcal{C}|)$. In each iteration of the inner loop, a set of three agents is added to a set $T$, which can be performed in constant time, using an appropriate data structure for $T$. In each iteration of the outer loop, at most two agents are added to $S$, which can also be performed in constant time. It follows that the running time of Subroutine~\algorithmfont{nonC4Components} is $O(|\bigcup \mathcal{C}|)$.

By the pseudocode, it is straightforward to show that $\bigcup \mathcal{C} = S \cup \bigcup T$.
\end{proof}

The second subroutine is Subroutine~\algorithmfont{oneC4TwoSingles}, shown in Algorithm~\ref{alg:3defr_wje_subroutine_oneC4TwoSingles}. This subroutine takes as input three connected components in $(N, E)$. The first, $R$, is a $4$-cycle in $(N, E)$. The second and third, $w_1$ and $w_2$, are other agents in $(N, E)$. It returns two triples in $C$, each of which contains two agents in $R$ and either $w_1$ or $w_2$.

\input{algorithms/threed_efr_as/wj_max_degree_two_oneC4TwoSingles}

\begin{lem}
\label{lem:threed_efr_as_max_degree_2_subgraph_oneC4TwoSingles}
Consider an arbitrary $4$-cycle $R$ and two other arbitrary agents $w_1, w_2$ in $(N, E)$. Suppose $T$ is returned by a call $\algorithmfont{oneC4TwoSingles}(R, w_1, w_2)$. If $M$ is a matching in $(N, E)$ where $T \subseteq M$ then no agent in $R \cup \{ w_1, w_2 \}$ is wj-envied in $M$.
\end{lem}
\begin{proof}
By the design of Subroutine~\algorithmfont{oneC4TwoSingles}, it must be that $T = \{ \{ w_1, r_1, r_2  \}, \{ w_2, r_3, r_4 \} \}$, for some labelling of $R$ where $R= ( r_1, r_2, r_3, r_4 )$. Suppose for a contradiction that some agent $\alpha_k \in N$ has wj-envy for some agent in $R \cup \{ w_1, w_2 \}$. By symmetry, we need only consider two cases: either $\alpha_k$ has wj-envy for $r_1$ or $\alpha_k$ has wj-envy for $w_1$. 

If $\alpha_k$ has wj-envy for $r_1$ then consider $r_2$. Since $r_2 \in M(r_1)$ it must be that $v_{r_2}(\alpha_k) \geq v_{r_2}(r_1) = 1$ and thus that $v_{r_2}(\alpha_k)=1$. The only possibility is that $\alpha_k = r_3$. This is a contradiction since $u_{r_3}(M) = 1 = u_{r_3}(\{ r_1, w_2 \})$, so $r_3$ does not have wj-envy for $r_1$. 

If $\alpha_k$ has wj-envy for $w_1$ then it must be that $u_{\alpha_k}(\{ r_1, r_2 \}) \geq 1$. The only possibility is that either $\alpha_k = r_3$ or $\alpha_k = r_4$. Since $u_{r_3}(M) = 1 = u_{r_3}(\{ r_1, r_2 \})$ and $u_{r_4}(M) = 1 = u_{r_4}(\{ r_1, r_2 \})$ it follows that neither $r_3$ nor $r_4$ have wj-envy for $w_1$, which is a contradiction.
\end{proof}

\begin{lem}
\label{lem:threed_efr_as_wjenvy_max_degree_2_subroutine_oneC4TwoSingles_running_times}
Subroutine~\algorithmfont{oneC4TwoSingles} terminates in constant time.
\end{lem}
\begin{proof}
A suitable choice of data structure for $M$ allows the asymptotic running time of this subroutine to be $O(1)$.
\end{proof}

The third subroutine is Subroutine~\algorithmfont{multipleOfThreeC4s}, shown in Algorithm~\ref{alg:3defr_wje_subroutine_multipleOfThreeC4s}. It takes as input a set $\mathcal{R}$ of $4$-cycles in $(N, E)$, where the number of $4$-cycles is $3q$ for some integer $q \geq 1$. It returns $4q$ triples, each of which contains two agents in one $4$-cycle and one agent in a different $4$-cycle. The agents in each $4$-cycle are assigned to either two or four triples.

\input{algorithms/threed_efr_as/wj_max_degree_two_multipleOfThreeC4s}

\begin{lem}
\label{lem:threed_efr_as_max_degree_2_subgraph_multipleOfThreeC4s}
Consider an arbitrary set $\mathcal{R}$ of $4$-cycles in $(N, E)$ where $|\mathcal{R}|=3q$ for some $q\geq 1$. Suppose $T$ is returned by a call $\algorithmfont{multipleOfThreeC4s}(\mathcal{R})$. If $M$ is a matching in $(N, E)$ where $T \subseteq M$ then no agent in $\bigcup \mathcal{R}$ is wj-envied in $M$.
\end{lem}
\begin{proof}
Suppose to the contrary that some agent in $\bigcup \mathcal{R}$ is wj-envied in $M$. By the design of Subroutine~\algorithmfont{multipleOfThreeC4s} it must be that some such agent was labelled $r_j^i$ for some $i$ where $1\leq i \leq 4$ and $j$ where $1\leq j \leq 3q$. By the pseudocode of this subroutine, it must be that some triple containing $r_i^j$ was added to $T$ in the $d'\textsuperscript{th}$ iteration of the for loop, where $d' = \lceil j / 3 \rceil$. We show that no agent in any of the four triples added to $T$ in this iteration is wj-envied.

In fact, by the symmetric construction of the four triples in $T$ in this $d'\textsuperscript{th}$ iteration of the loop, it suffices to consider only the triple $\{ r_{3d'-2}^1, r_{3d'-2}^2, r_{3d'-1}^1 \}$. 

Suppose first that some agent $\alpha_k \in N$ has wj-envy for $r_{3d'-2}^1$. Since $v_{r_{3d'-2}^2}(r_{3d'-2}^1) = 1$ it must be that $v_{r_{3d'-2}^2}(\alpha_k) = 1$. The only possibility is that $\alpha_k = r_{3d'-2}^3$. This is a contradiction since, by construction, $M(r_{3d'-2}^3) = \{ r_{3d'-2}^3, r_{3d'-2}^4, r_{3d'-1}^4 \}$ so $u_{r_{3d'-2}^3}(M) = 1 = u_{r_{3d'-2}^3}(\{ r_{3d'-2}^2, r_{3d'-1}^1 \})$ and thus $r_{3d'-2}^3$ does not have wj-envy for $r_{3d'-2}^1$. A symmetric argument shows that if some $\alpha_k$ has wj-envy for $r_{3d'-2}^2$ then it must be that $\alpha_k = r_{3d'-2}^4$, which leads to a contradiction since $u_{r_{3d'-2}^4}(M)=1$.

Suppose finally that that some agent $\alpha_k \in N$ has wj-envy for $r_{3d'-1}^1$. It follows that $u_{\alpha_k}(\{ r_{3d'-2}^1, r_{3d'-2}^2 \}) \geq 1$ so either $\alpha_k = r_{3d'-2}^3$ or $\alpha_k = r_{3d'-2}^4$. If $\alpha_k = r_{3d'-2}^3$ then by construction $M(r_{3d'-2}^3) = \{ r_{3d'-2}^3, r_{3d'-2}^4, r_{3d'-1}^4 \}$ and thus $u_{r_{3d'-2}^3}(M) = 1$. Since $u_{r_{3d'-2}^3}(M) = 1 = u_{r_{3d'-2}^3}(\{ r_{3d'-2}^1, r_{3d'-2}^2 \})$ it follows that $r_{3d'-2}^3$ does not in fact wj-envy $r_{3d'-1}^1$, which is a contradiction. A similar argument applies if $\alpha_k = r_{3d'-2}^4$.
\end{proof}

\begin{lem}
\label{lem:threed_efr_as_wjenvy_max_degree_2_subroutine_oneC4TwoSingles_running_time}
Subroutine~\algorithmfont{multipleOfThreeC4s} terminates in $O(|\mathcal{R}|)$ time.
\end{lem}
\begin{proof}
Consider the for loop in the subroutine. In each iteration, four triples are added to $T$, which can be performed in constant time. Since there are exactly $q$ iterations it follows that the running time of this subroutine is $O(|\mathcal{R}|)$.
\end{proof}

The fourth subroutine is Subroutine~\algorithmfont{configureSurplusAgents}, shown in Algorithm~\ref{alg:3defr_wje_subroutine_configureSurplusAgents}. It takes as input a set of agents $\hat{S} \subseteq N$ of agents where $|\hat{S}|$ is divisible by three and the subgraph induced by $\hat{S}$ in $(N, E)$ has maximum degree $1$. It returns a set $T$ of $|\hat{S}|/3$ triples. In the context of Algorithm~\algorithmfont{wjPathsCycles}, this subroutine will be called with a subset of the surplus agents in the second element of the tuple returned by Subroutine~\algorithmfont{nonC4Components}. We remark that Subroutine~\algorithmfont{configureSurplusAgents} is essentially the same procedure as one used in Algorithm~\algorithmfont{findStableUW}, shown in Algorithm~\ref{alg:threed_sr_as_approximationalgo} in Chapter~\ref{c:threed_sr_as}.

\input{algorithms/threed_efr_as/wj_max_degree_two_configureSurplusAgents}

\begin{lem}
\label{lem:threed_efr_as_max_degree_2_subgraph_configureSurplusAgents}
Consider an arbitrary set $\hat{S}\subseteq N$ where $3$ divides $\hat{S}$ and the maximum degree of the subgraph induced by $\hat{S}$ in $(N, E)$ is $1$. Suppose $T$ is returned by a call $\algorithmfont{configureSurplusAgents}(\hat{S})$. If $M$ is a matching in $(N, E)$ where $T \subseteq M$ then no agent in $\hat{S}$ has wj-envy in $M$ for any other agent in $\hat{S}$.
\end{lem}
\begin{proof}
From the pseudocode, it is straightforward to show that Subroutine~\algorithmfont{configureSurplusAgents} is bound to terminate and must return a set $T$ of disjoint triples where $\bigcup T = \hat{S}$.

Suppose for a contradiction that some agent $\alpha_{j_1} \in \hat{S}$ has wj-envy for an agent $\alpha_{k_1} \in \hat{S}$ where $M(\alpha_{j_1}) = \{ \alpha_{j_1}, \alpha_{j_2}, \alpha_{j_3} \}$ and $M(\alpha_{k_1}) = \{ \alpha_{k_1}, \alpha_{k_2}, \alpha_{k_3} \}$. It follows that $u_{\alpha_{j_1}}(\{  \alpha_{k_2}, \alpha_{k_3} \}) \geq 1$. Without loss of generality assume that $v_{\alpha_{j_1}}(\alpha_{k_2})=1$, or equivalently that $\{ \alpha_{j_1}, \alpha_{k_2} \} \in E$. By the definition of $\mathcal{Q}$, it must be that $\{ \alpha_{j_1}, \alpha_{k_2} \} \in \mathcal{Q}$.

Note that if $|\mathcal{Q}| < |\hat{S}|/3$ then, by the pseudocode, for every $\{ q_a, q_b \} \in \mathcal{Q}$ it must be that $q_a \in M(q_b)$. Since $\{ \alpha_{j_1}, \alpha_{k_2} \} \in \mathcal{Q}$ and $\alpha_{j_1} \notin M(\alpha_{k_2})$ it follows that $|\mathcal{Q}| \geq |\hat{S}|/3$. 

By the pseudocode, for each triple $r \in T$ there exists some pair $\{ q_a, q_b \} \in \mathcal{Q}$ where $\{ q_a, q_b \} \subset r$. Since $\mathcal{Q}$ is agent-disjoint (and we established that $\{ \alpha_{j_1}, \alpha_{k_2} \} \in \mathcal{Q}$) the only possibility is that $\{ \alpha_{k_1}, \alpha_{k_3} \} \in \mathcal{Q}$. By the definition of $\mathcal{Q}$ it must be that $v_{\alpha_{k_1}}(\alpha_{k_3})=1$. Since $\alpha_{j_1}$ has wj-envy for $\alpha_{k_1}$ it must be that $v_{\alpha_{k_3}}(\alpha_{j_1}) \geq v_{\alpha_{k_3}}(\alpha_{k_1})$ so it follows that $v_{\alpha_{k_3}}(\alpha_{j_1}) = 1$. Now $\{ \alpha_{k_3}, \alpha_{k_1} \} \in E$ and $\{ \alpha_{k_3}, \alpha_{j_1} \} \in E$ so $\alpha_{k_3}$ has degree $2$ in the subgraph induced by $\hat{S}$ in $(N, E)$, which is a contradiction.
\end{proof}

\begin{lem}
\label{lem:threed_efr_as_wjenvy_max_degree_2_subroutine_configureSurplusAgents_running_time}
Subroutine~\algorithmfont{configureSurplusAgents} terminates in $O(|\hat{S}|)$ time.
\end{lem}
\begin{proof}
The sets of pairs $\mathcal{X}$ and $\mathcal{W}$, the set of agents $Y$, and the returned set of triples $T$ can all be constructed in $O(|\hat{S}|)$ time.
\end{proof}

The fifth subroutine is Subroutine~\algorithmfont{pickLowDegree}. This subroutine takes as input a set $S$ and integer $k\geq 1$, such that the maximum degree of the subgraph induced by $S$ in $(N, E)$ is $1$. It returns a set of $k$ agents in $S$ such that the sum of the degrees of the agents returned in the subgraph induced by $S$ in $(N, E)$ is minimised. Since the maximum degree of the subgraph induced by $S$ in $(N, E)$ is $1$, this subroutine can be implemented to run in $O(|N|)$ time.

We now present Algorithm~\algorithmfont{wjPathsCycles}, shown in Algorithm~\ref{alg:3defr_wje_paths_cycles}. The overall strategy of this algorithm is as follows. First Subroutine~\algorithmfont{nonC4Components} is used to break up connected components that are not $4$-cycles into a set of triples $T$, in which each triple contains three successively adjacent agents, and a set $S$ of surplus agents. 
The algorithm then constructs a set $\mathcal{R}$ of all $4$-cycles in $(N, E)$. If $|\mathcal{R}|$ is divisible by three, Subroutine~\algorithmfont{multipleOfThreeC4s} is called and all agents in $\bigcup \mathcal{R}$, i.e.\ all agents belonging to $4$-cycles, are assigned to triples in $M$. If $|\mathcal{R}|$ is not divisible by three then there are two cases. In the first, the instance is identified as belonging to \iwjnomaxdegreetwofamily/. In the second, a set of surplus agents in $S$ are used to assign the agents belonging to either one or two $4$-cycles to triples in $M$. This set is chosen using Subroutine~\algorithmfont{pickLowDegree}, which (as we shall see later) ensures that no agent in this set will be wj-envious in $M$. Next, the remaining $4$-cycles (the number of which is divisible by three) are then added to triples in $M$ using Subroutine~\algorithmfont{multipleOfThreeC4s}.
The final step (in the case in which the instance does not belong to \iwjnomaxdegreetwofamily/) is a call to Subroutine~\algorithmfont{configureSurplusAgents} and the assignment of all remaining agents in $S$ to triples in $M$.

\input{algorithms/threed_efr_as/wj_max_degree_two_wjPathsCycles}

\begin{prop}
\label{prop:threed_efr_as_wjenvy_maxdeg2_algo_ucupbigcuppi}
In Algorithm~\algorithmfont{wjPathsCycles}, $S \cup \bigcup T$ is the set of agents that do not belong to $4$-cycles in $(N, E)$.
\end{prop}
\begin{proof}
This follows immediately by Lemma~\ref{lem:threed_efr_as_wjenvy_max_degree_2_nonC4s}.
\end{proof}

We first prove two propositions that show that, in two specific cases, $S$ is large enough to extract the number of agents required.

\begin{prop}
\label{prop:threed_efr_as_wjenvy_maxdeg2_algo_rmod3is2_specialcase}
In Algorithm~\algorithmfont{wjPathsCycles}, after initialising $\mathcal{R}$, if $|\mathcal{R}| \bmod 3 = 2$ and $|S| < 4$ then $|S| = 1$.
\end{prop}
\begin{proof}
Suppose $|\mathcal{R}| \bmod 3 = 2$ and $|S| < 4$ after initialising $\mathcal{R}$. Then there exists some constant $k_1 \geq 0$ such that $|\mathcal{R}| = 3{k_1} + 2$, so the number of agents in $N$ that belong to $4$-cycles is $4|\mathcal{R}| = 12{k_1} + 8$. It follows that the number of agents in $N$ that do not belong to $4$-cycles is $3n - 12{k_1} - 8$. Since $(3n - 12{k_1} - 8) \bmod 3 = 1$ there exists some constant $k_2 \geq 0$ such that the number of agents in $N$ that do not belong to $4$-cycles is $3{k_2} + 1$. By Proposition~\ref{prop:threed_efr_as_wjenvy_maxdeg2_algo_ucupbigcuppi}, $S \cup \bigcup T$ is the set of agents that do not belong to $4$-cycles. Since $|S \cup \bigcup T| = 3{k_2} + 1$, $T$ is a set of disjoint triples, and $|S| < 4$, it must be that ${k_2}=0$ and $|S|=1$.
\end{proof}

\begin{prop}
\label{prop:threed_efr_as_wjenvy_maxdeg2_algo_mod3is1}
In Algorithm~\algorithmfont{wjPathsCycles}, after initialising $\mathcal{R}$, if $|\mathcal{R}| \bmod 3 = 1$ then $|S| \geq 2$.
\end{prop}
\begin{proof}
Suppose $|\mathcal{R}| \bmod 3 = 1$ after initialising $\mathcal{R}$. Then there exists some constant ${k_1} \geq 0$ such that $|\mathcal{R}| = 3{k_1} + 1$, so the number of agents in $N$ that belong to $4$-cycles is $4|\mathcal{R}| = 12{k_1} + 4$. It follows that the number of agents in $N$ that do not belong to $4$-cycles is $3n - 12{k_1} - 4$. Since $(3n - 12{k_1} - 4) \bmod 3 = 2$ there exists some constant ${k_2} \geq 0$ where the number of agents in $N$ that do not belong to $4$-cycles is $3{k_2} + 2$. By Proposition~\ref{prop:threed_efr_as_wjenvy_maxdeg2_algo_ucupbigcuppi}, $S \cup \bigcup T$ is the set of agents that do not belong to $4$-cycles. Since $|S \cup \bigcup T| = 3{k_2} + 2$ and $T$ is a set of disjoint triples it must be that $|S| \geq 2$.
\end{proof}

We now show that Algorithm~\algorithmfont{wjPathsCycles} is bound to terminate and has a linear running time with respect to the number of agents.

\begin{lem}
\label{lem:threed_efr_as_wjenvy_max_degree_2_algo_valid_runningtime}
Algorithm~\algorithmfont{wjPathsCycles} terminates in $O(|N|)$ time.
\end{lem}
\begin{proof}
The pseudocode describes the algorithm at a high level. To analyse the worst-case asymptotic time complexity we describe one possible system of data structures and analyse the algorithm with respect to the number of basic operations on these data structures. We begin the analysis at the start of the pseudocode.

The initialisation of $M$, $\hat{T}$ and $\hat{S}$ can be performed in constant time. The set of connected components $\mathcal{C}$ that are not $4$-cycles can be identified in $O(|N|)$ time using breadth-first search, since the maximum degree of $(N, E)$ is two.

By Lemma~\ref{lem:threed_efr_as_wjenvy_max_degree_2_nonC4s}, the call to Subroutine~\algorithmfont{nonC4Components} takes $O(|\bigcup \mathcal{C}|) = O(|N|)$ time.

Like $\mathcal{C}$, the set of connected components $\mathcal{R}$ that are $4$-cycles can be constructed in $O(|N|)$ time. Each nested branch of the if/else statement involves removing a constant number of elements from $S$, at most two calls to Subroutine~\algorithmfont{oneC4TwoSingles} (which has constant running time by Lemma~\ref{lem:threed_efr_as_wjenvy_max_degree_2_subroutine_oneC4TwoSingles_running_time}), and an assignment to $\hat{T}$ and $\hat{S}$ (which can be performed in $O(|N|)$ time). It follows that the total running time of the if/else statement is $O(|N|)$.

By Lemma~\ref{lem:threed_efr_as_wjenvy_max_degree_2_subroutine_oneC4TwoSingles_running_time}, Subroutine~\algorithmfont{multipleOfThreeC4s} has $O(|\mathcal{R}|) = O(|N|)$ running time. By Lemma~\ref{lem:threed_efr_as_wjenvy_max_degree_2_subroutine_configureSurplusAgents_running_time}, Subroutine~\algorithmfont{configureSurplusAgents} has $O(|S|) = O(|N|)$ running time. It follows that the asymptotic worst-case running time of Algorithm~\algorithmfont{wjPathsCycles} is $O(|N|)$.
\end{proof}

Having established that Algorithm~\algorithmfont{wjPathsCycles} is bound to terminate, we prove its correctness using a sequence of lemmas. First we show that if $(N, E)$ belongs to \iwjnomaxdegreetwofamily/ then the algorithm correctly identifies it as such.

\begin{lem}
\label{lem:threed_efr_as_wjenvy_max_degree_2_algo_valid_part1_ino}
If $(N, E)$ belongs to \iwjnomaxdegreetwofamily/ then Algorithm~\algorithmfont{wjPathsCycles} returns ``$(N, E)$ belongs to \iwjnomaxdegreetwofamily/''.
\end{lem}
\begin{proof}
Suppose $(N, E)$ belongs to \iwjnomaxdegreetwofamily/. In the algorithm, the set of connected components $\mathcal{C}$ that are not $4$-cycles contains exactly one element $C_1$ where $C_1$ contains a single agent $c_1$ in $(N, E)$. By Lemma~\ref{lem:threed_efr_as_wjenvy_max_degree_2_nonC4s}, Subroutine~\algorithmfont{nonC4Components} must return $(\varnothing, \{ c_1 \})$ so $T = \varnothing$ and $S = \{ c_1 \}$. Consider the outermost if/else statement in the algorithm. By Definition~\ref{def:threed_efr_as_wjef_max_deg_2_ino}, it must be that $3n = 4|\mathcal{R}| + 1$ so $4|\mathcal{R}| + 1 \bmod 3 = 0$. This implies that $4|\mathcal{R}| + 4 \bmod 3 = 0$ so $4(|\mathcal{R}| + 1) \bmod 3 = 0$. It follows that that $|\mathcal{R}| + 1 \bmod 3 = 0$ and thus that $|\mathcal{R}| \bmod 3 = 2$. It follows that the algorithm enters the first branch of the outermost \algorithmfont{if/else} statement. Since $|S| = 1 < 4$ and $T = \varnothing$ the algorithm must then return ``$(N, E)$ belongs to \iwjnomaxdegreetwofamily/''.
\end{proof}

We now consider the case in which $(N, E)$ does not belong to \iwjnomaxdegreetwofamily/. We first show that in this case Algorithm~\algorithmfont{wjPathsCycles} returns a matching.

\begin{lem}
\label{lem:threed_efr_as_wjenvy_max_degree_2_algo_valid_part1pointfive_pit}
If $(N, E)$ does not belong to \iwjnomaxdegreetwofamily/ then Algorithm~\algorithmfont{wjPathsCycles} returns a matching $M$.
\end{lem}
\begin{proof}
Consider an arbitrary connected component $C$ in $(N, E)$. We show that each agent in $C$ is added to exactly one triple in $M$. 

Suppose $C$ is not a $4$-cycle. By the design of Algorithm~\algorithmfont{wjPathsCycles}, exactly one call is made to Subroutine~\algorithmfont{nonC4} with argument $C$. Consider an arbitrary agent $c_i \in C$. There are two cases: either $i \leq \lfloor |C|/3 \rfloor$ or $i > \lfloor |C|/3 \rfloor$. In the former case, exactly one triple containing $c_i$ is added to $T$ in Subroutine~\algorithmfont{nonC4}, which is then added to $M$ in the main algorithm. In the latter case, $c_i$ is eventually added to $S$. We can see from the pseudocode of Subroutine~\algorithmfont{configureSurplusAgents} that $c_i$ is therefore eventually added to exactly one triple in $M$.

Suppose $C$ is a $4$-cycle, so by definition $C \in \mathcal{R}$. If $C \in \mathcal{R}'$ then each agent in $C$ is added to exactly one triple in $M$ in some call to Subroutine~\algorithmfont{multipleOfThreeC4s}. If $C \notin \mathcal{R}'$ then some call to Subroutine~\algorithmfont{oneC4TwoSingles} was made with the first argument equal to $C$ and the returned set of two triples was then added to $M$. It follows that each agent in each $4$-cycle is added to exactly one triple in $M$.
\end{proof} 

We now show that if $(N, E)$ does not belong to \iwjnomaxdegreetwofamily/ then the algorithm returns a matching $M$ that is wj-envy-free. In the next four lemmas we consider subsets of $N$ and show that in each subset no agent is wj-envied in $M$. The results of these lemmas are then combined in Lemma~\ref{lem:threed_efr_as_wjenvy_max_degree_2_algo_valid_part2_EF}, in which we show that if the algorithm returns a matching $M$ then $M$ is wj-envy-free.

\begin{lem}
\label{lem:threed_efr_as_wjenvy_max_deg_2_no_agent_in_bigcupT_is_wjenvied}
If Algorithm~\algorithmfont{wjPathsCycles} returns a matching $M$ then no agent in $\bigcup T$ is wj-envied in $M$.
\end{lem}
\begin{proof}
Suppose Algorithm~\algorithmfont{wjPathsCycles} has returned some matching~$M$. Consider an arbitrary triple $t \in T$. By the pseudocode of Algorithm~\algorithmfont{wjPathsCycles} there are two possibilities: either $t \in \hat{T}$ or $t$ was labelled $\hat{t}$. If $t \in \hat{T}$ then by Lemma~\ref{lem:threed_efr_as_max_degree_2_subgraph_nonC4s} no agent in $t$ is wj-envied in $M$. Suppose then that $t$ was labelled $\hat{t}$. By the pseudocode of Algorithm~\algorithmfont{wjPathsCycles}, for any agent $c_i$ in $\hat{t}$ it must be that some call was made to Subroutine~\algorithmfont{oneC4TwoSingles} in which the second or third argument was equal to $c_i$ and then the two triples returned by the subroutine were added to $M$. By Lemma~\ref{lem:threed_efr_as_max_degree_2_subgraph_oneC4TwoSingles}, it follows that no agent in $\hat{t}$ is wj-envied in $M$.
\end{proof}

\begin{lem}
\label{lem:threed_efr_as_wjenvy_max_deg_2_no_agent_in_Uhat_is_wjenvied}
If Algorithm~\algorithmfont{wjPathsCycles} returns a matching $M$ then no agent in $\hat{S}$ is wj-envied in $M$.
\end{lem}
\begin{proof}
Suppose Algorithm~\algorithmfont{wjPathsCycles} has returned some matching $M$ in which some agent $\alpha_i \in N$ has wj-envy for some $\hat{s}_{j_1} \in \hat{S}$. By the pseudocode, it must be that $M(\hat{s}_{j_1})$ contains three agents in $\hat{S}$ so we label $M(\hat{s}_{j_1}) = \{ \hat{s}_{j_1}, \hat{s}_{j_2}, \hat{s}_{j_3} \}$. Note that since $|\hat{S}| > 0$ it must be that $\hat{T} = T$.

Since $\alpha_i$ has wj-envy for $\hat{s}_{j_1}$ it must be that $u_{\alpha_i}(\{ \hat{s}_{j_2}, \hat{s}_{j_3} \}) \geq 1$ so without loss of generality assume that $\{ \alpha_i, \hat{s}_{j_2} \} \in E$.  We now consider two possibilities: $\alpha_i \in S$ and $\alpha_i \notin S$.

First, suppose $\alpha_i \in S$. If $\alpha_i \in \hat{S}$ then Lemma~\ref{lem:threed_efr_as_max_degree_2_subgraph_configureSurplusAgents} is contradicted, so it must be that $\alpha_i \in S \setminus \hat{S}$. By the pseudocode, $\alpha_i$ was labelled either $w_1$, $w_2$, $w_3$ or $w_4$ during algorithm execution and must belong to some set of agents returned by a call to Subroutine~\algorithmfont{pickLowDegree}. Since $\{ \alpha_i, \hat{s}_{j_2}  \} \subset S$ the degree of $\alpha_i$ in the subgraph induced by $S$ in $(N, E)$ is $1$. By the definition of Subroutine~\algorithmfont{pickLowDegree} it must be that the degree of each agent in $\hat{S}$ is also $1$. With this in mind, consider the call $\algorithmfont{configureSurplusAgents}(\hat{S})$. It must be that $|\mathcal{Q}| = |\hat{S}|/2$. It follows that $\mathcal{X} \subset \mathcal{Q}$. By the pseudocode of Subroutine~\algorithmfont{configureSurplusAgents}, We shall consider the values of the variables in Subroutine~\algorithmfont{configureRemainingAgents} inside this call. It must be that for each triple in the set of triples returned by this subroutine contains two agents that are adjacent in $(N, E)$. If $\{ \hat{s}_{j_2}, \hat{s}_{j_1} \} \in E$ or $\{ \hat{s}_{j_2}, \hat{s}_{j_3} \} \in E$ then the degree of $\hat{s}_{j_2}$ in the subgraph induced by $S$ in $(N, E)$ is $2$, which is a contradiction. It remains that $\{ \hat{s}_{j_1}, \hat{s}_{j_3} \} \in E$. Since $w_k$ has wj-envy for $\hat{s}_{j_1}$ it must be that $v_{\hat{s}_{j_3}}(\hat{s}_{k_1}) \geq v_{\hat{s}_{j_3}}(\hat{s}_{j_1}) = 1$. It follows that $\hat{s}_{j_3}$ has degree $2$ in the subgraph induced by $S$ in $(N, E)$, which is a contradiction.

Second, suppose $\alpha_i \notin S$. Since $\{ \alpha_i, \hat{s}_{j_2} \} \in E$, by the pseudocode of Algorithm~\algorithmfont{wjPathsCycles} it must be that $\alpha_i$ belongs to the same connected component in $(N, E)$ as $\hat{s}_{j_2}$. Since $\hat{s}_{j_2}\in S$ it must be that the connected component that contains $\alpha_i$ and $\hat{s}_{j_2}$ is not a $4$-cycle and belongs to $\mathcal{C}$. Since $\alpha_i \notin S$ it must also be that $M(\alpha_i) \in T$. Since $\hat{T} = T$ it follows that $M(\alpha_i) \in \hat{T}$. Lemma~\ref{lem:threed_efr_as_max_degree_2_subgraph_nonC4s_part0} now implies that $\alpha_i$ is not wj-envious in $M$, which is a contradiction.
\end{proof}

\begin{lem}
\label{lem:threed_efr_as_wjenvy_max_deg_2_no_agent_in_U_is_wjenvied}
If Algorithm~\algorithmfont{wjPathsCycles} returns a matching $M$ then no agent in $S$ is wj-envied in $M$.
\end{lem}
\begin{proof}
Suppose Algorithm~\algorithmfont{wjPathsCycles} has returned some matching $M$. Consider an arbitrary agent $s_i \in S$. If $s_i \in \hat{S}$ then by Lemma~\ref{lem:threed_efr_as_wjenvy_max_deg_2_no_agent_in_Uhat_is_wjenvied} it must be that $s_i$ is not wj-envied in $M$. 

Suppose then that $s_i \notin \hat{S}$. There are three cases: either $|\mathcal{R}|\bmod 3 = 2$, $|S|\geq 4$, and $s_i$ was labelled $w_1$, $w_2$, $w_3$, or $w_4$; $|\mathcal{R}|\bmod 3 = 2$, $|T| \geq 1$, and $s_i$ was labelled $w_1$; or $|\mathcal{R}|\bmod 3 = 1$ and $s_i$ was labelled either $w_1$ or $w_2$. In each of the three cases, some call was then made to Subroutine~\algorithmfont{oneC4TwoSingles} in which the second or third argument was equal to $s_i$ and then two triples returned by the subroutine were added to $M$. By Lemma~\ref{lem:threed_efr_as_max_degree_2_subgraph_oneC4TwoSingles} it follows that $s_i$ is not wj-envied in $M$.
\end{proof}

\begin{lem}
\label{lem:threed_efr_as_wjenvy_max_deg_2_no_agent_in_bigcupR_is_wjenvied}
If Algorithm~\algorithmfont{wjPathsCycles} returns a matching $M$ then no agent in $\bigcup \mathcal{R}$, i.e.\ in some $4$-cycle in $(N, E)$, is wj-envied in $M$.
\end{lem}
\begin{proof}
Consider an arbitrary $R_j \in \mathcal{R}$ where $1\leq j \leq |\mathcal{R}|$. We show that no agent in $R_j$ is wj-envied in $M$. In this case, let $l'$ be the final value assigned to the variable $l$ before the algorithm terminated. There are two possibilities: either $j>l'$ or $j\leq l'$.

Suppose $j>l'$. It must be that $R_j \in \mathcal{R}'$, by the construction of $\mathcal{R}'$ in Algorithm~\algorithmfont{wjPathsCycles}. By Lemma~\ref{lem:threed_efr_as_max_degree_2_subgraph_multipleOfThreeC4s} it follows that no agent in $R_j$ is wj-envied in $M$.

Suppose $j\leq l'$. By the design of Algorithm~\algorithmfont{wjPathsCycles} there are two possibilities: either $|\mathcal{R}| \bmod 3 = 2$ and $l'=2$, or $|\mathcal{R}| \bmod 3 = 1$ and $l'=1$. In either case, by the pseudocode of Algorithm~\algorithmfont{wjPathsCycles} it must be that some call to Subroutine~\algorithmfont{oneC4TwoSingles} was made with the first argument equal to $R_j$ and the returned set of two triples was then added to $M$. It follows by Lemma~\ref{lem:threed_efr_as_max_degree_2_subgraph_oneC4TwoSingles} that no agent in $R_j$ is wj-envied in $M$.
\end{proof}

\begin{lem}
\label{lem:threed_efr_as_wjenvy_max_degree_2_algo_valid_part2_EF}
If $(N, E)$ does not belong to \iwjnomaxdegreetwofamily/ then Algorithm~\algorithmfont{wjPathsCycles} returns a matching $M$ that is wj-envy-free.
\end{lem}
\begin{proof}
By definition, $\mathcal{C} \cup \mathcal{R}$ is the set of all connected components in $(N, E)$. By Lemma~\ref{lem:threed_efr_as_wjenvy_max_degree_2_algo_valid_part1pointfive_pit}, Algorithm~\algorithmfont{wjPathsCycles} returns a matching $M$ in $N$. By Proposition~\ref{prop:threed_efr_as_wjenvy_maxdeg2_algo_ucupbigcuppi}, the set of agents $\bigcup \mathcal{C} = S \cup \bigcup T$. By Lemma~\ref{lem:threed_efr_as_wjenvy_max_deg_2_no_agent_in_bigcupT_is_wjenvied}, no agent in $\bigcup T$ is wj-envied in $M$. By Lemma~\ref{lem:threed_efr_as_wjenvy_max_deg_2_no_agent_in_U_is_wjenvied}, no agent in $S$ is wj-envied in $M$. By Lemma~\ref{lem:threed_efr_as_wjenvy_max_deg_2_no_agent_in_bigcupR_is_wjenvied}, no agent in $\bigcup \mathcal{R}$ is wj-envied in $M$.
\end{proof}

We now prove our main theorem on wj-envy-free matchings and instances with binary and symmetric preferences and maximum degree $2$.

\begin{thm}
\label{thm:threed_efr_as_wjef_algowjpathscycles}
Consider an instance of 3DR-AS with binary and symmetric preferences and maximum degree $2$. There exists an $O(|N|)$-time algorithm that can either find a wj-envy-free matching in the instance or report that the instance belongs to \iwjnomaxdegreetwofamily/, and thus contains no wj-envy-free matching.
\end{thm}
\begin{proof}
Lemma~\ref{lem:threed_efr_as_wjenvy_max_degree_2_algo_valid_runningtime} shows that Algorithm~\algorithmfont{wjPathsCycles} terminates in $O(|N|)$ time. Lemmas~\ref{lem:threed_efr_as_wjenvy_max_degree_2_algo_valid_part1_ino} and~\ref{lem:threed_efr_as_wjenvy_max_degree_2_algo_valid_part2_EF} establish the correctness of this algorithm and show that the algorithm either returns a wj-envy-free matching or reports that ``$(N, E)$ belongs to \iwjnomaxdegreetwofamily/''. In the latter case, Lemma~\ref{lem:threed_efr_as_wjef_maxdeg2_ino} shows that the supplied instance contains no wj-envy-free matching.
\end{proof}



\subsection{Symmetric binary preferences with maximum degree three}

As before in Theorem~\ref{thm:threed_efr_as_regularenvy_npcomplete}, in this section we consider instances of 3DR-AS with binary and symmetric preferences and maximum degree $3$. Here we show that, in contrast with Theorem~\ref{thm:threed_efr_as_wjef_algowjpathscycles}, deciding if a given instance of 3DR-AS contains a wj-envy-free matching is $\NP$-complete, even when preferences are binary and symmetric and the maximum degree is $3$.

As before in Section~\ref{sec:threed_efr_as_envyfreeness_maxdeg3}, we reduce from \porschenxsatvariant/ (Problem~\ref{pr:xsatvariant}). Here we assume that the number of clauses $m$ satisfies $m=4l$ for some $l\geq 1$. We can show that the wj-envy-free existence problem remains $\NP$-complete under this restriction as follows. Construct four distinct copies of the set of variables $X(C)$ and formula $C$. Let the constructed formula $C'$ be the union of the four copies of $C$. It is straightforward to show that $C'$ is exactly satisfiable if and only if each of the four copies is exactly satisfiable, which is true if and only if the original formula $C$ is exactly satisfiable.

The main difference between this reduction and the reduction for the 3DR-AS envy-free decision problem is in the second direction, in which we show that $C$ is exactly satisfiable if a wj-envy-free matching $M$ exists in the constructed instance $(N, E)$. Here we associate true literals with variable gadgets that belong to one triple (rather than three) and false literals with variable gadgets that belong to three triples (rather than one). Another difference is that we construct a number of garbage collector gadgets. We will show that there is only one possible configuration of the triples of agents in the garbage collector gadgets in a wj-envy-free matching.

% Given an instance $C$ of \porschenxsatvariant/, the reduction constructs an instance of 3DR-AS with binary and symmetric preferences $(N, E)$ such that $C$ is exactly satisfiable if and only if a wj-envy-free matching exists in $(N, E)$. 

The reduction, illustrated in Figure~\ref{fig:threed_efr_as_wj_envy_free_reduction}, is as follows. Suppose $C$ is an arbitrary instance of \porschenxsatvariant/. We shall construct an instance $(N, E)$ of 3DR-AS.

For each variable $x_i \in X(C)$ construct a set of three agents $W_i = \{ w_i^1, w_i^2, w_i^3 \}$, which we refer to as the \emph{$i\textsuperscript{th}$ variable gadget}. Add the edges $\{ w_i^1, w_i^2 \}$, $\{ w_i^2, w_i^3 \}$, and $\{ w_i^3, w_i^1 \}$ to $E$. Next, for each clause $c_r$ in $C$ construct a set of four agents $D_r = \{ d_r^1, d_r^2, d_r^3, d_r^4 \}$,  which we refer to as the \emph{$r\textsuperscript{th}$ clause gadget}. Add the edges $\{ d_r^1, d_r^4 \}$, $\{ d_r^2, d_r^4 \}$, and $\{ d_r^3, d_r^4 \}$. Recall that $3$ divides $m$ and $m=4l$ for some integer $l>1$. Construct a set of $12l$ agents labelled $g_1, g_2, \dots, g_{12l}$. For any $i$ where $1\leq i\leq 3l$, we shall refer to $G_i = \{ g_{4i-3}, g_{4i-2}, g_{4i-1}, g_{4i} \}$ as the \emph{$i\textsuperscript{th}$ garbage collector gadget}. For each $1\leq i\leq 3l$, add the edges $\{ g_{4i}, g_{4i-1} \}$, $\{ g_{4i}, g_{4i-2} \}$, and $\{ g_{4i}, g_{4i-3} \}$ to $E$. Note that each garbage collector gadget $G_i$ is an isolated claw $K_{1,3}$. 
We shall connect the variable and clause gadgets are connected in the way as the reduction in Section~\ref{sec:threed_efr_as_envyfreeness_maxdeg3}. Consider each clause $c_r = \{ x_i, x_j, x_k \}$. If $c_r$ contains the $j\textsuperscript{th}$ occurrence of $x_i$ then add the edge $\{ d_r^1, w_i^j \}$. Similarly, add an edge between $d_r^2$ and an agent in $W_j$ depending on the index of the occurrence of $x_j$ in the clause $c_r$ and an edge between $d_r^3$ and an agent in $W_k$ depending on the index of the occurrence of $x_k$ in the clause $c_r$.

This completes the construction of $(N, E)$. Note that each agent in a variable gadget has degree $3$, $d_r^1, d_r^2, d_r^3$ for each $1 \leq r \leq m$ have degree $2$, $d_r^4$ for each $1 \leq r \leq m$ has degree $3$, $g_i$ where $1\leq i\leq 12l$ and $i$ is not divisible by four has degree $1$, and $g_j$ where $1\leq j\leq 12l$ and $j$ is divisible by four has degree $3$. It follows that the maximum degree of $(N, E)$ is $3$.

It is straightforward to show that this reduction can be performed in polynomial time. To prove that the reduction is correct we show that the 3DR-AS instance $(N, E)$ contains a wj-envy-free matching if and only if the \porschenxsatvariant/ instance $C$ is exactly satisfiable.

\begin{figure}
    \centering
    \input{figures/threed_efr_as/wj_envy_free_reduction.tikz}
    \caption[The reduction from \porschenxsatvariant/ to the problem of deciding if a given instance of 3DR-AS contains an wj-envy-free matching]{The reduction from \porschenxsatvariant/ to the problem of deciding if a given instance of 3DR-AS contains an wj-envy-free matching. A variable gadget $W_i$, clause gadget $D_r$, and garbage collector gadget $G_i$ are represented as undirected graphs.}
    \label{fig:threed_efr_as_wj_envy_free_reduction}
\end{figure}

We first show that if the \porschenxsatvariant/ instance $C$ is exactly satisfiable then the 3DR-AS instance $(N, E)$ contains a wj-envy-free matching.

\begin{lem}
\label{lem:threed_efr_as_wjenvy_firstdirection}
If $C$ is exactly satisfiable then $(N, E)$ contains a wj-envy-free matching.
\end{lem}
\begin{proof}
Suppose $f$ is an exact model in $C$. We shall construct a matching $M$ in $(N, E)$ that is wj-envy-free. For each variable $x_i$ in $X(C)$ where $f(x_i)$ is true, add $\{ w_i^1, w_i^2, w_i^3 \}$ to $M$. Next, consider each clause $c_r = \{ x_i, x_j, x_k \}$ and the corresponding clause gadget $D_r$, labelling $i, j, k$ such that $W_i$ contains an agent adjacent to $d_r^1$, $W_j$ contains an agent adjacent to $d_r^2$, and $W_k$ contains an agent adjacent to $d_r^3$. There are three cases: $f(x_i)$ is true while both $f(x_j)$ and $f(x_k)$ are false, $f(x_j)$ is true while both $f(x_i)$ and $f(x_k)$ are false, and $f(x_k)$ is true while both $f(x_i)$ and $f(x_j)$ are false. In the first case, suppose $c_r$ contains the $a\textsuperscript{th}$ occurrence of $x_j$ and the $b\textsuperscript{th}$ occurrence of $x_k$. Add the triples $\{ d_r^1, d_r^4, g_{3r} \}$, $\{ d_r^2, w_j^a, g_{3r-1} \}$, and $\{ d_r^3, w_k^b, g_{3r-2} \}$. The constructions in the second and third cases are symmetric: in the second case, suppose $c_r$ contains the $a\textsuperscript{th}$ occurrence of $x_i$ and the $b\textsuperscript{th}$ occurrence of $x_k$. Add the triples $\{ d_r^2, d_r^4, g_{3r} \}$, $\{ d_r^1, w_i^a, g_{3r-1} \}$, and $\{ d_r^3, w_k^b, g_{3r-2} \}$. In the third case, suppose $c_r$ contains the $a\textsuperscript{th}$ occurrence of $x_i$ and the $b\textsuperscript{th}$ occurrence of $x_j$. Add the triples $\{ d_r^3, d_r^4, g_{3r} \}$, $\{ d_r^1, w_i^a, g_{3r-1} \}$, and $\{ d_r^2, w_j^b, g_{3r-2} \}$. 

Note that for any triple $t\in M$, either $t=W_i$ for some variable gadget $W_i$; $t=\{ d_r^4, d_r^a, g_{3r} \}$ where $1\leq r\leq m$ and $1\leq a \leq 3$; or $t=\{ d_r^a, w_i^b, g_j \}$ where $1\leq i, r \leq m$, $1\leq a, b \leq 3$ and $1\leq j\leq 12l$ where $j$ is not divisible by three. We shall show that in each case $t$ does not contain an agent with wj-envy.

First, consider some triple $t\in M$ where $t=W_i$ for some variable gadget $W_i$. Since each agent in $t$ has utility $2$, no agent in $t$ is envious.

Second, consider some triple $t\in M$ where $t=\{ d_r^4, d_r^a, g_{3r} \}$, $1\leq r\leq m$, and $1\leq a \leq 3$. Since $M(d_r^1)\neq M(d_r^2)$, $M(d_r^1)\neq M(d_r^3)$, and $M(d_r^2)\neq M(d_r^3)$, it must be that $\sigma(N(d_r^4), M)=3$. Since $u_{d_r^4}(M)=1$, it follows by Lemma~\ref{lem:threed_efr_as_util1_neighbourhoodthreetriples} that $d_r^4$ is not envious. Similarly, since $M(d_r^a)=\{ d_r^4, d_r^a, g_{3r} \}$ it follows that $\sigma(N(d_r^a), M)=2$ so since $u_{d_r^a}(M)=1$ by Lemma~\ref{lem:threed_efr_as_util1_neighbourhoodthreetriples} it must be that $d_r^a$ is also not envious. Suppose for a contradiction that $g_{3r}$ wj-envies some agent $\alpha_j$. It must be that $u_{g_{3r}}(M(\alpha_j) \setminus \{ \alpha_j \})\geq 1$ so $M(\alpha_j)$ contains some $g_q$ where $\{ g_{3r}, g_q \} \in E$. Let $M(\alpha_j) = \{ \alpha_j, g_q, \alpha_k \}$. By construction of $M$, it must be that $\alpha_j, \alpha_k \subset D_s$ where $1\leq s\leq m$ and $u_{\alpha_j}(M)=u_{\alpha_k}(M)=1$. It follows that $\{ \alpha_k, g_q \} \notin E$ so $u_{\alpha_k}(M) < u_{\alpha_k}(\{ g_q, g_{3r} \})$, which contradicts the supposition that $g_{3r}$ wj-envies $\alpha_j$.

Third, consider some triple $t\in M$ where $t=\{ d_r^a, w_i^b, g_j \}$,  $1\leq i, r \leq m$, $a \in \{1, 2, 3 \}$, $b \in \{ 1, 2, 3 \}$ and $1\leq j\leq 12l$ where $j$ is not divisible by three. By construction, $\sigma(N(w_i^b), M)=3$ so by Lemma~\ref{lem:threed_efr_as_util1_neighbourhoodthreetriples} $w_i^b$ is not envious. Similarly, since $M(d_r^a)=\{ d_r^a, w_i^b, g_{j} \}$ it follows that $\sigma(N(d_r^a), M)=2$ so since $u_{d_r^a}(M)=1$ by Lemma~\ref{lem:threed_efr_as_util1_neighbourhoodthreetriples} it must be that $d_r^a$ is also not envious. As before, suppose for a contradiction that $g_{j}$ wj-envies some agent $\alpha_k$. It must be that $u_{g_{j}}(M(\alpha_k) \setminus \{ \alpha_k \})\geq 1$ so $M(\alpha_k)$ contains some $g_q$ where $\{ g_{j}, g_q \} \in E$. Let $M(\alpha_k) = \{ \alpha_k, g_q, \alpha_h \}$. By construction of $M$, it must be that $\alpha_k, \alpha_h \subset D_s$ for some $s$ where $1\leq s\leq m$ and $u_{\alpha_k}(M)=u_{\alpha_h}(M)=1$. It follows that $\{ \alpha_h, g_q \} \notin E$ so $u_{\alpha_h}(M) < u_{\alpha_h}(\{ g_q, g_{j} \})$, which contradicts the supposition that $g_{j}$ wj-envies $\alpha_k$.
\end{proof}

We now show, using a sequence of lemmas, that if the 3DR-AS instance $(N, E)$ contains an wj-envy-free matching then the \porschenxsatvariant/ instance $C$ is exactly satisfiable.

\begin{lem}
\label{lem:threed_efr_as_wjenvy_seconddirection_triangle_split_stay}
If $(N, E)$ contains an envy-free matching $M$ then for any variable gadget $W_i$, either $\sigma(W_i, M)=1$ or $\sigma(W_i, M)=3$.
\end{lem}
\begin{proof}
The proof is similar to Lemma~\ref{lem:threed_efr_as_regularenvy_seconddirection_triangle_split_stay}. If some triple in $M$ contains exactly two agents in $W_i$ then the third agent in $W_i$ is wj-envious.
\end{proof}

\begin{lem}
\label{lem:threed_efr_as_wjenvy_giscores0}
If $(N, E)$ contains an envy-free matching $M$ then $u_{g_i}(M)=0$ for any $i$ where $1\leq i\leq 12l$.
\end{lem}
\begin{proof}
Suppose $M$ is a wj-envy-free matching. By construction, the structure of $G_i$ for each $1\leq i\leq 3l$ is identical, so to simplify the proof we assume, without loss of generality, that $i=1$ and $G_i=\{ g_1, g_2, g_3, g_4 \}$. We shall prove that $\sigma(G_1, M)=4$, from which it follows directly that $u_{g_1}(M)=u_{g_2}(M)=u_{g_3}(M)=u_{g_4}(M)=0$. Since $|G_1|=4$ clearly $\sigma(G_1, M) \leq 4$. Suppose for a contradiction that $\sigma(G_1, M) \leq 3$. Then there exists two agents $g_a, g_b \in G_1$ where $g_a \in M(g_b)$. Label the third agent in $M(g_b)$ as $\alpha_j$. By symmetry, we need only consider two cases. In the first case, $a=1$ and $b=4$. In the second case, $a=1$ and $b=2$. We will show that in both cases it is relatively straightforward to identify an envious agent, which is a contradiction.

First, suppose $a=1$ and $b=4$. Since $\{ g_1, g_4 \} \subset M(g_4)$, by construction it must be that either $u_{g_2}(M)=0$ or $u_{g_3}(M)=0$. Assume without loss of generality that $u_{g_2}(M)=0$. It follows that $g_2$ wj-envies $g_1$, since $u_{g_4}(\{ g_1, \alpha_j \}) = u_{g_4}(\{ g_2, \alpha_j \})$ and, by construction of $G_1$, $u_{\alpha_j}(\{ g_4, g_1 \}) = u_{\alpha_j}(\{ g_4, g_2 \})$. 

Second, suppose $a=1$ and $b=2$. There are two cases: $g_4=\alpha_j$ or $g_4\neq \alpha_j$. If $g_4=\alpha_j$ then $g_3$ wj-envies $g_2$, since $u_{g_1}(\{ g_2, g_4 \}) = u_{g_1}(\{ g_3, g_4 \})$ and $u_{g_4}(\{ g_1, g_2 \}) = u_{g_4}(\{ g_2, g_3 \})$. If $g_4\neq \alpha_j$ then $u_{g_4}(M)\leq 1$ so $g_4$ wj-envies $\alpha_j$, since $u_{g_4}(\{ g_1, g_2 \})=2$, $u_{g_1}(M)=u_{g_2}(M)=0$, and $u_{g_1}(\{ g_2, g_4 \})=u_{g_2}(\{ g_1, g_4 \})=1$.
\end{proof}

\begin{lem}
\label{lem:threed_efr_as_wjenvy_gi011}
If $(N, E)$ contains an envy-free matching $M$ then $M(g_i) = \{ g_i, \alpha_a, \alpha_b \}$ where $\{ \alpha_a, \alpha_b \} \in E$ for any $1\leq i\leq 12l$.
\end{lem}
\begin{proof}
Suppose $M$ is a wj-envy-free matching. Consider an arbitrary $g_i$ where $1\leq i \leq 12l$. Let $g_j$ be some agent for which $\{ g_i, g_j \} \in E$. By Lemma~\ref{lem:threed_efr_as_wjenvy_giscores0}, $u_{g_i}(M)=u_{g_j}(M)=0$. It remains to show $\{ \alpha_a, \alpha_b \} \in E$. Suppose for a contradiction that $\{ \alpha_a, \alpha_b \} \notin E$. It follows that $u_{\alpha_a}(M)=u_{\alpha_b}(M)=u_{g_i}(M)=0$. Now $g_j$ wj-envies $\alpha_a$, since $u_{g_j}(\{ g_i, \alpha_b \})=1$, $u_{\alpha_b}(\{ g_j, g_i \})=u_{\alpha_b}(M)$, and $u_{g_i}(\{ g_j, \alpha_b \})=1$, which is a contradiction.
\end{proof}

\begin{lem}
\label{lem:threed_efr_as_3triplesperclausegadget}
Suppose $(N, E)$ contains a wj-envy-free matching $M$. For each $1\leq r \leq m$, there exist exactly three triples $t_1, t_2, t_3$ that each contains one agent in $G$ and at least one agent $d_r^w \in D_r$ where $u_{d_r^w}(M)=1$.
\end{lem}
\begin{proof}
By Lemma~\ref{lem:threed_efr_as_wjenvy_gi011}, $M(g_i) = \{ g_i, \alpha_a, \alpha_b \}$ where $\{ \alpha_a, \alpha_b \} \in E$ for any $i$ where $1\leq i\leq 12l$. It follows that there exists $12l$ triples in $M$ of the form $\{ g_i, \alpha_a, \alpha_b \}$ where $\{ \alpha_a, \alpha_b \} \in E$. Let $T$ be this set of triples. Since $\{ \alpha_a, \alpha_b \} \in E$, it must be that $u_{\alpha_a}(M)=u_{\alpha_b}(M)=1$. It follows that $\alpha_a \notin G$ and $\alpha_b \notin G$. It must be that for any $\{ g_i, \alpha_a, \alpha_b \} \in T$ there exists some $1\leq r\leq m$ where either $\alpha_a \in D_r$ or $\alpha_b \in D_r$, for otherwise the only possibility is that $\{ \alpha_a, \alpha_b \} \subset W_i$ for some variable gadget $W_i$, which would contradict Lemma~\ref{lem:threed_efr_as_wjenvy_seconddirection_triangle_split_stay}. We have now shown that each $t\in T$ comprises $\{ g_i, d_r^w, \alpha_b \}$ where $u_{d_r^w}(M)=u_{\alpha_b}(M)=1$ for some $1\leq r\leq m$, $1\leq w \leq 4$ and $\alpha_b\in N$. It remains to show that for a given $r$, there exist exactly three triples in $T$ where each triple contains at least one agent in $D_r$. 

Suppose for a contradiction that there exists some $1\leq r\leq m$ where the number of triples in $T$ that contain an agent in $D_r$ is not three. Recall that $|T|=12l$, $m=4l$, and each $t\in T$ contains at least one agent $d_r^w$ where $1\leq r\leq m$ and $1\leq w\leq 4$. A counting argument shows that there must exist some $1\leq s\leq m$ where there are at least four triples $t_1,t_2,t_3,t_4 \in T$ that each contain at least one agent in $D_s$. Without loss of generality, it must be that $d_s^1\in t_1$, $d_s^2\in t_2$, $d_s^3\in t_3$, and $d_s^4\in t_4$. Since we have previously established that $u_{d_s^4}(M)=1$ this leads to a contradiction since $N(d_s^4)=\{ d_s^1, d_s^2, d_s^3 \}$ but $d_s^1\notin t_4$, $d_s^2\notin t_4$, and $d_s^3\notin t_4$. It follows that for each $1\leq r\leq m$, there exists exactly three triples in $T$ where each triple contains at least one agent in $D_r$.
\end{proof}

\begin{lem}
\label{lem:threed_efr_as_wjenvy_seconddirection}
If $(N, E)$ contains an envy-free matching then $C$ is exactly satisfiable.
\end{lem}
\begin{proof}
Suppose $M$ is a wj-envy-free matching in $(N, E)$. By Lemma~\ref{lem:threed_efr_as_wjenvy_seconddirection_triangle_split_stay}, for any variable gadget $W_i$ either $\sigma(W_i, M)=1$ or $\sigma(W_i, M)=3$. Construct a truth assignment $f$ in $C$ by setting $f(x_i)$ to be true if $\sigma(W_i, M)=1$ and false otherwise. Each variable $x_i$ corresponds to exactly one variable gadget $W_i$ so it follows that $f$ is a valid truth assignment. By the construction of $(N, E)$, each clause $c_r$ corresponds to exactly one clause gadget $D_r$. Each clause gadget is adjacent to three variable gadgets that correspond to the three variables in that clause. To show that $f$ is an exact model of $C$, it is sufficient to show that for each clause gadget $D_r$ there exists exactly one variable gadget $W_i$ such that $D_r$ is adjacent to $W_i$ and $\sigma(W_i, M)=1$.

Consider an arbitrary clause gadget $D_r$ and the corresponding clause $c_r=\{ x_i, x_j, x_k\}$, labelling $i, j, k$ such that $d_r^1$ is adjacent to some agent $w_i^{a_1} \in W_i$, $d_r^2$ is adjacent to some agent $w_j^{a_2} \in W_j$ and $d_r^3$ is adjacent to some agent $w_k^{a_3} \in W_k$. By Lemma~\ref{lem:threed_efr_as_3triplesperclausegadget}, there exists exactly three triples $t_1, t_2, t_3$ such that $t_1$ contains one agent $g_{h_1}\in G$ and at least one agent $d_r^{b_1} \in D_r$ where $u_{d_r^{b_1}}(M)=1$; $t_2$ contains one agent $g_{h_2}\in G$ and at least one agent $d_r^{b_2} \in D_r$ where $u_{d_r^{b_2}}(M)=1$; and $t_3$ contains one agent in $g_{h_3}\in G$ and at least one agent $d_r^{b_3} \in D_r$ where $u_{d_r^{b_3}}(M)=1$. Consider $d_r^4$. If $u_{d_r^4}(M)=0$ then $\{ b_1, b_2, b_3 \} = \{ 1, 2, 3 \}$ so without loss of generality we may assume that  $t_1 = \{ d_r^1, w_i^{a_1}, g_{h_1} \}$, $t_2 = \{ d_r^2, w_j^{a_2}, g_{h_2} \}$, and $t_3 = \{ d_r^3, w_k^{a_3}, g_{h_3} \}$. In this configuration $d_r^4$ wj-envies $w_i^{a_1}$ since $u_{d_r^4}(\{ d_r^1, g_{h_1} \})=1$, $u_{d_r^1}(\{ d_r^4, g_{h_1} \}) = u_{d_r^1}(M)$, and $u_{g_{h_1}}(\{ d_r^4, d_r^1 \}) = u_{g_{h_1}}(M) = 0$. It follows that $u_{d_r^4}(M)>0$. Recalling our earlier observation (in this proof) on the contents of $t_1$, $t_2$, and $t_3$ it must be that $u_{d_r^4}(M)<2$ and so $u_{d_r^4}(M)=1$. There are three possible cases: either $d_r^1 \in M(d_r^4)$, $d_r^2 \in M(d_r^4)$, or $d_r^3 \in M(d_r^4)$. By the symmetry of the clause gadget, we describe only the case in which $d_r^1 \in M(d_r^4)$. Without loss of generality we may assume that $t_1 = \{ d_r^1, d_r^4, g_h^1 \}$. Furthermore, we may assume that $t_2 = \{ d_r^2, w_j^{a_2}, g_{h_2} \}$ and $t_3 = \{ d_r^3, w_k^{a_3}, g_{h_3} \}$. Consider $w_i^{a_1}$. If $u_{w_i^{a_1}}(M)=0$ then $w_i^{a_1}$ wj-envies $d_r^4$, since $u_{w_i^{a_1}}(\{ d_r^1, g_{h_1} \})=1$, $u_{d_r^1}(\{ w_i^{a_1}, g_{h_1} \}) = u_{d_r^1}(M)$, and $u_{g_{h_1}}(\{ w_i^{a_1}, d_r^1 \}) = u_{g_{h_1}}(M) = 0$. It must be that $u_{w_i^{a_1}}(M)>0$. If $u_{w_i^{a_1}}(M)=1$ then Lemma~\ref{lem:threed_efr_as_wjenvy_seconddirection_triangle_split_stay} is contradicted. It follows that $u_{w_i^{a_1}}(M)=2$ and thus $\sigma(W_i, M)=1$. Now consider $w_j^{a_2}$ and $w_k^{a_3}$. Since $u_{w_j^{a_2}}(M)=1$ it must be that $\sigma(W_j, M)=3$. Similarly, since $u_{w_k^{a_3}}(M)=1$ it must be that $\sigma(W_k, M)=3$. To recap, after we supposed that $d_r^1 \in M(d_r^4)$, we showed that $\sigma(W_i, M)=1$ and $\sigma(W_j, M)=\sigma(W_k, M)=3$, as required. The cases in which $d_r^2 \in M(d_r^4)$ or $d_r^3 \in M(d_r^4)$ are symmetric.
\end{proof}

We have now shown that the 3DR-AS instance $(N, E)$ contains a wj-envy-free matching if and only if the \porschenxsatvariant/ instance $C$ is exactly satisfiable. This shows that the reduction is correct.

\begin{thm}
\label{thm:threed_efr_as_wjef_npcomplete}
Deciding if a given instance of 3DR-AS contains a wj-envy-free matching is $\NP$-complete, even when preferences are binary and symmetric and maximum degree is $3$.
\end{thm}
\begin{proof}
It is straightforward to show that this decision problem belongs to $\NP$, since for any two agents $\alpha_i, \alpha_j \in N$ we can test if $\alpha_i$ wj-envies $\alpha_j$ in constant time. 

We have presented a polynomial-time reduction from \porschenxsatvariant/, which is $\NP$-complete \cite{PSSW14}. Given an arbitrary instance $C$ of \porschenxsatvariant/, the reduction constructs an instance $(N, E)$ of 3DR-AS with binary and symmetric preferences and maximum degree $3$. Lemmas~\ref{lem:threed_efr_as_wjenvy_firstdirection} and~\ref{lem:threed_efr_as_wjenvy_seconddirection} show that $(N, E)$ contains a wj-envy-free matching if and only if $C$ is exactly satisfiable and thus that this decision problem is $\NP$-hard.
\end{proof}



\section{Justified envy-freeness}
\label{sec:threed_efr_as_jef}
\subsection{Binary preferences}
It is straightforward to show that if a matching is stable then it is j-envy-free. A corollary is that if we are given an instance of 3DR-AS in which preferences are binary and symmetric then we can find a j-envy-free matching by finding a stable matching, which, as we showed in Chapter~\ref{c:threed_sr_as}, is bound to exist and can be found in polynomial time (Theorem~\ref{thm:threed_sr_as_symmetric_binary_construction} in Chapter~\ref{c:threed_sr_as}). We state this corollary as Observation~\ref{obs:threed_efr_as_jef_binary_symmetric_from_stability}.

\begin{observation}
\label{obs:threed_efr_as_jef_binary_symmetric_from_stability}
Given an instance of 3DR-AS with binary and symmetric preferences, a j-envy-free matching always exists and can be found in $O(|N|^3)$ time.
\end{observation}

We showed in Chapter~\ref{c:threed_sr_as} that when preferences are binary but not necessarily symmetric, a stable matching need not exist and the associated decision problem is $\NP$-complete. In contrast, we show that a j-envy-free matching is bound to exist and can be found in polynomial time.

\begin{thm}
\label{thm:threed_efr_as_jef_binary_algorithm}
Given an instance of 3DR-AS with binary preferences, a j-envy-free matching must exist and can be found in $O(|N|^3)$ time.
\end{thm}
\begin{proof}
Suppose $(N, V)$ is an instance of 3DR-AS with binary preferences. We describe an $O(|N|^3)$-time algorithm that can construct a j-envy-free matching $M$ in $(N, V)$, as follows.

First, the algorithm constructs from $(N, V)$ another instance $(N, V')$ of 3DR-AS, which has binary and symmetric preferences, by deleting asymmetric arcs.  Next, the algorithm calls Algorithm~\algorithmfont{findStableUW}, shown in Algorithm~\ref{alg:threed_sr_as_approximationalgo}, in Chapter~\ref{c:threed_sr_as}, on the instance $(N, V')$ to construct a matching $M$, which is returned.

Since $(N, V')$ can be constructed in $O(|N|^2)$ time, and the worst-case time complexity of Algorithm~\algorithmfont{findStableUW} is $O(|N|^3)$ (by Theorem~\ref{thm:threed_sr_as_approxratio} in Chapter~\ref{c:threed_sr_as}), the worst-case time complexity of this algorithm is also $O(|N|^3)$. 

It remains to show that $M$ is j-envy-free in $(N, V)$. In the proof, which follows, we refer to the variables and subroutines in Algorithm~\algorithmfont{findStableUW}. We shall also write $v$ in the context of the instance $(N, V')$, which has binary preferences, and write $\mathit{v'}$ in the context of the instance $(N, V)$, which has binary and symmetric preferences. Similarly, we shall use $u'_{\alpha_i}(M)$ to denote $u'_{\alpha_i}(S) = \sum_{{\alpha_j}\in S \setminus \{ \alpha_i \}} \mathit{v'}_{\alpha_i}({\alpha_j})$ for some set of agents $S\subseteq N$.

Note that by the design of Algorithm~\algorithmfont{findStableUW}, $M$ meets the following condition: if there exists a pair of agents $\alpha_i, \alpha_j \in N$ where $\mathit{v'}_{\alpha_i}(\alpha_j) = 1$ and $u'_{\alpha_i}(M)=u'_{\alpha_j}(M)=0$ then no triple contains three agents each with utility $0$ in $M$. 

% the edge j1-j2 is present
For a contradiction, suppose there exists some $\alpha_{i_1}\in N$ where $\alpha_{i_1}$ has j-envy in $M$ for some $\alpha_{j_1}\in N \setminus \{ \alpha_{i_1} \}$. 

Let $M(\alpha_{i_1}) = \{ \alpha_{i_1}, \alpha_{i_2}, \alpha_{i_3} \}$ and $M(\alpha_{j_1}) = \{ \alpha_{j_1}, \alpha_{j_2}, \alpha_{j_3} \}$. Since $\alpha_{i_1}$ has j-envy for $\alpha_{j_1}$ it must be that $u_{\alpha_{i_1}}(\{ \alpha_{j_2}, \alpha_{j_3} \}) > 0$ so either $v_{\alpha_{i_1}}(\alpha_{j_2})=1$, $v_{\alpha_{i_1}}(\alpha_{j_3})=1$, or both. Assume without loss of generality that $v_{\alpha_{i_1}}(\alpha_{j_2})=1$. Note that it must also be that $u_{\alpha_{j_2}}(\{ \alpha_{i_1}, \alpha_{j_3} \}) > u_{\alpha_{j_2}}(M)$ and $u_{\alpha_{j_3}}(\{ \alpha_{i_1}, \alpha_{j_2} \}) > u_{\alpha_{j_3}}(M)$. It follows that $v_{\alpha_{j_2}}(\alpha_{i_1}) > v_{\alpha_{j_2}}(\alpha_{j_1})$ and $v_{\alpha_{j_3}}(\alpha_{i_1}) > v_{\alpha_{j_3}}(\alpha_{j_1})$. Since preferences are binary, it must be that $v_{\alpha_{j_2}}(\alpha_{i_1})=v_{\alpha_{j_3}}(\alpha_{i_1})=1$ and $v_{\alpha_{j_2}}(\alpha_{j_1})=v_{\alpha_{j_3}}(\alpha_{j_1})=0$. It follows by the construction of $(N, V')$ that $\mathit{v'}_{\alpha_{j_2}}(\alpha_{j_1})=\mathit{v'}_{\alpha_{j_3}}(\alpha_{j_1})=0$. Since $v_{\alpha_{i_1}}(\alpha_{j_2})=1$ and $v_{\alpha_{j_2}}(\alpha_{i_1})=1$ it also follows that $\mathit{v'}_{\alpha_{i_1}}(\alpha_{j_2})=1$.

We first claim that $\{ \alpha_{j_1}, \alpha_{j_2}, \alpha_{j_3} \} \notin M_1$. Since $v_{\alpha_{j_2}}(\alpha_{j_1})=v_{\alpha_{j_3}}(\alpha_{j_1})=0$, it follows that $\mathit{v'}_{\alpha_{j_2}}(\alpha_{j_1})=\mathit{v'}_{\alpha_{j_3}}(\alpha_{j_1})=0$ and thus that $u'_{\alpha_{j_1}}(M_1)=0$. Since $M_1$ is a $P$\nobreakdash-matching, no triple in $M_1$ contains an agent with utility $0$, which shows that $\{ \alpha_{j_1}, \alpha_{j_2}, \alpha_{j_3} \}\notin M_1$ as required. 

Note that it follows that $\{ \alpha_{j_1}, \alpha_{j_2}, \alpha_{j_3} \} \subseteq U$ and thus that $u'_{\alpha_{j_2}}(M_1)=u'_{\alpha_{j_3}}(M_1)=0$.

Next, we claim that $u'_{\alpha_{i_1}}(M)=0$. Since $\alpha_{i_1}$ envies $\alpha_{j_1}$ in $(N, V)$ it must be that $u_{\alpha_{i_1}}(M)<2$ and thus that $u'_{\alpha_{i_1}}(M)<2$. It remains that $u'_{\alpha_{i_1}}(M) \in \{ 0, 1 \}$. Suppose for a contradiction that $u'_{\alpha_{i_1}}(M)=1$ and without loss of generality that $\mathit{v'}_{\alpha_{i_1}}(\alpha_{i_2})=1$. It follows that $v_{\alpha_{i_1}}(\alpha_{i_2})=1$ and thus that $u_{\alpha_{i_1}}(M) \geq 1$. In fact, since $\alpha_{i_1}$ has j-envy for $\alpha_{j_1}$ it must be that $u_{\alpha_{i_1}}(M) = 1$. Moreover, since $u_{\alpha_{i_1}}(\{ \alpha_{j_2}, \alpha_{j_3} \}) > u_{\alpha_{i_1}}(M)=1$, it must be that $v_{\alpha_{i_1}}(\alpha_{j_2})=v_{\alpha_{i_1}}(\alpha_{j_3})=1$. Recall that since $\alpha_{i_1}$ has justified envy for $\alpha_{j_1}$, it must be that $v_{\alpha_{j_2}}(\alpha_{j_1})=v_{\alpha_{j_3}}(\alpha_{j_1})=0$ and $v_{\alpha_{j_2}}(\alpha_{i_1})=v_{\alpha_{j_3}}(\alpha_{i_1})=1$. We have now shown that $\mathit{v'}_{\alpha_{j_2}}(\alpha_{i_1})=\mathit{v'}_{\alpha_{j_3}}(\alpha_{i_1})=1$. Recall our earlier note that $u'_{\alpha_{j_2}}(M_1)=u'_{\alpha_{j_3}}(M_1)=0$. Now, the triple $\{ \alpha_{i_1}, \alpha_{j_2}, \alpha_{j_3} \}$ blocks $M_1$ in $(N, V')$, since $u'_{\alpha_{i_1}}(M_1) = 1 < 2 = u'_{\alpha_{i_1}}(\{ \alpha_{j_2}, \alpha_{j_3} \})$, $u'_{\alpha_{j_2}}(M_1) = 0 < 1 \leq u'_{\alpha_{j_2}}(\{ \alpha_{i_1}, \alpha_{j_3} \})$, and $u'_{\alpha_{j_3}}(M_1) = 0 < 1 \leq u'_{\alpha_{j_3}}(\{ \alpha_{i_1}, \alpha_{j_2} \})$. Since $M_1$ is a stable matching in $(N, V')$, by the correctness of Algorithm~\algorithmfont{findStableUW} (shown in Theorem~\ref{thm:threed_sr_as_approxratio} in Chapter~\ref{c:threed_sr_as}) this is a contradiction. It remains that $u'_{\alpha_{i_1}}(M)=0$, as required.

Note that since $M_1$ is a $P$\nobreakdash-matching and $u'_{\alpha_{i_1}}(M)=0$ it must be that $\alpha_{i_1}$ is unmatched in $M_1$ so $\alpha_{i_1} \in U$.

We finally claim that $\mathit{v'}_{\alpha_{j_2}}(\alpha_{j_3})=0$. Suppose for a contradiction that $\mathit{v'}_{\alpha_{j_2}}(\alpha_{j_3}) = 1$. We established earlier that $\alpha_{j_1}, \alpha_{j_2}, \alpha_{j_3}$ are unmatched in $M_1$ so $u'_{\alpha_{j_2}}(M_1)=u'_{\alpha_{j_3}}(M_1)=0$. We also established that $u'_{\alpha_{i_1}}(M)=0$ and thus that $u'_{\alpha_{i_1}}(M_1)=0$. Recall also that, by assumption, $\mathit{v'}_{\alpha_{i_1}}(\alpha_{j_2})=1$, so since $\mathit{v'}_{\alpha_{j_2}}(\alpha_{j_3})=1$ it now follows that $\{ \alpha_{i_1}, \alpha_{j_2}, \alpha_{j_3} \}$ blocks $M_1$ in $(N, V')$, which is a contradiction since $M_1$ is stable in $(N, V')$. It remains that $\mathit{v'}_{\alpha_{j_2}}(\alpha_{j_3})=0$, as required.

Since $\mathit{v'}_{\alpha_{j_1}}(\alpha_{j_3})=0$, $\mathit{v'}_{\alpha_{j_2}}(\alpha_{j_1})=0$, and $\mathit{v'}_{\alpha_{j_2}}(\alpha_{j_3})=0$ it follows that $u'_{\alpha_{j_1}}(M)=u'_{\alpha_{j_2}}(M)=u'_{\alpha_{j_3}}(M)=0$. This violates our stated condition on $M$, since there exists a pair of agents $\alpha_{i_1}, \alpha_{j_2} \in N$ where $\mathit{v'}_{\alpha_{i_1}}(\alpha_{j_2})$ and $u'_{\alpha_{i_1}}(M)=u'_{\alpha_{j_2}}(M)=0$, but each agent in the triple $\{ \alpha_{j_1}, \alpha_{j_2}, \alpha_{j_3} \}$ has utility $0$ in $M$, which is a contradiction.
\end{proof}

\subsection{Ternary preferences}
\label{sec:threed_efr_as_jef_ternary}
A natural next step would be to ask if the polynomial-time algorithm described in Theorem~\ref{thm:threed_efr_as_jef_binary_algorithm} can be extended to the setting in which preferences are ternary, i.e.\ $v_{\alpha_i}(\alpha_j) \in \{ 0, 1, 2 \}$. We show that, assuming $\P \neq \NP$, this is not the case, and the problem of deciding if a given instance of 3DR-AS contains a j-envy-free matching is $\NP$-complete, even when preferences are ternary.

We present a polynomial-time reduction from a special case of \emph{Directed Triangle Packing} (DTP, Problem~\ref{pr:DTP}).

\begin{myproblem}[Directed Triangle Packing (DTP)]
\label{pr:DTP}\mysymbolfirstusedefinition{symboldef:dtp}{}
\begin{samepage}
\begin{adjustwidth}{8pt}{8pt}
\inp a simple directed graph $G=(W, A)$ where $W=\{ w_1, w_2, \dots, w_{3q} \}$ for some integer $q$\\
\ques Can the vertices of $G$ be partitioned into $q$ disjoint sets $X=\{X_1, X_2, \dots, X_q\}$, each set containing exactly three vertices, such that each $X_p=\{ w_i,w_j,w_k \}$ in $X$ is a directed $3$-cycle, i.e.\ the arcs $( w_i,w_j )$, $( w_j, w_k )$, and $( w_k, w_i )$ belong to $A$?
\end{adjustwidth}
\end{samepage}
\end{myproblem}

We claim that DTP is $\NP$-complete, even when $G$ is antisymmetric (i.e.\ it contains no bidirectional arcs). As noted by Cechl\'arov\'a, Fleiner, and Manlove \cite{CFM05}, the proof of this claim can be obtained using a simple modification to the reduction presented by Garey and Johnson for Partition Into Triangles \cite[Theorem~3.7]{GJ79}. It is this restricted variant of DTP, in which $G$ is antisymmetric, that we reduce from to show that deciding if a given instance of 3DR-AS with ternary (but not necessarily symmetric) preferences contains a j-envy-free matching.

We shall first describe the reduction from DTP in detail and then provide some intuition with respect to its design. 

The reduction, shown in Figure~\ref{fig:threed_efr_as_jef_terasym_reduction}, is as follows. Suppose $G=(W, A)$ is an arbitrary instance of DTP. We shall construct an instance $(N, V)$ of 3DR-AS. Unless otherwise specified, assume that $v_{\alpha_i}(\alpha_j)=0$ for any $\alpha_i, \alpha_j \in N$. To simplify the description of the valuations in the reduction, in this section we write $i \myoplus y$ to denote $((i + y - 1) \bmod 5) + 1$.

First construct a set of five agents $H = \{ h_1, h_2, h_3, h_4, h_5 \}$. For each $i$ where $1\leq i \leq 5$ let $v_{h_i}(h_{i \myoplus 1}) = v_{h_{i \myoplus 1}}(h_i) = 1$, $v_{h_i}(h_{i \myoplus 3}) = 1$, and $v_{h_i}(h_{i \myoplus 2}) = 2$. Next, construct a set $L = \{ l_1, l_2, l_3, l_4 \}$ of four agents. Let $v_{l_1}(l_2) = v_{l_2}(l_1) = v_{l_3}(l_4) = v_{l_4}(l_3) = 2$ and $v_{l_1}(l_3) = v_{l_1}(l_4) = v_{l_2}(l_3) = v_{l_2}(l_4) = v_{l_3}(l_1) = v_{l_3}(l_2) = v_{l_4}(l_1) = v_{l_4}(l_2) = 1$. Next, construct a set $C = \{ c_1, c_2, \dots, c_{3q} \}$ of $3q$ agents. For each $i$ where $1\leq i \leq 3q$ let $v_{c_i}(l_3) = v_{l_3}(c_i) = v_{l_4}(c_i) = 1$ and $v_{c_i}(l_4) = 2$. For each $i$ and $j$ where $1\leq i, j \leq 3q$ let $v_{c_i}(c_j) = 2$ if $( w_i, w_j ) \in A$ otherwise $1$. This completes the construction of $(N, V)$. Note that the structure of the valuations among the agents in $C$ reflects the directed graph $G$. 

We make some remarks on the design of the constructed instance. The design of $H$ is derived from a particular instance that contains no j-envy-free matching. This instance comprises $H$ as well as a single isolated agent $\alpha_{z}$, where $v_{\alpha_{z}}(h_i) = v_{h_i}(\alpha_{z}) = 0$ for each $i$ where $1\leq i \leq 5$. In fact, the proof that this instance contains no j-envy-free matching can be directly derived from the proof of a lemma that appears later in this section (Lemma~\ref{lem:threed_efr_as_jef_terasym_atleasttwotriplescontainoneagentinH}).

It is straightforward to show that the reduction runs in polynomial time. To prove that the reduction is correct we show that the 3DR-AS instance $(N, V)$ contains a j-envy-free matching if and only if the DTP instance $G$ contains a directed triangle packing.

\begin{figure}
    \centering
    \input{figures/threed_efr_as/j_envy_free_terasym_reduction.tikz}
    \caption{The reduction from DTP to the problem of deciding if a given instance of 3DR-AS with ternary preferences contains a j-envy-free matching}
    \label{fig:threed_efr_as_jef_terasym_reduction}
\end{figure}

We first show that if the DTP instance $G$ contains a directed triangle packing then the 3DR-AS instance $(N, V)$ contains a j-envy-free matching.

\begin{lem}
\label{lem:threed_efr_as_jef_terasym_first_direction}
If $G$ contains a directed triangle packing then $(N, V)$ contains a j-envy-free matching.
\end{lem}
\begin{proof}
Suppose $G$ contains a directed triangle packing $X = \{ X_1, X_2, \dots, X_q \}$. We shall construct a matching $M$ that is j-envy-free. First, add $\{ h_1, h_2, h_3 \}$, $\{ h_4, l_1, l_2 \}$, and $\{ h_5, l_3, l_4 \}$ to $M$. Next, for each directed $3$-cycle $X_p = \{ w_i, w_j, w_k \}$ in $X$, add $\{ c_i, c_j, c_k \}$ to $M$.

Suppose for a contradiction that some agent $\alpha_j$ exists where $\alpha_j$ has j-envy for some other agent $\alpha_{k_1}$ where $M(\alpha_{k_1}) = \{ \alpha_{k_1}, \alpha_{k_2}, \alpha_{k_3} \}$. Since $N = H \cup L \cup C$ it must be that either $\alpha_{k_1} \in H$, $\alpha_{k_1} \in L$, or $\alpha_{k_1} \in C$. We show that each case leads to a contradiction. It follows that no such $\alpha_j$ exists and thus that $M$ is j-envy-free.
\begin{itemize}
    \item Suppose $\alpha_{k_1} \in H$. By the construction of $M$ there are two possibilities: either $\alpha_{k_1} \in \{ h_1, h_2, h_3 \}$ or $\alpha_{k_1} \in \{ h_4, h_5 \}$. 
    \begin{itemize}
    \item Suppose firstly that $\alpha_{k_1} \in \{ h_4, h_5 \}$ then by the construction of $M$ either $\{ \alpha_{k_2}, \alpha_{k_3} \} = \{ l_1, l_2 \}$ or $\{ \alpha_{k_2}, \alpha_{k_3} \} = \{ l_3, l_4 \}$. Note that $u_{l_1}(M) = u_{l_2}(M) = u_{l_3}(M) = u_{l_4}(M) = 2$. Since $u_{l_1}(\{ l_3, l_4 \}) = 2$ and $u_{l_2}(\{ l_3, l_4 \}) = 2$, neither $l_1$ nor $l_2$ has j-envy for $\alpha_{k_1}$, so $\alpha_j \notin \{ l_1, l_2 \}$. Similarly, since $u_{l_3}(\{ l_1, l_2 \}) = 2$ and $u_{l_4}(\{ l_1, l_2 \}) = 2$ neither $l_3$ nor $l_4$ has j-envy for $\alpha_{k_1}$, so $\alpha_j \notin \{ l_3, l_4 \}$. Since $u_{c_i}(M) = 3$, $u_{c_i}(\{ l_1, l_2 \}) = 0$, and $u_{c_i}(\{ l_1, l_2 \}) = 2$ for any $i$ where $1\leq i \leq 3q$, it must be no agent in $C$ has j-envy for $\alpha_{k_1}$, so $\alpha_j \notin C$. It remains that $\alpha_{j} \in H$. Since this implies $v_{l_1}(\alpha_j) = v_{l_2}(\alpha_j) = v_{l_3}(\alpha_j) = v_{l_4}(\alpha_j) = 0$ it follows that $\alpha_j$ does not have j-envy for $\alpha_{k_1}$ and thus that $\alpha_{k_1} \notin \{ h_4, h_5 \}$.
    
    \item Suppose then that $\alpha_{k_1} \in \{ h_1, h_2, h_3 \}$. Since $\alpha_j$ has j-envy for $\alpha_{k_1}$ it must be that $v_{\alpha_j}(\alpha_{k_2}) \geq 1$ so it follows that $\alpha_j \in \{ h_4, h_5 \}$. If $\alpha_{k_1} = h_1$ and $\alpha_j = h_4$ then we reach a contradiction since $h_4$ has j-envy for $h_1$ but $v_{h_3}(h_1) = 1 = v_{h_3}(h_4)$. Similarly, if $\alpha_{k_1} = h_1$ and $\alpha_j = h_5$ then we reach a contradiction since $v_{h_2}(h_1) = 1 = v_{h_2}(h_5)$. If $\alpha_{k_1} = h_2$ or $\alpha_{k_1} = h_3$ then we also reach a contradiction since $v_{h_1}(h_4) = v_{h_1}(h_5) = 1 = v_{h_1}(h_2) < 2 = v_{h_1}(h_3)$.
    \end{itemize}
    
    \item Suppose $\alpha_{k_1} \in C$. By the construction of $M$ it must be that $\alpha_{k_2} \in C$ and $\alpha_{k_3} \in C$ so we label $\alpha_{k_1} = c_{i_1}$, $\alpha_{k_2} = c_{i_2}$, and $\alpha_{k_3} = c_{i_3}$. By the construction of $(N, V)$ in the reduction it follows that $v_{c_{i_2}}(c_{i_1}) \geq 1$ and $v_{c_{i_3}}(c_{i_1}) \geq 1$. Since $\alpha_j$ has j-envy for $c_{i_1}$ it follows then that $v_{c_{i_2}}(\alpha_j) = 2$ and $v_{c_{i_3}}(\alpha_j) = 2$. By the construction of the instance there are two possibilities: either $\alpha_j = l_4$ or $\alpha_j \in C$. If $\alpha_j = l_4$ then $u_{l_4}(\{ c_{i_2}, c_{i_3} \}) = 2$ which is a contradiction since by assumption $l_4$ has j-envy for $c_{i_1}$ but $u_{l_4}(M) = 2$. If $\alpha_j \in C$ then label $\alpha_j = c_{i_4}$. Since $v_{c_{i_2}}(c_{i_4}) = 2$ and $v_{c_{i_3}}(c_{i_4}) = 2$, by the construction of $C$ it must be that $( w_{i_2}, w_{i_4} ) \in A$ and $( w_{i_3}, w_{i_4} ) \in A$, where the vertices $w_{i_2}, w_{i_3}, w_{i_4}$ are the vertices in $W$ that correspond respectively to the agents $c_{i_2}, c_{i_3}, c_{i_4}$ in $C$. Since $G$ is antisymmetric, it follows that $( w_{i_4}, w_{i_2} ) \notin A$ and $( w_{i_4}, w_{i_3} ) \notin A$ so it must be that $v_{c_{i_4}}(c_{i_2}) = v_{c_{i_4}}(c_{i_3}) = 1$. This is also a contradiction since by assumption $c_{i_4}$ has j-envy for $c_{i_1}$ but $u_{c_{i_4}}(\{ c_{i_2}, c_{i_3} \}) = 2$ and by the construction of $M$ it must be that $u_{c_{i_4}}(M) = 3$.
    
    \item Suppose $\alpha_{k_1} \in L$. It must be that $\alpha_{k_1} = l_{i_1}$, $\alpha_{k_2} = l_{i_2}$, and $\alpha_{k_3} = h_{i_3}$, for some $i_1, i_2$ where $1 \leq i_1, i_2 \leq 4$ and $i_3 \in \{ 4, 5 \}$. If $\alpha_j \in H$ then it must be that $v_{l_{i_2}}(\alpha_j) = 0$ which contradicts the supposition that $\alpha_j$ has j-envy for $l_{i_1}$. Otherwise, if $\alpha_j \notin H$ then $v_{h_{i_3}}(\alpha_j) = 0$, which also contradicts the supposition that $\alpha_j$ has j-envy for $l_{i_1}$.
\end{itemize}
\end{proof}

We now show that if the 3DR-AS instance $(N, V)$ contains a j-envy-free matching then the DTP instance $G$ contains a directed triangle packing.

\begin{lem}
\label{lem:threed_efr_as_jef_terasym_hbelongsto3}
If $(N, V)$ contains a j-envy-free matching $M$ then $\sigma(H, M) \geq 3$.
\end{lem}
\begin{proof}
Since $|H|=5$ it must be that $\sigma(H, M) \geq 2$. Suppose for a contradiction that $\sigma(H, M) = 2$. It must be that one triple in $M$ contains three agents in $H$ and one triple in $M$ contains two agents in $H$. Suppose the former triple is $\{ h_{i_1}, h_{i_2}, h_{i_3} \}$ and the latter triple is $\{ h_{i_4}, h_{i_5}, \alpha_j \}$, where $1\leq i_1, i_2, \dots, i_5 \leq 5$ and $\alpha_j \in N \setminus H$. There are five symmetries in $H$ and $\binom{5}{2}=10$ possible assignments of $\{ h_{i_4}, h_{i_5} \}$ to two agents in $H$, so we need only consider the two assignments $i_4 = 1$, $i_5 = 2$ and $i_4 = 1$, $i_5 = 3$, which are not symmetric. If $i_4 = 1$ and $i_5 = 2$ then it remains that $\{ i_1, i_2, i_3 \} = \{ 3, 4, 5 \}$. In this case, $h_5$ has j-envy for $\alpha_j$ since $u_{h_5}(M) = 2 < 3 \leq u_{h_5}(\{ h_1, h_2 \})$, $v_{h_1}(\alpha_j) = 0 < 1 = v_{h_1}(h_5)$, and $v_{h_2}(\alpha_j) = 0 < 1 = v_{h_2}(h_5)$. If $i_4 = 1$ and $i_5 = 3$ then it remains that $\{ i_1, i_2, i_3 \} = \{ 2, 4, 5 \}$. In this case, $h_4$ has j-envy for $\alpha_j$ since $u_{h_4}(M) = 2 < 3 \leq u_{h_4}(\{ h_1, h_3 \})$, $v_{h_1}(\alpha_j) = 0 < 1 = v_{h_1}(h_4)$, and $v_{h_3}(\alpha_j) = 0 < 1 = v_{h_3}(h_4)$. 
\end{proof}

\begin{lem}
\label{lem:threed_efr_as_jef_terasym_atleasttwotriplescontainoneagentinH}
If $(N, V)$ contains a j-envy-free matching $M$  then at least two triples in $M$ each contain exactly one agent in $H$.
\end{lem}
\begin{proof}
By Lemma~\ref{lem:threed_efr_as_jef_terasym_hbelongsto3}, $\sigma(H, M) \geq 3$. If, contrary to the lemma statement, at most one triple in $M$ contains exactly one agent in $H$ then it must be that two triples in $M$ each contain two agents in $H$ and one triple in $M$ contains exactly one agent in $H$. Suppose one of the two former triples is $\{ h_{i_1}, h_{i_2}, \alpha_{j_1} \}$ and the latter triple is $\{ h_{i_3}, \alpha_{j_2}, \alpha_{j_3} \}$, where $1\leq i_1, i_2, i_3 \leq 5$ and $\alpha_{j_1}, \alpha_{j_2}, \alpha_{j_3} \in N \setminus H$. By the construction of the instance it must be that $v_{h_{i_1}}(\alpha_{j_1}) = v_{h_{i_2}}(\alpha_{j_1}) = 0$,  $v_{h_{i_1}}(h_{i_3}) \geq 1$ and $v_{h_{i_2}}(h_{i_3}) \geq 1$. It follows that $h_{i_3}$ has j-envy for $\alpha_{j_1}$ since $u_{h_{i_3}}(M) = 0 < 2 \leq u_{h_{i_3}}(\{ h_{i_1}, h_{i_2} \})$, $v_{h_{i_1}}(\alpha_j) = 0 < 1 \leq v_{h_{i_1}}(h_{i_3})$, and $v_{h_{i_2}}(\alpha_j) = 0 < 1 \leq v_{h_{i_2}}(h_{i_3})$. This contradicts the supposition that $M$ is j-envy-free.
\end{proof}

We have shown in Lemma~\ref{lem:threed_efr_as_jef_terasym_atleasttwotriplescontainoneagentinH} that if $(N, V)$ contains a j-envy-free matching $M$ then at least two triples in $M$ each contain exactly one agent in $H$. Suppose $t_{\beta}, t_{\gamma} \in M$ are two such triples and $t_{\beta} = \{ h_{a_1}, \alpha_{b_1}, \alpha_{b_2} \}$ and $t_{\gamma} = \{ h_{a_2}, \alpha_{b_3}, \alpha_{b_4} \}$.

\begin{lem}
\label{lem:threed_efr_as_jef_terasym_lequalsthefourisolated}
If $(N, V)$ contains a j-envy-free matching then $\{ \alpha_{b_1}, \alpha_{b_2}, \alpha_{b_3}, \alpha_{b_4} \} = L$.
\end{lem}
\begin{proof}
Suppose for a contradiction that $\{ \alpha_{b_1}, \alpha_{b_2}, \alpha_{b_3}, \alpha_{b_4} \} \neq L$.

By definition, $\{ \alpha_{b_1}, \alpha_{b_2}, \alpha_{b_3}, \alpha_{b_4} \} \cap H = \varnothing$ and $\{ \alpha_{b_1}, \alpha_{b_2}, \alpha_{b_3}, \alpha_{b_4} \} \neq L$ it must be that at least one agent in $\{ \alpha_{b_1}, \alpha_{b_2}, \alpha_{b_3}, \alpha_{b_4} \}$ belongs to $C$. Assume without loss of generality that $\alpha_{b_1} \in C$.

Note that by construction of the instance, the valuation of any agent not in $H$ for any other agent not in $H$ is at least $1$. 

Since $\alpha_{b_2} \notin H$, it must be that $u_{\alpha_{b_2}}(M) = v_{\alpha_{b_2}}(\alpha_{b_1})$. By the design of the instance, since $\alpha_{b_2} \notin H$ and $\alpha_{b_1} \notin H$ it must be that $v_{\alpha_{b_2}}(\alpha_{b_1}) \in \{ 1, 2 \}$. We consider each possibility of $u_{\alpha_{b_2}}(M) = v_{\alpha_{b_2}}(\alpha_{b_1})$.

Firstly, suppose $u_{\alpha_{b_2}}(M) = 1$. As noted earlier in this proof, since $\alpha_{b_2}, \alpha_{b_3}, \alpha_{b_4} \in N \setminus H$ it must be that $v_{\alpha_{b_2}}(\alpha_{b_3}) \geq 1$ and $v_{\alpha_{b_2}}(\alpha_{b_4}) \geq 1$. It follows that $\alpha_{b_2}$ has j-envy for $h_{a_2}$, since $u_{\alpha_{b_2}}(M) = 1 < 2 \leq u_{\alpha_{b_2}}(\{ \alpha_{b_3}, \alpha_{b_4} \})$, $v_{\alpha_{b_3}}(h_{a_2}) = 0 < 1 \leq v_{\alpha_{b_3}}(\alpha_{b_2})$, and $v_{\alpha_{b_4}}(h_{a_2}) = 0 < 1 \leq v_{\alpha_{b_4}}(\alpha_{b_2})$. This contradicts the supposition that $M$ is j-envy-free.

Suppose then that $u_{\alpha_{b_2}}(M) = 2$, so $v_{\alpha_{b_2}}(\alpha_{b_1}) = 2$. Since $\alpha_{b_1} \in C$ by assumption, by the design of the instance it must be $\alpha_{b_2} \in C$. For the remainder of this lemma only, label $\alpha_{b_1} = c_{i_1}$ and $\alpha_{b_2} = c_{i_2}$. Since $v_{c_{i_2}}(c_{i_1}) = 2$ it follows that $( w_{i_2}, w_{i_1} ) \in A$. Since $G$ is antisymmetric it must be that $( w_{i_1}, w_{i_2} ) \notin A$ and thus that $v_{c_{i_1}}(c_{i_2}) = 1$. Since $M(c_{i_1}) = \{ c_{i_1}, c_{i_2}, h_{a_1} \}$ it follows that $u_{c_{i_1}}(M) = v_{c_{i_1}}(c_{i_2}) = 1$. Now $\alpha_{b_1}$ has j-envy for $h_{a_2}$, since $u_{\alpha_{b_1}}(M) = 1 < 2 \leq u_{\alpha_{b_1}}(\{ \alpha_{b_3}, \alpha_{b_4} \})$, $v_{\alpha_{b_3}}(h_{a_2}) = 0 < 1 \leq v_{\alpha_{b_3}}(\alpha_{b_1})$, and $v_{\alpha_{b_4}}(h_{a_2}) = 0 < 1 \leq v_{\alpha_{b_4}}(\alpha_{b_1})$. This contradicts the supposition that $M$ is j-envy-free.
\end{proof}

\begin{lem}
\label{lem:threed_efr_as_jef_terasym_structureofL}
If $(N, V)$ contains a j-envy-free matching then $\{ \{ \alpha_{b_1}, \alpha_{b_2} \}, \{ \alpha_{b_3}, \alpha_{b_4} \} \} = \{ \{ l_1, l_2 \}, \{ l_3, l_4 \} \}$.
\end{lem}
\begin{proof}
By Lemma~\ref{lem:threed_efr_as_jef_terasym_lequalsthefourisolated}, $\{ \alpha_{b_1}, \alpha_{b_2}, \alpha_{b_3}, \alpha_{b_4} \} = L$. There are now three possibilities: first that $\{ \{ \alpha_{b_1}, \alpha_{b_2} \}, \{ \alpha_{b_3}, \alpha_{b_4} \} \} = \{ \{ l_1, l_3 \}, \{ l_2, l_4 \} \}$, second that $\{ \{ \alpha_{b_1}, \alpha_{b_2} \}, \{ \alpha_{b_3}, \alpha_{b_4} \} \} = \{ \{ l_1, l_4 \}, \{ l_2, l_3 \} \}$, and third that $\{ \{ \alpha_{b_1}, \alpha_{b_2} \}, \{ \alpha_{b_3}, \alpha_{b_4} \} \} = \{ \{ l_1, l_2 \}, \{ l_3, l_4 \} \}$.

First suppose $\{ \{ \alpha_{b_1}, \alpha_{b_2} \}, \{ \alpha_{b_3}, \alpha_{b_4} \} \} = \{ \{ l_1, l_3 \}, \{ l_2, l_4 \} \}$. Without loss of generality assume that $\alpha_{b_1} = l_1$. Now $l_1$ has j-envy for $h_{a_2}$ since $u_{l_1}(\{ h_{a_1}, l_3 \}) = 1 < 3 = u_{l_1}(\{ l_2, l_4 \})$, $v_{l_2}(h_{a_2}) = 0 < 2 = v_{l_2}(l_1)$, and $v_{l_4}(h_{a_2}) = 0 < 1 = v_{l_4}(l_1)$.

Second suppose $\{ \{ \alpha_{b_1}, \alpha_{b_2} \}, \{ \alpha_{b_3}, \alpha_{b_4} \} \} = \{ \{ l_1, l_4 \}, \{ l_2, l_3 \} \}$. Without loss of generality assume that $\alpha_{b_1} = l_1$. As before, $l_1$ has j-envy for $h_{a_2}$ since $u_{l_1}(\{ h_{a_1}, l_4 \}) = 1 < 3 = u_{l_1}(\{ l_2, l_3 \})$, $v_{l_2}(h_{a_2}) = 0 < 2 = v_{l_2}(l_1)$, and $v_{l_3}(h_{a_2}) = 0 < 1 = v_{l_3}(l_1)$.

It remains that $\{ \{ \alpha_{b_1}, \alpha_{b_2} \}, \{ \alpha_{b_3}, \alpha_{b_4} \} \} = \{ \{ l_1, l_2 \}, \{ l_3, l_4 \} \}$.
\end{proof}

By Lemma~\ref{lem:threed_efr_as_jef_terasym_structureofL}, either $\{ \alpha_{b_1}, \alpha_{b_2} \} = \{ l_1, l_2 \}$ or $\{ \alpha_{b_1}, \alpha_{b_2} \} = \{ l_3, l_4 \}$. Without loss of generality assume that $\{ \alpha_{b_1}, \alpha_{b_2} \} = \{ l_3, l_4 \}$.

\begin{lem}
\label{lem:threed_efr_as_jef_terasym_eachcigets3}
If $(N, V)$ contains a j-envy-free matching $M$ then $u_{c_i}(M) \geq 3$ for each $i$ where $1\leq i \leq 3q$.
\end{lem}
\begin{proof}
Suppose to the contrary that some $1\leq i \leq 3q$ exists where $u_{c_i}(M) < 3$. Then $c_i$ has j-envy for $h_{a_1}$ since $u_{c_i}(M) \leq 2 < 3 = u_{c_i}(\{ l_3, l_4 \})$, $v_{l_3}(h_{a_1}) = 0 < 1 = v_{l_3}(c_i)$, and $v_{l_4}(h_{a_1}) = 0 < 1 = v_{l_4}(c_i)$. This contradicts our supposition that $M$ is j-envy-free.
\end{proof}

\begin{lem}
\label{lem:threed_efr_as_jef_terasym_second_direction}
If $(N, V)$ contains a j-envy-free matching $M$ then $G$ contains a directed triangle packing.
\end{lem}
\begin{proof}
Suppose $(N, V)$ contains a j-envy-free matching $M$. Lemma~\ref{lem:threed_efr_as_jef_terasym_eachcigets3} shows that $u_{c_i}(M) \geq 3$ for each $i$ where $1\leq i \leq 3q$. By construction, it follows that $M(c_i)$ contains two agents $c_j, c_k$ such that $v_{c_i}(c_j) \geq 1$ and $v_{c_i}(c_k) = 2$. Hence $c_k$ corresponds to a vertex $w_k \in W$ where $( w_i, w_k ) \in A$ and, since $G$ is antisymmetric, $( w_k, w_i ) \notin A$. Since $c_i$ was chosen arbitrarily it follows that $\{ w_i, w_j, w_k \}$ is a directed $3$-cycle in $G$. It follows thus that there are exactly $q$ triples in $M$ each containing three agents $\{ c_i, c_j, c_k \}$ where the three corresponding vertices $\{ w_i, w_j, w_k \}$ form a directed $3$-cycle in $G$. From these triples a directed triangle packing $X$ can be easily constructed.
\end{proof}

% \paragraph{Correctness of the reduction: conclusion}

We have now shown that the 3DR-AS instance $(N, V)$ contains a j-envy-free matching if and only if the DTP instance $G$ contains a directed triangle packing. This shows that the reduction is correct.

\begin{thm}
\label{thm:threed_efr_as_jef_terasym_npcomplete}
Deciding if a given instance of 3DR-AS contains a j-envy-free matching is $\NP$-complete, even when preferences are ternary.
\end{thm}
\begin{proof}
It is straightforward to show that this decision problem belongs to $\NP$, since for any two agents $\alpha_i, \alpha_j \in N$ we can test if $\alpha_i$ j-envies $\alpha_j$ in constant time. 

We have presented a polynomial-time reduction from a restricted version of Directed Triangle Packing (DTP), which is $\NP$-complete \cite{CFM05}. Given a directed antisymmetric graph $G$, the reduction constructs an instance $(N, V)$ of 3DR-AS with ternary preferences. Lemmas~\ref{lem:threed_efr_as_jef_terasym_first_direction} and~\ref{lem:threed_efr_as_jef_terasym_second_direction} show that $(N, V)$ contains a j-envy-free matching if and only if $G$ contains a directed triangle packing and thus that this decision problem is $\NP$-hard.
\end{proof}

\subsection{Symmetric non-binary preferences}
\label{sec:threed_efr_as_jef_symmetricnonbinrary}
From Theorems~\ref{thm:threed_efr_as_jef_binary_algorithm} and~\ref{thm:threed_efr_as_jef_terasym_npcomplete}, a natural question arises: is it the symmetry of agents' preferences that guarantees the existence of a j-envy-free matching? 
In this section we show that this is not the case, and a j-envy-free matching may not exist even when agents' preferences are symmetric, and the associated existence problem is $\NP$-complete. We remark, however, that the instances shown that do not contain j-envy-free matchings are relatively contrived and involve non-negative integer valuations up to $6$. We leave open the case in which preferences are symmetric and the maximum valuation is strictly less than $6$.

To show that this existence problem is $\NP$-complete, we present a polynomial-time reduction from \emph{Partition into Triangles} (PIT, Problem~\ref{prob:pit}), which is $\NP$-complete \cite{GJ79}. This reduction is similar to the reduction we presented in Section~\ref{sec:threed_efr_as_jef_ternary} for the analogous problem involving ternary preferences that are not (necessarily) symmetric. Note that section we reduced from Directed Triangle Packing (DTP, Problem~\ref{pr:DTP}) but here we reduce from PIT.

We describe the reduction in detail and then provide some intuition with respect to its design. The reduction, illustrated in Figure~\ref{fig:threed_efr_as_jef_symmetric_reduction}, is as follows. Suppose $G$ is an arbitrary instance of PIT. We shall construct an instance $(N, V)$ of 3DR-AS with symmetric preferences and maximum valuation $6$. Since valuations are symmetric in $(N, V)$, we shall usually specify valuations in one direction only. For example, instead of writing ``let $v_{\alpha_i}(\alpha_j)=v_{\alpha_j}(\alpha_i)=1$'' we write ``let $v_{\alpha_i}(\alpha_j)=1$''. Unless otherwise specified assume that $v_{\alpha_i}(\alpha_j)=0$ for any $\alpha_i, \alpha_j \in N$. To simplify the description of the valuations in the reduction, in this section we write $i \myoplus y$ to denote $((i + y - 2) \bmod 10) + 2$. 

First, construct a set of eleven agents $H = \{ h_1, h_2, \dots, h_{11} \}$. For each $i$ where $2\leq i \leq 11$ let $v_{h_1}(h_i) = 2$. For each $i$ where $2\leq i \leq 11$, let:
\begin{itemize}
    \item $v_{h_i}(h_{i \myoplus 1}) = 4$ if $i$ is even otherwise $5$
    \item $v_{h_i}(h_{i \myoplus 2}) = 6$ if $i$ is even otherwise $3$
    \item $v_{h_i}(h_{i \myoplus 3}) = 1$
    \item $v_{h_i}(h_{i \myoplus 4}) = 1$
    \item $v_{h_i}(h_{i \myoplus 5}) = 3$.
\end{itemize}
Next, construct a set of four agents $L = \{ l_1, l_2, l_3, l_4 \}$. Let $v_{l_1}(l_2) = v_{l_3}(l_4) = 2$ and $v_{l_1}(l_3) = v_{l_1}(l_4) = v_{l_2}(l_3) = v_{l_2}(l_4) = 1$.

Next, construct a set of $3q$ agents $C = \{ c_1, c_2, \dots, c_{3q} \}$. Let $v_{c_i}(l_r) = 3$ for each $i$ and $r$ where $1\leq i \leq 3q$ and $1\leq r \leq 4$. For each $i$ and $j$ where $1\leq i, j \leq 3q$ let $v_{c_i}(c_j) = 3$ if $\{ w_i, w_j \} \in E$ otherwise $2$. This completes the construction of $(N, V)$. Note that the structure of the valuations among the agents in $C$ reflects the graph $G$.
%
\begin{figure}
    \centering
    \vspace*{0.2cm}
    \input{figures/threed_efr_as/j_envy_free_sym_reduction.tikz}
    \vspace*{0.5cm}
    \caption[The reduction from PIT to the problem of deciding if an instance of 3DR-AS with symmetric preferences contains a j-envy-free matching]{The reduction from PIT to the problem of deciding if an instance of 3DR-AS with symmetric preferences contains a j-envy-free matching. Valuation colour key: \textcolor{figurecolourschemewt6}{\figurecolorschemewtsixname} - 6, \textcolor{figurecolourschemewt5}{\figurecolorschemewtfivename} - 5, \textcolor{figurecolourschemewt4}{\figurecolorschemewtfourname} - 4, \textcolor{figurecolourschemewt3}{\figurecolorschemewtthreename} - 3, \textcolor{figurecolourschemewt2}{\figurecolorschemewttwoname} - 2, \textcolor{figurecolourschemewt1}{\figurecolorschemewtonename} - 1.}
    \label{fig:threed_efr_as_jef_symmetric_reduction}
\end{figure}

We make some remarks on the design of the constructed instance. Like before, in the reduction presented in Section~\ref{sec:threed_efr_as_jef_ternary}, the design of $H$ is derived from a particular instance that contains no j-envy-free matching. This instance comprises $H$ as well as a single isolated agent $\alpha_{z}$, where $v_{\alpha_{z}}(h_i) = 0$ for each $i$ where $1\leq i \leq 11$. In fact, the proof that this instance contains no j-envy-free matching can be directly derived from the proofs of lemmas appearing later in this section (Lemmas~\ref{lem:threed_efr_as_jef_3332_case_part1}--\ref{lem:threed_efr_as_jef_hopen}). These proofs involve lengthy case analyses, and we leave open the problem of finding a more intuitive or succinct argument (see Section~\ref{sec:threed_efr_as_conclusion} for more discussion on this).

It is straightforward to show that the reduction runs in polynomial time. To prove that the reduction is correct we show that the 3DR-AS instance $(N, V)$ contains a j-envy-free matching if and only if the PIT instance $G$ contains a partition into triangles.

We first show that if the PIT instance $G$ contains a partition into triangles then the 3DR-AS instance $(N, V)$ contains a j-envy-free matching.

\begin{lem}
\label{lem:threed_efr_as_jef_first_direction}
If $G$ contains a partition into triangles then $(N, V)$ contains a j-envy-free matching.
\end{lem}
\begin{proof}
Suppose $G$ contains a partition into triangles $X = \{ X_1, X_2, \dots, X_q \}$. We shall construct a matching $M$ in $(N, V)$ that is j-envy-free. First, add $\{ h_2, h_{10}, h_{11} \}$, $\{ h_5, h_6, h_8 \}$, $\{ h_1, h_9, h_4 \}$, $\{ h_3, l_1, l_2 \}$ and $\{ h_7, l_3, l_4 \}$ to $M$. Next, for each triangle $X_p = \{ w_i, w_j, w_k \}$ in $X$, add $\{ c_i, c_j, c_k \}$ to $M$.

Suppose for a contradiction that some agent $\alpha_j$ exists where $\alpha_j$ has j-envy for some other agent $\alpha_{k_1}$ where $M(\alpha_{k_1}) = \{ \alpha_{k_1}, \alpha_{k_2}, \alpha_{k_3} \}$. Since $N = H \cup L \cup C$ it must be that either $\alpha_{k_1} \in H$, $\alpha_{k_1} \in L$, or $\alpha_{k_1} \in C$. We show that each case leads to a contradiction. It follows that no such $\alpha_j$ exists and thus that $M$ is j-envy-free.
\begin{itemize}
    \item Suppose $\alpha_{k_1} \in H$. Either $\alpha_{k_1} \in \{ h_3, h_7 \}$ or $\alpha_{k_1} \in H \setminus \{ h_3, h_7 \}$.
    \begin{itemize}
        \item Suppose $\alpha_{k_1} \in \{ h_3, h_7 \}$. Then it must be that either $\{ \alpha_{k_2}, \alpha_{k_3} \} = \{ l_1, l_2 \}$ or $\{ \alpha_{k_2}, \alpha_{k_3} \} = \{ l_3, l_4 \}$. Suppose firstly that $\{ \alpha_{k_2}, \alpha_{k_3} \} = \{ l_1, l_2 \}$. We can see immediately that $\alpha_j \notin H$ since otherwise $u_{\alpha_j}(\{ l_1, l_2 \}) = 0$. It must also be that $\alpha_j \notin C$, since  $u_{c_{a}}(\{ l_1, l_2 \}) = 6 = u_{c_a}(M)$ for any $c_a \in C$. Similarly, $u_{l_3}(\{ l_1, l_2 \}) = 2 = u_{l_3}(M)$ and $u_{l_4}(\{ l_1, l_2 \}) = 2 = u_{l_4}(M)$ so $\alpha_j \neq l_3$ and $\alpha_j \neq l_4$. This shows $\alpha_j \notin L$. We have shown that $\alpha_j \notin H$, $\alpha_j \notin C$, and $\alpha_j \notin L$, which is a contradiction. The proof for the case in which $\{ \alpha_{k_2}, \alpha_{k_3} \} = \{ l_3, l_4 \}$ is symmetric and also leads to a contradiction.
        \item Suppose $\alpha_{k_1} \in H\setminus \{ h_3, h_7 \}$. If $\alpha_{k_1} = h_1$ then $\{ \alpha_{k_2}, \alpha_{k_3} \} = \{ h_4, h_9 \}$ so it must be that $v_{h_4}(\alpha_j) > 2 = v_{h_4}(h_1)$ and $v_{h_9}(\alpha_j) > 2 = v_{h_9}(h_1)$, which is impossible by the design of $H$. The proof for every other assignment of $\alpha_{k_1}$ is similar: if $\alpha_{k_1} = h_2$ then $v_{h_{11}}(\alpha_j) > 5$, which is impossible. If $\alpha_{k_1} = h_4$ then $v_{h_1}(\alpha_j) > 2$, which is impossible. If $\alpha_{k_1} = h_5$ then $v_{h_6}(\alpha_j) > 5$ and $v_{h_8}(\alpha_j) > 1$, which is impossible. If $\alpha_{k_1} = h_6$ then $v_{h_8}(\alpha_j) > 6$, which is impossible. If $\alpha_{k_1} = h_8$ then $v_{h_6}(\alpha_j) > 6$, which is impossible. If $\alpha_{k_1} = h_9$ then $v_{h_1}(\alpha_j) > 2$, which is impossible. If $\alpha_{k_1} = h_{10}$ then $v_{h_2}(\alpha_j) > 6$, which is impossible. If $\alpha_{k_1} = h_{11}$ then $v_{h_2}(\alpha_j) > 5$ and $v_{h_{10}}(\alpha_j) > 4$, which is impossible.
    \end{itemize}
    \item Suppose $\alpha_{k_1} \in C$. By construction, it must be that $\alpha_{k_1} = c_{i_1}$, $\alpha_{k_2} = c_{i_2}$, and $\alpha_{k_3} = c_{i_3}$ where $\{ c_{i_1}, c_{i_2}, c_{i_3} \} \subseteq C$, where the corresponding vertices $\{ w_{i_1}, w_{i_2}, w_{i_3} \}$ in $G$ are a triangle. It follows that  $v_{c_{i_2}}(c_{i_1}) = 3$. By assumption, $\alpha_j$ has j-envy for $c_{i_1}$ so it must be that $v_{c_{i_2}}(\alpha_j) > v_{c_{i_2}}(c_{i_1}) = 3$, which is impossible by the design of $C$.
    \item Suppose $\alpha_{k_1} \in L$. It must be that $\alpha_{k_1} = l_{i_1}$ for some $i_1$ where $1\leq i_1 \leq 4$, $\alpha_{k_2} = l_{i_2}$ for some $i_2$ where $1\leq i_2 \leq 4$ and $\alpha_{k_3} = h_{i_3}$ where $i_3 \in \{ 3, 7 \}$. If $\alpha_j \in H$ then $v_{l_{i_2}}(\alpha_j) = 0$ which contradicts the supposition that $\alpha_j$ has j-envy for $l_{i_1}$. Otherwise, if $\alpha_j \notin H$ then $v_{h_{i_3}}(\alpha_j) = 0$, which also contradicts the supposition that $\alpha_j$ has j-envy for $l_{i_1}$.
\end{itemize}
\end{proof}

% \paragraph{Correctness of the reduction: second direction}

We now show that if the 3DR-AS instance $(N, V)$ contains a j-envy-free matching then the PIT instance $G$ contains a partition into triangles.

In the first part of this proof (up to and including Lemma~\ref{lem:threed_efr_as_jef_hopen}) we focus on $H$. To begin, we define two possible configurations of $H$ in an arbitrary j-envy-free matching $M$. If some triple $t\in M$ contains exactly one agent in $H$ then we say that $H$ has an \emph{open configuration in $M$}. Otherwise, we say that $H$ has a \emph{closed configuration in $M$}. We shall eventually show, in Lemma~\ref{lem:threed_efr_as_jef_hopen}, that the only possible configuration of $H$ in $M$ is an open configuration. In Lemmas~\ref{lem:threed_efr_as_jef_two_and_one_in_h}--\ref{lem:threed_efr_as_jef_32222_case} we prove a sequence of intermediary results. 

\begin{lem}
\label{lem:threed_efr_as_jef_two_and_one_in_h}
If $(N, V)$ contains a j-envy-free matching $M$ then no triples $t_1, t_2$ in $M$ exist such that $t_1$ contains exactly two agents in $H$ and $t_2$ contains exactly one agent in $H$.
\end{lem}
\begin{proof}
Suppose for a contradiction that some such $t_1, t_2 \in M$ exist. Suppose $t_1 = \{ h_{i_1}, h_{i_2}, \alpha_{j_1} \}$ and $t_2 = \{ h_{i_3}, \alpha_{j_2}, \alpha_{j_3} \}$ where $1\leq i_1, i_2, i_3 \leq 11$ and $\alpha_{j_1}, \alpha_{j_2}, \alpha_{j_3} \in N \setminus H$. Now $h_{i_3}$ has j-envy for $\alpha_{j_1}$ since $u_{h_{i_3}}(M) = 0 < 2 \leq u_{h_{i_3}}(\{ h_{i_1}, h_{i_2} \})$, $v_{h_{i_1}}(\alpha_{j_1}) = 0 < 1 \leq v_{h_{i_1}}(h_{i_3})$ and $v_{h_{i_2}}(\alpha_{j_1}) = 0 < 1 \leq v_{h_{i_2}}(h_{i_3})$. This contradicts our supposition that $M$ is j-envy-free.
\end{proof}


\begin{lem}
\label{lem:threed_efr_as_jef_3332_case_part1}
If $(N, V)$ contains a j-envy-free matching $M$, $\sigma(H, M) = 4$, and $u_{h_1}(M) < 4$ then $H$ has an open configuration in $M$.
\end{lem}
\begin{proof}
\input{chapters/threed_efr_as/jef/jef_symmetric_result_lemma_3332_proof_part1}
\end{proof}

\begin{lem}
\label{lem:threed_efr_as_jef_3332_case_part2}
If $(N, V)$ contains a j-envy-free matching $M$, $\sigma(H, M) = 4$, and $u_{h_1}(M) = 4$ then $H$ has an open configuration in $M$.
\end{lem}
\begin{proof}
\input{chapters/threed_efr_as/jef/jef_symmetric_result_lemma_3332_proof_part2}
\end{proof}

\begin{lem}
\label{lem:threed_efr_as_jef_3332_case}
If $(N, V)$ contains a j-envy-free matching $M$ and $\sigma(H, M) = 4$ then $H$ has an open configuration in $M$.
\end{lem}
\begin{proof}
Suppose $\sigma(H, M) = 4$. Consider $u_{h_1}(M)$. By the design of $H$, it must be that $2 \leq u_{h_1}(M) \leq 4$. If $u_{h_1}(M) < 4$ then Lemma~\ref{lem:threed_efr_as_jef_3332_case_part1} shows that $H$ has an open configuration in $M$. If $u_{h_1}(M) = 4$ then Lemma~\ref{lem:threed_efr_as_jef_3332_case_part2} shows that $H$ has an open configuration in $M$.
\end{proof}

\begin{lem}
\label{lem:threed_efr_as_jef_32222_case}
If $(N, V)$ contains a j-envy-free matching $M$ and $\sigma(H, M) = 5$ then $H$ has an open configuration in $M$.
\end{lem}
\begin{proof}
\input{chapters/threed_efr_as/jef/jef_symmetric_result_lemma_32222}
\end{proof}

\begin{lem}
\label{lem:threed_efr_as_jef_hopen}
If $(N, V)$ contains a j-envy-free matching $M$ then $H$ has an open configuration in $M$.
\end{lem}
\begin{proof}
By definition, $4 \leq \sigma(H, M) \leq 11$. If $\sigma(H, M) \leq 5$ then $H$ has an open configuration in $M$, by Lemmas~\ref{lem:threed_efr_as_jef_3332_case} and~\ref{lem:threed_efr_as_jef_32222_case}. If $6 \leq \sigma(H, M) \leq 11$ then, by a counting argument, at least one triple in $M$ must contain exactly one agent in $H$. In other words, $H$ has an open configuration in $M$.
\end{proof}

We have shown, in Lemma~\ref{lem:threed_efr_as_jef_hopen}, that if $(N, V)$ contains a j-envy-free matching $M$ then $H$ has an open configuration in $M$. By definition, some triple $t_{\beta}$ in $M$ contains exactly one agent in $H$. Since $|H|=11$, if $t_{\beta}$ is the only triple in $M$ to contain exactly one agent in $H$ then there must exist some triple in $M$ that contains exactly two agents in $H$. By Lemma~\ref{lem:threed_efr_as_jef_two_and_one_in_h}, this is a contradiction. It follows that at least two triples in $M$ exist that each contain exactly one agent in $H$. Suppose $t_{\beta}, t_{\gamma} \in M$ are two such triples and $t_{\beta} = \{ h_{a_1}, \alpha_{b_1}, \alpha_{b_2} \}$ and $t_{\gamma} = \{ h_{a_2}, \alpha_{b_3}, \alpha_{b_4} \}$.

\begin{lem}
\label{lem:threed_efr_as_jef_lequalsthefourisolated}
If $(N, V)$ contains a j-envy-free matching then $\{ \alpha_{b_1}, \alpha_{b_2}, \alpha_{b_3}, \alpha_{b_4} \} = L$.
\end{lem}
\begin{proof}
Suppose for a contradiction that $\{ \alpha_{b_1}, \alpha_{b_2}, \alpha_{b_3}, \alpha_{b_4} \} \neq L$.

By definition, $\{ \alpha_{b_1}, \alpha_{b_2}, \alpha_{b_3}, \alpha_{b_4} \} \cap H = \varnothing$ and $\{ \alpha_{b_1}, \alpha_{b_2}, \alpha_{b_3}, \alpha_{b_4} \} \neq L$ it must be that at least one agent in $\{ \alpha_{b_1}, \alpha_{b_2}, \alpha_{b_3}, \alpha_{b_4} \}$ belongs to $C$. Assume without loss of generality that $\alpha_{b_1} \in C$.

We have already shown that $t_{\beta}$ contains exactly one agent in $H$. Since $v_{\alpha_{b_1}}(h_{a_1}) = 0$, by the design of the instance it must be that $u_{\alpha_{b_1}}(M) = v_{\alpha_{b_1}}(\alpha_{b_2}) \leq 3$. By the design of the instance $v_{\alpha_{b_1}}(\alpha_{b_3}) \geq 2$ and $v_{\alpha_{b_1}}(\alpha_{b_4}) \geq 2$ so $u_{\alpha_{b_1}}(\{ \alpha_{b_3}, \alpha_{b_4} \}) \geq 4$. Now $\alpha_{b_1}$ has j-envy for $h_{a_2}$ since $u_{\alpha_{b_1}}(M) \leq 3 < 4 \leq u_{\alpha_{b_1}}(\{ \alpha_{b_3}, \alpha_{b_4} \})$, $v_{\alpha_{b_3}}(h_{a_2}) = 0 < 2 \leq v_{\alpha_{b_3}}(\alpha_{b_1})$, and $v_{\alpha_{b_4}}(h_{a_2}) = 0 < 2 \leq v_{\alpha_{b_4}}(\alpha_{b_1})$.
\end{proof}

\begin{lem}
\label{lem:threed_efr_as_jef_structureofL}
If $(N, V)$ contains a j-envy-free matching then $\{ \{ \alpha_{b_1}, \alpha_{b_2} \}, \{ \alpha_{b_3}, \alpha_{b_4} \} \} = \{ \{ l_1, l_2 \}, \{ l_3, l_4 \} \}$.
\end{lem}
\begin{proof}
By Lemma~\ref{lem:threed_efr_as_jef_lequalsthefourisolated}, $\{ \alpha_{b_1}, \alpha_{b_2}, \alpha_{b_3}, \alpha_{b_4} \} = L$. There are now three possibilities: first that $\{ \{ \alpha_{b_1}, \alpha_{b_2} \}, \{ \alpha_{b_3}, \alpha_{b_4} \} \} = \{ \{ l_1, l_3 \}, \{ l_2, l_4 \} \}$, second that $\{ \{ \alpha_{b_1}, \alpha_{b_2} \}, \{ \alpha_{b_3}, \alpha_{b_4} \} \} = \{ \{ l_1, l_4 \}, \{ l_2, l_3 \} \}$, and third that $\{ \{ \alpha_{b_1}, \alpha_{b_2} \}, \{ \alpha_{b_3}, \alpha_{b_4} \} \} = \{ \{ l_1, l_2 \}, \{ l_3, l_4 \} \}$.

First suppose $\{ \{ \alpha_{b_1}, \alpha_{b_2} \}, \{ \alpha_{b_3}, \alpha_{b_4} \} \} = \{ \{ l_1, l_3 \}, \{ l_2, l_4 \} \}$. Now $l_1$ has j-envy for $h_{a_2}$ since $u_{l_1}(\{ h_{a_1}, l_3 \}) = 1 < 3 \leq u_{l_1}(\{ l_2, l_4 \})$, $v_{l_2}(h_{a_2}) = 0 < 2 = v_{l_2}(l_1)$, and $v_{l_4}(h_{a_2}) = 0 < 1 \leq v_{l_4}(l_1)$.

Second suppose $\{ \{ \alpha_{b_1}, \alpha_{b_2} \}, \{ \alpha_{b_3}, \alpha_{b_4} \} \} = \{ \{ l_1, l_4 \}, \{ l_2, l_3 \} \}$. As before, $l_1$ has j-envy for $h_{a_2}$ since $u_{l_1}(\{ h_{a_1}, l_4 \}) = 1 < 3 \leq u_{l_1}(\{ l_2, l_3 \})$, $v_{l_2}(h_{a_2}) = 0 < 2 = v_{l_2}(l_1)$, and $v_{l_3}(h_{a_2}) = 0 < 1 \leq v_{l_3}(l_1)$.

It remains that $\{ \{ \alpha_{b_1}, \alpha_{b_2} \}, \{ \alpha_{b_3}, \alpha_{b_4} \} \} = \{ \{ l_1, l_2 \}, \{ l_3, l_4 \} \}$.
\end{proof}

By Lemma~\ref{lem:threed_efr_as_jef_structureofL}, either $\{ \alpha_{b_1}, \alpha_{b_2} \} = \{ l_1, l_2 \}$ or $\{ \alpha_{b_1}, \alpha_{b_2} \} = \{ l_3, l_4 \}$. Without loss of generality assume that $\{ \alpha_{b_1}, \alpha_{b_2} \} = \{ l_1, l_2 \}$.

\begin{lem}
\label{lem:threed_efr_as_jef_eachpigets6}
If $(N, V)$ contains a j-envy-free matching then $u_{c_i}(M) = 6$ for each $i$ where $1\leq i \leq 3q$.
\end{lem}
\begin{proof}
Suppose to the contrary that some $1\leq i \leq 3q$ exists where $u_{c_i}(M) < 6$. Then $c_i$ has j-envy for $h_{a_1}$ since $u_{c_i}(M) \leq 5 < 6 \leq u_{c_i}(\{ l_1, l_2 \})$, $v_{l_1}(h_{a_1}) = 0 < 3 = v_{l_1}(c_i)$, and $v_{l_2}(h_{a_1}) = 0 < 3 = v_{l_2}(c_i)$. This contradicts our supposition that $M$ is j-envy-free.
\end{proof}

\begin{lem}
\label{lem:threed_efr_as_jef_second_direction}
If $(N, V)$ contains a j-envy-free matching then $G$ contains a partition into triangles.
\end{lem}
\begin{proof}
Suppose $(N, V)$ contains a j-envy-free partition into triangles $M$. Lemma~\ref{lem:threed_efr_as_jef_eachpigets6} shows that $u_{c_i}(M) = 6$ for each $i$ where $1\leq i \leq 3q$. By construction, it follows that $M(c_i)$ contains two agents $c_j, c_k$ such that $v_{c_i}(c_j) = v_{c_i}(c_k) = 3$. By construction, $c_j$ and $c_k$ therefore correspond to vertices $w_j, w_k \in W$ where $\{ w_i, w_j \} \in E$ and $\{ w_i, w_k \} \in E$. It follows thus that there are exactly $q$ triples in $M$ each containing three agents $\{ c_i, c_j, c_k \}$, where the three corresponding vertices $w_i, w_j, w_k$ are pairwise adjacent in $G$. From these triples a partition into triangles $X$ can be easily constructed.
\end{proof}

% \paragraph{Correctness of the reduction: conclusion}

We have now shown that the 3DR-AS instance $(N, V)$ contains a j-envy-free matching if and only if the PIT instance $G$ contains a partition into triangles. This shows that the reduction is correct.

\begin{thm}
\label{thm:threed_efr_as_jef_symmetric_6_npcomplete}
Deciding if a given instance of 3DR-AS contains a j-envy-free matching is $\NP$-complete, even when preferences are symmetric and the maximum possible valuation is~$6$.
\end{thm}
\begin{proof}
It is straightforward to show that this decision problem belongs to $\NP$, since for any two agents $\alpha_i, \alpha_j \in N$ we can test if $\alpha_i$ j-envies $\alpha_j$ in constant time. 

We have presented a polynomial-time reduction from Partition Into Triangles (PIT, Problem~\ref{prob:pit}), which is $\NP$-complete \cite{GJ79}. Given a graph $G$, the reduction constructs an instance $(N, V)$ of 3DR-AS with symmetric preferences in which the maximum valuation is $6$. Lemmas~\ref{lem:threed_efr_as_jef_first_direction} and~\ref{lem:threed_efr_as_jef_second_direction} show that $(N, V)$ contains a j-envy-free matching if and only if $G$ contains a partition into triangles and thus that this decision problem is $\NP$-hard.
\end{proof}

\section{Summary and open problems}
\label{sec:threed_efr_as_conclusion}
In this chapter we considered the existence of envy-free, wj-envy-free, and j-envy-free matchings in 3DR-AS and the complexity of the associated decision and construction problems. For each of the three solution concepts, we considered various restrictions on the agents' valuations.

We first showed that an arbitrary instance of 3DR-AS may not contain an envy-free matching, even when preferences are binary and symmetric and the maximum degree of the underlying graph is $2$. We described a polynomial-time algorithm for this case that can, in a given instance of 3DR-AS, either construct an envy-free matching or report that no such matching exists. We then contrasted this result by showing that the corresponding existence problem is $\NP$-complete even when the maximum degree of the underlying graph is $3$.

Next, we considered wj-envy-freeness. Our results for wj-envy-freeness were similar to those for envy-freeness.  We first showed that, as in the case of envy-freeness, a wj-envy-free matching may not exist even when preferences are binary and symmetric and the maximum degree of the underlying graph is $2$. We described a slightly more complex polynomial-time algorithm for this case, compared to the corresponding algorithm for envy-freeness, which either constructs a wj-envy-free matching or reports that no such matching exists. We also showed that the corresponding existence problem is $\NP$-complete even when the maximum degree of the underlying graph is $3$. 

We then considered j-envy-freeness. We showed that if preferences are binary but not necessarily symmetric, a j-envy-free matching must exist and can be found in polynomial time. We then considered two restrictions of 3DR-AS, in which valuations are ternary but not symmetric, and non-binary and symmetric. In both restrictions, we showed that a given instance of 3DR-AS may not contain a j-envy-free matching and the associated existence problem is $\NP$-complete.

We summarise our new existence and complexity results in Table~\ref{tab:introduction_3dsras_mainresults}, which also includes the corresponding results for stability from Chapter~\ref{c:threed_sr_as}. In the table, for a given solution concept and preference restriction, ``must exist?'' refers to whether an arbitrary such instance of 3DR-AS must contain a matching that satisfies that solution concept, and ``search'' refers to the complexity class of the associated construction problem. From this table (and the associated theorems) we can identify a general trend in our results that for successively weaker solution concepts, existence and polynomial-time solvability hold under successively weaker restrictions on the agents' preferences.

\begin{table}[ht]
\begin{center}
% \resizebox{\textwidth}{!}{
% \begin{tabular}{c@{\hspace{1pt}}c@{\hspace{1pt}}c@{\hspace{3pt}}c@{\hspace{1pt}}}\\\noalign{\hrule}
\begin{tabular}{ccccc}\\\noalign{\hrule}
\multicolumn{2}{c}{input settings}            
& \multicolumn{2}{c}{results} 
\\
solution concept & preference restriction &  must exist? & search & Theorem\\
\noalign{\hrule}
\noalign{\hrule}
stability & binary and symmetric & \checkmark & \P & \ref{thm:threed_sr_as_symmetric_binary_construction}\\
'' & binary & \xmark & \NP-h. &  \ref{thm:threed_sr_as_binary_reduction}\\
'' & ternary and symmetric & \xmark & \NP-h. & \ref{thm:threed_sr_as_symmetric_ternary_reduction}\\[4.5pt]
envy & binary and symmetric, $\Delta=2$  & \xmark & \P & \ref{thm:threed_efr_as_ef_algorithm}\\
'' & binary and symmetric, $\Delta=3$  & \xmark & \NP-h. & \ref{thm:threed_efr_as_regularenvy_npcomplete}\\[4.5pt]
weakly justified envy & binary and symmetric, $\Delta=2$ & \xmark & \P & \ref{thm:threed_efr_as_wjef_algowjpathscycles}\\
'' & binary and symmetric, $\Delta=3$  & \xmark & \NP-h. & \ref{thm:threed_efr_as_wjef_npcomplete}\\[4.5pt]
justified envy & binary and symmetric & \checkmark & \P & Obs.~\ref{obs:threed_efr_as_jef_binary_symmetric_from_stability}\\
'' & binary & \checkmark & \P & \ref{thm:threed_efr_as_jef_binary_algorithm}\\
'' & ternary & \xmark & \NP-h. & \ref{thm:threed_efr_as_jef_terasym_npcomplete}\\
'' & symmetric and $0 \leq v_i(j) \leq 6$ & \xmark & \NP-h. & \ref{thm:threed_efr_as_jef_symmetric_6_npcomplete}\\
\noalign{\hrule}
\end{tabular}
% }
\end{center}
\caption{Our complexity results for 3DR-AS (from Chapters~\ref{c:threed_sr_as} and~\ref{c:threed_efr_as}). In restrictions involving binary and symmetric preferences, $\Delta$ refers to the maximum degree of the underlying graph.}
\label{tab:introduction_3dsras_mainresults}
\end{table}

We now present some open problems specifically involving envy-freeness, wj-envy-freeness, and j-envy-freeness in 3DR-AS. More general problems, involving solution concepts that do not involve envy and other models of fixed-size coalitions, are discussed in Chapter~\ref{c:conclusion}.

The immediate open problem relates to j-envy-freeness and preferences that are ternary and symmetric. Specifically, it would be interesting to resolve the computational complexity of the problem of deciding if a given instance 3DR-AS with ternary preferences contains j-envy-free matching. The first step in this direction would be to determine whether every instance of 3DR-AS with ternary preferences contains a j-envy-free matching. 

In Theorem~\ref{thm:threed_efr_as_jef_symmetric_6_npcomplete} we showed that there exist instances of 3DR-AS with symmetric preferences that do not contain a j-envy-free matching. The proof involved a gadget $H$ from which such an instance can be directly derived, by adding a single ``isolated'' agent. This specific instance was discovered by sequential search, using an integer programming \cite{IPbook} model to test candidate solutions (similar techniques have been used in the context of hedonic games~\cite{bullinger21}). In order to reduce the search space, certain assumptions were made about the design of the instance, such as the existence of an isolated agent (an ``undesired guest'' \cite{BJ02,GS62}). Although this technique was effective, it is hard to provide any intuition as to why this specific instance contains no j-envy-free matching. It is also open whether this instance is minimal, or if a smaller instance of 3DR-AS exists, with fewer than $12$ agents, that contains no j-envy-free matching. 

As we noted in Chapter~\ref{c:threed_sr_as}, another open problem is whether our results apply to a setting involving more general definitions of binary and ternary. For example, whether our results for binary preferences hold in a more general setting in which $v_{\alpha_i}(\alpha_j) \in \{ a, b \}$ for any non-negative integers $a$ and $b$ where $a < b$.

% . A similar technique has been previously applied in a model of a hedonic game \cite{bullinger21}. In order to reduce the search space, certain assumptions were made. For example, one assumption was that any candidate instance would be highly symmetric. Another was the existence of an ``undesired guest'' (i.e.\ $h_1$) \cite{BJ02,GS62}. Although this technique was effective, it is hard to provide any intuition why this particular instance contains no j-envy-free matching. It also remains open whether this instance is minimal, or if a smaller instance of 3DR-AS exists, with fewer than $12$ agents, that contains no j-envy-free matching. 

% We believe the apparent difficulty of constructing such instances leads to interesting directions for future work. For example, identifying a minimal such instance might provide a deeper insight into the structure of 3DR-AS. 
% interesting open direction of work is to either prove 
% To our knowledge, such instances are non-trivial, and in fact the instance presented 
% A central contribution of this theorem is that it proves the existence of an instance of 3DR-AS with no j-envy-free 
% of this proof, shown in Lemmas~\ref{lem:threed_efr_as_jef_two_and_one_in_h}--\ref{lem:threed_efr_as_jef_hopen}, is a lengthy case analysis. This part of the proof is essentially equivalent 

As we also noted in Chapter~\ref{c:threed_sr_as}, it might be interesting to identify other restrictions of 3DR-AS in which an envy-free, wj-envy-free or j-envy-free matching can be found in polynomial time. The gadgets used in our reductions are highly regular and it might be that there exist interesting classes of instances that must contain a stable matching. Alternatively, we could study these problems from the perspective of parameterised complexity. For example, in the case of binary and symmetric preferences, one could consider the tree-width \cite{Robertson84} of the instance.

It might be also interesting to estimate the probability that a random instance of 3DR-AS contains an envy-free, wj-envy-free, or j-envy-free matching, or to estimate the same probability in a random instance of 3DR-AS with binary or ternary preferences. Our complexity results indicate that, among instances with binary and symmetric preferences and maximum degree $2$, the set of instances that contain a j-envy-free matching (i.e.\ all instances) is larger than the set of instances that contain a wj-envy-free matching, which is in turn larger than the set of instances that contain an envy-free matching. We conjecture that, in a general instance of 3DR-AS, the probability that a given instance contains an envy-free matching is smaller than the probability that it contains a wj-envy-free matching, which is in turn smaller than the probability that it contains a j-envy-free matching. 
In this direction, it might be possible to apply probabilistic techniques from graph theory, such as the Erd\H{o}s-R\'enyi model of a random graph. Of course, the probabilistic events in which agents have envy for other agents are not independent, which complicates the analysis. Alternatively an empirical approach might be informative, for example by formulating the problem as an integer program \cite{IPbook}.

% As we also noted in Chapter~\ref{c:threed_sr_as}, the connection between 3DR-AS and graph theory gives rise to a number of natural parameters. In particular, recall that the problem of finding an envy-free or wj-envy-free matching in a given instance of 3DR-AS is $\NP$-complete, even when preferences are binary and symmetric and the maximum degree of the underlying graph is $3$. It might be natural to study both problems with respect to other parameters of the instance, such as tree-width \cite{Robertson84}.
\chapter{The \texorpdfstring{$K_r$}{Kr}-packing Problem in bounded degree graphs}
\label{c:kr_packing}

\section{Introduction}
\label{sec:krpacking_intro}
\subsection{Background}
\label{sec:krpacking_background}

In this chapter we consider two problems related to clique packings in undirected graphs. In particular, finding a maximum-cardinality set of $r$-cliques, for some fixed $r \geq 3$, subject to the cliques in that set being either vertex disjoint or edge disjoint. We refer to such a set as a \emph{$K_r$-packing}, and to the two problems as the \emph{Vertex-Disjoint $K_r$-Packing Problem} (\vdkr) and the \emph{Edge-Disjoint $K_r$-Packing Problem} (\edkr). Note that $r$ is a fixed constant and does not form part of the problem input. If $r$ is not fixed then both problems generalise the well-studied problem of finding a clique of a given size \cite{GJ79}.

Most existing research relating to either vertex- or edge-disjoint $K_r$-packings covers either more restricted or more general cases. Two well-known special cases of $K_r$-packing are \vdktwo, also known as two-dimensional matching; and \vdkthree, for which an associated decision problem is known as \emph{Partition Into Triangles} (PIT, Problem~\ref{prob:pit}). 

\vdktwo is a central problem of graph theory and algorithmics. A classical result is that a maximum-cardinality two-dimensional matching can be found in polynomial time \cite{Edm65, MV80}. \vdkthree and its associated decision problem have also been the subject of much research. In 1975, Karp \cite{Karp75} noted that PIT, i.e.\ deciding if a give graph contains a $K_3$-packing of cardinality $|V|/3$, was $\NP$-complete. In this chapter we call such a $K_r$-packing, with cardinality $|V|/r$, \emph{perfect}, although it is sometimes known as a \emph{$K_r$-factor} \cite{guruswami_k_2001}.

In 2002, Caprara and Rizzi \cite{caprara_packing_2002} considered \vdkthree and \edkthree in the setting of a fixed maximum degree $\Delta$. They showed that \vdkthree is solvable in polynomial time if $\Delta = 3$ and $\APX$-hard even when $\Delta = 4$, and \edkthree is solvable in polynomial time if $\Delta = 4$ and $\APX$-hard even when $\Delta = 5$. They also showed that \vdkthree is $\NP$-hard for planar graphs even when $\Delta = 4$ and \edkthree is $\NP$-hard for planar graphs with $\Delta = 5$. In 2013, van Rooij et al.\ \cite{van_rooij_partition_2013} established an equivalence between \vdkthree when $\Delta = 4$ and \emph{Exact 3-Satisfiability} (X3SAT). They then used this equivalence to devise an $O(1.02220^n)$-time algorithm for PIT (i.e.\ the perfect \vdkthree decision problem) when $\Delta = 4$.

The approximability of \vdkthree and \edkthree has also been of interest. In 1989, Hurkens and Schrijver \cite{hs89} presented a new result relating to \emph{Systems of Distinct Representatives}, which they used to construct an algorithm, based on ``local improvement'', for a class of packing problems that includes \vdkr and \edkr. Specifically, for either \vdkr or \edkr and for any fixed constant $\varepsilon > 0$, Hurkens and Schrijver identify a polynomial-time $(r/2 + \varepsilon)$-approximation algorithm. In 1995, Halld\'{o}rsson \cite{Hal95} presented an alternative proof of this result and considered algorithms based on local improvement for other types of packing problems. In 2005, Mani\'c and Wakabayashi \cite{Eurocomb05} described approximation algorithms that improve on this approximation ratio for the restricted cases of \vdkthree in which $\Delta=4$, and \edkthree in which $\Delta=5$. They also presented a linear-time algorithm for \vdkthree on so-called indifference graphs.

A generalisation of \vdkr is \emph{Vertex-Disjoint $H$-Packing} (also called \emph{Vertex-Disjoint $G$-Packing}), where $H$ is an arbitrary but fixed undirected graph. In 1982, Takamizawa et al.\ \cite{takamizawa1982linear} showed that an optimisation problem related to Vertex-Disjoint $H$-Packing is solvable in polynomial time on a subclass of planar graphs known as \emph{series-parallel} graphs. In 1983, Kirkpatrick and Hell \cite{KH83} reviewed the literature of Vertex-Disjoint $H$-Packing and classify the complexity of an array of packing problems, some of which generalise \vdkr. In particular, they showed that if $H$ contain any connected component with three vertices then the perfect vertex-disjoint $H$-packing decision problem is $\NP$-complete. Pantel's 1999 thesis provides a comprehensive survey of $H$-packing \cite{Pantel99}.

The restriction of $H$-packing to $K_r$-packing has received comparatively less attention in the literature. In 1998, Dahlhaus and Karpinski \cite{DAHLHAUS199879} proved that a perfect vertex-disjoint $K_r$-packing can be found in polynomial time, if it exists, in chordal and strongly chordal graphs. In 2001, Guruswami et al.\ \cite{guruswami_k_2001} showed that, for any $r \geq 3$, the \vdkr decision problem is $\NP$-complete for chordal graphs, planar graphs (assuming $r<5$), line graphs, and total graphs. Their $\NP$-completeness result involving $K_r$-packing on chordal graphs, for any $r \geq 3$, resolved an open question of Dahlhaus and Karpinski. Guruswami et al.\ also described polynomial-time algorithms for \vdkthree and the perfect \vdkr decision problem on split graphs, and \vdkr on cographs. Their result for cographs was later extended by Pedrotti and de Mello \cite{KrPackingP4sparse} for so-called $P_4$-sparse graphs.

From the converse perspective of graphs with a fixed minimum degree, Hajnal and Szemer\`edi \cite{hajnal_szemeredi} proved in 1970 that if the minimum degree of a graph is greater than or equal to $(1 - 1/r)|V|$ then a perfect vertex-disjoint $K_r$-packing must exist. Kierstead and Kostochka \cite{KIERSTEAD2008226} later generalised this result to show that, in this case, such a packing can be constructed in polynomial time. A growing area of research work studies the existence of perfect $K_r$- and $H$-packings with respect to conditions involving vertex degree \cite{TreglownThesis, KD09}.

\subsection{Our contribution}

In 2002, Caprara and Rizzi \cite{caprara_packing_2002} showed that \vdkthree is solvable in polynomial time if $\Delta=3$ and $\APX$-hard if $\Delta=4$; and \edkthree is solvable in polynomial time if $\Delta=4$ and $\APX$-hard if $\Delta=5$. In this chapter we extend some of their techniques in order to generalise their results and fully classify the complexity of both \vdkr and \edkr for any $\Delta \geq 1$ and any fixed $r \geq 3$. We summarise this classification in Table~\ref{tab:krpacking_results}.

\begin{table}[h]
\centering
% \def\arraystretch{1.4}
\begin{tabular}{clp{0.0cm}lp{0.0cm}l}\noalign{\hrule}
\Tstrut\vspace*{0.2em}
& \multicolumn{3}{c}{is solvable in} & \\
& \multicolumn{1}{c}{linear time if} & \multicolumn{2}{c}{polynomial time if} & \multicolumn{2}{c}{is $\APX$-hard if} \\
\hline \vdkr & $\Delta < 3r/2 - 1$ & & $\Delta < 5r/3 - 1$ & & $\Delta \geq \lceil 5r/3 \rceil - 1$\Tstrut\\[0.4em]
\multirow{2}{*}{\edkr} & \multirow{2}{*}{$\Delta < 3r/2 - 1$} & \ldelim\{{2}{*} & $\Delta \leq 2r - 2$ if $r \leq 5$ &  \ldelim\{{2}{*} &  $\Delta > 2r - 2$ if $r \leq 5$\\
& & & $\Delta < 5r/3 - 1$ otherwise & & $\Delta \geq \lceil 5r/3 \rceil - 1$ otherwise
\vspace*{0.3em}
\Bstrut\\\noalign{\hrule}
\end{tabular}
\caption{Our complexity results for \vdkr and \edkr}
\label{tab:krpacking_results}
\end{table}

In the next section, Section~\ref{sec:krpacking_prelims}, we define some additional notation and make an observation on the coincidence of vertex- and edge-disjoint $K_r$-packings.

In Section~\ref{sec:krpacking_lineartime}, we consider the case when $\Delta < 3r/2 - 1$. We show that in this case, any maximal vertex- or edge-disjoint $K_r$-packing is also maximum (Theorem~\ref{thm:krpacking_r_maximal_is_maximum}), and devise a linear-time algorithm for \vdkr and \edkr in this setting (Theorem~\ref{thm:krpacking_vdkr_3r2minus1} and Corollary~\ref{cor:krpacking_edkr_3r2minus2}). 

In Section~\ref{sec:krpacking_ptime}, we present our solvability results, showing that \vdkr can be solved in polynomial time if $\Delta < 5r/3 - 1$ (Theorem~\ref{thm:krpacking_vdkr_polytime_5r3minus1}); and \edkr can be solved in polynomial time if either $r\leq 5$ and $\Delta \leq 2r - 2$ (Theorem~\ref{thm:krpacking_edkr345_polytime}), or $r\geq 6$ and $\Delta < 5r/3 - 1$ (Theorem~\ref{thm:krpacking_edkrgeq6_polytime}). Our proof uses a similar technique to that of Caprara and Rizzi's \cite{caprara_packing_2002}, which involves finding a maximum independent set in a graph that represents the intersection of $K_r$s.

In Section~\ref{sec:krpacking_apxhardness}, we show that our solvability results are in a sense best possible, unless $\P \neq \NP$. Specifically, we show that \vdkr is $\APX$-hard if $\Delta \geq \lceil 5r/3 \rceil - 1$ (Theorem~\ref{thm:krpacking_vdkr_apxhard}); and \edkr is $\APX$-hard if either $r \leq 5$ and $\Delta > 2r - 2$ (Theorem~\ref{thm:krpacking_edkr345apxhard}), or $r \geq 6$ and $\Delta \geq \lceil 5r/3 \rceil - 1$ (Theorem~\ref{thm:krpacking_edkr_rgeq6_apxhard}). In other words, we prove that there exists some fixed constants $\varepsilon > 1$ and $\varepsilon' > 1$ such that no polynomial-time $\varepsilon$-approximation algorithm exists for \vdkr if $\Delta \geq \lceil 5r/3 \rceil - 1$; and no polynomial-time $\varepsilon'$-approximation algorithm exists for \edkr if either $r \leq 5$ and $\Delta > 2r - 2$, or $r \geq 6$ and $\Delta \geq \lceil 5r/3 \rceil - 1$. 

In Section~\ref{sec:krpacking_conclusion} we recap on our results and consider directions for future work.

\subsection{Preliminaries}
\label{sec:krpacking_prelims}
In this section we first clarify our terminology and notation and then make a preliminary observation on the coincidence of vertex- and edge-disjoint $K_r$-packings.

Let $G = (V, E)$ be a simple undirected graph. We denote the \emph{closed neighbourhood} of some vertex $v \in V$ as $N_{G}[v] = N_{G}(v) \cup \{ v \}$. We denote by $\deg_{G}(v) = |N_{G}[v]| - 1$ the \emph{degree} of $v$ in $G$ and by $\Delta(G) = \max_{v\in V} \deg_{G}(v)$ the \emph{maximum degree} of $G$. If the graph in question is clear from context then we just write $\Delta$. If $\deg_{G}(v) = \delta$ for all $v \in V$ then we say that $G$ is \emph{$\delta$-regular}. For any subset of vertices $U \subseteq V$, we denote by $G[U]$ the subgraph of $G$ induced by $U$. 
Let $K_r$ denote a clique of size $r$, for some integer $r \geq 1$, and $K_r^G$ be the set of $K_r$s in $G$. We say that $T$ is a \emph{$K_r$-packing} in $G$ if $T\subseteq K_r^G$. The \emph{cardinality} of a $K_r$-packing is the number of $K_r$s that it contains. We say that a $K_r$-packing $T$ is \emph{vertex disjoint} if any two $K_r$s in $T$ have no vertex in common and \emph{edge disjoint} if any two $K_r$s in $T$ intersect by at most one vertex. The \emph{Vertex-Disjoint $K_r$-Packing Problem} (\mysymbolfirstusedefinition{symboldef:vdkr}{\vdkr}) is the following optimisation problem: given a simple undirected graph $G$, find a vertex-disjoint $K_r$-packing of maximum cardinality. The \emph{Edge-Disjoint $K_r$-Packing Problem} (\mysymbolfirstusedefinition{symboldef:edkr}{\edkr}) is defined analogously.

If $G$ contains four vertices $v_i, v_{j_1}, v_{j_2}, v_{j_3}$ where $\{ v_i, v_{j_a} \} \in E$ for each $a \in \{ 1, 2, 3 \}$, $\{ v_{j_1}, v_{j_2} \} \notin E$, $\{ v_{j_1}, v_{j_3} \} \notin E$, and $\{ v_{j_2}, v_{j_3} \} \notin E$, then we say that these four vertices form a \emph{claw}. Otherwise, we say that $G$ is \emph{claw-free}. For example, line graphs are claw-free \cite{MINTY1980284}. 

% If $G$ contains four vertices $v_i, v_{j_1}, v_{j_2}, v_{j_3}$ where $\{ v_i, v_{j_a} \} \in E$ for each $a \in \{ 1, 2, 3 \}$, $\{ v_{j_1}, v_{j_2} \} \notin E$, $\{ v_{j_1}, v_{j_3} \} \notin E$, and $\{ v_{j_2}, v_{j_3} \} \notin E$, then we say that these four vertices form a \emph{claw} (shown in Figure~\ref{fig:krpacking_claw}). Otherwise, we say that $G$ is \emph{claw-free}. For example, line graphs are claw-free \cite{MINTY1980284}. 
% \begin{wrapfigure}{r}{0.5\textwidth} 
%     \centering
%     \input{figures/kr_packing/claw.tikz}
%     \caption{A claw}
%     \label{fig:krpacking_claw}
% \end{wrapfigure}

For any maximisation problem $P$, instance $I$ of $P$, and feasible solution $S$ of $I$, let $\textrm{m}_{P}(I, S)$ denote the measure of $S$. Let $\textrm{opt}_{P}(I) = \max_{S \in \mathcal{F}(I)} \textrm{m}_{P}(I, S)$, where $\mathcal{F}(I)$ is the set of feasible solutions of $I$.

For technical purposes we define the \emph{$K_r$-vertex intersection graph} $\mathcal{K}_r^G = (K_r^G, E_{\mathcal{K}_r^G})$ of $G$, in which $\{ U, W \} \in E_{\mathcal{K}_r^G}$ if $|U \cap W| \geq 1$ for any $U, W \in K_r^G$. Similarly, we define the \emph{$K_r$-edge intersection graph} ${\mathcal{K}'}_r^G = (K_r^G, E_{{\mathcal{K}'}_r^G})$ of $G$ in which $\{ U, W \} \in E_{{\mathcal{K}'}_r^G}$ if $|U \cap W| \geq 2$ for any $U, W \in K_r^G$. We now make a preliminary observation.

\begin{observation}
\label{obs:krpacking_edkr_is_also_vdkr}
If $\Delta < 2r - 2$ then any edge-disjoint $K_r$-packing is also vertex disjoint.
\end{observation}
\begin{proof}
Any two $K_r$s in $G$ that share at least one vertex must in fact share at least two vertices, otherwise that vertex has degree at least $2r - 2$ in $G$.
\end{proof}

\section{Linear-time solvability}
\label{sec:krpacking_lineartime}
In this section we present Algorithm~\algorithmfont{greedyCliques}, which can solve \vdkr and \edkr in linear time if $\Delta < 3r/2 - 1$. This algorithm generalises an algorithm of van Rooij et al.\ \cite{van_rooij_partition_2013} that can solve \vdkthree in linear time.

The key insight is that if $\Delta < 3r/2 - 1$ then any maximal vertex-disjoint $K_r$-packing is also a maximum vertex-disjoint $K_r$-packing. The proof of this is stated in Theorem~\ref{thm:krpacking_r_maximal_is_maximum}, which we prove using a sequence of lemmas. In what follows, suppose $G$ is a simple undirected graph in which $\Delta(G) < 3r/2 - 1$.

\begin{lem}
\label{lem:krpacking_r_lem1}
For any $U_1,U_2\in K_r^G$, if $\{ U_1, U_2 \} \in E_{\mathcal{K}_r^G}$ then $|U_1 \cap U_2| > r/2$.
\end{lem}
\begin{proof}
Consider any $U_1,U_2\in K_r^G$ where $\{ U_1, U_2 \} \in E_{\mathcal{K}_r^G}$. By definition of $\mathcal{K}_r^G$, there exists at least one $u\in V$ where $u\in U_1 \cap U_2$. Since $3r/2 - 1 > \Delta \geq \deg_{G}(u) \geq |U_1 \cup U_2| - 1 = |U_1| + |U_2| - |U_1 \cap U_2| - 1 = 2r - |U_1 \cap U_2| - 1$ it follows that $|U_1 \cap U_2| > 2r - 3r/2 = r/2$.
\end{proof}

\begin{lem}
\label{lem:krpacking_r_lem3}
$\mathcal{K}_r^G$ is a disjoint union of cliques (i.e.\ a cluster graph \cite{clustergraphcitation}).
\end{lem}
\begin{proof}
It suffices to show that for any three vertices $U_i, U_j, U_k \in K_r^G$, if $\{ U_i, U_j \} \in E_{\mathcal{K}_r^G}$ and $\{ U_j, U_k \} \in E_{\mathcal{K}_r^G}$ then $\{ U_i, U_k \} \in E_{\mathcal{K}_r^G}$. Consider some such $U_i, U_j, U_k$. If $\{ U_i, U_j \} \in E_{\mathcal{K}_r^G}$ and $\{ U_j, U_k \} \in E_{\mathcal{K}_r^G}$ then by Lemma~\ref{lem:krpacking_r_lem1} it must be that $|U_i \cap U_j| > r/2$ and $|U_j \cap U_k|>r/2$. Since $|U_j|=r$ it follows that $|U_i \cap U_k| > 0$ and thus that $\{ U_i, U_k \} \in E_{\mathcal{K}_r^G}$.
\end{proof}

\begin{thm}
\label{thm:krpacking_r_maximal_is_maximum}
If $T$ is a maximal vertex-disjoint $K_r$-packing then $T$ is a maximum vertex-disjoint $K_r$-packing.
\end{thm}
\begin{proof}
Suppose $T$ is a maximal vertex-disjoint $K_r$-packing in $G$, which by definition corresponds to a maximal independent set in $\mathcal{K}_r^G$. Since $\mathcal{K}_r^G$ is the disjoint union of cliques (by Lemma~\ref{lem:krpacking_r_lem3}), any two maximal independent sets in $\mathcal{K}_r^G$ have the same cardinality so $T$ is also maximum.
\end{proof}

We have shown in Theorem~\ref{thm:krpacking_r_maximal_is_maximum} that any maximal vertex-disjoint $K_r$-packing is also a maximum vertex-disjoint $K_r$-packing. It follows immediately that \vdkr can be solved in $O(|V|^r)$ time by constructing the $K_r$-vertex intersection graph $\mathcal{K}_r^G$ and greedily selecting an independent set. In fact, the explicit construction of $\mathcal{K}_r^G$ can be avoided by exploring $G$ and greedily selecting $K_r$s. We present Algorithm~\algorithmfont{greedyCliques}, shown in Algorithm \ref{alg:krpacking_r}, and show that it requires $O(|V|)$ time.

\begin{algorithm}
\textbf{Input:} a fixed $r\geq 3$, a simple undirected graph $G=(V, E)$ where $\Delta(G) < 3r/2 - 1$\\
\textbf{Output:} a maximum $K_r$-packing $T$
\smallskip
\begin{algorithmic}
\caption{Algorithm~\algorithmfont{greedyCliques}\label{alg:krpacking_r}} 
\State $T\gets\varnothing$
\While{$|V| > 0$}
    \State $v\gets \text{any element of }V$
    \If{$\deg_{G}(v) \geq r - 1$}
        \State $K\gets \varnothing$
        \For{each subset $W$ of $N_{G}(v)$ of size $r-1$}
            \If{$G[W]$ has $\binom{r-1}{2}$ edges}
                \LineComment{so $W$ is a clique of size $r-1$ in $G$}
                \State $K \gets W \cup \{ v \}$
            \EndIf
            \State \textbf{end if}
        \EndFor
        \State \textbf{end for}
        \smallskip
        
        \If{$K\neq\varnothing$}
            \State $G \gets G[V \setminus K]$
            \State $T \gets T \cup \{ \{ K \} \}$
        \Else
            \State $G \gets G[V \setminus \{ v \}]$
        \EndIf
        \State \textbf{end if}
    \Else
        \State $G \gets G[V \setminus \{ v \}]$
    \EndIf
    \State \textbf{end if}
\EndWhile
\State \textbf{end while}
\smallskip

\State \Return $T$
\end{algorithmic}
\end{algorithm}

\begin{lem}
\label{lem:krpacking_lineartimealgorunningtime}
Algorithm~\algorithmfont{greedyCliques} requires $O(|V|)$ time.
\end{lem}
\begin{proof}
In any iteration of the outermost while loop, either $v$ and its incident edges are removed from $G$ or a set of vertices $K$ where $v\in K$ and incident edges are removed. It follows that the algorithm terminates after at most $|V|$ iterations of this loop. It remains to show that one iteration of the loop can be performed in constant time.

In each iteration, either $\deg_{G}(v) \geq r-1$ or $\deg_{G}(v) < r - 1$. Computing $\deg_{G}(v)$ requires $O(r)$ time, since $\Delta < 3r/2 - 1$.  
Consider the first branch of the outermost if statement. There are $\binom{\Delta}{r-1} < \binom{3r/2-1}{r-1}=O(2^r)$ iterations of the for loop. In each iteration, the algorithm tests if $G[W]$ contains $\binom{r-1}{2}$ edges. This can be performed in $O(r^2)$ time. Removing $K$ from $G$ and adding $K$ to $T$, if $K\neq \varnothing$, can be done in $O(r^2)$ time. In both the else branch in which $K=\varnothing$ and the second branch of the outermost if statement, $v$ can be removed from $G$ in $O(r)$ time.
\end{proof}

\begin{thm}
\label{thm:krpacking_vdkr_3r2minus1}
If $\Delta < 3r/2 - 1$ then \vdkr can be solved in linear time.
\end{thm}
\begin{proof}
By Lemma~\ref{lem:krpacking_lineartimealgorunningtime}, Algorithm~\algorithmfont{greedyCliques} terminates in $O(2^r |V|)$ time. By Theorem~\ref{thm:krpacking_r_maximal_is_maximum}, it suffices to show that it returns a maximal vertex-disjoint $K_r$-packing $T$ in $G$. Suppose $K'$ is an arbitrary $K_r$ in $G$. We show that either $K'$ is added to $T$ or at least one vertex in $K'$ belongs to some other $K_r$ in $T$. By the pseudocode, the algorithm removes at least one vertex in each iteration of the while loop, which ends once there are no remaining vertices. Consider the first iteration of the while loop in which a vertex $v$ in $K'$ is removed. Let $G'$ be the subgraph of $G$ at the beginning of this iteration. It must be that every vertex of $K'$ is present in $G'$, including $v$. Moreover, it must be that $\deg_{G'}(v) \geq r - 1$. It follows that $v$ was not deleted from $G'$ by the second branch of the outermost if statement. Similarly, $v$ cannot have been deleted from $G'$ by the second branch of the innermost if statement, since $v$ belongs to $K'$, which is a clique of size $r$ in $G'$. The only possibility is thus that $v$ was deleted from $G'$ as a result of $v$ being part of a clique $K''$ of size $r$ in $G'$, where $K''$ was then added to $T$.
\end{proof}

\begin{cor}
\label{cor:krpacking_edkr_3r2minus2}
If $\Delta < 3r/2 - 1$ then Algorithm~\algorithmfont{greedyCliques} can solve \edkr in linear time.
\end{cor}
\begin{proof}
Since $\Delta < 3r/2 - 1 < 2r - 2$, by Observation~\ref{obs:krpacking_edkr_is_also_vdkr} any edge-disjoint $K_r$-packing is also vertex-disjoint. It follows that any vertex-disjoint $K_r$-packing by Algorithm~\algorithmfont{greedyCliques}is also a maximum edge-disjoint $K_r$-packing.
\end{proof}


\section{Polynomial-time solvability}
\label{sec:krpacking_ptime}
\subsection{Vertex-disjoint packing} 
\label{sec:krpacking_vdkr_polytimesolvability}

In this section we show that \vdkr is solvable in polynomial time if $\Delta < 5r/3 - 1$. The proof involves finding an independent set in the $K_r$-vertex intersection graph $\mathcal{K}_r^G$. We build on the technique of Caprara and Rizzi \cite{caprara_packing_2002} and show that if $\Delta(G) < 5r/3 - 1$ then $\mathcal{K}_r^G$ is claw-free. It follows that a a maximum independent set in $\mathcal{K}_r^G$ can be found in polynomial time (a result of Minty \cite{MINTY1980284} and Sbihi \cite{SBIHI198053}), which corresponds directly to a maximum vertex-disjoint $K_r$-packing.

In his paper on claw-free graphs, Minty \cite{MINTY1980284} remarked that an algorithm to find a maximum cardinality matching, or vertex-disjoint $K_2$-packing, can be used to find a maximum independent set in a line graph (the $K_2$ vertex-intersection graph). Here, like Caprara and Rizzi~\cite{caprara_packing_2002}, we make use of the converse relationship, that if the corresponding intersection graph is claw-free then \vdkr and \edkr can be solved in polynomial time. A general relationship between packing problems and independent sets in intersection graphs is also noted by Kann~\cite{kann91}.

In what follows, suppose $\Delta(G) < {5r}/{3}-1$. In Lemma~\ref{lem:krpacking_polytime_overlap} we place a lower bound on the size of the intersection of two non-vertex intersecting $K_r$s in $G$.

\begin{lem}
\label{lem:krpacking_polytime_overlap}
$|U_i \cap U_j| > r/3$ for any $\{ U_i, U_j \} \in E_{\mathcal{K}_r^G}$.
\end{lem}
\begin{proof}
Consider an arbitrary vertex $u_l \in |U_i \cap U_j|$. By definition, $u_l$ is adjacent to each vertex in $U_i \cup U_j$. By the principle of inclusion-exclusion, $|U_i \cup U_j| = 2r - |U_i \cap U_j|$ so $\deg_{G}(u_l) \geq 2r - |U_i \cap U_j| - 1$. Since $\deg_{G}(u_l) \leq \Delta(G) < {5r}/{3}-1$, it must be that $2r - |U_i \cap U_j| - 1 < {5r}/{3}-1$. Rearranging gives $|U_i \cap U_j| > r/3$.
\end{proof}

\begin{lem}
$\mathcal{K}_r^G$ is claw-free.
\end{lem}
\begin{proof}
Consider some $U_i, U_{j_1}, U_{j_2}, U_{j_3} \in K_r^G$ exist where $\{ U_i, U_{j_a} \} \in E_{\mathcal{K}_r^G}$ for each $a \in \{1, 2, 3\}$. By Lemma~\ref{lem:krpacking_polytime_overlap}, it must be that $|U_i \cap U_{j_1}| > r/3$, $|U_i \cap U_{j_2}| > r/3$, and $|U_i \cap U_{j_3}| > r/3$. Since $|U_i|=r$ it follows by the pigeonhole principle that either $U_{j_1} \cap U_{j_2} \neq \varnothing$, $U_{j_2} \cap U_{j_3} \neq \varnothing$, or $U_{j_1} \cap U_{j_3} \neq \varnothing$ and thus $\{ U_i, U_{j_1}, U_{j_2}, U_{j_3} \}$ is not a claw in $\mathcal{K}_r^G$.
\end{proof}

\begin{thm}
\label{thm:krpacking_vdkr_polytime_5r3minus1}
If $\Delta(G)<5r/3-1$ then \vdkr can be solved in polynomial time.
\end{thm}
\begin{proof}
First, construct the $K_r$-vertex intersection graph $\mathcal{K}_r^G = (K_r^G, E_{\mathcal{K}_r^G})$. The set $K_r^G$ can be constructed in $O(\binom{|V|}{r})=O(|V|^r)$ by considering every possible set of $r$ vertices in $V$. The set $E_{\mathcal{K}_r^G}$ can then be constructed in $O(|V|^{2r})$ time. Next, find a maximum independent set in $\mathcal{K}_r^G$, which can be accomplished using Minty's algorithm (other algorithms have since been developed with improved worst-case time complexity \cite{FOS11}).
\end{proof}

We remark that Minty's algorithm also extends to the problem in which the vertices are weighted and the goal is to find an independent set of maximum total weight \cite{Nakamura01}. It might be interesting to study a weighted generalisation of \vdkr, in which a weight function exists on either the vertices or edges of the graph, and see if this approach can be used to derive other polynomial-time solvability results.

\subsection{Edge-disjoint packing} 

In this section we consider \edkr. Using Theorem~\ref{thm:krpacking_vdkr_polytime_5r3minus1} and Observation~\ref{obs:krpacking_edkr_is_also_vdkr} it is straightforward to show by that if $r \geq 4$ then \edkr can be solved in polynomial time. We state this result as Theorem~\ref{thm:krpacking_edkrgeq6_polytime}.

% NOTE - NOT A TYPO, this result is true for r \geq 4, but we only care about the case when r \geq 6, since we improve it for 4, 5.
\begin{thm}
\label{thm:krpacking_edkrgeq6_polytime}
If $r \geq 4$ and $\Delta(G) < 5r/3 - 1$ then \edkr can be solved in polynomial time.
\end{thm}
\begin{proof}
If $r > 3$ and $\Delta(G) < 5r/3 - 1$ then we can find a maximum vertex-disjoint $K_r$-packing in polynomial time by Theorem~\ref{thm:krpacking_vdkr_polytime_5r3minus1}. Such a packing is also a is also a maximum edge-disjoint $K_r$-packing, by Observation~\ref{obs:krpacking_edkr_is_also_vdkr}.
\end{proof}

We now show that this upper bound on $\Delta(G)$ can be improved if $r \in \{ 4, 5 \}$. The key insight in this case is that if $r \in \{ 4, 5 \}$ and $\Delta \leq 2r - 2$ then the $K_r$-edge intersection graph ${\mathcal{K}'}_r^G$ is claw-free.

\begin{lem}
\label{lem:krpacking_edgedisjoint_r45}
If $r\in \{ 4, 5 \}$ and $\Delta(G) \leq 2r - 2$ then the $K_r$-edge intersection graph ${\mathcal{K}'}_r^G$ is claw-free.
\end{lem}
\begin{proof}
Suppose $U_i, U_{j_1}, U_{j_2}, U_{j_3} \in K_r^G$ exist where $\{ U_i, U_{j_a} \} \in E_{{\mathcal{K}'}_r^G}$ for each $a \in \{1, 2, 3\}$. Suppose for a contradiction that $U_i, U_{j_1}, U_{j_2}, U_{j_3}$ is a claw in ${\mathcal{K}'}_r^G$, i.e.\ $|U_{j_a} \cap U_{j_b}| < 2$ for each $a, b \in \{ 1, 2, 3 \}$. By the definition of $E_{{\mathcal{K}'}_r^G}$, it follows that $|U_i \cap U_{j_a}| \geq 2$ for each $a \in \{ 1, 2, 3 \}$. Since $|U_i| = r \leq 5$, we assume without loss of generality that $|U_i \cap U_{j_1} \cap U_{j_2}| \geq 1$. Furthermore, it must be that $|U_i \cap U_{j_1} \cap U_{j_2}| = 1$, otherwise $|U_{j_1} \cap U_{j_2}| > 1$ which is a contradiction.  Let $v_r$ be the single vertex in $U_i \cap U_{j_1} \cap U_{j_2}$. Since $U_{j_1}$ and $U_{j_2}$ are $K_r$s in $G$ and $\deg_{G}(v_r) \leq 2r-2$ it must be that $N_{G}[v_r]=U_{j_1} \cup U_{j_2}$. Since $U_i$ is also a $K_r$ it follows that $U_i \subset U_{j_1} \cup U_{j_2}$. 

Now consider $U_i \cap U_{j_1}$ and $U_i \cap U_{j_2}$. If $|U_i \cap U_{j_1}| + |U_i \cap U_{j_2}| \geq r+2$ then since $|U_i|=r$ it follows that $|U_{j_1} \cap U_{j_2}| \geq 2$ which is a contradiction. It follows that $|U_i \cap U_{j_1}| + |U_i \cap U_{j_2}| \leq r + 1$ and either $|U_i \cap U_{j_1}| \leq (r+1)/2$ or $|U_i \cap U_{j_2}| \leq (r+1)/2$. Assume without loss of generality that $|U_i \cap U_{j_1}| \leq (r+1)/2$. 

Now consider $U_{j_3}$. Since $U_i, U_{j_1}, U_{j_2}, U_{j_3}$ is a claw in ${\mathcal{K}'}_r^G$, it must be that $|U_{j_3} \cap U_{i}| \geq 2$, $|U_{j_3} \cap U_{j_1}| \leq 1$, and $|U_{j_3} \cap U_{j_2}| \leq 1$. Since $U_i \subset U_{j_1} \cup U_{j_2}$, the only possibility is that $|U_{j_3} \cap U_i| = 2$, $|U_{j_3} \cap U_i \cap U_{j_2}| = 1$ and $|U_{j_3} \cap U_i \cap U_{j_1}| = 1$. 

Let $v_s$ be the single vertex in $U_{i} \cap U_{j_1} \cap U_{j_3}$. Since $U_i$, $U_{j_1}$, and $U_{j_3}$ are $K_r$s in $G$ it follows that $v_s$ is adjacent to every other vertex in $U_{j_3} \cup U_{i} \cup U_{j_1}$ so
\begin{align*}
    \deg_{G}(v_s) &\geq |U_i \cup U_{j_1} \cup U_{j_3}| - 1\\
    &= 3r - |U_i \cap U_{j_1}| - |U_{i} \cap U_{j_3}| - |U_{j_1} \cap U_{j_3}| + |U_i \cap U_{j_1} \cap U_{j_3}| - 1\enspace.
\end{align*}
Recall that since $U_i, U_{j_1}, U_{j_2}, U_{j_3}$ is a claw in ${\mathcal{K}'}_r^G$, $|U_{j_1} \cap U_{j_3}| \leq 1$. We deduced earlier in this proof that $|U_i \cap U_{j_1}| \leq (r+1)/2$, $|U_i \cap U_{j_3}| = 2$, and $|U_{j_3} \cap U_i \cap U_{j_1}| = 1$. Since $r \geq 4$ it follows that $\deg_{G}(v_s) \geq (5r-7)/2 > 2r - 2$, which contradicts the fact that $\Delta(G) \leq 2r - 2$.
\end{proof}

\begin{thm}
\label{thm:krpacking_edkr345_polytime}
If $r \leq 5$ and $\Delta \leq 2r - 2$ then \edkr can be solved in polynomial time.
\end{thm}
\begin{proof}
Caprara and Rizzi \cite{caprara_packing_2002} prove the case when $r=3$ and $\Delta \leq 4$. If $r \in \{ 4, 5 \}$ and $\Delta(G) \leq 2r - 2$ then Lemma~\ref{lem:krpacking_edgedisjoint_r45} shows that the $K_r$-edge intersection graph ${\mathcal{K}'}_r^G$ is claw-free. It follows that a maximum edge-disjoint $K_r$-packing can be found in polynomial time by constructing ${\mathcal{K}'}_r^G$, in $O(\binom{|V|}{2r})$ time, and finding in it a maximum independent set, which can also be accomplished in time polynomial in the size of ${\mathcal{K}'}_r^G$ \cite{FOS11}.
\end{proof}

\section{\texorpdfstring{$\APX$}{APX}-hardness}
\label{sec:krpacking_apxhardness}
\subsection{Vertex-disjoint packing} 

We show that if $\Delta \geq \lceil 5r/3 \rceil - 1$ then \vdkr is $\APX$-hard. In other words, there exists some fixed constant $\varepsilon > 0$ such that no polynomial-time $\varepsilon$-approximation algorithm exists for \vdkr, unless $\P=\NP$. To do this, we use an \emph{$L$-reduction} \cite{Crescenzi97}, which is a type of approximability-preserving reduction. An $L$-reduction from an optimisation problem $A$ to another optimisation problem $B$ implies that if $B$ admits a polynomial-time approximation scheme, then so does $A$.

We reduce from the problem of finding a Maximum Independent Set (MIS) in graphs that are $3$-regular and are triangle-free, which we refer to as \mistfvariant/ (Problem~\ref{pr:mistfvariant}).

\begin{myproblem}[Maximum Independent Set in $3$-regular Triangle-Free graphs (\mistfvariant/)]
\label{pr:mistfvariant}\mysymbolfirstusedefinition{symboldef:mistfvariant}{}
\begin{samepage}
\begin{adjustwidth}{8pt}{8pt}
\instance an undirected graph $G=(V, E)$ that is $3$-regular and triangle-free\\
\solution a set $S \subseteq V$ such that $\{ v_i, v_j \} \notin E$ for any $v_i, v_j \in S$\\
\measure $|S|$
\end{adjustwidth}
\end{samepage}
\end{myproblem}
Berman and Karpinski \cite{BK99} show that \mistfvariant/ is $\APX$-hard, providing an explicit lower bound on the approximation ratio (specifically, they showed that it is $\NP$-hard to approximate \mistfvariant/ within $140/139-\varepsilon$, for any $\varepsilon > 0$).

The reduction from \mistfvariant/ is as follows. Our goal is to construct a new graph $G'=(V', E')$ where each $K_r$ in $G'$ corresponds to exactly one vertex in $V$ and each vertex in $V$ corresponds to exactly one $K_r$ in $G'$. For any two adjacent vertices in $G$, the intersection of the two corresponding $K_r$s in $G'$ will contain exactly $\lfloor r/3 \rfloor$ vertices. 

To do this, first construct a set of $|V|$ disjoint $K_r$s in $G'$, labelled $\mathcal{U} = \{ U_1, U_2, \dots, U_{|V|} \}$ where $U_i = \{ u_i^1, u_i^2, \dots, u_i^r \}$. Next, consider each edge $\{ v_i, v_j \} \in E$. let $U'_i = \{ u_i^{a_1}, u_i^{a_2}, \dots, u_i^{a_{\lfloor r/3 \rfloor}} \}$ be any set of $\lfloor r/3 \rfloor$ vertices in $U_i$ with degree $r-1$ and $U'_j = \{ u_j^{b_1}, u_j^{b_2}, \dots, u_j^{b_{\lfloor r/3 \rfloor}} \}$ be any set of $\lfloor r/3 \rfloor$ vertices in $U_j$ with degree $r-1$. For each $q$ from $1$ to $\lfloor r/3 \rfloor$ inclusive, identify $u_i^{a_q}$ and $u_j^{b_q}$ to create a single vertex labelled $u_{ij}^{a_q}$. Label $U'_j = U'_i$ as $W_{ij}$. % Note that since $\Delta(G) = 3$, for any edge $\{ v_i, v_j \} \in E$ considered there must exist some set of $\lfloor r/3 \rfloor$ vertices in $U_i$ with degree $r-1$ and some set of $\lfloor r/3 \rfloor$ vertices in $U_j$ with degree $r-1$.

Finally, for each vertex $v_i \in V$ let $X_i$ be the set of (at least $r \bmod 3$) vertices in $U_i$ with degree $r-1$. Note that any vertex in $G'$ either belongs to some set $W_{ij}$ where $\{ v_i, v_j \} \in E$ or some set $X_i$ where $v_i \in V$.

We first show that the set of $K_r$s in $G'$ is $\mathcal{U}$.

\begin{lem}
\label{lem:krpacking_lissetofkrs}
$\mathcal{U} = K_r^{G'}$.
\end{lem}
\begin{proof}
By definition, $\mathcal{U} \subseteq K_r^{G'}$ so it remains to show that each $K_r$ in $G'$ belongs to $\mathcal{U}$. Suppose $K$ is an arbitrary $K_r$ in $G'$. By definition, any vertex in any set $X_i$ has degree $r-1$ in $G'$ and thus belongs to exactly one $K_r$ in $G'$, namely $U_i$, which belongs to $K_r^{G'}$. The only other possibility is that each vertex in $K$ belongs to some set $W_ij$ where $1\leq i, j \leq |V|$. Since $|W_ij| = \lfloor r/3 \rfloor$ it must be that either there exist three sets $W_{i_1, j_1}, W_{i_2, j_2}, W_{i_3, j_3}$ where $1\leq i_1, i_2, \dots, j_3 \leq |V|$ and $K \subseteq W_{i_1, j_1} \cup W_{i_2, j_2} \cup W_{i_3, j_3}$, or there exist four sets $W_{i_1, j_1}, W_{i_2, j_2}, W_{i_3, j_3}, W_{i_4, j_4}$ where $1\leq i_1, i_2, \dots, j_4 \leq |V|$ and $K \subseteq W_{i_1, j_1} \cup W_{i_2, j_2} \cup W_{i_3, j_3} \cup W_{i_4, j_4}$. In the latter case, we may assume without loss of generality that $\{ i_1, j_1 \} \cap \{ i_2, j_2 \} = \varnothing$. By the construction of $G'$ it follows that no edge exists between any vertex in $W_{i_1, j_1}$ and any vertex in $W_{i_2, j_2}$ which contradicts the supposition that $K$ is a $K_r$ in $G'$. It remains that there exist three sets $W_{i_1, j_1}, W_{i_2, j_2}, W_{i_3, j_3}$ where $K \subseteq W_{i_1, j_1} \cup W_{i_2, j_2} \cup W_{i_3, j_3}$ and $1\leq i_1, i_2, \dots, j_3 \leq |V|$.

By construction, the closed neighbourhood of any vertex in $W_{i_1, j_1}$ is $U_{i_1} \cup U_{j_1}$ so since $K$ is a $K_r$, without loss of generality assume that $i_1 \in \{ i_2, j_2 \}$ and $j_1 \in \{ i_3, j_3 \}$. A symmetric argument shows that $i_2 \in \{ i_1, j_1 \}$ and $j_2 \in \{ i_3, j_3 \}$, and $i_3 \in \{ i_1, j_1 \}$ and $j_3 \in \{ i_2, j_2 \}$. By symmetry, we need only consider the two cases, in which $i_1 = i_2 = i_3$ and in which $K = W_{i_1, i_2} \cup W_{i_2, i_3} \cup W_{i_3, i_1}$. In the former case, $K$ must be labelled $U_{i_1}$ and thus belongs to $K_r^G$. In the latter case, by the construction of $G'$ the three vertices $\{ v_{i_1}, v_{i_2}, v_{i_3} \}$ in $G$ form a triangle, which is a contradiction.
\end{proof}

\begin{lem}
\label{lem:krpacking_vdkr_reduction_degree}
$\Delta(G') = \lceil 5r/3 \rceil - 1$.
\end{lem}
\begin{proof}
By definition, any vertex in $X_i$ has degree $r-1$. Any vertex in $W_{ij}$ has degree $|U_i| + |U_j| - |W_{ij}| - 1 = 2r - \lfloor r/3 \rfloor - 1$, for any $1\leq i, j \leq |V|$.
\end{proof}

\begin{thm}
\label{thm:krpacking_vdkr_apxhard}
For any simple undirected graph $G'$, if $\Delta(G') \geq \lceil 5r/3 \rceil - 1$ then \vdkr is $\APX$-hard.
\end{thm}
\begin{proof}
We first show that a vertex-disjoint $K_r$-packing of size $B$ exists in $G'$ if and only if an independent set of size $B$ exists in $G$. By Lemma~\ref{lem:krpacking_lissetofkrs}, any $U_i, U_j \in K_r^G$ that are not vertex disjoint in $G'$ correspond to two vertices $v_i, v_j \in V$, which by the design of the reduction must be adjacent. Conversely, for any two $v_i, v_j \in V$ where $\{ v_i, v_j \} \in E$, by the design of the reduction it must be that the two corresponding $K_r$s, $U_i, U_j \in K_r^G$ are not vertex disjoint in $G'$. 
It follows that, for any graph $G$, $\textrm{opt}_{\textrm{\mistfvariant/}}(G) = \textrm{opt}_{\textrm{\vdkr}}(G')$. Moreover, for any vertex-disjoint $K_r$-packing $T$ in $G'$, there exists a corresponding independent set $S$ in $G$ where $|S|=|T|$, and thus that $\textrm{m}_{\textrm{\mistfvariant/}}(G, S) = \textrm{m}_{\textrm{\textrm{\vdkr}}}(G', T)$.

It follows that reduction from \mistfvariant/ to \vdkr is an $L$-reduction with $\alpha=\beta=1$ (also called a \emph{strict reduction} \cite{Crescenzi97}) and thus that \vdkr is $\APX$-hard even when $\Delta(G) = \lceil 5r/3 \rceil - 1$ (shown in Lemma~\ref{lem:krpacking_vdkr_reduction_degree}). To show that \vdkr is $\APX$-hard even when $\Delta(G) \geq \lceil 5r/3 \rceil - 1$, one can add to $G'$ a disconnected star.
% To prove that the reduction is correct we show that the tuple $( f, g, \alpha, \beta )$ meets the five conditions of a correct $L$-reduction \cite{ausielloetal99}. First, the reduction can be performed in polynomial time. Second, if $G$ is a triangle-free undirected graph with maximum degree $3$ and there exists an independent set in $G$ then by Lemma~\ref{lem:krpacking_reductioniff} there also exists a $K_r$-packing in $G'$. Third, given any undirected graph $G$ and any $K_r$-packing in the constructed graph $G'$, a corresponding independent set $S$ in $G$ can be easily constructed in polynomial time. Fourth, if the size of a maximum $K_r$-packing $S'$ in $G'$ is $B$ then by Lemma~\ref{lem:krpacking_reductioniff} the size of the maximum independent set in $G$ is also $B$. Fifth, also by Lemma~\ref{lem:krpacking_reductioniff}, for any $K_r$-packing $S'$ in the constructed graph $G'$, the difference between the the size of a maximum independent set $S$ in $G$ and the independent set in $G$ corresponding to $S'$ is the same as the difference between the size of a maximum $K_r$-packing in $G'$ and the size of $S'$.
\end{proof}

\subsection{Edge-disjoint packing} 

\subsubsection{Edge-disjoint \texorpdfstring{$K_r$}{Kr}-packing when \texorpdfstring{$r \geq 6$}{r >= 6}}

If $r \geq 6$ and $\Delta = \lceil 5r/3 \rceil - 1$ then it must be that $\Delta < 2r - 2$, so by Observation~\ref{obs:krpacking_edkr_is_also_vdkr}, any edge-disjoint $K_r$-packing is also vertex disjoint. It follows by Theorem~\ref{thm:krpacking_vdkr_apxhard} that \edkr is $\APX$-hard for any $r \geq 6$ even when $\Delta = \lceil 5r/3 \rceil - 1$. We generalise this result in Theorem~\ref{thm:krpacking_edkr_rgeq6_apxhard}.

% In this section we show that \edkr if $\APX$-hard for any $r \geq 6$ if $\Delta \geq \lceil 5r/3 \rceil - 1$. The proof uses the fact that if $r \geq 6$ then $\lceil 5r/3 \rceil - 1 < 2r - 2$ and thus by Observation~\ref{obs:krpacking_edkr_is_also_vdkr}, any edge-disjoint $K_r$-packing is also vertex-disjoint.

\begin{thm}
\label{thm:krpacking_edkr_rgeq6_apxhard}
If $r \geq 6$ and $\Delta \geq \lceil 5r/3 \rceil - 1$ then \edkr is $\APX$-hard.
\end{thm}
\begin{proof}
Suppose $\Delta = \lceil 5r/3 \rceil - 1$. By definition, any vertex-disjoint $K_r$-packing is also edge disjoint. Since $r \geq 6$ it follows that $\Delta = \lceil 5r/3 \rceil - 1 < 2r - 2$ so by Observation~\ref{obs:krpacking_edkr_is_also_vdkr}, any edge-disjoint $K_r$-packing is also vertex disjoint. We have shown that an edge-disjoint $K_r$-packing of size $B$ exists in $G$ if and only if a vertex-disjoint $K_r$-packing of size $B$ exists in $G$. This fact constitutes an $L$-reduction with $\alpha=\beta=1$ from the restricted case of \vdkr in which $\Delta = \lceil 5r/3 \rceil - 1$. The lemma follows by Theorem~\ref{thm:krpacking_vdkr_apxhard}. As in the proof of Theorem~\ref{thm:krpacking_vdkr_apxhard}, to show that \edkr is $\APX$-hard if $r\geq 6$ even when $\Delta(G) \geq \lceil 5r/3 \rceil - 1$, one can add to $G'$ a disconnected star.
\end{proof}

\subsubsection{Edge-disjoint \texorpdfstring{$K_4$}{K4}-packing} 
\label{sec:krpacking_edkfour}

We present an $L$-reduction from a variant of the \emph{Maximum Satisfiability} problem to \edkfour when $\Delta=7$, by extending the $L$-reduction of Caprara and Rizzi \cite{caprara_packing_2002} for \edkthree when $\Delta=5$.

An instance of Maximum Satisfiability is a boolean formula $\phi$ in conjunctive normal form with \emph{clauses} $C$ and variable set $X$. Each clause contains a set of \emph{literals}. Each literal is formed by either a variable or its negation. A \emph{truth assignment} $\mathfrak{f}$ is a function $\mathfrak{f} : X \mapsto \{ \text{true}, \text{false} \}$. A clause is \emph{satisfied} by $\mathfrak{f}$ if any of its literals are true. The goal is to find a truth assignment that satisfies the maximum number of clauses. We reduce from the special case of Maximum Satisfiability in which each clause contains at most two literals and each variable occurs in at most three clauses. We shall refer to this special case as \maxtwosatthree/ (Problem~\ref{pr:maxtwosatthree}).

\begin{myproblem}[\maxtwosatthree/]
\label{pr:maxtwosatthree}\mysymbolfirstusedefinition{symboldef:maxtwosatthree}{}
\begin{samepage}
\begin{adjustwidth}{8pt}{8pt}
\instance a boolean formula $\phi$ in conjunctive normal form, represented as a set of clauses $C = \{ c_1, c_2, \dots, c_{|C|} \}$ and a set of variables $X = \{ x_1, x_2, \dots, x_{|X|} \}$, in which each clause contains at most two literals and each variable occurs in at most three clauses\\
\solution a truth assignment $\mathfrak{f} : X \mapsto \{ \text{true}, \text{false} \}$\\
\measure the number of clauses in $\phi$ satisfied by $\mathfrak{f}$
\end{adjustwidth}
\end{samepage}
\end{myproblem}

Let $m_i$ be the number of occurrences of variable $x_i$ in $\phi$ for each variable $x_i \in X$. We assume that $2\leq m_i \leq 3$ for each $x_i \in X$, since if some variable $x_i$ occurs in exactly one clause it can be set to the value satisfying that clause. \maxtwosatthree/ is $\APX$-hard \cite{ACGKMP99}.

Given an instance $\phi$ of \maxtwosatthree/, we construct a graph $G$ such that a truth assignment for $\phi$ exists that satisfies at least $k$ clauses if and only if there exists an edge-disjoint $K_4$-packing of size at least $\sum_{i=1}^{|X|} 3 m_i + k$. As in the case of the reduction presented for \edkthree by Caprara and Rizzi, the reduction here is one of local replacement \cite{GJ79}. As they remark, the construction and connection of variable and clause gadgets is a standard technique when reducing from variants of Maximum Satisfiability. The reduction, shown in Figure~\ref{fig:krpacking_edkfour}, is as follows.

\begin{figure}
    \centering
    \newcommand\Kfourdraw[4]{%
    \draw (#1) -- (#2) -- (#3) -- (#4) -- (#1);
    \draw (#1) -- (#3);
    \draw (#2) -- (#4);
}

\tikzset{gradientpath/.style n args={3}{
    postaction={
    decorate,
    decoration={
    markings,
    mark=between positions 0 and \pgfdecoratedpathlength step 0.2pt with {
    \pgfmathsetmacro\myval{multiply(
        divide(
        \pgfkeysvalueof{/pgf/decoration/mark info/distance from start}, \pgfdecoratedpathlength
        ),
        100
    )};
    \pgfsetfillcolor{#3!\myval!#2};
    \pgfpathcircle{\pgfpointorigin}{#1};
    \pgfusepath{fill};}
}}}}


\begin{tikzpicture}

% \filldraw[color=red, fill=none](0.0, 0.0) circle (\innerradius);
% \filldraw[color=red, fill=none](0.0, 0.0) circle (\outerradius);

\begin{scope}[shift={(-4.0, 0.0)}, every node/.style={thick, circle, draw, minimum size=2.4mm, fill=white}]

\def\innerradius{4.4}
\def\outerradius{6.4}

\node[draw=none] (hijminus1) at (105:{\outerradius}) {};
\node[draw=none, label={[label distance=0.4cm]90:$a_i^j$}] (aij) at (90:{\outerradius}) {};
\node[label={[label distance=0.4cm]80:$b_i^j$}] (bij) at (75:{\outerradius}) {};
\node[draw=none, label={[label distance=0.4cm]70:$c_i^j$}] (cij) at (60:{\outerradius}) {};
\node[draw=none, label={[label distance=0.4cm]60:$d_i^j$}] (dij) at (45:{\outerradius}) {};
\node[label={[label distance=0.4cm]50:$e_i^j$}] (eij) at (30:{\outerradius}) {};
\node[draw=none, label={[label distance=0.4cm]40:$h_i^j$}] (hij) at (15:{\outerradius}) {};
\node[draw=none] (aij1) at (0:{\outerradius}) {};

\node[draw=none] (yijminus1) at (112.5:{\innerradius}) {};
\node[label={[label distance=0.4cm]270:$u_i^j$}] (uij) at (90:{\innerradius}) {};
\node[label={[label distance=0.4cm]247.5:$v_i^j$}] (vij) at (67.5:{\innerradius}) {};
\node[label={[label distance=0.4cm]225:$w_i^j$}] (wij) at (45:{\innerradius}) {};
\node[label={[label distance=0.4cm]202.5:$y_i^j$}] (yij) at (22.5:{\innerradius}) {};
\node[draw=none] (uij1) at (0:{\innerradius}) {};

\path[gradientpath={0.2pt}{black}{white}] (aij) -- (hijminus1);
\path[gradientpath={0.2pt}{black}{white}] (aij) -- (yijminus1);
\path[gradientpath={0.2pt}{black}{white}] (uij) -- (yijminus1);
\path[gradientpath={0.2pt}{black}{white}] (uij) -- (hijminus1);
% \draw (yijminus1) -- (hijminus1);
\Kfourdraw{aij}{bij}{uij}{vij}
\Kfourdraw{cij}{vij}{dij}{wij}
\draw (bij) -- (cij);
\draw (dij) -- (eij);
\draw (wij) -- (yij);
\draw (dij) -- (yij);
\draw (wij) -- (eij);
\draw (eij) -- (yij);
\draw (eij) -- (hij);
\draw (yij) -- (hij);

\path[gradientpath={0.2pt}{black}{white}] (hij) -- (aij1);
\path[gradientpath={0.2pt}{black}{white}] (yij) -- (uij1);
\path[gradientpath={0.2pt}{black}{white}] (hij) -- (uij1);
\path[gradientpath={0.2pt}{black}{white}] (yij) -- (aij1);

\draw plot [smooth] coordinates {(aij) ($ (bij) !.4! (vij) $) (cij)};
\draw plot [smooth] coordinates {(dij) ($ (eij) !.4! (yij) $) (hij)};

% redraw a,c,d,h
\node (aijfake) at (aij) {};
\node (cijfake) at (cij) {};
\node (dijfake) at (dij) {};
\node (hijfake) at (hij) {};

\begin{scope}[shift={(22.5:3.8)}]
\begin{scope}[shift={(-67.5:2.6)}]
\node[label={[label distance=0.4cm]-67.5:$w_r$}] (wr) at (30:{\outerradius}) {};
\end{scope}
\end{scope}

\begin{scope}[shift={(22.5:2.6)}]
\node[label={[label distance=0.4cm]112.5:$s_r^1$}] (sr1) at (30:{\outerradius}) {};
\node[label={[label distance=0.4cm]-67.5:$t_r^1$}] (tr1) at (15:{\outerradius}) {};
\end{scope}

\begin{scope}[shift={(22.5:5.0)}]
\node[label={[label distance=0.4cm]112.5:$s_r^2$}] (sr2) at (30:{\outerradius}) {};
\node[label={[label distance=0.4cm]-67.5:$t_r^2$}] (tr2) at (15:{\outerradius}) {};
\end{scope}

\draw (sr1) -- (tr1) -- (wr) -- (tr2) -- (sr2) -- cycle;
\draw (tr1) -- (sr2);
\draw (tr2) -- (sr1);
% alternate idea for curved lines
% \path (wr) edge [bend right=10] (sr1);
% \path (wr) edge [bend left=10] (sr2);
\draw (wr) -- (sr1);
\draw (wr) -- (sr2);
\draw (sr1) -- (sr2);
\end{scope}


% clause gadget

% \begin{scope}[shift={(4.0, 0.0)}]
% \draw[help lines] (0,0) grid (4,4);
% \end{scope}


% draw the gradient, based on https://tex.stackexchange.com/q/606045
% \begin{scope}
% \draw[red] (0,0) ++(90:\outerradius) arc (90:105:\outerradius);
% \def\startangle{90}
% \def\changeangle{22.5}
% \def\inter{1}
% \begin{scope}
%     \foreach \i in {0,\inter,...,\changeangle}
%         {
%         \pgfmathsetmacro\ix{\i+\startangle}
%         \pgfmathsetmacro\colorvalue{\i/\changeangle}
%         \definecolor{slicecolor}{rgb}{\colorvalue,\colorvalue,\colorvalue}
%         \pgfmathsetmacro\jx{\i+\inter+\startangle}
%         \filldraw[thin,red,fill opacity=\colorvalue, draw=none] (0:0) -- ((\ix:\outerradius) arc (\ix:\jx:\outerradius) -- (0:0) -- cycle;
%         }
% \end{scope}
% \end{scope}


\end{tikzpicture}
    \caption{The reduction from \maxtwosatthree/ to \edkfour}
    \label{fig:krpacking_edkfour}
\end{figure}

For each variable $x_i$, construct a variable gadget of $10 m_i$ vertices, labelled $R_i = \{ a_i^j, b_i^j, c_i^j, d_i^j, e_i^j, h_i^j, u_i^j, v_i^j, w_i^j, y_i^j \}$ for each $j$ where $1\leq j \leq m_i$. For each $j$ where $1\leq j \leq m_i$, add an edge (if it does not exist already) between each pair of vertices in $\{ a_i^j, b_i^j, u_i^j, v_i^j \}$; $\{ a_i^j, b_i^j, c_i^j, v_i^j \}$; $\{ c_i^j, v_i^j, d_i^j, w_i^j \}$; $\{ d_i^j, w_i^j, e_i^j, y_i^j \}$; $\{ d_i^j, e_i^j, h_i^j, y_i^j \}$; and finally $\{ h_i^j, a_i^{j+1}, y_i^j, u_i^{j+1} \}$ if $j < m_i$ and otherwise $\{ h_i^j, a_i^1, y_i^j, u_i^1 \}$.

We shall refer to $\{ a_i^j, b_i^j, u_i^j, v_i^j \}$, $\{ c_i^j, v_i^j, d_i^j, w_i^j \}$, and $\{ d_i^j, e_i^j, h_i^j, y_i^j \}$ as the \emph{even $K_4$s in $R_i$}, and $\{ a_i^j, b_i^j, c_i^j, v_i^j \}$, $\{ d_i^j, w_i^j, e_i^j, y_i^j \}$, and $\{ h_i^j, a_i^{j+1}, y_i^j, u_i^{j+1} \}$ (and $\{ h_i^j, a_i^1, y_i^j, u_i^1 \}$) as the \emph{odd $K_4$s in $R_i$}. Note that at this point in construction, $\deg_{G}(a_i^j) = \deg_{G}(v_i^j) = \deg_{G}(d_i^j) = \deg_{G}(y_i^j) = 6$, $\deg_{G}(u_i^j) = \deg_{G}(c_i^j) = \deg_{G}(w_i^j) = \deg_{G}(h_i^j) = 5$, and $\deg_{G}(b_i^j) = \deg_{G}(e_i^j) = 4$ for each $j$ where $1\leq j \leq m_i$. 

We shall now construct the clause gadgets. For each clause $c_r$, construct a clause gadget of $5$ vertices labelled $S_r = \{ s_r^1, t_r^1, s_r^2, t_r^2, w_r \}$. Add an edge (if it does not exist already) between each pair of vertices in $\{ s_r^1, t_r^1, s_r^2, w_r \}$ and $\{ s_r^1, s_r^2, t_r^2, w_r \}$. We shall refer to $\{ s_r^1, t_r^1, s_r^2, w_r \}$ and $\{ s_r^1, s_r^2, t_r^2, w_r \}$ as $P_i^r$ and $P_j^r$ supposing the variables of the first and second literals in $c_r$ are $x_i$ and $x_j$. Note that at this point in construction, $\deg_{G}(s_r^1) = \deg_{G}(s_r^2) = \deg_{G}(w_r) = 4$ and $\deg_{G}(t_r^1) = \deg_{G}(t_r^2) = 3$.

We shall now connect the variable and clause gadgets. For each clause $c_r$, suppose $x_i$ is the variable of some literal in $c_r$ where $c_r$ contains the $j\textsuperscript{th}$ occurrence of $x_i$ in $\phi$. If $x_i$ is the first literal in $c_r$ and occurs positively in $c_r$ then identify $b_i^j$ and $s_r^1$, and $c_i^j$ and $t_r^1$. We shall hereafter refer to the first identified vertex as either $b_i^j$ or $s_r^1$ and the second identified vertex as either $c_i^j$ or $t_r^1$. Note that now $\deg_{G}(b_i^j) = \deg_{G}(c_i^j) = 7$. Similarly, if $x_i$ is the first literal in $c_r$ and occurs negatively in $c_r$ then identify $e_i^j$ and $s_r^1$, and $h_i^j$ and $t_r^1$. In this case $\deg_{G}(e_i^j) = \deg_{G}(h_i^j) = 7$. If $x_i$ is the second literal in $c_r$ and occurs positively in $c_r$ then identify $b_i^j$ and $s_r^2$, and $c_i^j$ and $t_r^2$. Similarly, if $x_i$ is the second literal in $c_r$ and occurs negatively in $c_r$ then identify $e_i^j$ and $s_r^2$, and $h_i^j$ and $t_r^2$. This completes the construction of $G$. Observe that $\Delta = 7$.

It is straightforward that the reduction can be performed in polynomial time. We now prove that the reduction is correct in the first direction. By construction, no $K_4$ exists in $G$ that contains at least one vertex in a variable gadget and at least one vertex in a clause gadget. Thus, we shall say that some $K_4$ is \emph{in} a variable or clause gadget if it is a strict subset of that gadget.

\begin{lem}
\label{lem:krpacking_kfourpacking_firstdirection}
If a truth assignment $\mathfrak{f}$ for $\phi$ satisfies at least $k$ clauses then an edge-disjoint $K_4$-packing $T$ exists in $G$ where $|T| \geq \sum_{i=1}^{|X|} 3 m_i + k$.
\end{lem}
\begin{proof}
Suppose $\mathfrak{f}$ is a truth assignment for $\phi$ that satisfies at least $k$ clauses. We shall construct an edge-disjoint $K_4$-packing $T$ where $|T| \geq \sum_{i=1}^{|X|} 3 m_i + k$.

For each variable $x_i$, if $\mathfrak{f}(x_i)$ is true then add the set of even $K_4$s in $R_i$ to $T$. Similarly, if $\mathfrak{f}(x_i)$ is false then add the set of odd $K_4$s in $R_i$ to $T$. Now $|T|=\sum_{i=1}^{|X|} 3 m_i$.
For each clause gadget $c_r$ that is satisfied by $\mathfrak{f}$, it must be that there exists some variable $x_i$ where either $\mathfrak{f}(x_i)$ is true and $x_i$ occurs positively in $c_r$ or $\mathfrak{f}(x_i)$ is false and $x_i$ occurs negatively in $c_r$. In either case, add $P_i^r$ to $T$. Now, $T$ contains exactly $\sum_{i=1}^{|X|} 3m_i$ $K_4$s in variable gadgets and at least $k$ $K_4$s in clause gadgets.
It remains to show that $T$ is edge disjoint. By the construction of $G$, any two $K_4$s in $T$ in the same variable gadget are edge disjoint. Consider an arbitrary $P_r^i$ in some clause gadget $c_r$ that belongs to $T$. It must be that either $\mathfrak{f}(x_i)$ is true and $x_i$ occurs positively in $c_r$ or $\mathfrak{f}(x_i)$ is false and $x_i$ occurs negatively in $c_r$. In the former case, $T$ contains the set of even $K_4$s in $R_i$ so since $P_i^r \cap R_i = \{ b_i^j, c_i^j \}$ where $1\leq j\leq 3$ it follows that $T$ is edge disjoint. In the latter case, $T$ contains the set of odd $K_4$s in $R_i$ so since $P_i^r \cap R_i = \{ e_i^j, h_i^j \}$ where $1\leq j\leq 3$ it also follows that $T$ is edge disjoint.
\end{proof}

We now prove that the reduction is correct in the second direction. We say that some edge-disjoint $K_4$-packing $T$ in $G$ is \emph{canonical} if for any variable gadget $R_i$, $T$ contains either the set of even $K_4$s in $R_i$ or the set of odd $K_4$s in $R_i$. By the construction of $G$, no edge-disjoint $K_4$-packing can contain all even $K_4$s and all odd $K_4$s.

We first show that for any variable gadget $R_i$ and edge-disjoint $K_r$-packing $T$, if $T$ neither contains all even $K_4$s in $R_i$ nor all odd $K_4$s in $R_i$ then the number of $K_4$s in $T$ is at most $3 m_i - 1$.

\begin{prop}
\label{prop:krpacking_kfour_evenoddareonlymaximum}
Suppose $T$ is an arbitrary edge-disjoint $K_4$-packing in $G$. For any variable gadget $R_i$, if $T$ neither contains all even $K_4$s in $R_i$ nor all odd $K_4$s in $R_i$ then the number of $K_4$s in $T$ is at most $3 m_i - 1$.
\end{prop}
\begin{proof}
By the construction of $G$, each even $K_4$ in $R_i$ intersects exactly two odd $K_4$s in $R_i$ by at least two vertices and each odd $K_4$ in $R_i$ intersects exactly two even $K_4$s in $R_i$ by at least two vertices. 

It follows that the $K_4$-edge intersection graph ${\mathcal{K}'}_r^G$ contains a cycle of $6 m_i$ vertices corresponding to the $6 m_i$ $K_4$s in $R_i$. It then follows that any edge-disjoint $K_4$-packing that contains $3 m_i$ $K_4$s in $R_i$ corresponds to an independent set of size $3 m_i$ in ${\mathcal{K}'}_r^G$, and thus is either the set of even $K_4$s in $R_i$ or the set of odd $K_4$s in $R_i$. Since $T$ neither contains all even $K_4$s in $R_i$ nor all odd $K_4$s in $R_i$ it follows that $|T| < 3m_i$.
\end{proof}

We can now prove that for any edge-disjoint $K_4$-packing in $G$ that is not canonical, there exists a canonical edge-disjoint $K_4$-packing in $G$ of at least the same cardinality.

\begin{lem}
If $T$ is an edge-disjoint $K_4$-packing then there exists a canonical edge-disjoint $K_4$-packing $T'$ where $|T'| \geq |T|$.
\label{lem:krpacking_four_canonical}
\end{lem}
\begin{proof}
If $T$ is already canonical then let $T'=T$. Otherwise, by the definition of canonical, there must exist at least one variable gadget $i$ such that $T$ neither contains all even $K_4$s in $R_i$ nor all odd $K_4$s in $R_i$. For any such $i$ where $1\leq i \leq |X|$, we show how to modify $T$ to ensure that it either contains the set of even $K_4$s in $R_i$ or the set of odd $K_4$s in $R_i$ and the cardinality of $T$ does not decrease. It follows that there exists a canonical edge-disjoint $K_4$-packing $T'$ where $|T'| \geq |T|$. 

By Proposition~\ref{prop:krpacking_kfour_evenoddareonlymaximum}, the number of $K_4$s in $R_i$ in $T$ is at most $3 m_i - 1$. 

Suppose the variable $x_i$ corresponding to $R_i$ occurs in clauses $c_{r_1}, c_{r_2}, \dots, c_{r_{m_i}}$, corresponding to the sets $P_i^{r_1}, P_i^{r_2}, \dots, P_i^{r_{m_i}}$. It must be that either at most one $K_4$ in $\{ P_i^{r_1}, P_i^{r_2}, \dots, P_i^{r_{m_i}} \}$ exists in $T$ where the corresponding occurrence of $x_i$ is positive; or at most one $K_4$ in $\{ P_i^{r_1}, P_i^{r_2}, \dots, P_i^{r_{m_i}} \}$ exists in $T$ where the corresponding occurrence of $x_i$ is negative. Suppose the former case is true. Remove the $K_4$ in $\{ P_i^{r_1}, P_i^{r_2}, \dots, P_i^{r_{m_i}} \}$ in $T$ where the corresponding occurrence of $x_i$ is positive. Next, remove any even $K_4$s in $R_i$ in $T$ and add the set of odd $K_4$s in $R_i$ not already in $T$. The number of $K_4$s in $R_i$ in $T$ is now $3 m_i$ so since at most one $K_4$ was removed, which was not in $R_i$, it follows that the cardinality of $T$ has not decreased. To see that $T$ is still edge-disjoint, observe that any $K_4$ in $\{ P_i^{r_1}, P_i^{r_2}, \dots, P_i^{r_{m_i}} \}$ in $T$ intersects any odd $K_4$ in $R_i$ by at most one vertex. The construction and proof in the latter case are symmetric.
\end{proof}

\begin{lem}
\label{lem:krpacking_kfourpacking_seconddirection}
If $T$ is an edge-disjoint $K_4$-packing where $|T| = \sum_{i=1}^{|X|} 3 m_i + k$ for some integer $k\geq 1$ then exists a truth assignment $\mathfrak{f}$ that satisfies at least $k$ clauses.
\end{lem}
\begin{proof}
Assume by Lemma~\ref{lem:krpacking_four_canonical} that $T$ is canonical. It follows that $T$ contains exactly $\sum_{i=1}^{|X|} 3 m_i$ $K_4$s in variable gadgets and at least $k$ $K_4$s in clause gadgets. For each variable $x_i$, set $\mathfrak{f}(x_i)$ to be true if $T$ contains all even $K_4$s in $R_i$ and false otherwise. Now consider each clause gadget $c_r$ where $S_r$ contains some $K_4$ in $T$, denoted $P_i^r$. Suppose $x_i$ occurs positively in $c_r$. It follows that $P_i^r$ contains $b_i^j, c_i^j$ for some $j$ where $1\leq j\leq 3$. Since $T$ is canonical and edge-disjoint it follows that $T$ contains the set of even $K_4$s in $R_i$. By the construction of $\mathfrak{f}$ it follows that $\mathfrak{f}(x_i)$ is true and thus $c_r$ is satisfied. The proof for when $x_i$ occurs negatively in $c_r$ is symmetric. It follows that at least $k$ clauses are satisfied by $\mathfrak{f}$.
\end{proof}

\begin{lem}
\label{lem:krpacking_edkr_req4_apxhard}
If $r=4$ and $\Delta=7$ then \edkr is $\APX$-hard.
\end{lem}
\begin{proof}
We shall describe an $L$-reduction from \maxtwosatthree/ (which is $\APX$-hard \cite{ACGKMP99}) to \edkfour when $\Delta = 7$, using the definition of Crescenzi \cite{Crescenzi97}. An $L$-reduction from optimisation problem $Q$ to an optimisation problem $P$ shows that if there exists a $(1+\delta)$-approximation algorithm for $P$ then there exists a $(1 + \alpha\beta\delta)$-approximation algorithm for $Q$. For compactness we abbreviate \maxtwosatthree/ when appearing in a subscript to \maxtwosatthreeshort/.

An $L$-reduction is characterised by a pair $(f, g)$ of functions that can be computed in polynomial time. Here, $f$ is the reduction described at the start of the start of this section (Section~\ref{sec:krpacking_edkfour}) in which an instance $G$ of \edkfour is constructed from an arbitrary instance $\phi$ of \maxtwosatthree/. It is straightforward to show that $f$ can be computed in polynomial time.

The function $g$ is described by Lemma~\ref{lem:krpacking_kfourpacking_seconddirection}. For any instance $\phi$ of \maxtwosatthree/ and edge-disjoint $K_4$-packing in $f(\phi)$, $g$ computes a truth assignment $\mathfrak{f}$ for $\phi$. It is also straightforward to show that $g$ can be computed in polynomial time.

To show that $f, g$ constitute a valid $L$-reduction, we must show that there exists fixed constants $\alpha, \beta$ such that for any instance $\phi$ of \maxtwosatthree/,
\begin{align*}
    \textrm{opt}_{\textrm{\edkfour}}(f(\phi)) \leq \alpha \textrm{opt}_{\textrm{\maxtwosatthreeshort/}}(\phi)
\end{align*}
and that for any instance $\phi$ and any edge-disjoint $K_4$-packing $T$ in $f(\phi)$,
\begin{align*}
    \textrm{opt}_{\textrm{\maxtwosatthreeshort/}}(\phi) - \textrm{m}_{\textrm{\maxtwosatthreeshort/}}(\phi, g(\phi, T)) \leq \beta(\textrm{opt}_{\textrm{\edkfour}}(f(\phi)) - \textrm{m}_{\textrm{\edkfour}}(f(\phi), T))\enspace.
\end{align*}
We shall now demonstrate the existence of some such $\alpha$ and $\beta$. Recall that in the instance of \maxtwosatthree/, $X$ is the set of variables, $C$ is the set of clauses, and $m_i$ is the number of occurrences of each variable $x_i$. Note that by the definition of \maxtwosatthree/, $\sum_{i=1}^{|X|} m_i$ is the total number of literals, which must be at most $2|C|$. Note also that for any instance $\phi$ of \maxtwosatthree/, it must be that $\textrm{opt}_{\textrm{\maxtwosatthreeshort/}}(\phi) \geq |C|/2$. This is because a truth assignment satisfying $|C|/2$ clauses can be found using a greedy algorithm that in each step assigns a truth value to a variable occurring in the maximum number of clauses \cite{approximationvazirani}. We can now show that
\begin{align*}
    \textrm{opt}_{\textrm{\edkfour}}(f(\phi)) &\leq \sum\limits_{i=1}^{|X|} 3 m_i + \textrm{opt}_{\textrm{\maxtwosatthreeshort/}}(\phi) && \mbox{by Lemma~\ref{lem:krpacking_kfourpacking_seconddirection}}\\
    &= 3 \sum\limits_{i=1}^{|X|} m_i + \textrm{opt}_{\textrm{\maxtwosatthreeshort/}}(\phi)\\[-0.8em]
    &\leq 6|C| + \textrm{opt}_{\textrm{\maxtwosatthreeshort/}}(\phi) && \mbox{since $2|C| \geq \sum_{i=1}^{|X|} m_i$}\\[-1.0em]
    &\leq 13 \textrm{opt}_{\textrm{\maxtwosatthreeshort/}}(\phi)  && \mbox{since $\textrm{opt}_{\textrm{\maxtwosatthreeshort/}}(\phi) \geq |C|/2$}
\end{align*}
so $\alpha = 13$. We can also show that for any instance $\phi$ and any edge-disjoint $K_4$-packing $T$ in $f(\phi)$, 
\begin{align*}
    \textrm{opt}_{\textrm{\maxtwosatthreeshort/}}(\phi) - \textrm{m}_{\textrm{\maxtwosatthreeshort/}}(\phi, g(\phi, T)) &\leq \textrm{opt}_{\textrm{\maxtwosatthreeshort/}}(\phi) - \left(|T| - \sum\limits_{i=1}^{|X|} 3 m_i \right) && \mbox{by Lemma~\ref{lem:krpacking_kfourpacking_seconddirection}}\\
    &= \sum\limits_{i=1}^{|X|} 3 m_i + \textrm{opt}_{\textrm{\maxtwosatthreeshort/}}(\phi) - |T|\\[-0.8em]
    &\leq \textrm{opt}_{\textrm{\edkfour}}(f(\phi)) - |T| && \mbox{by Lemma~\ref{lem:krpacking_kfourpacking_firstdirection}}\\[0.2em]
    &= \textrm{opt}_{\textrm{\edkfour}}(f(\phi)) - \textrm{m}_{\textrm{\edkfour}}(f(\phi), T)
\end{align*}
since $\textrm{m}_{\textrm{\edkfour}}(f(\phi), T) = |T|$, which shows that $\beta = 1$.
\end{proof}

\subsubsection{Edge-disjoint \texorpdfstring{$K_5$}{K5}-packing} 

In this section we show that \edkfive is $\APX$-hard even when $\Delta=9$. The proof uses an $L$-reduction that follows the same pattern as the one shown in Section~\ref{sec:krpacking_edkfour} for \edkfour, extending the $L$-reduction of Caprara and Rizzi \cite{caprara_packing_2002}. The reduction, shown in Figure~\ref{fig:krpacking_edkfive}, is as follows.
\begin{figure}
    \centering
    \tikzset{
  laser beam action/.style={
    line width=\pgflinewidth+1.0pt,draw opacity=.12,draw=#1,
  },
  laser beam recurs/.code 2 args={%
    \pgfmathtruncatemacro{\level}{#1-1}%
    \ifthenelse{\equal{\level}{0}}%
    {\tikzset{preaction={laser beam action=#2}}}%
    {\tikzset{preaction={laser beam action=#2,laser beam recurs={\level}{#2}}}}
  },
  laser beam/.style={preaction={laser beam recurs={30}{#1}},draw opacity=1,draw=#1},
}

\newcommand\Kfivedraw[5]{%
    \draw (#1) -- (#2) -- (#3) -- (#4) -- (#5) -- (#1);
    \draw (#1) -- (#3);
    \draw (#1) -- (#4);
    \draw (#2) -- (#4);
    \draw (#2) -- (#5);
    \draw (#3) -- (#5);
}

\tikzset{gradientpath/.style n args={3}{
    postaction={
    decorate,
    decoration={
    markings,
    mark=between positions 0 and \pgfdecoratedpathlength step 0.2pt with {
    \pgfmathsetmacro\myval{multiply(
        divide(
        \pgfkeysvalueof{/pgf/decoration/mark info/distance from start}, 
        \pgfdecoratedpathlength
        ),
        100
    )};
    \pgfsetfillcolor{#3!\myval!#2};
    \pgfpathcircle{\pgfpointorigin}{#1};
    \pgfusepath{fill};}
}}}}

% fpu reciprocal from https://tex.stackexchange.com/a/537016, seemingly helps avoid 'dimension too large' errors
\begin{tikzpicture}[use fpu reciprocal]

\begin{scope}[shift={(-4.0, 0.0)}, every node/.style={thick, circle, draw, minimum size=2.4mm, fill=white}]

% \def\innerradius{4.4}
% \def\middleradius{5.4}
% \def\outerradius{6.4}

\begin{scope}[scale=2.4]

\node[draw=none, label={[label distance=0.4cm]197.5:$v_i^j$}] (vij) at (0.0, 0.0) {};
\node[draw=none, label={[label distance=0.4cm]0:$c_i^j$}] (cij) at (1.0, 0.0) {};
\node[label={[label distance=0.4cm]180:$u_i^{j+1}$}] (uij1) at (0.0, -1.0) {};
\node[label={[label distance=0.4cm]0:$h_i^j$}] (hij) at (0.5, -0.5) {};
\node[draw=none, label={[label distance=0.4cm]0:$d_i^j$}] (dij) at (1.0, -1.0) {};

\begin{scope}[rotate=35]
\node[draw=none, label={[label distance=0.4cm]35:$b_i^j$}] (bij) at (1.0, 0.0) {};
\node[draw=none, label={[label distance=0.4cm]215:$u_i^j$}] (uij) at (0.0, 1.0) {};
\node[label={[label distance=0.4cm]35:$e_i^j$}] (eij) at (0.5, 0.5) {};
\node[draw=none, label={[label distance=0.4cm]35:$a_i^j$}] (aij) at (1.0, 1.0) {};
\begin{scope}[shift={(0.0, 1.0)}]
\begin{scope}[rotate=35]
\node[draw=none] (vijminus1) at (0.0, 1.0) {};
\node[draw=none] (hijminus1) at (0.5, 0.5) {};
\node[draw=none] (dijminus1) at (1.0, 0.0) {};
\node[draw=none] (cijminus1) at (1.0, 1.0) {};
\end{scope}
\end{scope}
\end{scope}

\begin{scope}[shift={(0.0, -1.0)}]
\begin{scope}[rotate=-35]
\node[draw=none, label={[label distance=0.4cm]-35:$a_i^{j+1}$}] (aij1) at (1.0, 0.0) {};
\node (vij1) at (0.0, -1.0) {};
\node[label={[label distance=0.4cm]-35:$e_i^{j+1}$}] (eij1) at (0.5, -0.5) {};
\node[label={[label distance=0.4cm]-35:$b_i^{j+1}$}] (bij1) at (1.0, -1.0) {};

\begin{scope}[shift={(0.0, -1.0)}]
\begin{scope}[rotate=-35]
\node[draw=none] (cij1) at (1.0, 0.0) {};
% \node (uij2) at (0.0, -1.0) {};
\node[draw=none] (hij1) at (0.5, -0.5) {};
% \node (bij12) at (1.0, -1.0) {};
\end{scope}
\end{scope}
\end{scope}
\end{scope}


\draw (aij) -- (bij) -- (vij) -- (uij) -- (aij);
\draw (vij) -- (cij) -- (dij) -- (hij);
\draw (aij) -- (eij) -- (bij);
\draw (uij) -- (eij) -- (vij);

\draw (vij) -- (hij) -- (cij);
\draw (eij) -- (hij);
\draw (bij) -- (cij);

\draw (bij) -- (hij);
\draw (eij) -- (cij);

\draw plot [smooth] coordinates {(aij) ($ (eij) !.4! (uij) $) (vij)};
\draw plot [smooth] coordinates {(bij) ($ (eij) !.4! (vij) $) (uij)};

\draw (aij) -- (hijminus1);

\draw (vij) -- (uij1);
\draw (dij) -- (aij1);
\draw (hij) -- (uij1);
\draw (dij) -- (uij1);
\draw (vij1) -- (uij1) -- (eij1) -- (vij1) -- (bij1) -- (eij1) -- (aij1) -- (bij1);

\draw plot [smooth] coordinates {(aij1) ($ (eij1) !.4! (uij1) $) (vij1)};
\draw plot [smooth] coordinates {(bij1) ($ (eij1) !.4! (vij1) $) (uij1)};

\draw (eij) -- (hijminus1);
\draw (eij) -- (dijminus1);

\draw (aij) -- (dijminus1);
\draw (uij) -- (vijminus1);
\draw (uij) -- (vijminus1);
\draw (uij) -- (dijminus1);
\draw (uij) -- (hijminus1);

\draw (hij) -- (eij1);
\draw (dij) -- (eij1);
\draw (hij) -- (aij1);
\draw (uij1) -- (aij1);

\draw (eij1) -- (hij1);
\draw (eij1) -- (cij1);

\draw (bij1) -- (hij1);
\draw (bij1) -- (cij1);

\draw [smooth] plot coordinates {(cij) ($ (hij) !.4! (vij) $) (uij1)};
\draw [smooth] plot coordinates {(dij) ($ (hij) !.4! (uij1) $) (vij)};
\draw [smooth] plot coordinates {(uij) ($ (hijminus1) !.4! (vijminus1) $) (cijminus1)};

\node (aijfake) at (aij) {};
\node (bijfake) at (bij) {};
\node (cijfake) at (cij) {};
\node (dijfake) at (dij) {};
\node (uijfake) at (uij) {};
\node (vijfake) at (vij) {};
\node (uij1fake) at (uij1) {};
\node (aij1fake) at (aij1) {};
\node (bij1fake) at (bij1) {};
\node (vij1fake) at (vij1) {};
\node (dijminus1fake) at (dijminus1) {};

\begin{scope}[rotate=17.5]
\begin{scope}[shift={(1.9, 0.0)}]
\node[label={[label distance=0.4cm]-72.5:$s_r^1$}] (sr1) at (0.0, -0.3) {};
\node[label={[label distance=0.4cm]107.5:$t_r^1$}] (tr1) at (0.0, 0.3) {};

\node[label={[label distance=0.4cm]107.5:$w_r^1$}] (wr1) at (0.4, 0.7) {};

\node[label={[label distance=0.4cm]-72.5:$w_r^2$}] (wr2) at (0.8, -0.3) {};
\node[label={[label distance=0.4cm]107.5:$w_r^3$}] (wr3) at (0.8, 0.3) {};

\node[label={[label distance=0.4cm]107.5:$w_r^4$}] (wr4) at (1.2, 0.7) {};

\node[label={[label distance=0.4cm]-72.5:$s_r^2$}] (sr2) at (1.6, -0.3) {};
\node[label={[label distance=0.4cm]107.5:$t_r^2$}] (tr2) at (1.6, 0.3) {};

\Kfivedraw{sr1}{tr1}{wr1}{wr2}{wr3}
\draw (wr3) -- (wr4) -- (sr2) -- (tr2) -- (wr2) -- (sr2) -- (wr3) -- (tr2) -- (wr4) -- (wr2);


% \shade[top color=red, path fading=we] (aij.center) -- ($(aij.center) + (-0.15, 0.7)$) -- ($(uij.center) + (-0.6, 0.3)$) -- (uij.center) -- cycle;

\end{scope}
\end{scope}
\end{scope}

\end{scope}


% do the shading nonsense
\begin{scope}
\path [laser beam=white] ($(aij.center) + (-0.2, 1.0)$) -- ($(uij.center) + (-2.0, -0.3)$);
\fill [white] ($(aij.center) + (-0.2, 1.0)$) -- ($(uij.center) + (-2.0, -0.3)$) -- ($(uij.center) + (-2.5, -0.0)$) -- ($(aij.center) + (-4.0, 2.0)$);

\path [laser beam=white] ($(uij1.center) + (-1.0, -0.4)$) -- ($(bij1.center) + (-0.9, -0.9)$);
\fill [white] ($(uij1.center) + (-1.0, -0.4)$) -- ($(uij1.center) + (-2.0, -0.5)$) -- ($(bij1.center) + (-3.0, -0.9)$) -- ($(bij1.center) + (-1.0, -1.0)$);
\end{scope}

\end{tikzpicture}
    \caption{The reduction from \maxtwosatthree/ to \edkfive}
    \label{fig:krpacking_edkfive}
\end{figure}
As before, we reduce from \maxtwosatthree/ (Problem~\ref{pr:maxtwosatthree}) and construct a set of variable and clause gadgets. For each variable $x_i$, construct a variable gadget of $8 m_i$ vertices, labelled $R_i = \{ a_i^j, b_i^j, c_i^j, d_i^j, e_i^j, h_i^j, u_i^j, v_i^j \}$ for each $j$ where $1\leq j \leq m_i$. For each $j$ where $1\leq j \leq m_i$, add an edge (if it does not exist already) between each pair of vertices in $\{ a_i^j, b_i^j, e_i^j, u_i^j, v_i^j \}$; $\{ b_i^j, c_i^j, e_i^j, h_i^j, v_i^j \}$; and finally $\{ c_i^j, d_i^j, h_i^j, v_i^j, u_i^{j+1} \}$ and $\{ d_i^j, a_i^{j+1}, h_i^j, e_i^{j+1}, u_i^{j+1} \}$ if $j < m_i$, otherwise $\{ c_i^j, d_i^j, h_i^j, v_i^j, u_i^1 \}$ and $\{ d_i^j, a_i^1, h_i^j, e_i^1, u_i^1 \}$. We shall refer to $\{ a_i^j, b_i^j, e_i^j, u_i^j, v_i^j \}$ and $\{ c_i^j, d_i^j, h_i^j, v_i^j, u_i^{j+1} \}$ (and $\{ c_i^j, d_i^j, h_i^j, v_i^j, u_i^1 \}$) as \emph{odd $K_5$s in $R_i$}, and $\{ b_i^j, c_i^j, e_i^j, h_i^j, v_i^j \}$ and $\{ d_i^j, a_i^{j+1}, h_i^j, e_i^{j+1}, u_i^{j+1} \}$ (and $\{ d_i^j, a_i^1, h_i^j, e_i^1, u_i^1 \}$) as \emph{even $K_5$s in $R_i$}. At this point $\deg_{G}(a_i^j) = \deg_{G}(b_i^j) = \deg_{G}(c_i^j) = \deg_{G}(d_i^j) = 6$ and $\deg_{G}(e_i^j) = \deg_{G}(h_i^j) = \deg_{G}(u_i^j) = \deg_{G}(v_i^j) = 8$ for any $j$ where $1\leq j \leq m_i$. 

% We begin by constructing the variable gadgets. For each variable $x_i$, construct $10m_i$ vertices labelled $a_i^1, b_i^1, c_i^1, d_i^1, e_i^1, h_i^1, u_i^1, v_i^1, a_i^2, b_i^2, \dots, u_i^{m_i}, v_i^{m_i}$. We shall refer to these vertices as the \emph{$i\textsuperscript{th}$ variable gadget}. For each $j$ where $1\leq j \leq m_i$, add an edge (if it does not exist already) between each pair of vertices in $\{ a_i^j, b_i^j, e_i^j, u_i^j, v_i^j \}$; $\{ b_i^j, c_i^j, e_i^j, h_i^j, v_i^j \}$; and finally $\{ c_i^j, d_i^j, h_i^j, v_i^j, u_i^{j+1} \}$ and $\{ d_i^j, a_i^{j+1}, h_i^j, e_i^{j+1}, u_i^{j+1} \}$ if $j < m_i$, otherwise $\{ c_i^j, d_i^j, h_i^j, v_i^j, u_i^1 \}$ and $\{ d_i^j, a_i^1, h_i^j, e_i^1, u_i^1 \}$. We shall refer to $\{ a_i^j, b_i^j, e_i^j, u_i^j, v_i^j \}$ and $\{ c_i^j, d_i^j, h_i^j, v_i^j, u_i^{j+1} \}$ (and $\{ c_i^j, d_i^j, h_i^j, v_i^j, u_i^1 \}$) as \emph{even} $K_5$s, and $\{ b_i^j, c_i^j, e_i^j, h_i^j, v_i^j \}$ and $\{ d_i^j, a_i^{j+1}, h_i^j, e_i^{j+1}, u_i^{j+1} \}$ (and $\{ d_i^j, a_i^1, h_i^j, e_i^1, u_i^1 \}$) as \emph{odd} $K_5$s. Note that at this point of construction, $\deg_{G}(a_i^j) = \deg_{G}(b_i^j) = \deg_{G}(c_i^j) = \deg_{G}(d_i^j) = 6$ and $\deg_{G}(e_i^j) = \deg_{G}(h_i^j) = \deg_{G}(u_i^j) = \deg_{G}(v_i^j) = 8$ for any $j$ where $1\leq j \leq m_i$. 

We shall now construct the clause gadgets. For each clause $c_r$, construct a clause gadget of $7$ vertices labelled $S_r = \{ s_r^1, t_r^1, s_r^2, t_r^2, w_r^1, w_r^2, w_r^3, w_r^4 \}$. Add an edge (if it does not exist already) between each pair of vertices in $\{ s_r^1, t_r^1, w_r^1, w_r^2, w_r^3 \}$ and $\{ s_r^2, t_r^2, w_r^2, w_r^3, w_r^4 \}$. Label $\{ s_r^1, t_r^1, w_r^1, w_r^2, w_r^3 \}$ and $\{ s_r^2, t_r^2, w_r^2, w_r^3, w_r^4 \}$ as $P_i^r$ and $P_j^r$, where the variables of the literals in $c_r$ are $x_i$ and $x_j$.

% We shall now construct the clause gadgets. For each clause $c_r$, construct $7$ vertices labelled $s_r^1, t_r^1, s_r^2, t_r^2, w_r^1, w_r^2, w_r^3, w_r^4$. We shall refer to these vertices as the \emph{$r\textsuperscript{th}$ clause gadget}. Add an edge (if it does not exist already) between each pair of vertices in $\{ s_r^1, t_r^1, w_r^1, w_r^2, w_r^3 \}$ and $\{ s_r^2, t_r^2, w_r^2, w_r^3, w_r^4 \}$. We shall refer to $\{ s_r^1, t_r^1, w_r^1, w_r^2, w_r^3 \}$ and $\{ s_r^2, t_r^2, w_r^2, w_r^3, w_r^4 \}$ as $P_i^r$ and $P_j^r$ where the variables of the literals in $c_r$ are $x_i$ and $x_j$.

The connection of variable and clause gadgets follows the same pattern as for \edkfour. For each clause $c_r$, suppose $x_i$ is the variable of some literal in $c_r$ where $c_r$ contains the $j\textsuperscript{th}$ occurrence of $x_i$ in $\phi$. If $x_i$ is the first literal in $c_r$ and occurs positively in $c_r$ then identify $a_i^j$ and $s_r^1$, and $b_i^j$ and $t_r^1$. Now $\deg_{G}(a_i^j) = \deg_{G}(b_i^j) = 9$. Similarly, if $x_i$ is the first literal in $c_r$ and occurs negatively in $c_r$ then identify $b_i^j$ and $s_r^1$, and $c_i^j$ and $t_r^1$. If $x_i$ is the second literal in $c_r$ and occurs positively in $c_r$ then identify $a_i^j$ and $s_r^2$, and $b_i^j$ and $t_r^2$. If $x_i$ is the second literal in $c_r$ and occurs negatively in $c_r$ then identify $b_i^j$ and $s_r^2$, and $c_i^j$ and $t_r^2$. Now $\Delta = 9$.

% We shall now connect the variable and clause gadgets. For each clause $c_r$, suppose $x_i$ is the variable of the some literal in $c_r$ where $c_r$ contains the $j\textsuperscript{th}$ occurrence of $x_i$ in $\phi$. If $x_i$ is the first literal in $c_r$ and occurs positively in $c_r$ then identify $a_i^j$ and $s_r^1$, and $b_i^j$ and $t_r^1$. We shall hereafter refer to the first identified vertex as either $a_i^j$ or $s_r^1$ and the second identified vertex as either $b_i^j$ or $t_r^1$. Note that now $\deg_{G}(a_i^j) = \deg_{G}(b_i^j) = 9$. Similarly, if $x_i$ is the first literal in $c_r$ and occurs negatively in $c_r$ then identify $b_i^j$ and $s_r^1$, and $c_i^j$ and $t_r^1$. If $x_i$ is the second literal in $c_r$ and occurs positively in $c_r$ then identify $a_i^j$ and $s_r^2$, and $b_i^j$ and $t_r^2$. If $x_i$ is the second literal in $c_r$ and occurs negatively in $c_r$ then identify $b_i^j$ and $s_r^2$, and $c_i^j$ and $t_r^2$. This completes the construction of $G$. Observe that $\Delta = 9$.

As before, the reduction can be performed in polynomial time.  We now prove correctness in the first direction.

\begin{lem}
\label{lem:krpacking_kfivepacking_firstdirection}
If a truth assignment $\mathfrak{f}$ for $\phi$ satisfies at least $k$ clauses then an edge-disjoint $K_5$-packing $T$ exists in $G$ where $|T| \geq \sum_{i=1}^{|X|} 2 m_i + k$.
\end{lem}
\begin{proof}
Suppose $\mathfrak{f}$ is a truth assignment for $\phi$ that satisfies at least $k$ clauses. We shall construct an edge-disjoint $K_5$-packing $T$ where $|T| \geq \sum_{i=1}^{|X|} 2 m_i + k$.
For each variable $x_i$, add to $T$ the set of even $K_5$s in $R_i$ if $\mathfrak{f}(x_i)$ is true and otherwise the set of odd $K_5$s in $R_i$. Now $|T|=\sum_{i=1}^{|X|} 3 m_i$.
For each clause $c_r$ satisfied by $\mathfrak{f}$, it must be that there exists some variable $x_i$ where $\mathfrak{f}(x_i)$ is true and $x_i$ occurs positively in $c_r$, or there exists some variable $x_i$ where $\mathfrak{f}(x_i)$ is false and $x_i$ occurs negatively in $c_r$. As before, in either case add $P_i^r$ to $T$. Now $|T| = \sum_{i=1}^{|X|} 2m_i + k$. The proof that $T$ is edge disjoint is analogous to the proof in Lemma~\ref{lem:krpacking_kfourpacking_firstdirection}.
\end{proof}

% We now prove that the reduction is correct in the second direction. In the remainder of this section, assume that $T$ is an edge-disjoint $K_r$-packing where $|T|=\sum_{i=1}^{|X|} 2m_i + k$ for some integer $k\geq 1$. We shall eventually construct a truth assignment $\mathfrak{f}$ that satisfies at least $|T|=\sum_{i=1}^{|X|} 2m_i + k$ clauses.

We now prove the second direction. Like before, we say that some edge-disjoint $K_5$-packing $T$ in $G$ is \emph{canonical} if for any $R_i$, $T$ contains either the set of even $K_5$s in $R_i$ or the set of odd $K_5$s in $R_i$.

\begin{lem}
If $T$ is an edge-disjoint $K_5$-packing then there exists a canonical edge-disjoint $K_5$-packing $T'$ where $|T'| \geq |T|$.
\label{lem:krpacking_five_canonical}
\end{lem}
\begin{proof}
The proof is analogous to the proof of Lemma~\ref{lem:krpacking_four_canonical}. Here we describe the modification of a single variable gadget $R_i$ where $T$ neither contains all even $K_5$s nor all odd $K_5$s in $R_i$. It must be that the number of $K_5$s in $R_i$ in $T$ is at most $2 m_i - 1$.

% By the definition of canonical, there exists at least one variable gadget $i$ such that $T$ neither contains all even $K_5$s in the $i\textsuperscript{th}$ variable gadget nor all odd $K_5$s in the $i\textsuperscript{th}$ variable gadget. For any such $i$, we show how to modify $T$ to construct $T'$ such that $|T'|\geq |T|$ and $T'$ contains either all even $K_5$s in $i\textsuperscript{th}$ variable gadget or all odd $K_5$s in the $i\textsuperscript{th}$ variable gadget. It follows from this that there exists some canonical edge-disjoint $K_r$-packing $T'$ where $|T'| \geq |T|$. 

% In the remainder of this proof we shall consider only one variable gadget, namely the $i\textsuperscript{th}$. It must be that at most $2m_i - 1$ $K_5$s that are subsets of this variable gadget belong to $T$.

Suppose $x_i$ occurs in clauses $c_{r_1}, c_{r_2}, \dots, c_{r_{m_i}}$, corresponding to the sets $P_i^{r_1}, P_i^{r_2}, \dots, P_i^{r_{m_i}}$. It must be that either at most one $K_5$ in $\{ P_i^{r_1}, P_i^{r_2}, \dots, P_i^{r_{m_i}} \}$ exists in $T$ where the corresponding occurrence of $x_i$ is positive, or at most one $K_5$ in $\{ P_i^{r_1}, P_i^{r_2}, \dots, P_i^{r_{m_i}} \}$ exists in $T$ where the corresponding occurrence of $x_i$ is negative. In the former case, remove the $K_5$ in $\{ P_i^{r_1}, P_i^{r_2}, \dots, P_i^{r_{m_i}} \}$ where the corresponding occurrence of $x_i$ is positive as well as any even $K_5$s in $R_i$ in $T$, then add the set of odd $K_5$s not already in $T$. The number of $K_5$s in $R_i$ is now $2 m_i$ so since at most one $K_5$ was removed, which was not in $R_i$, it follows that the cardinality of $T$ has not decreased. To see that $T$ is still edge-disjoint, observe that any $K_5$ in $\{ P_i^{r_1}, P_i^{r_2}, \dots, P_i^{r_{m_i}} \}$ in $T$ intersects any odd $K_5$ by at most one vertex. The construction and proof in the latter case is symmetric.
\end{proof}

\begin{lem}
\label{lem:krpacking_kfivepacking_seconddirection}
If $T$ is an edge-disjoint $K_5$-packing where $|T| = \sum_{i=1}^{|X|} 2 m_i + k$ for some integer $k\geq 1$ then exists a truth assignment $\mathfrak{f}$ that satisfies at least $k$ clauses.
\end{lem}
\begin{proof}
Assume by Lemma~\ref{lem:krpacking_five_canonical} that $T$ is canonical. It follows that $T$ contains exactly $\sum_{i=1}^{|X|} 2 m_i$ $K_5$s in variable gadgets and at least $k$ $K_5$s in clause gadgets. For each variable $x_i$, set $\mathfrak{f}(x_i)$ to be true if $T$ contains all even $K_5$s in $R_i$ and false otherwise. Now consider each clause gadget $c_r$ where $S_r$ contains some $K_5$ in $T$, which we label $P_i^r$. Suppose $x_i$ occurs positively in $c_r$. It follows that $P_i^r$ contains $a_i^j, b_i^j$ for some $j$ where $1\leq j\leq 3$. Since $T$ is edge disjoint it follows that $T$ contains the even $K_5$s in $R_i$. By the construction of $\mathfrak{f}$ it follows that $\mathfrak{f}(x_i)$ is true and thus $c_r$ is satisfied. The proof when $x_i$ occurs negatively in $c_r$ is symmetric. It follows thus that at least $k$ clauses are satisfied by $\mathfrak{f}$.
\end{proof}

\begin{lem}
\label{lem:krpacking_edkr_req5_apxhard}
If $r=5$ and $\Delta=9$ then \edkr is $\APX$-hard.
\end{lem}
\begin{proof}
The reduction described runs in polynomial time, and Lemma~\ref{lem:krpacking_kfivepacking_seconddirection} shows how to construct a truth assignment $\mathfrak{f}$ that satisfies $k$ clauses given an edge-disjoint $K_5$-packing of cardinality $\sum_{i=1}^{|X|} 3 m_i + k$ where $k \geq 1$. By Lemmas~\ref{lem:krpacking_kfivepacking_firstdirection} and~\ref{lem:krpacking_kfivepacking_seconddirection}, in the reduction a truth assignment $\mathfrak{f}$ for $\phi$ exists that satisfies at least $k$ clauses if and only if there exists an edge-disjoint $K_5$-packing of size at least $\sum_{i=1}^{|X|} 3 m_i + k$. This reduction is thus an $L$-reduction with $\alpha=9$ and $\beta=1$.
\end{proof}

We now combine Lemmas~\ref{lem:krpacking_edkr_req4_apxhard} and \ref{lem:krpacking_edkr_req4_apxhard} with the existing result of Caprara and Rizzi \cite{caprara_packing_2002} in Theorem~\ref{thm:krpacking_edkr345apxhard}.

\begin{thm}
\label{thm:krpacking_edkr345apxhard}
If $r \leq 5$ and $\Delta > 2r - 2$ then \edkr is $\APX$-hard.
\end{thm}
\begin{proof}
Caprara and Rizzi \cite{caprara_packing_2002} prove the case when $r = 3$ and $\Delta = 5$. In Lemma~\ref{lem:krpacking_edkr_req4_apxhard} we prove the case when $r=4$ and $\Delta=7$. In Lemma~\ref{lem:krpacking_edkr_req5_apxhard} we prove the case when $r=5$ and $\Delta=9$.
\end{proof}

% \subsubsection{Conclusion} 

% We now combine Lemmas~\ref{lem:krpacking_edkr_req4_apxhard}, \ref{lem:krpacking_edkr_req5_apxhard}, and \ref{lem:krpacking_edkr_rgeq6_apxhard} in Theorem~\ref{thm:krpacking_edkr_conclusion_apxhard}.

% \begin{thm}
% \label{thm:krpacking_edkr_conclusion_apxhard}
% For any simple undirected graph $G$, If either $r \leq 5$ and $\Delta > 2r - 2$, or $r > 5$ and $\Delta \geq \lceil 5r/3 \rceil - 1$, then \edkr is $\APX$-hard.
% \end{thm}
% \begin{proof}
% Lemma~\ref{lem:krpacking_edkr_rgeq6_apxhard} shows the case when $r\geq 6$ and $\Delta \geq \lceil 5r/3 \rceil - 1$.
% \end{proof}



%Construct $r|V| - r/3|E|$ vertices in $V'$ labelled $w_1, w_2, \dots, w_{|V'|}$. 

\section{Summary and future work}
\label{sec:krpacking_conclusion}
To recap, we considered the problem of finding a maximum-cardinality $K_r$-packing in an undirected graph of fixed maximum degree $\Delta$, subject to the set of selected $K_r$s being either vertex disjoint (\vdkr) or edge disjoint (\edkr). It is known that \vdkthree is solvable in linear time if $\Delta=3$ but $\APX$-hard if $\Delta \geq 4$, and \edkthree is solvable in linear time if $\Delta=4$ but $\APX$-hard if $\Delta \geq 5$ \cite{caprara_packing_2002}. We generalised these results and presented a full complexity classification for both \vdkr and \edkr. 
We first showed that \vdkr is solvable in linear time if $\Delta < 3r/2 - 1$ (Theorem~\ref{thm:krpacking_vdkr_3r2minus1}), solvable in polynomial time if $\Delta < 5r/3 - 1$ (Theorem~\ref{thm:krpacking_vdkr_polytime_5r3minus1}), and $\APX$-hard if $\Delta \geq \lceil 5r/3 \rceil - 1$ (Theorem~\ref{thm:krpacking_vdkr_apxhard}). 

We also showed that if $r\geq 6$ then \edkr is also solvable in linear time if $\Delta < 3r/2 - 1$, solvable in polynomial time if  $\Delta < 5r/3 - 1$, and $\APX$-hard if $\Delta \geq \lceil 5r/3 \rceil - 1$. If $r \leq 5$, then \edkr is solvable in linear time if $\Delta < 3r/2 - 1$, solvable in polynomial time if $\Delta \leq 2r - 2$, and $\APX$-hard if $\Delta > 2r - 2$.

Some of our polynomial-time algorithms involved finding a maximum independent set in a corresponding intersection graph. In each case, we showed that this intersection graph was claw-free, from which it follows that a maximum independent set in the intersection graph can be found in polynomial time \cite{MINTY1980284,SBIHI198053}. As we noted in 
Section~\ref{sec:krpacking_vdkr_polytimesolvability}, in a more general setting in which graph vertices have weights, it is possible to find an independent set of maximum weight \cite{MINTY1980284,Nakamura01}. It might be interesting to use this result to derive polynomial-time algorithms for weighted versions of \vdkr and \edkr.

Another direction for future work is to generalise other known results for \vdkthree and \edkthree to \vdkr and \edkr where $r \geq 3$. For example, Mani\'c and Wakabayashi \cite{Eurocomb05} showed that the known approximation ratio of $(3/2 + \varepsilon)$ for \vdkthree and \edkthree with can be improved upon in the restricted settings where $\Delta = 4$ and $\Delta = 5$, respectively. It might be possible to show a similar improvement of the corresponding approximation ratio for \vdkr and \edkr (of $r/2 + \varepsilon$, as discussed in Section~\ref{sec:krpacking_intro}) in the setting of an arbitrary fixed maximum degree.

% The second direction is to consider other problems involving clique packing. For example, Chataigner et al.~\cite{CHATAIGNER20091396} study the approximability of the $\mathcal{K}_r$-packing problem, in which the goal is to find a vertex-disjoint set of cliques, each with size \emph{at most} $r$, that covers the maximum number of edges in the input graph.
\chapter{Conclusion}
\label{c:conclusion}

In this chapter we recap on the contribution of this thesis and discuss some future work relating to 3DR as well as more general problems of coalition formation.

%  Some of our results can be generalised to other problems involving fixed-size coalitions but, in any case, they give us a suggestive insight into the algorithmics of fixed-size coalitions. 

Our main contribution related to \emph{Three-Dimensional Roommates} (3DR). For two models involving $\mathscr{B}$- and $\mathscr{W}$-preferences (3DR-B and 3DR-W), we considered the existence of matchings that are stable. We first showed that both associated existence problems are $\NP$-complete. Next, in each model we considered the optimisation problem in which the objective is to construct a matching with the maximum number of non-blocking triples. We showed that an existing result led to a $9/4$-approximation algorithm in both models and a simple algorithm based on serial dictatorship led to a $3/2$-approximation in 3DR-B.

In a model of 3DR with additively separable preferences (3DR-AS), we studied stable and envy-free matchings, for three successively weaker definitions of envy-freeness. We considered various restrictions on the agents' valuations and gave a comprehensive complexity classification based on these restrictions. Interestingly, we identified a general trend that shows, for successively weaker solution concepts, either existence or polynomial-time solvability holds under successively weaker preference restrictions. Building on our new result that any instance of 3DR-AS with binary and symmetric preferences must contain a stable matching, we also developed a $2$-approximation algorithm for the problem of finding a stable matching with maximum utilitarian welfare in such an instance.

We also presented new results relating to Three-Dimensional Stable Matching with Cyclic Preferences (3-DSM-CYC). In particular, we considered the optimisation problem of finding a matching with the maximum number of non-blocking families. We first presented two different approximation algorithms for this problem in the general case. We then considered a situation in which the preferences of some agents are sufficiently similar to some master list, and showed that the approximation ratio of one algorithm can be improved in relation to a particular similarity measure (specifically the Kendall tau distance \cite{KendallTauCitation}).

Finally, we considered a general problem in graph theory that generalises the notion of assigning agents to coalitions of a fixed size, known as the \emph{$K_r$-packing problem}. In particular, we studied the restricted case of this problem in which the graph has a fixed maximum degree $\Delta$. It is known for $r=3$ that the vertex-disjoint (edge-disjoint) variant is solvable in linear time if $\Delta=3$ ($\Delta=4$) but $\APX$-hard if $\Delta \geq 4$ ($\Delta \geq 5$). We generalised these results to an arbitrary but fixed $r \geq 3$, and provided a full complexity classification for both the vertex- and edge-disjoint variants in graphs of maximum degree $\Delta$, for all $r \geq 3$.

At the end of each chapter of this thesis we summarised our new results in detail and discussed some closely related angles of possible future work. We shall now discuss some more general directions for future work related to both 3DR and the wider topic of coalition formation. 

An immediate open question is to what extent our results relating to 3DR generalise to problems of multidimensional roommates and more general models of coalition formation. We conjecture that our $\NP$-completeness reductions relating to 3DR-B, 3DR-W, and 3DR-AS can all be generalised to a model of $k$-dimensional roommates ($k$DR) where $k \geq 3$. We also conjecture that our approximation algorithms for 3-DSM-CYC and 3DR-B can also be generalised, without too much extra work, to $k$-DSM-CYC and $k$DR-B, where $k \geq 3$, with the same approximation ratios. In our opinion the most interesting question here concerns our polynomial-time algorithm for the restriction of 3DR-AS in which preferences are binary and symmetric. It is unclear if either a similar algorithm exists for the same restriction in 4DR-AS or if instances of 4DR-AS exist that do not contain a stable matching.

In Chapter~\ref{c:three_dsm_cyc}, we devised a $9/4$-approximation algorithm for 3-DSM-CYC-MSM, which involved first constructing a corresponding instance of 3GSM and then using Rosenbaum's~\cite{rosenbaum16} $9/4$-approximation algorithm for 3GSM-MSM to find a matching with at least $9n^3/4$ non-blocking families. We also proved similar results in Chapters \ref{c:threed_sr_b} and~\ref{c:threed_sr_w} for 3DR-B-MSM and 3DR-W-MSM respectively, making use of Rosenbaum's algorithm for 3PSA-MSM. We believe that this approach can be easily generalised to other optimisation problems in variants of either 3GSM or 3PSA. Specifically, we believe that this approach can be generalised for any such variant in which each agent's preference over triples can be expressed as a poset. It follows that a linear extension of each agent's preferences exists, which was the central component of our proofs for 3-DSM-CYC-MSM, 3DR-B-MSM, and 3DR-W-MSM. For example, we believe that it will be straightforward to identify, along these lines, a $9/4$-approximation algorithm for the corresponding problem in the model of 3DR proposed by Iwama et al.~\cite{IMO07} in 2007.

We saw in Chapters~\ref{c:threed_sr_b} and~\ref{c:threed_sr_w} that deciding if a given instance of 3DR-B or 3DR-W contains a stable matching is $\NP$-complete, which contrasts with the analogous models in which coalitions need not have a fixed size, wherein a stable matching is bound to exist and can be found in polynomial time \cite{CR01,CH04}. It seems intuitive that the added restriction of fixed coalition size makes both problems somehow harder to solve. It would be interesting to identify other problem models that exhibit a similar behaviour, or identify models that counter this intuition. 
In this direction, one could also explore other restrictions of coalition size, such as flatmate games \cite{Brandt2020FindingAR} or lower and upper bounds. 

Many existing works relating to fixed-size coalition formation, and in particular those involving multidimensional roommates, propose new models and study related problems in a relatively ad-hoc way. In this thesis we used the common framework of 3DR to formalise three related models of fixed-size coalition formation, and in each one studied the existence of, and complexity of finding, feasible matchings. This approach allowed us to compare analogous results between problems that relate to different systems of preference representation and different solution concepts. For example, we noted in Chapter~\ref{c:threed_sr_w} that it seems difficult to construct an approximation algorithm for 3DR-W-MSM with the same performance guarantee as the algorithm for 3DR-B-MSM. We believe that such a systematic approach helps us explore the interplay between the system of preference representation, solution concept, and coalition size, both in the setting of 3DR as well as in more general models of coalition formation.

As we saw in Chapter~\ref{c:lit_review} (and in Figure~\ref{fig:lit_review_hgsolutionconcepts}) a multitude of solution concepts and systems of preference representation have been explored in the setting of hedonic games (in which coalitions generally need not have a fixed size) \cite{HedonicGamesHOCSC}. It remains open to what extent many of these systems and concepts can be transposed either to 3DR or other models involving coalitions of a restricted size. For example, as noted by Bil\`o et al.~\cite{Bilo22}, it seems unclear whether Nash stability, which involves the individual deviation of agents, can be meaningfully defined in some models of fixed-size coalitions. More generally, it would be interesting to see to what extent the hierarchy of solution concepts defined in the setting of hedonic games~\cite{HedonicGamesHOCSC,AZIZ2013316,BY19} (which we discussed in Chapter~\ref{c:lit_review}) can be redefined in a model involving fixed-size coalitions. 


\chapter*{\glossarychaptertitle/}
\addcontentsline{toc}{chapter}{\glossarychaptertitle/}
\markboth{GLOSSARY OF ABBREVIATIONS}{GLOSSARY OF ABBREVIATIONS}

\newcommand{\mysymbolformatting}[1]{%
    \textbf{#1}
}
\newcommand{\myglossarydefinitionspace}{%
    \hspace*{1mm}
}

\begin{center}
\begingroup
\def\arraystretch{1.15}
\begin{longtable}{>{\raggedright\arraybackslash}p{\textwidth-1mm}}
\mysymbolformatting{3DR} \myglossarydefinitionspace Three-Dimensional Roommates \myglossarydotfill \mysymbolpageref{symboldef:threedr}\\
\mysymbolformatting{3DR-AS} \myglossarydefinitionspace Three-Dimensional Roommates with Additively Separable preferences \myglossarydotfill \mysymbolpageref{symboldef:threedr_as}\\
\mysymbolformatting{3DR-AS-SMUW} \myglossarydefinitionspace 3DR-AS Stable Maximum Utilitarian Welfare problem \myglossarydotfill \mysymbolpageref{symboldef:threedr_as_smuw}\\
\mysymbolformatting{3DR-B} \myglossarydefinitionspace Three-Dimensional Roommates with $\mathscr{B}$-preferences \myglossarydotfill \mysymbolpageref{symboldef:threedr_b}\\
\mysymbolformatting{3DR-B-MSM} \myglossarydefinitionspace 3DR-B Maximally Stable Matching problem \myglossarydotfill \mysymbolpageref{symboldef:threedr_b_msm}\\
\mysymbolformatting{3DR-W} \myglossarydefinitionspace Three-Dimensional Roommates with $\mathscr{W}$-preferences \myglossarydotfill \mysymbolpageref{symboldef:threedr_w}\\
\mysymbolformatting{3DR-W-MSM} \myglossarydefinitionspace 3DR-W Maximally Stable Matching problem \myglossarydotfill \mysymbolpageref{symboldef:threedr_w_msm}\\
\mysymbolformatting{3-DSM-CYC} \myglossarydefinitionspace Three-Dimensional Stable Matching with Cyclic preferences \myglossarydotfill \mysymbolpageref{symboldef:threedsmcyc}\\
\mysymbolformatting{3-DSM-CYC-MSM} \myglossarydefinitionspace 3-DSM-CYC Maximally Stable Matching problem \myglossarydotfill \mysymbolpageref{symboldef:threedsmcyc_msm}\\
\mysymbolformatting{3GSM} \myglossarydefinitionspace Three-Gender Stable Marriage problem \myglossarydotfill \mysymbolpageref{symboldef:threegsm}\\
% 3GSM-MSM \myglossarydefinitionspace 3GSM Maximally Stable Matching problem \myglossarydotfill \mysymbolpageref{symboldef:threegsm_msm}\\
\mysymbolformatting{3PSA} \myglossarydefinitionspace Three-Person Stable Assignment problem \myglossarydotfill \mysymbolpageref{symboldef:threepsa}\\
% 3PSA-MSM \myglossarydefinitionspace 3PSA Maximally Stable Matching problem \myglossarydotfill \mysymbolpageref{symboldef:threepsa_msm}\\
% 3PSA-MSS \myglossarydefinitionspace 3PSA Maximal Stable Submatching problem \myglossarydotfill \mysymbolpageref{symboldef:threepsa_mss}\\
\mysymbolformatting{ASHG} \myglossarydefinitionspace Additively Separable Hedonic Game \myglossarydotfill \mysymbolpageref{symboldef:ashg}\\
\mysymbolformatting{DTP} \myglossarydefinitionspace Directed Triangle Packing \myglossarydotfill \mysymbolpageref{symboldef:ashg}\\
\mysymbolformatting{\edkr} \myglossarydefinitionspace Edge-Disjoint $K_r$-packing problem \myglossarydotfill \mysymbolpageref{symboldef:edkr}\\
% FHG \myglossarydefinitionspace Fractional Hedonic Game \myglossarydotfill \mysymbolpageref{symboldef:fhg}\\
% HR \myglossarydefinitionspace Hospitals-Residents problem \myglossarydotfill \mysymbolpageref{symboldef:hr}\\
\mysymbolformatting{IRLC} \myglossarydefinitionspace Individually Rational Lists of Coalitions \myglossarydotfill \mysymbolpageref{symboldef:irlc}\\
% \end{tabular}
% \begin{tabular}{ c c c }
% Symbol \myglossarydefinitionspace Meaning \myglossarydefinitionspace Defined on page \\ 
% \hline
\mysymbolformatting{LC} \myglossarydefinitionspace Lists of Coalitions \myglossarydotfill \mysymbolpageref{symboldef:lc}\\
\mysymbolformatting{\maxtwosatthree/} \myglossarydefinitionspace a restriction of Maximum Satisfiability \myglossarydotfill \mysymbolpageref{symboldef:maxtwosatthree}\\
% \mysymbolformatting{MIS} \myglossarydefinitionspace Maximum Independent Set \myglossarydotfill \mysymbolpageref{symboldef:mis}\\
\mysymbolformatting{MIS-3-TF} \myglossarydefinitionspace Maximum Independent Set in $3$-regular Triangle-Free graphs \myglossarydotfill \mysymbolpageref{symboldef:mistfvariant}\\
\mysymbolformatting{PIT} \myglossarydefinitionspace Partition into Triangles \myglossarydotfill \mysymbolpageref{symboldef:pit}\\
% PLS \myglossarydefinitionspace Polynomial Local Search \myglossarydotfill \mysymbolpageref{symboldef:pit}\\
% PON \myglossarydefinitionspace Precedence by Ordinal Number \myglossarydotfill \mysymbolpageref{symboldef:pon}\\
% SAT \myglossarydefinitionspace Satisfiability \myglossarydotfill \mysymbolpageref{symboldef:sat}\\
\mysymbolformatting{SDR} \myglossarydefinitionspace System of Distinct Representatives \myglossarydotfill \mysymbolpageref{symboldef:sdr}\\
\mysymbolformatting{SM} \myglossarydefinitionspace Stable Marriage problem \myglossarydotfill \mysymbolpageref{symboldef:sm}\\
\mysymbolformatting{SR} \myglossarydefinitionspace Stable Roommates problem \myglossarydotfill \mysymbolpageref{symboldef:sr}\\
\mysymbolformatting{\vdkr} \myglossarydefinitionspace Vertex-Disjoint $K_r$-packing problem \myglossarydotfill \mysymbolpageref{symboldef:vdkr}\\
\mysymbolformatting{\porschenxsatvariant/} \myglossarydefinitionspace a restriction of Exact 3-Satisfiability \myglossarydotfill \mysymbolpageref{symboldef:porschenxsatvariant}
% LC\\
% MIS\\
% MIS-3-TF\\
% PIT\\
% PIT-SDR\\
% PLS\\
% PON\\
% SAT\\
% SDR\\
% SM\\
% SR\\
% VDK\\
\end{longtable}
\endgroup
\end{center}

% ***
% IF THIS GOES ONTO A SECOND PAGE, FIND OUT HOW TO MAKE THE HEADER TITLE UPPERCASE ONLY ON THE FIRST PAGE!
% ***
% \appendix
% \chapter{An Appendix}
\label{a:appendix}

This is an appendix.


\newpage
\bibliographystyle{bibliography/vancouver_modified}
\bibliography{bibliography/final}
\addcontentsline{toc}{chapter}{Bibliography}
\end{document}
