\chapter{Conclusion}
\label{c:conclusion}

In this chapter we recap on the contribution of this thesis and discuss some future work relating to 3DR as well as more general problems of coalition formation.

%  Some of our results can be generalised to other problems involving fixed-size coalitions but, in any case, they give us a suggestive insight into the algorithmics of fixed-size coalitions. 

Our main contribution related to \emph{Three-Dimensional Roommates} (3DR). For two models involving $\mathscr{B}$- and $\mathscr{W}$-preferences (3DR-B and 3DR-W), we considered the existence of matchings that are stable. We first showed that both associated existence problems are $\NP$-complete. Next, in each model we considered the optimisation problem in which the objective is to construct a matching with the maximum number of non-blocking triples. We showed that an existing result led to a $9/4$-approximation algorithm in both models and a simple algorithm based on serial dictatorship led to a $3/2$-approximation in 3DR-B.

In a model of 3DR with additively separable preferences (3DR-AS), we studied stable and envy-free matchings, for three successively weaker definitions of envy-freeness. We considered various restrictions on the agents' valuations and gave a comprehensive complexity classification based on these restrictions. Interestingly, we identified a general trend that shows, for successively weaker solution concepts, either existence or polynomial-time solvability holds under successively weaker preference restrictions. Building on our new result that any instance of 3DR-AS with binary and symmetric preferences must contain a stable matching, we also developed a $2$-approximation algorithm for the problem of finding a stable matching with maximum utilitarian welfare in such an instance.

We also presented new results relating to Three-Dimensional Stable Matching with Cyclic Preferences (3-DSM-CYC). In particular, we considered the optimisation problem of finding a matching with the maximum number of non-blocking families. We first presented two different approximation algorithms for this problem in the general case. We then considered a situation in which the preferences of some agents are sufficiently similar to some master list, and showed that the approximation ratio of one algorithm can be improved in relation to a particular similarity measure (specifically the Kendall tau distance \cite{KendallTauCitation}).

Finally, we considered a general problem in graph theory that generalises the notion of assigning agents to coalitions of a fixed size, known as the \emph{$K_r$-packing problem}. In particular, we studied the restricted case of this problem in which the graph has a fixed maximum degree $\Delta$. It is known for $r=3$ that the vertex-disjoint (edge-disjoint) variant is solvable in linear time if $\Delta=3$ ($\Delta=4$) but $\APX$-hard if $\Delta \geq 4$ ($\Delta \geq 5$). We generalised these results to an arbitrary but fixed $r \geq 3$, and provided a full complexity classification for both the vertex- and edge-disjoint variants in graphs of maximum degree $\Delta$, for all $r \geq 3$.

At the end of each chapter of this thesis we summarised our new results in detail and discussed some closely related angles of possible future work. We shall now discuss some more general directions for future work related to both 3DR and the wider topic of coalition formation. 

An immediate open question is to what extent our results relating to 3DR generalise to problems of multidimensional roommates and more general models of coalition formation. We conjecture that our $\NP$-completeness reductions relating to 3DR-B, 3DR-W, and 3DR-AS can all be generalised to a model of $k$-dimensional roommates ($k$DR) where $k \geq 3$. We also conjecture that our approximation algorithms for 3-DSM-CYC and 3DR-B can also be generalised, without too much extra work, to $k$-DSM-CYC and $k$DR-B, where $k \geq 3$, with the same approximation ratios. In our opinion the most interesting question here concerns our polynomial-time algorithm for the restriction of 3DR-AS in which preferences are binary and symmetric. It is unclear if either a similar algorithm exists for the same restriction in 4DR-AS or if instances of 4DR-AS exist that do not contain a stable matching.

In Chapter~\ref{c:three_dsm_cyc}, we devised a $9/4$-approximation algorithm for 3-DSM-CYC-MSM, which involved first constructing a corresponding instance of 3GSM and then using Rosenbaum's~\cite{rosenbaum16} $9/4$-approximation algorithm for 3GSM-MSM to find a matching with at least $9n^3/4$ non-blocking families. We also proved similar results in Chapters \ref{c:threed_sr_b} and~\ref{c:threed_sr_w} for 3DR-B-MSM and 3DR-W-MSM respectively, making use of Rosenbaum's algorithm for 3PSA-MSM. We believe that this approach can be easily generalised to other optimisation problems in variants of either 3GSM or 3PSA. Specifically, we believe that this approach can be generalised for any such variant in which each agent's preference over triples can be expressed as a poset. It follows that a linear extension of each agent's preferences exists, which was the central component of our proofs for 3-DSM-CYC-MSM, 3DR-B-MSM, and 3DR-W-MSM. For example, we believe that it will be straightforward to identify, along these lines, a $9/4$-approximation algorithm for the corresponding problem in the model of 3DR proposed by Iwama et al.~\cite{IMO07} in 2007.

We saw in Chapters~\ref{c:threed_sr_b} and~\ref{c:threed_sr_w} that deciding if a given instance of 3DR-B or 3DR-W contains a stable matching is $\NP$-complete, which contrasts with the analogous models in which coalitions need not have a fixed size, wherein a stable matching is bound to exist and can be found in polynomial time \cite{CR01,CH04}. It seems intuitive that the added restriction of fixed coalition size makes both problems somehow harder to solve. It would be interesting to identify other problem models that exhibit a similar behaviour, or identify models that counter this intuition. 
In this direction, one could also explore other restrictions of coalition size, such as flatmate games \cite{Brandt2020FindingAR} or lower and upper bounds. 

Many existing works relating to fixed-size coalition formation, and in particular those involving multidimensional roommates, propose new models and study related problems in a relatively ad-hoc way. In this thesis we used the common framework of 3DR to formalise three related models of fixed-size coalition formation, and in each one studied the existence of, and complexity of finding, feasible matchings. This approach allowed us to compare analogous results between problems that relate to different systems of preference representation and different solution concepts. For example, we noted in Chapter~\ref{c:threed_sr_w} that it seems difficult to construct an approximation algorithm for 3DR-W-MSM with the same performance guarantee as the algorithm for 3DR-B-MSM. We believe that such a systematic approach helps us explore the interplay between the system of preference representation, solution concept, and coalition size, both in the setting of 3DR as well as in more general models of coalition formation.

As we saw in Chapter~\ref{c:lit_review} (and in Figure~\ref{fig:lit_review_hgsolutionconcepts}) a multitude of solution concepts and systems of preference representation have been explored in the setting of hedonic games (in which coalitions generally need not have a fixed size) \cite{HedonicGamesHOCSC}. It remains open to what extent many of these systems and concepts can be transposed either to 3DR or other models involving coalitions of a restricted size. For example, as noted by Bil\`o et al.~\cite{Bilo22}, it seems unclear whether Nash stability, which involves the individual deviation of agents, can be meaningfully defined in some models of fixed-size coalitions. More generally, it would be interesting to see to what extent the hierarchy of solution concepts defined in the setting of hedonic games~\cite{HedonicGamesHOCSC,AZIZ2013316,BY19} (which we discussed in Chapter~\ref{c:lit_review}) can be redefined in a model involving fixed-size coalitions. 
