In this section we first clarify our terminology and notation and then make a preliminary observation on the coincidence of vertex- and edge-disjoint $K_r$-packings.

Let $G = (V, E)$ be a simple undirected graph. We denote the \emph{closed neighbourhood} of some vertex $v \in V$ as $N_{G}[v] = N_{G}(v) \cup \{ v \}$. We denote by $\deg_{G}(v) = |N_{G}[v]| - 1$ the \emph{degree} of $v$ in $G$ and by $\Delta(G) = \max_{v\in V} \deg_{G}(v)$ the \emph{maximum degree} of $G$. If the graph in question is clear from context then we just write $\Delta$. If $\deg_{G}(v) = \delta$ for all $v \in V$ then we say that $G$ is \emph{$\delta$-regular}. For any subset of vertices $U \subseteq V$, we denote by $G[U]$ the subgraph of $G$ induced by $U$. 
Let $K_r$ denote a clique of size $r$, for some integer $r \geq 1$, and $K_r^G$ be the set of $K_r$s in $G$. We say that $T$ is a \emph{$K_r$-packing} in $G$ if $T\subseteq K_r^G$. The \emph{cardinality} of a $K_r$-packing is the number of $K_r$s that it contains. We say that a $K_r$-packing $T$ is \emph{vertex disjoint} if any two $K_r$s in $T$ have no vertex in common and \emph{edge disjoint} if any two $K_r$s in $T$ intersect by at most one vertex. The \emph{Vertex-Disjoint $K_r$-Packing Problem} (\mysymbolfirstusedefinition{symboldef:vdkr}{\vdkr}) is the following optimisation problem: given a simple undirected graph $G$, find a vertex-disjoint $K_r$-packing of maximum cardinality. The \emph{Edge-Disjoint $K_r$-Packing Problem} (\mysymbolfirstusedefinition{symboldef:edkr}{\edkr}) is defined analogously.

If $G$ contains four vertices $v_i, v_{j_1}, v_{j_2}, v_{j_3}$ where $\{ v_i, v_{j_a} \} \in E$ for each $a \in \{ 1, 2, 3 \}$, $\{ v_{j_1}, v_{j_2} \} \notin E$, $\{ v_{j_1}, v_{j_3} \} \notin E$, and $\{ v_{j_2}, v_{j_3} \} \notin E$, then we say that these four vertices form a \emph{claw}. Otherwise, we say that $G$ is \emph{claw-free}. For example, line graphs are claw-free \cite{MINTY1980284}. 

% If $G$ contains four vertices $v_i, v_{j_1}, v_{j_2}, v_{j_3}$ where $\{ v_i, v_{j_a} \} \in E$ for each $a \in \{ 1, 2, 3 \}$, $\{ v_{j_1}, v_{j_2} \} \notin E$, $\{ v_{j_1}, v_{j_3} \} \notin E$, and $\{ v_{j_2}, v_{j_3} \} \notin E$, then we say that these four vertices form a \emph{claw} (shown in Figure~\ref{fig:krpacking_claw}). Otherwise, we say that $G$ is \emph{claw-free}. For example, line graphs are claw-free \cite{MINTY1980284}. 
% \begin{wrapfigure}{r}{0.5\textwidth} 
%     \centering
%     \begin{tikzpicture}
\def\clawsize{1.6}
% \node[draw=none] (casenumber) at (-1.5, 3.0) {\emph{Case 7}};
% \draw[help lines,step=0.5] (0,0) grid (14,4);
\begin{scope}[every node/.style={circle,draw, minimum size=2.4mm}, scale=1.0]
\node[label={[label distance=0.4cm]30:$v_i$}] (vi) at (0.0, 0.0) {};

\node[label={[label distance=0.4cm]90:$v_{j_1}$}] (vj1) at ({90:\clawsize}) {};
\node[label={[label distance=0.4cm]270:$v_{j_2}$}] (vj2) at ({210:\clawsize}) {};
\node[label={[label distance=0.4cm]270:$v_{j_3}$}] (vj3) at ({330:\clawsize}) {};

\foreach \from/\to in {vi/vj1, vi/vj2, vi/vj3}
    \draw [thick] (\from) -- (\to);

\end{scope}
\end{tikzpicture}
%     \caption{A claw}
%     \label{fig:krpacking_claw}
% \end{wrapfigure}

For any maximisation problem $P$, instance $I$ of $P$, and feasible solution $S$ of $I$, let $\textrm{m}_{P}(I, S)$ denote the measure of $S$. Let $\textrm{opt}_{P}(I) = \max_{S \in \mathcal{F}(I)} \textrm{m}_{P}(I, S)$, where $\mathcal{F}(I)$ is the set of feasible solutions of $I$.

For technical purposes we define the \emph{$K_r$-vertex intersection graph} $\mathcal{K}_r^G = (K_r^G, E_{\mathcal{K}_r^G})$ of $G$, in which $\{ U, W \} \in E_{\mathcal{K}_r^G}$ if $|U \cap W| \geq 1$ for any $U, W \in K_r^G$. Similarly, we define the \emph{$K_r$-edge intersection graph} ${\mathcal{K}'}_r^G = (K_r^G, E_{{\mathcal{K}'}_r^G})$ of $G$ in which $\{ U, W \} \in E_{{\mathcal{K}'}_r^G}$ if $|U \cap W| \geq 2$ for any $U, W \in K_r^G$. We now make a preliminary observation.

\begin{observation}
\label{obs:krpacking_edkr_is_also_vdkr}
If $\Delta < 2r - 2$ then any edge-disjoint $K_r$-packing is also vertex disjoint.
\end{observation}
\begin{proof}
Any two $K_r$s in $G$ that share at least one vertex must in fact share at least two vertices, otherwise that vertex has degree at least $2r - 2$ in $G$.
\end{proof}