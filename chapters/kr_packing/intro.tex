\subsection{Background}
\label{sec:krpacking_background}

In this chapter we consider two problems related to clique packings in undirected graphs. In particular, finding a maximum-cardinality set of $r$-cliques, for some fixed $r \geq 3$, subject to the cliques in that set being either vertex disjoint or edge disjoint. We refer to such a set as a \emph{$K_r$-packing}, and to the two problems as the \emph{Vertex-Disjoint $K_r$-Packing Problem} (\vdkr) and the \emph{Edge-Disjoint $K_r$-Packing Problem} (\edkr). Note that $r$ is a fixed constant and does not form part of the problem input. If $r$ is not fixed then both problems generalise the well-studied problem of finding a clique of a given size \cite{GJ79}.

Most existing research relating to either vertex- or edge-disjoint $K_r$-packings covers either more restricted or more general cases. Two well-known special cases of $K_r$-packing are \vdktwo, also known as two-dimensional matching; and \vdkthree, for which an associated decision problem is known as \emph{Partition Into Triangles} (PIT, Problem~\ref{prob:pit}). 

\vdktwo is a central problem of graph theory and algorithmics. A classical result is that a maximum-cardinality two-dimensional matching can be found in polynomial time \cite{Edm65, MV80}. \vdkthree and its associated decision problem have also been the subject of much research. In 1975, Karp \cite{Karp75} noted that PIT, i.e.\ deciding if a give graph contains a $K_3$-packing of cardinality $|V|/3$, was $\NP$-complete. In this chapter we call such a $K_r$-packing, with cardinality $|V|/r$, \emph{perfect}, although it is sometimes known as a \emph{$K_r$-factor} \cite{guruswami_k_2001}.

In 2002, Caprara and Rizzi \cite{caprara_packing_2002} considered \vdkthree and \edkthree in the setting of a fixed maximum degree $\Delta$. They showed that \vdkthree is solvable in polynomial time if $\Delta = 3$ and $\APX$-hard even when $\Delta = 4$, and \edkthree is solvable in polynomial time if $\Delta = 4$ and $\APX$-hard even when $\Delta = 5$. They also showed that \vdkthree is $\NP$-hard for planar graphs even when $\Delta = 4$ and \edkthree is $\NP$-hard for planar graphs with $\Delta = 5$. In 2013, van Rooij et al.\ \cite{van_rooij_partition_2013} established an equivalence between \vdkthree when $\Delta = 4$ and \emph{Exact 3-Satisfiability} (X3SAT). They then used this equivalence to devise an $O(1.02220^n)$-time algorithm for PIT (i.e.\ the perfect \vdkthree decision problem) when $\Delta = 4$.

The approximability of \vdkthree and \edkthree has also been of interest. In 1989, Hurkens and Schrijver \cite{hs89} presented a new result relating to \emph{Systems of Distinct Representatives}, which they used to construct an algorithm, based on ``local improvement'', for a class of packing problems that includes \vdkr and \edkr. Specifically, for either \vdkr or \edkr and for any fixed constant $\varepsilon > 0$, Hurkens and Schrijver identify a polynomial-time $(r/2 + \varepsilon)$-approximation algorithm. In 1995, Halld\'{o}rsson \cite{Hal95} presented an alternative proof of this result and considered algorithms based on local improvement for other types of packing problems. In 2005, Mani\'c and Wakabayashi \cite{Eurocomb05} described approximation algorithms that improve on this approximation ratio for the restricted cases of \vdkthree in which $\Delta=4$, and \edkthree in which $\Delta=5$. They also presented a linear-time algorithm for \vdkthree on so-called indifference graphs.

A generalisation of \vdkr is \emph{Vertex-Disjoint $H$-Packing} (also called \emph{Vertex-Disjoint $G$-Packing}), where $H$ is an arbitrary but fixed undirected graph. In 1982, Takamizawa et al.\ \cite{takamizawa1982linear} showed that an optimisation problem related to Vertex-Disjoint $H$-Packing is solvable in polynomial time on a subclass of planar graphs known as \emph{series-parallel} graphs. In 1983, Kirkpatrick and Hell \cite{KH83} reviewed the literature of Vertex-Disjoint $H$-Packing and classify the complexity of an array of packing problems, some of which generalise \vdkr. In particular, they showed that if $H$ contain any connected component with three vertices then the perfect vertex-disjoint $H$-packing decision problem is $\NP$-complete. Pantel's 1999 thesis provides a comprehensive survey of $H$-packing \cite{Pantel99}.

The restriction of $H$-packing to $K_r$-packing has received comparatively less attention in the literature. In 1998, Dahlhaus and Karpinski \cite{DAHLHAUS199879} proved that a perfect vertex-disjoint $K_r$-packing can be found in polynomial time, if it exists, in chordal and strongly chordal graphs. In 2001, Guruswami et al.\ \cite{guruswami_k_2001} showed that, for any $r \geq 3$, the \vdkr decision problem is $\NP$-complete for chordal graphs, planar graphs (assuming $r<5$), line graphs, and total graphs. Their $\NP$-completeness result involving $K_r$-packing on chordal graphs, for any $r \geq 3$, resolved an open question of Dahlhaus and Karpinski. Guruswami et al.\ also described polynomial-time algorithms for \vdkthree and the perfect \vdkr decision problem on split graphs, and \vdkr on cographs. Their result for cographs was later extended by Pedrotti and de Mello \cite{KrPackingP4sparse} for so-called $P_4$-sparse graphs.

From the converse perspective of graphs with a fixed minimum degree, Hajnal and Szemer\`edi \cite{hajnal_szemeredi} proved in 1970 that if the minimum degree of a graph is greater than or equal to $(1 - 1/r)|V|$ then a perfect vertex-disjoint $K_r$-packing must exist. Kierstead and Kostochka \cite{KIERSTEAD2008226} later generalised this result to show that, in this case, such a packing can be constructed in polynomial time. A growing area of research work studies the existence of perfect $K_r$- and $H$-packings with respect to conditions involving vertex degree \cite{TreglownThesis, KD09}.

\subsection{Our contribution}

In 2002, Caprara and Rizzi \cite{caprara_packing_2002} showed that \vdkthree is solvable in polynomial time if $\Delta=3$ and $\APX$-hard if $\Delta=4$; and \edkthree is solvable in polynomial time if $\Delta=4$ and $\APX$-hard if $\Delta=5$. In this chapter we extend some of their techniques in order to generalise their results and fully classify the complexity of both \vdkr and \edkr for any $\Delta \geq 1$ and any fixed $r \geq 3$. We summarise this classification in Table~\ref{tab:krpacking_results}.

\begin{table}[h]
\centering
% \def\arraystretch{1.4}
\begin{tabular}{clp{0.0cm}lp{0.0cm}l}\noalign{\hrule}
\Tstrut\vspace*{0.2em}
& \multicolumn{3}{c}{is solvable in} & \\
& \multicolumn{1}{c}{linear time if} & \multicolumn{2}{c}{polynomial time if} & \multicolumn{2}{c}{is $\APX$-hard if} \\
\hline \vdkr & $\Delta < 3r/2 - 1$ & & $\Delta < 5r/3 - 1$ & & $\Delta \geq \lceil 5r/3 \rceil - 1$\Tstrut\\[0.4em]
\multirow{2}{*}{\edkr} & \multirow{2}{*}{$\Delta < 3r/2 - 1$} & \ldelim\{{2}{*} & $\Delta \leq 2r - 2$ if $r \leq 5$ &  \ldelim\{{2}{*} &  $\Delta > 2r - 2$ if $r \leq 5$\\
& & & $\Delta < 5r/3 - 1$ otherwise & & $\Delta \geq \lceil 5r/3 \rceil - 1$ otherwise
\vspace*{0.3em}
\Bstrut\\\noalign{\hrule}
\end{tabular}
\caption{Our complexity results for \vdkr and \edkr}
\label{tab:krpacking_results}
\end{table}

In the next section, Section~\ref{sec:krpacking_prelims}, we define some additional notation and make an observation on the coincidence of vertex- and edge-disjoint $K_r$-packings.

In Section~\ref{sec:krpacking_lineartime}, we consider the case when $\Delta < 3r/2 - 1$. We show that in this case, any maximal vertex- or edge-disjoint $K_r$-packing is also maximum (Theorem~\ref{thm:krpacking_r_maximal_is_maximum}), and devise a linear-time algorithm for \vdkr and \edkr in this setting (Theorem~\ref{thm:krpacking_vdkr_3r2minus1} and Corollary~\ref{cor:krpacking_edkr_3r2minus2}). 

In Section~\ref{sec:krpacking_ptime}, we present our solvability results, showing that \vdkr can be solved in polynomial time if $\Delta < 5r/3 - 1$ (Theorem~\ref{thm:krpacking_vdkr_polytime_5r3minus1}); and \edkr can be solved in polynomial time if either $r\leq 5$ and $\Delta \leq 2r - 2$ (Theorem~\ref{thm:krpacking_edkr345_polytime}), or $r\geq 6$ and $\Delta < 5r/3 - 1$ (Theorem~\ref{thm:krpacking_edkrgeq6_polytime}). Our proof uses a similar technique to that of Caprara and Rizzi's \cite{caprara_packing_2002}, which involves finding a maximum independent set in a graph that represents the intersection of $K_r$s.

In Section~\ref{sec:krpacking_apxhardness}, we show that our solvability results are in a sense best possible, unless $\P \neq \NP$. Specifically, we show that \vdkr is $\APX$-hard if $\Delta \geq \lceil 5r/3 \rceil - 1$ (Theorem~\ref{thm:krpacking_vdkr_apxhard}); and \edkr is $\APX$-hard if either $r \leq 5$ and $\Delta > 2r - 2$ (Theorem~\ref{thm:krpacking_edkr345apxhard}), or $r \geq 6$ and $\Delta \geq \lceil 5r/3 \rceil - 1$ (Theorem~\ref{thm:krpacking_edkr_rgeq6_apxhard}). In other words, we prove that there exists some fixed constants $\varepsilon > 1$ and $\varepsilon' > 1$ such that no polynomial-time $\varepsilon$-approximation algorithm exists for \vdkr if $\Delta \geq \lceil 5r/3 \rceil - 1$; and no polynomial-time $\varepsilon'$-approximation algorithm exists for \edkr if either $r \leq 5$ and $\Delta > 2r - 2$, or $r \geq 6$ and $\Delta \geq \lceil 5r/3 \rceil - 1$. 

In Section~\ref{sec:krpacking_conclusion} we recap on our results and consider directions for future work.

\subsection{Preliminaries}
\label{sec:krpacking_prelims}
In this section we first clarify our terminology and notation and then make a preliminary observation on the coincidence of vertex- and edge-disjoint $K_r$-packings.

Let $G = (V, E)$ be a simple undirected graph. We denote the \emph{closed neighbourhood} of some vertex $v \in V$ as $N_{G}[v] = N_{G}(v) \cup \{ v \}$. We denote by $\deg_{G}(v) = |N_{G}[v]| - 1$ the \emph{degree} of $v$ in $G$ and by $\Delta(G) = \max_{v\in V} \deg_{G}(v)$ the \emph{maximum degree} of $G$. If the graph in question is clear from context then we just write $\Delta$. If $\deg_{G}(v) = \delta$ for all $v \in V$ then we say that $G$ is \emph{$\delta$-regular}. For any subset of vertices $U \subseteq V$, we denote by $G[U]$ the subgraph of $G$ induced by $U$. 
Let $K_r$ denote a clique of size $r$, for some integer $r \geq 1$, and $K_r^G$ be the set of $K_r$s in $G$. We say that $T$ is a \emph{$K_r$-packing} in $G$ if $T\subseteq K_r^G$. The \emph{cardinality} of a $K_r$-packing is the number of $K_r$s that it contains. We say that a $K_r$-packing $T$ is \emph{vertex disjoint} if any two $K_r$s in $T$ have no vertex in common and \emph{edge disjoint} if any two $K_r$s in $T$ intersect by at most one vertex. The \emph{Vertex-Disjoint $K_r$-Packing Problem} (\mysymbolfirstusedefinition{symboldef:vdkr}{\vdkr}) is the following optimisation problem: given a simple undirected graph $G$, find a vertex-disjoint $K_r$-packing of maximum cardinality. The \emph{Edge-Disjoint $K_r$-Packing Problem} (\mysymbolfirstusedefinition{symboldef:edkr}{\edkr}) is defined analogously.

If $G$ contains four vertices $v_i, v_{j_1}, v_{j_2}, v_{j_3}$ where $\{ v_i, v_{j_a} \} \in E$ for each $a \in \{ 1, 2, 3 \}$, $\{ v_{j_1}, v_{j_2} \} \notin E$, $\{ v_{j_1}, v_{j_3} \} \notin E$, and $\{ v_{j_2}, v_{j_3} \} \notin E$, then we say that these four vertices form a \emph{claw}. Otherwise, we say that $G$ is \emph{claw-free}. For example, line graphs are claw-free \cite{MINTY1980284}. 

% If $G$ contains four vertices $v_i, v_{j_1}, v_{j_2}, v_{j_3}$ where $\{ v_i, v_{j_a} \} \in E$ for each $a \in \{ 1, 2, 3 \}$, $\{ v_{j_1}, v_{j_2} \} \notin E$, $\{ v_{j_1}, v_{j_3} \} \notin E$, and $\{ v_{j_2}, v_{j_3} \} \notin E$, then we say that these four vertices form a \emph{claw} (shown in Figure~\ref{fig:krpacking_claw}). Otherwise, we say that $G$ is \emph{claw-free}. For example, line graphs are claw-free \cite{MINTY1980284}. 
% \begin{wrapfigure}{r}{0.5\textwidth} 
%     \centering
%     \begin{tikzpicture}
\def\clawsize{1.6}
% \node[draw=none] (casenumber) at (-1.5, 3.0) {\emph{Case 7}};
% \draw[help lines,step=0.5] (0,0) grid (14,4);
\begin{scope}[every node/.style={circle,draw, minimum size=2.4mm}, scale=1.0]
\node[label={[label distance=0.4cm]30:$v_i$}] (vi) at (0.0, 0.0) {};

\node[label={[label distance=0.4cm]90:$v_{j_1}$}] (vj1) at ({90:\clawsize}) {};
\node[label={[label distance=0.4cm]270:$v_{j_2}$}] (vj2) at ({210:\clawsize}) {};
\node[label={[label distance=0.4cm]270:$v_{j_3}$}] (vj3) at ({330:\clawsize}) {};

\foreach \from/\to in {vi/vj1, vi/vj2, vi/vj3}
    \draw [thick] (\from) -- (\to);

\end{scope}
\end{tikzpicture}
%     \caption{A claw}
%     \label{fig:krpacking_claw}
% \end{wrapfigure}

For any maximisation problem $P$, instance $I$ of $P$, and feasible solution $S$ of $I$, let $\textrm{m}_{P}(I, S)$ denote the measure of $S$. Let $\textrm{opt}_{P}(I) = \max_{S \in \mathcal{F}(I)} \textrm{m}_{P}(I, S)$, where $\mathcal{F}(I)$ is the set of feasible solutions of $I$.

For technical purposes we define the \emph{$K_r$-vertex intersection graph} $\mathcal{K}_r^G = (K_r^G, E_{\mathcal{K}_r^G})$ of $G$, in which $\{ U, W \} \in E_{\mathcal{K}_r^G}$ if $|U \cap W| \geq 1$ for any $U, W \in K_r^G$. Similarly, we define the \emph{$K_r$-edge intersection graph} ${\mathcal{K}'}_r^G = (K_r^G, E_{{\mathcal{K}'}_r^G})$ of $G$ in which $\{ U, W \} \in E_{{\mathcal{K}'}_r^G}$ if $|U \cap W| \geq 2$ for any $U, W \in K_r^G$. We now make a preliminary observation.

\begin{observation}
\label{obs:krpacking_edkr_is_also_vdkr}
If $\Delta < 2r - 2$ then any edge-disjoint $K_r$-packing is also vertex disjoint.
\end{observation}
\begin{proof}
Any two $K_r$s in $G$ that share at least one vertex must in fact share at least two vertices, otherwise that vertex has degree at least $2r - 2$ in $G$.
\end{proof}