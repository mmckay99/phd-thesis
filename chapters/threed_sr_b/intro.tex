In this chapter we study a new model of fixed-size coalition formation, which we call the \emph{Three-Dimensional Roommates with $\mathscr{B}$-preferences} (\mysymbolfirstusedefinition{symboldef:threedr_b}{3DR-B}). This model is closely related to some of the existing models of Three-Dimensional Roommates (3DR) discussed in Chapter~\ref{c:lit_review}, and in particular the \emph{Three-Person Stable Assignment problem} (3PSA) \cite{NH91} and the model of Iwama et al.~\cite{IMO07}. As in the model of Iwama et al., in 3DR-B each agent has a strict preference list over all other agents. A specific set extension rule is then used to infer from each agent's preference list that agent's preferences over coalitions. In 3DR-B, that set extension rule is known as \emph{$\mathscr{B}$-preferences} \cite{CH02}. Using $\mathscr{B}$-preferences, any agent prefers some triple $S$ to another triple $T$ if the most-preferred agent in $S$ is preferred to the most-preferred agent in $T$. In this chapter we consider in 3DR-B the existence of, and complexity of finding, matchings that are stable.

% There is an interesting motivation underlying this model: in 

We first show, in Section~\ref{sec:threed_sr_b_hardness}, that a given instance of 3DR-B may not contain a stable matching and that the associated decision problem is $\NP$-complete (Theorem~\ref{thm:threed_sr_b_existence}). This contrasts with an analogous model in which coalitions need not have a fixed size, in which a stable matching must always exist and can be found in polynomial time \cite{CR01}. It may seem intuitive that the additional requirement of fixed-size coalitions makes this particular problem harder to solve, and this result gives an example of a model in which that intuition holds. It also leads to interesting directions for subsequent work, for example involving approximately stable matchings, or alternative constraints on the size of feasible coalitions.

We then consider, in Section~\ref{sec:threed_sr_b_approximation}, a closely related optimisation problem in which the objective is to construct a matching that is, in terms of a specific measure, as stable as possible. We begin by showing that an existing approximation algorithm for a different model of 3DR can be used to devise a $9/4$-approximation algorithm for the 3DR-B problem (Theorem~\ref{thm:threed_sr_b_approxalgofournine}). We then show that a simple algorithm based on serial dictatorship gives an improved approximation ratio of $3/2$ (Theorem~\ref{thm:threed_sr_b_approxalgo}). Interestingly, this algorithm can be viewed as an adaptation of the algorithm developed by Cechl\'{a}rov\'{a} and Romero-Medina \cite{CR01} that can be used to construct a stable matching in the analogous model in which coalitions need not have a fixed size.

Next, in Section~\ref{sec:threed_sr_b_structural}, we consider the problem of identifying the smallest instance of 3DR-B that contains no stable matching. We show that such an instance contains at least $9$ (Theorem~\ref{thm:threed_sr_b_ifnis6thensmexists}) and at most $15$ agents (Theorem~\ref{thm:threed_sr_b_ifnis15thensmdoesnotexist}) but leave determining the precise number of agents as an open problem.

%By the reduction that we present in Section~\ref{sec:threed_sr_b_hardness}, such an instance must contain at most $63$ agents. We provide a lower bound and show that such an instance must contain at least $9$ agents (Theorem~\ref{thm:threed_sr_b_ifnis6thensmexists}).

Finally, in Section~\ref{sec:threed_sr_b_conclusion}, we recap on our results and discuss some directions for future work.

We proceed with some formal definitions and notation. An instance of 3DR-B comprises a set $N$ of $3n$ agents and a preference list of each agent $\alpha_i$, labelled $P_{\alpha_i}$, that describes a strict order over all agents in $N \setminus \{ \alpha_i \}$. We say that an agent $\alpha_i$ \emph{prefers} $\alpha_j$ to $\alpha_k$, denoted $\alpha_j \succ_{\alpha_i} \alpha_k$, if $\alpha_j$ precedes $\alpha_k$ in $P_{\alpha_i}$. A \emph{triple} is an unordered set of three agents. In order to compare triples, agents in an instance of 3DR-B use \emph{$\mathscr{B}$-preferences}, which are defined as follows. For any agent $\alpha_i$ and set of agents $S \subseteq N$ we denote by $\mathscr{B}_{\alpha_i}(S)$ the most-preferred agent in $S \setminus \{ \alpha_i \}$ according to $\alpha_i$. For any agent $\alpha_i$ and any two triples $r$ and $s$, we say that $\alpha_i$ \emph{prefers} $r$ to $s$, denoted $r \succ_{\alpha_i} s$, if $\mathscr{B}_{\alpha_i}(r) \succ_{\alpha_i} \mathscr{B}_{\alpha_i}(s)$. A \emph{matching} is a partition of $N$ into $n$ triples. Given an agent $\alpha_i$ and a matching $M$, we denote by $M(\alpha_i)$ the triple in $M$ that contains $\alpha_i$. Given a matching $M$, we say that a triple $t$ is \emph{blocking} if each agent $\alpha_i$ in $t$ prefers $t$ to $M(\alpha_i)$. A matching is \emph{stable} if it does not contain a blocking triple. Let $\mathscr{B}_{\alpha_i}(M)$ be short for $\mathscr{B}_{\alpha_i}(M(\alpha_i))$. Let $P$ be the collection of preference lists  $P_{\alpha_i}$ for each agent $\alpha_i$. For any instance $(N, P)$ of 3DR-B and any matching $M$, we denote by $\textrm{bt}(M, (N, P)) \subseteq \binom{N}{3}$ the set of triples that block $M$ in $(N, P)$. Conversely, we denote by $\textrm{nbt}(M, (N, P)) = \binom{N}{3} \setminus \textrm{bt}(M, (N, P))$ the set of triples that do not block $M$ in $(N, P)$. When the instance in question is clear from context, we simply write $\textrm{bt}(M)$ or $\textrm{nbt}(M)$. 