In this chapter we study the approximability of the Three-Dimensional Stable Matching problem with Cyclic Preferences (\mysymbolfirstusedefinition{symboldef:threedsmcyc}{3-DSM-CYC}, also known as 3DSM \cite{BM10} or c3DSM \cite{Pashkovich20}).

As we saw in Chapter~\ref{c:lit_review}, the question of whether every instance of 3-DSM-CYC contains a stable matching was open for several decades. It was only in 2019 that Lam and Plaxton~\cite{Plaxton3DSMCYCJournal} showed that a given instance of 3-DSM-CYC need not contain a stable matching and that the associated decision problem is $\NP$-complete. A natural next step is to consider approximately stable matchings, as we do here. To our knowledge, we present the first theoretical results on the approximability of 3-DSM-CYC. 

In this chapter we consider the optimisation problem of finding a matching with the maximum number of non-blocking families, which we call the {3-DSM-CYC Maximally Stable Matching problem} (\mysymbolfirstusedefinition{symboldef:threedsmcyc_msm}{3-DSM-CYC-MSM}).

We begin, in Section~\ref{sec:three_dsm_cyc_unrestricted_preferences}, by showing that an existing approximation algorithm for 3GSM-MSM, which is a closely related problem, can be used to devise a $9/4$-approximation algorithm for 3-DSM-CYC-MSM (Theorem~\ref{thm:three_dsm_cyc_ninefour}). We then show that a simple algorithm based on serial dictatorship gives an improved approximation ratio of $6/5$ (Theorem~\ref{thm:three_dsm_cyc_unrestricted}).

Next, in Section~\ref{sec:three_dsm_cyc_masterlist}, we consider a situation in which there exists a master preference list over all agents of one type, and quantify the similarity between the preference list of any agent of the previous type and that master list, in terms of a specific distance metric. We extend the approximation algorithm for 3-DSM-CYC-MSM and show that if the maximum distance is sufficiently small then as it is further reduced, the approximation ratio of the algorithm decreases from $6/5$ to $1$ (Theorem~\ref{thm:three_dsm_cyc_ml}). As a corollary, we show that if every agent of one type has the same preference list then the algorithm returns a matching that is stable (Corollary~\ref{cor:three_dsm_cyc_ml_stab}, which is also implied by a result of Escamocher and O'Sullivan \cite{Escamocher2018}).

Finally, in Section~\ref{sec:three_dsm_cyc_conclusion}, we recap on our results and discuss some directions for future work.

We proceed with some formal definitions and notation. An instance of 3-DSM-CYC comprises a set $N$ of $3n$ agents and a strict preference list for each agent $\alpha_i$, labelled $P_{\alpha_i}$. Each agent in $N$ has one of three \emph{types}, which we call \emph{man}, \emph{woman}, and \emph{dog}. There are $n$ agents of each type, and the agents of each type are labelled $U = \{ u_1, u_2, \dots, u_n \}$, $W = \{ w_1, w_2, \dots, w_n \}$, and $D =  \{ d_1, d_2, \dots, d_n \}$. The types have a cyclic order in which $W$ follows $U$, $D$ follows $W$, and $U$ follows $D$. Each agent's preference list $P_{\alpha_i}$ describes a strict order all agents in the next type. We say that an agent $\alpha_i$ \textit{prefers} $\beta_j$ to $\beta_k$ if $\beta_j$ precedes $\beta_k$ in the preference list $P_{\alpha_i}$ of $\alpha_i$. A \textit{family} is a $3$-tuple $( u_i, w_j, d_k ) \in U\times W\times D$. A \textit{matching} is a set of families where each agent in $N$ is contained in exactly one family. Given an agent $\alpha_i$ and a matching $M$, we denote by $M(\alpha_i)$ the family in $M$ that contains $\alpha_i$. Given a matching $M$, we say that a family $f$ is \textit{blocking} if each agent $\alpha_i$ in $f$ prefers the agent of the next type in $f$ to the agent of the next type in $M(\alpha_i)$. A matching is \textit{stable} if it does not contain a blocking family. Let $P$ be the collection of preference lists  $P_{\alpha_i}$ for each agent $\alpha_i$. For any instance $(N, P)$ of 3-DSM-CYC and any matching $M$, we denote by $\textrm{bf}(M, (N, P)) \subseteq U \times W \times D$ the set of families that block $M$ in $(N, P)$. Conversely, we denote by $\textrm{nbf}(M, (N, P)) = (U \times W\times D) \setminus \textrm{bf}(M, (N, P))$ the set of families that do not block $M$ in $(N, P)$. When the instance in question is clear from context, we simply write $\textrm{bf}(M)$ or $\textrm{nbf}(M)$. Formally, 3-DSM-CYC-MSM is the optimisation variant of 3-DSM-CYC in which the objective is to maximise $|\textrm{nbf}(M, (N, P))|$.

% The decision problem associated with 3-DSM-CYC is to test if a given instance contains a stable matching. We define the \emph{3-DSM-CYC Maximally Stable Matching problem} (3-DSM-CYC-MSM) as follows: given an instance of 3-DSM-CYC, construct a matching with the maximum number of non-blocking families. 


