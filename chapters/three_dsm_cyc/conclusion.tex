In this chapter we considered the approximability of the 3-DSM-CYC Maximally Stable Matching problem (3-DSM-CYC-MSM). We first presented a $9/4$-approximation algorithm based on an existing algorithm for 3GSM-MSM, which is a closely related problem \cite{rosenbaum16}. We then presented a $6/5$-approximation algorithm based on serial dictatorship, and showed that our analysis is tight asymptotically. Finally, we considered a situation in which the preference lists of all agents of at least one type are derived from some master list, and modified the aforementioned serial dictatorship algorithm for this setting. We considered the maximum Kendall tau distance $c$ between any such agent's list and the master list, and showed that if $c \leq n$ then the modified algorithm has an approximation ratio of $6 / (6 - (3d^2 - 2d^3))$, where $d = c/n$, which is tight asymptotically.

As we saw in Chapter~\ref{c:lit_review}, the history of 3-DSM-CYC spans several decades and it continues to generate interesting research. Given the recent result of Lam and Plaxton \cite{Plaxton3DSMCYCJournal} (who showed that stable matchings need not exist in general, and the associated decision problem is $\NP$-complete), it seems most natural to consider either the approximability of 3-DSM-CYC, as we do here, or parameterised complexity. 

In the optimisation version of 3-DSM-CYC that we studied here, the objective is to maximise the number of non-blocking families. Of course, one could also define a complementary problem in which the objective is to minimise the number of blocking families, which is arguably more natural. In fact, similar optimisation problems, in which the objective involves minimising the number of blocking coalitions, have been studied in relation to other problems of matching under preferences \cite{BMM10, ABM06, Hamada16}. Another possibility is to construct a sub-matching of maximum cardinality such that no three agents in families in the sub-matching form a blocking family in the sub-matching. Rosenbaum \cite{rosenbaum16} refers to the analogous problem for 3GSM as the \emph{3G Maximum Stable Sub-matching problem} (3G-MSS). 

Although the definitions of the maximisation and minimisation variants of 3-DSM-CYC are complementary, it seems as if the difficulty of characterising instances with no stable matching makes tackling the latter variant more challenging.

% It is of course possible that the approximation algorithms we presented here for 3-DSM-CYC-MSM can be improved upon. At a high level, our approximation algorithm in the case of a master list demonstrates one possible strategy when the preference lists of all agents of at least one type are sufficiently similar. One  be possible to augment this algorithm to handle a scenario  

