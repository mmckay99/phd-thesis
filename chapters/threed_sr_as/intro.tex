In this chapter we formalise a model of Three-Dimensional Roommates (3DR) involving additively separable preferences, which we call \mysymbolfirstusedefinition{symboldef:threedr_as}{3DR-AS}. We consider in 3DR-AS the existence of, and complexity of finding, matchings that are stable, under three possible restrictions of the agents' preferences.

A strong motivation exists for a model of 3DR with additively separable preferences. The first such model, of which 3DR-AS is a generalisation, was first proposed by Huang \cite{Huang07conference} in 2007, who noted the natural definition and relative practicality of additively separable preferences compared to other possible systems of preference representation. For example, such a model could be applied to a social network graph involving a symmetric ``friendship'' relation between users \cite{Sless18}. Another special case of 3DR-AS is \emph{Geometric 3D-SR} \cite{ABEOMP09} (see Chapter~\ref{c:lit_review}). In a sense, all of these models can also be considered as a special type of additively separable hedonic game \cite{AZIZ2013316,SUNG2010635}, which have received much attention in the literature (and are also discussed in Chapter~\ref{c:lit_review}).

We begin in Section~\ref{sec:threed_sr_as_model} with some preliminary definitions and results. 

We then show, in Section~\ref{sec:threed_sr_as_symmetricbinary}, that any instance of 3DR-AS with binary and symmetric preferences must contain a stable matching, and present a polynomial-time algorithm that can construct a stable matching in a given such instance (Theorem~\ref{thm:threed_sr_as_symmetric_binary_construction}). We then consider the problem of finding a stable matching with maximum utilitarian welfare, given an instance in which preferences are binary and symmetric. We show that this optimisation problem is $\NP$-hard (Theorem~\ref{thm:threed_sr_as_maxutilstable_hard}) but also that the algorithm for constructing a stable matching in this setting can be modified to yield a $2$-approximation algorithm (Theorem~\ref{thm:threed_sr_as_approxratio}).

Next, we complement the previous tractability results with two hardness results. The first, shown in Section~\ref{sec:threed_sr_as_generalbinary}, is that a stable matching need not exist in general, and the associated decision problem is $\NP$-complete even when preferences are binary and not necessarily symmetric (Theorem~\ref{thm:threed_sr_as_binary_reduction}). The second, shown in Section~\ref{sec:threed_sr_as_symmetricternary}, is that the same decision problem is $\NP$-complete even when preferences are ternary and symmetric (Theorem~\ref{thm:threed_sr_as_symmetric_ternary_reduction}).

Finally, in Section~\ref{sec:threed_sr_as_conclusion}, we recap on our contribution and discuss some directions for future work.