From Theorems~\ref{thm:threed_efr_as_jef_binary_algorithm} and~\ref{thm:threed_efr_as_jef_terasym_npcomplete}, a natural question arises: is it the symmetry of agents' preferences that guarantees the existence of a j-envy-free matching? 
In this section we show that this is not the case, and a j-envy-free matching may not exist even when agents' preferences are symmetric, and the associated existence problem is $\NP$-complete. We remark, however, that the instances shown that do not contain j-envy-free matchings are relatively contrived and involve non-negative integer valuations up to $6$. We leave open the case in which preferences are symmetric and the maximum valuation is strictly less than $6$.

To show that this existence problem is $\NP$-complete, we present a polynomial-time reduction from \emph{Partition into Triangles} (PIT, Problem~\ref{prob:pit}), which is $\NP$-complete \cite{GJ79}. This reduction is similar to the reduction we presented in Section~\ref{sec:threed_efr_as_jef_ternary} for the analogous problem involving ternary preferences that are not (necessarily) symmetric. Note that section we reduced from Directed Triangle Packing (DTP, Problem~\ref{pr:DTP}) but here we reduce from PIT.

We describe the reduction in detail and then provide some intuition with respect to its design. The reduction, illustrated in Figure~\ref{fig:threed_efr_as_jef_symmetric_reduction}, is as follows. Suppose $G$ is an arbitrary instance of PIT. We shall construct an instance $(N, V)$ of 3DR-AS with symmetric preferences and maximum valuation $6$. Since valuations are symmetric in $(N, V)$, we shall usually specify valuations in one direction only. For example, instead of writing ``let $v_{\alpha_i}(\alpha_j)=v_{\alpha_j}(\alpha_i)=1$'' we write ``let $v_{\alpha_i}(\alpha_j)=1$''. Unless otherwise specified assume that $v_{\alpha_i}(\alpha_j)=0$ for any $\alpha_i, \alpha_j \in N$. To simplify the description of the valuations in the reduction, in this section we write $i \myoplus y$ to denote $((i + y - 2) \bmod 10) + 2$. 

First, construct a set of eleven agents $H = \{ h_1, h_2, \dots, h_{11} \}$. For each $i$ where $2\leq i \leq 11$ let $v_{h_1}(h_i) = 2$. For each $i$ where $2\leq i \leq 11$, let:
\begin{itemize}
    \item $v_{h_i}(h_{i \myoplus 1}) = 4$ if $i$ is even otherwise $5$
    \item $v_{h_i}(h_{i \myoplus 2}) = 6$ if $i$ is even otherwise $3$
    \item $v_{h_i}(h_{i \myoplus 3}) = 1$
    \item $v_{h_i}(h_{i \myoplus 4}) = 1$
    \item $v_{h_i}(h_{i \myoplus 5}) = 3$.
\end{itemize}
Next, construct a set of four agents $L = \{ l_1, l_2, l_3, l_4 \}$. Let $v_{l_1}(l_2) = v_{l_3}(l_4) = 2$ and $v_{l_1}(l_3) = v_{l_1}(l_4) = v_{l_2}(l_3) = v_{l_2}(l_4) = 1$.

Next, construct a set of $3q$ agents $C = \{ c_1, c_2, \dots, c_{3q} \}$. Let $v_{c_i}(l_r) = 3$ for each $i$ and $r$ where $1\leq i \leq 3q$ and $1\leq r \leq 4$. For each $i$ and $j$ where $1\leq i, j \leq 3q$ let $v_{c_i}(c_j) = 3$ if $\{ w_i, w_j \} \in E$ otherwise $2$. This completes the construction of $(N, V)$. Note that the structure of the valuations among the agents in $C$ reflects the graph $G$.
%
\begin{figure}
    \centering
    \vspace*{0.2cm}
    \definecolor{figurecolourschemewt1_adjusted}{rgb}{0.8,0.8,0.8}
\definecolor{figurecolourschemewt1}{rgb}{0.5,0.5,0.5}
\definecolor{figurecolourschemewt2}{rgb}{0.5,0.0,0.5}
\definecolor{figurecolourschemewt3}{rgb}{0,0,0}
\definecolor{figurecolourschemewt4}{rgb}{0,0,1}
\definecolor{figurecolourschemewt5}{rgb}{0,1,0}
\definecolor{figurecolourschemewt6}{rgb}{1,0,0}

\begin{tikzpicture}

\begin{scope}[every node/.style={circle,draw, minimum size=2.4mm}, scale=1.0]
    \begin{scope}
        % the lines from h1 to H
        \def\honedist{4.8}
        \def\honeangle{90} % formerly 36
        \node[thick, circle, label={[label distance=0.4cm]270:$h_1$}] (h1) at ({\honedist*sin(\honeangle)*-1.0 - 1.6}, {\honedist*cos(\honeangle)}) {};
        
        % virtual nodes to show the "complete bipartiteness"
        % \def\honeaspacing{2.0}
        % \node[thick, circle] (h1a) at ({\honeaspacing*sin(90-\honeangle)*-1.0}, {\honeaspacing*cos(90-\honeangle)*-1.0}) {};
        % \node[thick, circle] (h1b) at ({\honeaspacing*sin(90-\honeangle)*-0.5}, {\honeaspacing*cos(90-\honeangle)*-0.5}) {};
        % \node[thick, circle] (h1c) at ({\honeaspacing*sin(90-\honeangle)*0.33}, {\honeaspacing*cos(90-\honeangle)*0.33}) {};
        % \node[thick, circle] (h1d) at ({\honeaspacing*sin(90-\honeangle)*1.0}, {\honeaspacing*cos(90-\honeangle)*1.0}) {};
        
        % \draw[color=figurecolourschemewt2, very thick] (h1) -- (h1a) -- (h1b) -- (h1) -- (h1d);
        % \path (h1) -- (h1c) node [draw=none, pos=0.18] {$\dots$};
        \draw[color=figurecolourschemewt2, very thick] (h1) -- (0.0, 0.0);
    
        \begin{scope}[shift={(-1.5, 0.0)}]
            \def\scalefactor{0.25}
            \def\hlabeldist{0.4cm}
            
            \filldraw[color=enclosure_color, fill=white](-0.05, 0) circle (\scalefactor*15);
            
            \node[thick, circle, label={[label distance=\hlabeldist]76:$h_3$}] (h3) at (\scalefactor*3.09,\scalefactor*9.51) {};
            \node[thick, circle, label={[label distance=\hlabeldist]94:$h_2$}] (h2) at (\scalefactor*-3.09,\scalefactor*9.51) {};
            \node[thick, circle, label={[label distance=\hlabeldist]-225:$h_{11}$}] (h11) at (\scalefactor*-8.09,\scalefactor*5.88) {};
            \node[thick, circle, label={[label distance=\hlabeldist]-180:$h_{10}$}] (h10) at (\scalefactor*-10.0,\scalefactor*0.0) {};
            \node[thick, circle, label={[label distance=\hlabeldist]-135:$h_9$}] (h9) at (\scalefactor*-8.09,\scalefactor*-5.88) {};
            \node[thick, circle, label={[label distance=\hlabeldist]-94:$h_8$}] (h8) at (\scalefactor*-3.09,\scalefactor*-9.51) {};
            \node[thick, circle, label={[label distance=\hlabeldist]-76:$h_7$}] (h7) at (\scalefactor*3.09,\scalefactor*-9.51) {};
            \node[thick, circle, label={[label distance=\hlabeldist]-45:$h_6$}] (h6) at (\scalefactor*8.09,\scalefactor*-5.88) {};
            \node[thick, circle, label={[label distance=\hlabeldist]0:$h_5$}] (h5) at (\scalefactor*10.0,\scalefactor*0.0) {};
            \node[thick, circle, label={[label distance=\hlabeldist]45:$h_4$}] (h4) at (\scalefactor*8.09,\scalefactor*5.88) {};
        \end{scope}
        
        
        \begin{scope}[shift={(5.0, 0.0)}]
            \draw[color=figurecolourschemewt3] (0.0,-2.0) -- (0.0, 2.0);
        
            % the edges from L to C
            % \draw[color=figurecolourschemewt3] (-0.45,-2.0) -- (-0.45, 2.0);
            % \draw[color=figurecolourschemewt3] (-0.25,-2.0) -- (-0.25, 2.0);
            % % \draw[color=figurecolourschemewt3] (0.25,-2.0) -- (0.25, 2.0);
            % \node[draw=none] (dots1) at (0.13, 0.1) {$\dots$};
            % \draw[color=figurecolourschemewt3] (0.45,-2.0) -- (0.45, 2.0);
        \end{scope}
        
        \begin{scope}[shift={(5.0, -1.4)}]
            % \filldraw[color=enclosure_color, fill=white](0, 0) circle (1.8);
            \filldraw[color=enclosure_color, fill=white, rounded corners=0.5cm] (-1.7, -0.7) rectangle (1.7, 0.7) {};
            \node[draw=none] (clabel) at (0.0, -0.05) {$c_1, c_2, \dots, c_{3q}$};
            % \node[draw=none] (clabel) at (0.0, -1.0) {$w_{c_i}(c_j) \in \{ 2, 3 \}$};
            
            % \node[draw=none] (clabel) at (0.0, -1.0) {$w_{c_i}(c_j) =
            % \begin{cases}
            %     3 & \text{if } \{ w_i, w_j \} \in E\\
            %     2 & \text{otherwise}
            % \end{cases}$};
        \end{scope}
        
        \begin{scope}[shift={(5.0, 1.4)}]
            \def\scalefactorl{1.2}
            \filldraw[color=enclosure_color, fill=white, rounded corners=0.5cm] (\scalefactorl*-1.5,\scalefactorl*-1.0) rectangle (\scalefactorl*1.5,\scalefactorl*1.0) {};
            \node[thick, circle, label={[label distance=0.4cm]180:$l_1$}] (l1) at (\scalefactorl*-0.5,\scalefactorl*0.5) {};
            \node[thick, circle, label={[label distance=0.4cm]0:$l_2$}] (l2) at (\scalefactorl*0.5,\scalefactorl*0.5) {};
            \node[thick, circle, label={[label distance=0.4cm]180:$l_3$}] (l3) at (\scalefactorl*-0.5,\scalefactorl*-0.5) {};
            \node[thick, circle, label={[label distance=0.4cm]0:$l_4$}] (l4) at (\scalefactorl*0.5,\scalefactorl*-0.5) {};
        \end{scope}
        
        \begin{scope}
            \draw[color=figurecolourschemewt1, thick] (l1) -- (l3) -- (l2) -- (l4) -- (l1);
            \draw[color=figurecolourschemewt2, very thick] (l1) -- (l2);
            \draw[color=figurecolourschemewt2, very thick] (l3) -- (l4);
        \end{scope}
        
        % draw the edges inside H
        \begin{scope}
            \foreach \i in {2,3,4,5,6,7,8,9,10,11} \draw[color=figurecolourschemewt1_adjusted] let \n1={int(mod(\i+1,10)+2)} in (h\i) -- (h\n1);
            \foreach \i in {2,3,4,5,6,7,8,9,10,11} \draw[color=figurecolourschemewt1_adjusted] let \n1={int(mod(\i+2,10)+2)} in (h\i) -- (h\n1);
            
            % figurecolourschemewt 3
            \foreach \i in {3,5,7,9,11} \draw[color=figurecolourschemewt3] let \n1={int(mod(\i,10) + 2)} in (h\i) -- (h\n1);
            \foreach \i in {2,3,4,5,6} \draw[color=figurecolourschemewt3] let \n1={int(mod(\i,10) + 5)} in (h\i) -- (h\n1);
            
            % figurecolourschemewt 4
            \foreach \i in {2,4,6,8,10} \draw[color=figurecolourschemewt4, thick] let \n1={int(mod(\i,11) + 1)} in (h\i) -- (h\n1);
            
            % figurecolourschemewt 5
            \foreach \i in {3,5,7,9,11} \draw[color=figurecolourschemewt5, thick] let \n1={int(mod(\i,10) + 1)} in (h\i) -- (h\n1);
            
            % figurecolourschemewt 6
            \foreach \i in {2,4,6,8,10} \draw[color=figurecolourschemewt6, thick] let \n1={int(mod(\i,10) + 2)} in (h\i) -- (h\n1);
        \end{scope}
    \end{scope}
\end{scope}
\begin{scope}
\end{scope}
\end{tikzpicture}
    \vspace*{0.5cm}
    \caption[The reduction from PIT to the problem of deciding if an instance of 3DR-AS with symmetric preferences contains a j-envy-free matching]{The reduction from PIT to the problem of deciding if an instance of 3DR-AS with symmetric preferences contains a j-envy-free matching. Valuation colour key: \textcolor{figurecolourschemewt6}{\figurecolorschemewtsixname} - 6, \textcolor{figurecolourschemewt5}{\figurecolorschemewtfivename} - 5, \textcolor{figurecolourschemewt4}{\figurecolorschemewtfourname} - 4, \textcolor{figurecolourschemewt3}{\figurecolorschemewtthreename} - 3, \textcolor{figurecolourschemewt2}{\figurecolorschemewttwoname} - 2, \textcolor{figurecolourschemewt1}{\figurecolorschemewtonename} - 1.}
    \label{fig:threed_efr_as_jef_symmetric_reduction}
\end{figure}

We make some remarks on the design of the constructed instance. Like before, in the reduction presented in Section~\ref{sec:threed_efr_as_jef_ternary}, the design of $H$ is derived from a particular instance that contains no j-envy-free matching. This instance comprises $H$ as well as a single isolated agent $\alpha_{z}$, where $v_{\alpha_{z}}(h_i) = 0$ for each $i$ where $1\leq i \leq 11$. In fact, the proof that this instance contains no j-envy-free matching can be directly derived from the proofs of lemmas appearing later in this section (Lemmas~\ref{lem:threed_efr_as_jef_3332_case_part1}--\ref{lem:threed_efr_as_jef_hopen}). These proofs involve lengthy case analyses, and we leave open the problem of finding a more intuitive or succinct argument (see Section~\ref{sec:threed_efr_as_conclusion} for more discussion on this).

It is straightforward to show that the reduction runs in polynomial time. To prove that the reduction is correct we show that the 3DR-AS instance $(N, V)$ contains a j-envy-free matching if and only if the PIT instance $G$ contains a partition into triangles.

We first show that if the PIT instance $G$ contains a partition into triangles then the 3DR-AS instance $(N, V)$ contains a j-envy-free matching.

\begin{lem}
\label{lem:threed_efr_as_jef_first_direction}
If $G$ contains a partition into triangles then $(N, V)$ contains a j-envy-free matching.
\end{lem}
\begin{proof}
Suppose $G$ contains a partition into triangles $X = \{ X_1, X_2, \dots, X_q \}$. We shall construct a matching $M$ in $(N, V)$ that is j-envy-free. First, add $\{ h_2, h_{10}, h_{11} \}$, $\{ h_5, h_6, h_8 \}$, $\{ h_1, h_9, h_4 \}$, $\{ h_3, l_1, l_2 \}$ and $\{ h_7, l_3, l_4 \}$ to $M$. Next, for each triangle $X_p = \{ w_i, w_j, w_k \}$ in $X$, add $\{ c_i, c_j, c_k \}$ to $M$.

Suppose for a contradiction that some agent $\alpha_j$ exists where $\alpha_j$ has j-envy for some other agent $\alpha_{k_1}$ where $M(\alpha_{k_1}) = \{ \alpha_{k_1}, \alpha_{k_2}, \alpha_{k_3} \}$. Since $N = H \cup L \cup C$ it must be that either $\alpha_{k_1} \in H$, $\alpha_{k_1} \in L$, or $\alpha_{k_1} \in C$. We show that each case leads to a contradiction. It follows that no such $\alpha_j$ exists and thus that $M$ is j-envy-free.
\begin{itemize}
    \item Suppose $\alpha_{k_1} \in H$. Either $\alpha_{k_1} \in \{ h_3, h_7 \}$ or $\alpha_{k_1} \in H \setminus \{ h_3, h_7 \}$.
    \begin{itemize}
        \item Suppose $\alpha_{k_1} \in \{ h_3, h_7 \}$. Then it must be that either $\{ \alpha_{k_2}, \alpha_{k_3} \} = \{ l_1, l_2 \}$ or $\{ \alpha_{k_2}, \alpha_{k_3} \} = \{ l_3, l_4 \}$. Suppose firstly that $\{ \alpha_{k_2}, \alpha_{k_3} \} = \{ l_1, l_2 \}$. We can see immediately that $\alpha_j \notin H$ since otherwise $u_{\alpha_j}(\{ l_1, l_2 \}) = 0$. It must also be that $\alpha_j \notin C$, since  $u_{c_{a}}(\{ l_1, l_2 \}) = 6 = u_{c_a}(M)$ for any $c_a \in C$. Similarly, $u_{l_3}(\{ l_1, l_2 \}) = 2 = u_{l_3}(M)$ and $u_{l_4}(\{ l_1, l_2 \}) = 2 = u_{l_4}(M)$ so $\alpha_j \neq l_3$ and $\alpha_j \neq l_4$. This shows $\alpha_j \notin L$. We have shown that $\alpha_j \notin H$, $\alpha_j \notin C$, and $\alpha_j \notin L$, which is a contradiction. The proof for the case in which $\{ \alpha_{k_2}, \alpha_{k_3} \} = \{ l_3, l_4 \}$ is symmetric and also leads to a contradiction.
        \item Suppose $\alpha_{k_1} \in H\setminus \{ h_3, h_7 \}$. If $\alpha_{k_1} = h_1$ then $\{ \alpha_{k_2}, \alpha_{k_3} \} = \{ h_4, h_9 \}$ so it must be that $v_{h_4}(\alpha_j) > 2 = v_{h_4}(h_1)$ and $v_{h_9}(\alpha_j) > 2 = v_{h_9}(h_1)$, which is impossible by the design of $H$. The proof for every other assignment of $\alpha_{k_1}$ is similar: if $\alpha_{k_1} = h_2$ then $v_{h_{11}}(\alpha_j) > 5$, which is impossible. If $\alpha_{k_1} = h_4$ then $v_{h_1}(\alpha_j) > 2$, which is impossible. If $\alpha_{k_1} = h_5$ then $v_{h_6}(\alpha_j) > 5$ and $v_{h_8}(\alpha_j) > 1$, which is impossible. If $\alpha_{k_1} = h_6$ then $v_{h_8}(\alpha_j) > 6$, which is impossible. If $\alpha_{k_1} = h_8$ then $v_{h_6}(\alpha_j) > 6$, which is impossible. If $\alpha_{k_1} = h_9$ then $v_{h_1}(\alpha_j) > 2$, which is impossible. If $\alpha_{k_1} = h_{10}$ then $v_{h_2}(\alpha_j) > 6$, which is impossible. If $\alpha_{k_1} = h_{11}$ then $v_{h_2}(\alpha_j) > 5$ and $v_{h_{10}}(\alpha_j) > 4$, which is impossible.
    \end{itemize}
    \item Suppose $\alpha_{k_1} \in C$. By construction, it must be that $\alpha_{k_1} = c_{i_1}$, $\alpha_{k_2} = c_{i_2}$, and $\alpha_{k_3} = c_{i_3}$ where $\{ c_{i_1}, c_{i_2}, c_{i_3} \} \subseteq C$, where the corresponding vertices $\{ w_{i_1}, w_{i_2}, w_{i_3} \}$ in $G$ are a triangle. It follows that  $v_{c_{i_2}}(c_{i_1}) = 3$. By assumption, $\alpha_j$ has j-envy for $c_{i_1}$ so it must be that $v_{c_{i_2}}(\alpha_j) > v_{c_{i_2}}(c_{i_1}) = 3$, which is impossible by the design of $C$.
    \item Suppose $\alpha_{k_1} \in L$. It must be that $\alpha_{k_1} = l_{i_1}$ for some $i_1$ where $1\leq i_1 \leq 4$, $\alpha_{k_2} = l_{i_2}$ for some $i_2$ where $1\leq i_2 \leq 4$ and $\alpha_{k_3} = h_{i_3}$ where $i_3 \in \{ 3, 7 \}$. If $\alpha_j \in H$ then $v_{l_{i_2}}(\alpha_j) = 0$ which contradicts the supposition that $\alpha_j$ has j-envy for $l_{i_1}$. Otherwise, if $\alpha_j \notin H$ then $v_{h_{i_3}}(\alpha_j) = 0$, which also contradicts the supposition that $\alpha_j$ has j-envy for $l_{i_1}$.
\end{itemize}
\end{proof}

% \paragraph{Correctness of the reduction: second direction}

We now show that if the 3DR-AS instance $(N, V)$ contains a j-envy-free matching then the PIT instance $G$ contains a partition into triangles.

In the first part of this proof (up to and including Lemma~\ref{lem:threed_efr_as_jef_hopen}) we focus on $H$. To begin, we define two possible configurations of $H$ in an arbitrary j-envy-free matching $M$. If some triple $t\in M$ contains exactly one agent in $H$ then we say that $H$ has an \emph{open configuration in $M$}. Otherwise, we say that $H$ has a \emph{closed configuration in $M$}. We shall eventually show, in Lemma~\ref{lem:threed_efr_as_jef_hopen}, that the only possible configuration of $H$ in $M$ is an open configuration. In Lemmas~\ref{lem:threed_efr_as_jef_two_and_one_in_h}--\ref{lem:threed_efr_as_jef_32222_case} we prove a sequence of intermediary results. 

\begin{lem}
\label{lem:threed_efr_as_jef_two_and_one_in_h}
If $(N, V)$ contains a j-envy-free matching $M$ then no triples $t_1, t_2$ in $M$ exist such that $t_1$ contains exactly two agents in $H$ and $t_2$ contains exactly one agent in $H$.
\end{lem}
\begin{proof}
Suppose for a contradiction that some such $t_1, t_2 \in M$ exist. Suppose $t_1 = \{ h_{i_1}, h_{i_2}, \alpha_{j_1} \}$ and $t_2 = \{ h_{i_3}, \alpha_{j_2}, \alpha_{j_3} \}$ where $1\leq i_1, i_2, i_3 \leq 11$ and $\alpha_{j_1}, \alpha_{j_2}, \alpha_{j_3} \in N \setminus H$. Now $h_{i_3}$ has j-envy for $\alpha_{j_1}$ since $u_{h_{i_3}}(M) = 0 < 2 \leq u_{h_{i_3}}(\{ h_{i_1}, h_{i_2} \})$, $v_{h_{i_1}}(\alpha_{j_1}) = 0 < 1 \leq v_{h_{i_1}}(h_{i_3})$ and $v_{h_{i_2}}(\alpha_{j_1}) = 0 < 1 \leq v_{h_{i_2}}(h_{i_3})$. This contradicts our supposition that $M$ is j-envy-free.
\end{proof}


\begin{lem}
\label{lem:threed_efr_as_jef_3332_case_part1}
If $(N, V)$ contains a j-envy-free matching $M$, $\sigma(H, M) = 4$, and $u_{h_1}(M) < 4$ then $H$ has an open configuration in $M$.
\end{lem}
\begin{proof}
Suppose to the contrary that $\sigma(H, M) = 4$, $u_{h_1}(M) < 4$, and $H$ has a closed configuration in $M$. Since $\sigma(H, M) = 4$ it must be that three triples in $M$ each contain exactly three agents in $H$ and one triple in $M$ contains exactly two agents in $H$. Suppose then that $t_1, t_2, t_3 \in M$ each contain exactly three agents in $H$ and $t_4 \in M$ contains exactly two agents in $H$. Since $u_{h_1}(M) < 4$ by assumption, by the design of $H$ it follows that $M(h_1)$ contains at most one agent in $H \setminus \{ h_1 \}$ and therefore $h_1 \in t_4$. It follows that $t_4 = \{ h_1, h_{i_1}, \alpha_{j} \}$ where $2 \leq i_1 \leq 11$ and $\alpha_{j} \in N \setminus H$. We use a case analysis to prove a contradiction occurs for each possible assignment of $i_1$. 

\begin{itemize}
    \item Suppose first $i_1 = 2$. If $u_{h_4}(M) \leq 7$ then $h_4$ has j-envy for $\alpha_{j}$, since $u_{h_4}(M) \leq 7 < 8 = u_{h_4}(\{ h_1, h_2 \})$, $v_{h_1}(\alpha_{j}) = 0 < 2 = v_{h_1}(h_4)$, and $v_{h_2}(\alpha_{j}) = 0 < 6 = v_{h_2}(h_4)$. It follows that $u_{h_4}(M) \geq 8$. By the assumptions regarding the structure of $M$, it must be that $M(h_4)$ contains three agents in $H$. Since $t_4 = \{ h_1, h_2, \alpha_{j} \}$ it must be that $M(h_4) = \{ h_4, h_{i_2}, h_{i_3} \}$ where $\{ i_2, i_3 \} \subset \{ 3, 5, 6, 7, 8, 9, 10, 11 \}$. Recall that $v_{h_4}(h_3)=5$, $v_{h_4}(h_5)=4$, $v_{h_4}(h_6)=6$, $v_{h_4}(h_7)=1$, $v_{h_4}(h_8)=1$, $v_{h_4}(h_9)=3$, $v_{h_4}(h_{10})=1$, and $v_{h_4}(h_{11})=1$. Since we established $u_{h_4}(M) \geq 8$ it follows that there are 5 possibilities: $\{ i_2, i_3 \} = \{ 3, 9 \}$, $\{ i_2, i_3 \} = \{ 3, 5 \}$, $\{ i_2, i_3 \} = \{ 3, 6 \}$, $\{ i_2, i_3 \} = \{ 5, 6 \}$, and $\{ i_2, i_3 \} = \{ 6, 9 \}$, which we shall now consider.
\begin{itemize}
    \item Suppose $\{ i_2, i_3 \} = \{ 3, 9 \}$. It follows that $h_2$ has j-envy for $h_9$, since $u_{h_2}(M) = 2 < 10 = u_{h_2}(\{ h_3, h_4 \})$, $v_{h_3}(h_9) = 1 < 4 = v_{h_3}(h_2)$, and $v_{h_4}(h_9) = 3 < 6 = v_{h_4}(h_2)$. This contradicts the supposition that $M$ is j-envy-free.
    \item Suppose $\{ i_2, i_3 \} = \{ 3, 5 \}$. It follows that $h_2$ has j-envy for $h_5$, since $u_{h_2}(M) = 2 < 10 = u_{h_2}(\{ h_3, h_4 \})$, $v_{h_3}(h_5) = 3 < 4 = v_{h_3}(h_2)$, and $v_{h_4}(h_5) = 4 < 6 = v_{h_4}(h_2)$. 
    \item Suppose $\{ i_2, i_3 \} = \{ 3, 6 \}$. Consider $h_{11}$. If $u_{h_{11}}(M) \leq 6$ then $h_{11}$ has j-envy for $\alpha_{j}$, since $u_{h_{11}}(M) \leq 6 < 7 = u_{h_{11}}(\{ h_1, h_2 \})$, $v_{h_1}(\alpha_{j}) = 0 < 2 = v_{h_1}(h_{11})$, and $v_{h_2}(\alpha_{j}) = 0 < 5 = v_{h_2}(h_{11})$. It follows that $u_{h_{11}}(M) \geq 7$. We have established that $h_2 \notin M(h_{11})$, $h_3 \notin M(h_{11})$, and $h_6 \notin M(h_{11})$ so, by the design of $H$, it must be that $M(h_{11}) = \{ h_9, h_{10}, h_{11} \}$. Now $h_2$ has j-envy for $h_9$, since $u_{h_2}(M) = 2 < 11 = u_{h_2}(\{ h_{10}, h_{11} \})$, $v_{h_{10}}(h_9) = 5 < 6 = v_{h_{10}}(h_2)$, and $v_{h_{11}}(h_9) = 3 < 5 = v_{h_{11}}(h_2)$.
    \item Suppose $\{ i_2, i_3 \} = \{ 5, 6 \}$. Consider $h_3$. If $u_{h_3}(M) \leq 5$ then $h_3$ has j-envy for $\alpha_{j}$, since $u_{h_3}(M) \leq 5 < 6 = u_{h_3}(\{ h_1, h_2 \})$, $v_{h_1}(\alpha_{j}) = 0 < 2 = v_{h_1}(h_3)$, and $v_{h_2}(\alpha_{j}) = 0 < 4 = v_{h_2}(h_3)$. It follows that $u_{h_3}(M) \geq 6$. We have established that $h_2 \notin M(h_3)$, $h_4 \notin M(h_3)$, and $h_5 \notin M(h_3)$ so, by the design of $H$, it must be that $M(h_3) = \{ h_3, h_8, h_{11} \}$. Now $h_{11}$ has j-envy for $\alpha_{j}$, since $u_{h_{11}}(M) = 4 < 7 = u_{h_{11}}(\{ h_1, h_2 \})$, $v_{h_1}(\alpha_{j}) = 0 < 2 = v_{h_1}(h_{11})$, and $v_{h_2}(\alpha_{j}) = 0 < 5 = v_{h_2}(h_{11})$.
    \item Suppose $\{ i_2, i_3 \} = \{ 6, 9 \}$. Consider $h_{11}$. If $u_{h_{11}}(M) \leq 6$ then $h_{11}$ has j-envy for $\alpha_{j}$, since $u_{h_{11}}(M) \leq 6 < 7 = u_{h_{11}}(\{ h_1, h_2 \})$, $v_{h_1}(\alpha_{j}) = 0 < 2 = v_{h_1}(h_{11})$, and $v_{h_2}(\alpha_{j}) = 0 < 5 = v_{h_2}(h_{11})$. It follows that $u_{h_{11}}(M) \geq 7$. We have established that $h_2 \notin M(h_{11})$, $h_6 \notin M(h_{11})$, and $h_9 \notin M(h_{11})$ so, by the design of $H$, it must be that $M(h_{11}) = \{ h_3, h_{10}, h_{11} \}$. Now $h_2$ has j-envy for $h_{10}$, since $u_{h_2}(M) = 2 < 9 = u_{h_2}(\{ h_3, h_{11} \})$, $v_{h_3}(h_{10}) = 1 < 4 = v_{h_3}(h_2)$, and $v_{h_{11}}(h_{10}) = 4 < 5 = v_{h_{11}}(h_2)$.
\end{itemize}
\item Suppose next $i_1 = 3$. If $u_{h_4}(M) \leq 6$ then $h_4$ has j-envy for $\alpha_{j}$, since $u_{h_4}(M) \leq 6 < 7 = u_{h_4}(\{ h_1, h_3 \})$, $v_{h_1}(\alpha_{j}) = 0 < 2 = v_{h_1}(h_4)$, and $v_{h_3}(\alpha_{j}) = 0 < 5 = v_{h_3}(h_4)$. It follows that $u_{h_4}(M) \geq 7$. Since $M(h_4) \neq t_4$ it must be that $M(h_4) = \{ h_4, h_{i_2}, h_{i_3} \}$ where $\{ i_2, i_3 \} \subset \{ 2, 5, 6, 7, 8, 9, 10, 11 \}$. Recall that $v_{h_4}(h_2)=6$, $v_{h_4}(h_5)=4$, $v_{h_4}(h_6)=6$, $v_{h_4}(h_7)=1$, $v_{h_4}(h_8)=1$, $v_{h_4}(h_9)=3$, $v_{h_4}(h_{10})=1$, and $v_{h_4}(h_{11})=1$. Since we established $u_{h_4}(M) \geq 7$ it follows that there are $14$ possibilities: $\{ i_2, i_3 \} = \{ 2, 5 \}$, $\{ i_2, i_3 \} = \{ 2, 6 \}$, $\{ i_2, i_3 \} = \{ 2, 7 \}$, $\{ i_2, i_3 \} = \{ 2, 8 \}$, $\{ i_2, i_3 \} = \{ 2, 9 \}$, $\{ i_2, i_3 \} = \{ 2, 10 \}$, $\{ i_2, i_3 \} = \{ 2, 11 \}$, $\{ i_2, i_3 \} = \{ 5, 6 \}$, $\{ i_2, i_3 \} = \{ 5, 9 \}$, $\{ i_2, i_3 \} = \{ 6, 7 \}$, $\{ i_2, i_3 \} = \{ 6, 8 \}$, $\{ i_2, i_3 \} = \{ 6, 9 \}$, $\{ i_2, i_3 \} = \{ 6, 10 \}$, and $\{ i_2, i_3 \} = \{ 6, 11 \}$, which we shall now consider.
\begin{itemize}
    \item Suppose $\{ i_2, i_3 \} = \{ 2, 5 \}$. It follows that $h_3$ has j-envy for $h_5$, since $u_{h_3}(M) = 2 < 9 = u_{h_3}(\{ h_2, h_4 \})$, $v_{h_2}(h_5) = 1 < 4 = v_{h_2}(h_3)$, and $v_{h_4}(h_5) = 4 < 5 = v_{h_4}(h_3)$.
    \item Suppose $\{ i_2, i_3 \} = \{ 2, 6 \}$. If $u_{h_5}(M) \leq 4$ then $h_5$ has j-envy for $\alpha_{j}$, since $u_{h_5}(M) \leq 4 < 5 = u_{h_5}(\{ h_1, h_3 \})$, $v_{h_1}(\alpha_{j}) = 0 < 2 = v_{h_1}(h_5)$, and $v_{h_3}(\alpha_{j}) = 0 < 3 = v_{h_3}(h_5)$. It follows that $u_{h_5}(M) \geq 5$. We have established that $h_3 \notin M(h_5)$, $h_4 \notin M(h_5)$, and $h_6 \notin M(h_5)$ so, by the design of $H$, it must be that $M(h_5) = \{ h_5, h_7, h_{10} \}$. It remains that $M(h_{11}) = \{ h_8, h_9, h_{11} \}$. Now $h_{11}$ has j-envy for $\alpha_{j}$, since $u_{h_{11}}(M) = 4 < 5 = u_{h_{11}}(\{ h_1, h_3 \})$, $v_{h_1}(\alpha_{j}) = 0 < 2 = v_{h_1}(h_{11})$, and $v_{h_3}(\alpha_{j}) = 0 < 3 = v_{h_3}(h_{11})$.
    \item Suppose $\{ i_2, i_3 \} = \{ 2, 7 \}$. It follows that $h_3$ has j-envy for $h_7$, since $u_{h_3}(M) = 2 < 9 = u_{h_3}(\{ h_2, h_4 \})$, $v_{h_2}(h_7) = 3 < 4 = v_{h_2}(h_3)$, and $v_{h_4}(h_7) = 1 < 5 = v_{h_4}(h_3)$.
    \item Suppose $\{ i_2, i_3 \} = \{ 2, 8 \}$. It follows that $h_3$ has j-envy for $h_8$, since $u_{h_3}(M) = 2 < 9 = u_{h_3}(\{ h_2, h_4 \})$, $v_{h_2}(h_8) = 1 < 4 = v_{h_2}(h_3)$, and $v_{h_4}(h_8) = 1 < 5 = v_{h_4}(h_3)$.
    \item Suppose $\{ i_2, i_3 \} = \{ 2, 9 \}$. It follows that $h_3$ has j-envy for $h_9$, since $u_{h_3}(M) = 2 < 9 = u_{h_3}(\{ h_2, h_4 \})$, $v_{h_2}(h_9) = 1 < 4 = v_{h_2}(h_3)$, and $v_{h_4}(h_9) = 3 < 5 = v_{h_4}(h_3)$.
    \item Suppose $\{ i_2, i_3 \} = \{ 2, 10 \}$. If $u_{h_{11}}(M) \leq 4$ then $h_{11}$ has j-envy for $\alpha_{j}$, since $u_{h_{11}}(M) \leq 4 < 5 = u_{h_5}(\{ h_1, h_3 \})$, $v_{h_1}(\alpha_{j}) = 0 < 2 = v_{h_1}(h_{11})$, and $v_{h_3}(\alpha_{j}) = 0 < 3 = v_{h_3}(h_{11})$. It follows that $u_{h_{11}}(M) \geq 5$. We have established that $h_2 \notin M(h_{11})$, $h_3 \notin M(h_{11})$, and $h_{10} \notin M(h_{11})$ so, by the design of $H$, it must be that $M(h_{11}) = \{ h_6, h_9, h_{11} \}$. Since $h_5 \notin t_4$, it must be that $M(h_5)$ contains three agents in $H$ and thus $M(h_5) = \{ h_5, h_7, h_8 \}$. Now $h_6$ has j-envy for $h_5$, since $u_{h_6}(M) = 4 < 10 = u_{h_6}(\{ h_7, h_8 \})$, $v_{h_7}(h_5) = 3 < 4 = v_{h_7}(h_6)$, and $v_{h_8}(h_5) = 1 < 6 = v_{h_8}(h_6)$.
    \item Suppose $\{ i_2, i_3 \} = \{ 2, 11 \}$. If $u_{h_5}(M) \leq 4$ then $h_5$ has j-envy for $\alpha_{j}$, since $u_{h_5}(M) \leq 4 < 5 = u_{h_5}(\{ h_1, h_3 \})$, $v_{h_1}(\alpha_{j}) = 0 < 2 = v_{h_1}(h_5)$, and $v_{h_3}(\alpha_{j}) = 0 < 3 = v_{h_3}(h_5)$. It follows that $u_{h_5}(M) \geq 5$. We have established that $h_3 \notin M(h_5)$ and $h_4 \notin M(h_5)$ so, by the design of $H$, there are three possibilities: either $M(h_5) = \{ h_5, h_6, h_7 \}$, $M(h_5) = \{ h_5, h_6, h_{10} \}$, or $M(h_5) = \{ h_5, h_7, h_{10} \}$.
    \begin{itemize}
        \item If $M(h_5) = \{ h_5, h_6, h_7 \}$ then $h_4$ has j-envy for $h_7$, since $u_{h_4}(M) = 7 < 10 = u_{h_4}(\{ h_5, h_6 \})$, $v_{h_5}(h_7) = 3 < 4 = v_{h_5}(h_4)$, and $v_{h_6}(h_7) = 4 < 6 = v_{h_6}(h_4)$.
        \item If $M(h_5) = \{ h_5, h_6, h_{10} \}$ then $h_4$ has j-envy for $h_{10}$, since $u_{h_4}(M) = 7 < 10 = u_{h_4}(\{ h_5, h_6 \})$, $v_{h_5}(h_{10}) = 3 < 4 = v_{h_5}(h_4)$, and $v_{h_6}(h_{10}) = 1 < 6 = v_{h_6}(h_4)$.
        \item If $M(h_5) = \{ h_5, h_7, h_{10} \}$ then it remains that $M(h_6) = \{ h_6, h_8, h_9 \}$. Now $h_6$ has j-envy for $h_{10}$, since $u_{h_6}(M) = 7 < 9 = u_{h_6}(\{ h_5, h_7 \})$, $v_{h_5}(h_{10}) = 3 < 5 = v_{h_5}(h_6)$, and $v_{h_7}(h_{10}) = 1 < 4 = v_{h_7}(h_6)$.
    \end{itemize}
    \item Suppose $\{ i_2, i_3 \} = \{ 5, 6 \}$. If $u_{h_2}(M) \leq 5$ then $h_2$ has j-envy for $\alpha_{j}$, since $u_{h_2}(M) \leq 5 < 6 = u_{h_2}(\{ h_1, h_3 \})$, $v_{h_1}(\alpha_{j}) = 0 < 2 = v_{h_1}(h_2)$, and $v_{h_3}(\alpha_{j}) = 0 < 4 = v_{h_3}(h_2)$. It follows that $u_{h_2}(M) \geq 6$. Since $h_2 \notin t_4$ it must be that $M(h_2)$ contains three agents in $H$. Since $t_4 = \{ h_1, h_3, \alpha_{j} \}$ and $M(h_4) = \{ h_4, h_5, h_6 \}$ it follows that $M(h_2) = \{ h_2, h_{i_4}, h_{i_5} \}$ where $\{ i_4, i_5 \} \subset \{ 7, 8, 9, 10, 11 \}$. Recall that $v_{h_2}(h_7)=3$, $v_{h_2}(h_8)=1$, $v_{h_2}(h_9)=1$, $v_{h_2}(h_{10})=6$, and $v_{h_2}(h_{11})=5$. Since we established $u_{h_2}(M) \geq 6$ it follows that there are $7$ possibilities: $\{ h_{i_4}, h_{i_5} \} = \{ 7, 10 \}$, $\{ h_{i_4}, h_{i_5} \} = \{ 7, 11 \}$, $\{ h_{i_4}, h_{i_5} \} = \{ 8, 10 \}$, $\{ h_{i_4}, h_{i_5} \} = \{ 8, 11 \}$, $\{ h_{i_4}, h_{i_5} \} = \{ 9, 10 \}$, $\{ h_{i_4}, h_{i_5} \} = \{ 9, 11 \}$, and $\{ h_{i_4}, h_{i_5} \} = \{ 10, 11 \}$, which we shall now consider.
    \begin{itemize}
        \item If $\{ h_{i_4}, h_{i_5} \} = \{ 7, 10 \}$ then it remains that $M(h_{11}) = \{ h_8, h_9, h_{11} \}$. Now $h_{11}$ has j-envy for $h_7$, since $u_{h_{11}}(M) = 4 < 9 = u_{h_{11}}(\{ h_2, h_{10} \})$, $v_{h_2}(h_7) = 3 < 5 = v_{h_2}(h_{11})$, and $v_{h_{10}}(h_7) = 1 < 4 = v_{h_{10}}(h_{11})$.
        \item If $\{ h_{i_4}, h_{i_5} \} = \{ 7, 11 \}$ then $h_3$ has j-envy for $h_7$, since $u_{h_3}(M) = 2 < 7 = u_{h_3}(\{ h_2, h_{11} \})$, $v_{h_2}(h_7) = 3 < 4 = v_{h_2}(h_3)$, and $v_{h_{11}}(h_7) = 1 < 3 = v_{h_{11}}(h_3)$.
        \item If $\{ h_{i_4}, h_{i_5} \} = \{ 8, 10 \}$ then it remains that $M(h_7) = \{ h_7, h_9, h_{11} \}$. Now $h_8$ has j-envy for $h_{11}$, since $u_{h_8}(M) = 7 < 9 = u_{h_8}(\{ h_7, h_9 \})$, $v_{h_7}(h_{11}) = 1 < 5 = v_{h_7}(h_8)$, and $v_{h_9}(h_{11}) = 3 < 4 = v_{h_9}(h_8)$.
        \item If $\{ h_{i_4}, h_{i_5} \} = \{ 8, 11 \}$ then $h_3$ has j-envy for $h_8$, since $u_{h_3}(M) = 2 < 7 = u_{h_3}(\{ h_2, h_{11} \})$, $v_{h_2}(h_8) = 1 < 4 = v_{h_2}(h_3)$, and $v_{h_{11}}(h_8) = 1 < 3 = v_{h_{11}}(h_3)$.
        \item If $\{ h_{i_4}, h_{i_5} \} = \{ 9, 10 \}$ then it remains that $M(h_7) = \{ h_7, h_8, h_{11} \}$. Now $h_9$ has j-envy for $h_{11}$, since $u_{h_9}(M) = 6 < 7 = u_{h_9}(\{ h_7, h_8 \})$, $v_{h_7}(h_{11}) = 1 < 3 = v_{h_7}(h_9)$, and $v_{h_8}(h_{11}) = 1 < 4 = v_{h_8}(h_9)$.
        \item If $\{ h_{i_4}, h_{i_5} \} = \{ 9, 11 \}$ then it remains that $M(h_{10}) = \{ h_7, h_8, h_{10} \}$. Now $h_{10}$ has j-envy for $h_9$, since $u_{h_{10}}(M) = 7 < 10 = u_{h_{10}}(\{ h_2, h_{11} \})$, $v_{h_2}(h_9) = 1 < 6 = v_{h_2}(h_{10})$, and $v_{h_{11}}(h_9) = 3 < 4 = v_{h_{11}}(h_{10})$.
        \item If $\{ h_{i_4}, h_{i_5} \} = \{ 10, 11 \}$ then it remains that $M(h_7) = \{ h_7, h_8, h_9 \}$. Now $h_{10}$ has j-envy for $h_7$, since $u_{h_{10}}(M) = 10 < 11 = u_{h_{10}}(\{ h_8, h_9 \})$, $v_{h_8}(h_7) = 5 < 6 = v_{h_8}(h_{10})$, and $v_{h_9}(h_7) = 3 < 5 = v_{h_9}(h_{10})$. 
    \end{itemize}
    \item Suppose $\{ i_2, i_3 \} = \{ 5, 9 \}$. It follows that $h_3$ has j-envy for $h_9$, since $u_{h_3}(M) = 2 < 8 = u_{h_3}(\{ h_4, h_5 \})$, $v_{h_4}(h_9) = 3 < 5 = v_{h_4}(h_3)$, and $v_{h_5}(h_9) = 1 < 3 = v_{h_5}(h_3)$.
    \item Suppose $\{ i_2, i_3 \} = \{ 6, 7 \}$. Consider $h_5$. Since $h_3 \notin M(h_5)$, $h_4 \notin M(h_5)$, $h_6 \notin M(h_5)$, and $h_7 \notin M(h_5)$ by the design of $H$ it must be that $u_{h_5}(M) \leq 4$. It follows that $h_5$ has j-envy for $h_7$, since $u_{h_5}(M) \leq 4 < 9 = u_{h_5}(\{ h_4, h_6 \})$, $v_{h_4}(h_7) = 1 < 4 = v_{h_4}(h_5)$, and $v_{h_6}(h_7) = 4 < 5 = v_{h_6}(h_5)$.
    \item Suppose $\{ i_2, i_3 \} = \{ 6, 8 \}$. Consider $h_5$. If $u_{h_5}(M) \leq 4$ then $h_5$ has j-envy for $\alpha_{j}$, since $u_{h_5}(M) \leq 4 < 5 = u_{h_2}(\{ h_1, h_3 \})$, $v_{h_1}(\alpha_{j}) = 0 < 2 = v_{h_1}(h_5)$, and $v_{h_3}(\alpha_{j}) = 0 < 3 = v_{h_3}(h_5)$. It follows that $u_{h_5}(M) \geq 5$. Since $h_3 \notin M(h_5)$, $h_4 \notin M(h_5)$, and $h_6 \notin M(h_5)$, the only possibility is that $M(h_5) = \{ h_5, h_7, h_{10} \}$. It remains that $M(h_2) = \{ h_2, h_9, h_{11} \}$. Now $h_{10}$ has j-envy for $h_9$, since $u_{h_{10}}(M) = 4 < 10 = u_{h_{10}}(\{ h_2, h_{11} \})$, $v_{h_2}(h_9) = 1 < 6 = v_{h_2}(h_{10})$, and $v_{h_{11}}(h_9) = 3 < 4 = v_{h_{11}}(h_{10})$. 
    \item Suppose $\{ i_2, i_3 \} = \{ 6, 9 \}$. Consider $h_5$. Since $h_3 \notin M(h_5)$, $h_4 \notin M(h_5)$, and $h_6 \notin M(h_5)$ by the design of $H$ it must be that $u_{h_5}(M) \leq 6$. It follows that $h_5$ has j-envy for $h_9$, since $u_{h_5}(M) \leq 6 < 9 = u_{h_5}(\{ h_4, h_6 \})$, $v_{h_4}(h_9) = 3 < 4 = v_{h_4}(h_5)$, and $v_{h_6}(h_9) = 1 < 5 = v_{h_6}(h_5)$.
    \item Suppose $\{ i_2, i_3 \} = \{ 6, 10 \}$. Consider $h_5$. Since $h_3 \notin M(h_5)$, $h_4 \notin M(h_5)$, $h_6 \notin M(h_5)$, and $h_{10} \notin M(h_5)$ by the design of $H$ it must be that $u_{h_5}(M) \leq 4$. It follows that $h_5$ has j-envy for $h_{10}$, since $u_{h_5}(M) \leq 4 < 9 = u_{h_5}(\{ h_4, h_6 \})$, $v_{h_4}(h_{10}) = 1 < 4 = v_{h_4}(h_5)$, and $v_{h_6}(h_{10}) = 1 < 5 = v_{h_6}(h_5)$.
    \item Suppose $\{ i_2, i_3 \} = \{ 6, 11 \}$. Consider $h_5$. Since $h_3 \notin M(h_5)$, $h_4 \notin M(h_5)$, and $h_6 \notin M(h_5)$ by the design of $H$ it must be that $u_{h_5}(M) \leq 6$. It follows that $h_5$ has j-envy for $h_{11}$, since $u_{h_5}(M) \leq 6 < 9 = u_{h_5}(\{ h_4, h_6 \})$, $v_{h_4}(h_{11}) = 1 < 4 = v_{h_4}(h_5)$, and $v_{h_6}(h_{11}) = 3 < 5 = v_{h_6}(h_5)$.
\end{itemize}
\end{itemize}


\end{proof}

\begin{lem}
\label{lem:threed_efr_as_jef_3332_case_part2}
If $(N, V)$ contains a j-envy-free matching $M$, $\sigma(H, M) = 4$, and $u_{h_1}(M) = 4$ then $H$ has an open configuration in $M$.
\end{lem}
\begin{proof}
Suppose to the contrary that $\sigma(H, M) = 4$, $u_{h_1}(M) = 4$, and $H$ has an open configuration in $M$. Since $\sigma(H, M) = 4$ it must be that three triples in $M$ each contain exactly three agents in $H$ and one triple in $M$ contains exactly two agents in $H$. Suppose $t_1, t_2, t_3 \in M$ each contain exactly three agents in $H$ and $t_4 \in M$ contains exactly two agents in $H$. Since $u_{h_1}(M) = 4$, by the design of $H$ it follows that $M(h_1)$ contains two agents in $H \setminus \{ h_1 \}$ and therefore $h_1 \notin t_4$. It follows that $t_4 = \{ h_{i_1}, h_{i_2}, \alpha_{j} \}$ where $2 \leq i_1, i_2 \leq 11$ and $\alpha_{j} \in (N \setminus H)$.  We use a case analysis to prove a contradiction occurs for each possible assignment of $\{ h_{i_1}, h_{i_2} \}$ where $2\leq i_1, i_2 \leq 11$.

As before, in the proof of Lemma~\ref{lem:threed_efr_as_jef_3332_case_part1}, the symmetries of $H$ allow us to shorten the case analysis. Recall that the structure of the valuations between agents $H \setminus \{ h_1 \}$ has five symmetries. It follows that for any possible assignment of $\{ {i_1}, {i_2} \}$ there exist in total five symmetric assignments, where the case analysis for one assignment is symmetric to the case analysis for the other four assignments. Since there are $\binom{10}{2}=45$ possible assignments of $\{ {i_1}, {i_2} \}$ where $2\leq i_1, i_2 \leq 11$ and five symmetries we need only consider the nine assignments $\{ 2, 3 \}$, $\{ 2, 4 \}$, $\{ 2, 5 \}$, $\{ 2, 6 \}$, $\{ 2, 7 \}$, $\{ 3, 4 \}$, $\{ 3, 5 \}$, $\{ 3, 6 \}$, $\{ 3, 7 \}$, of which no two are symmetric.
\begin{itemize}
    \item Suppose $\{ {i_1}, {i_2} \} = \{ 2, 3 \}$. Since $h_2 \notin M(h_4)$ and $h_3 \notin M(h_4)$ it must be that $u_{h_4}(M) \leq 10$. It follows that $h_4$ has j-envy for $\alpha_{j}$ since $u_{h_4}(M) \leq 10 < 12 = u_{h_4}(\{ h_2, h_3 \})$, $v_{h_2}(\alpha_{j}) = 0 < 6 = v_{h_2}(h_4)$, and $v_{h_3}(\alpha_{j}) = 0 < 5 = v_{h_3}(h_4)$. This contradicts the supposition that $M$ is j-envy-free. The following proofs concerning other possible assignments of $\{ {i_1}, {i_2} \}$ are similar in technique although in some cases we must make further deductions about the utilities of agents in $H$.
    \item Suppose $\{ {i_1}, {i_2} \} = \{ 2, 4 \}$. Since $h_2 \notin M(h_3)$ and $h_4 \notin M(h_3)$ it follows that $u_{h_3}(M) \leq 6$. It then follows that $h_3$ has j-envy for $\alpha_{j}$ since $u_{h_3}(M) \leq 6 < 9 = u_{h_3}(\{ h_2, h_4 \})$, $v_{h_2}(\alpha_{j}) = 0 < 4 = v_{h_2}(h_3)$, and $v_{h_4}(\alpha_{j}) = 0 < 5 = v_{h_4}(h_3)$.
    \item Suppose $\{ {i_1}, {i_2} \} = \{ 2, 5 \}$. If $u_{h_4}(M) \leq 9$ then $h_4$ has j-envy for $\alpha_{j}$, since $u_{h_4}(M) \leq 9 < 10 = u_{h_4}(\{ h_2, h_5 \})$, $v_{h_2}(\alpha_{j}) = 0 < 6 = v_{h_2}(h_4)$, and $v_{h_5}(\alpha_{j}) = 0 < 4 = v_{h_5}(h_4)$. It follows that $u_{h_4}(M) \geq 10$. The only possibility is that $M(h_4) = \{ h_3, h_4, h_6 \}$. Now $h_3$ has j-envy for $\alpha_{j}$ since $u_{h_3}(M) = 6 < 7 = u_{h_3}(\{ h_2, h_5 \})$, $v_{h_2}(\alpha_{j}) = 0 < 4 = v_{h_2}(h_3)$, and $v_{h_5}(\alpha_{j}) = 0 < 3 = v_{h_5}(h_3)$.
    \item Suppose $\{ {i_1}, {i_2} \} = \{ 2, 6 \}$. Since $h_2 \notin M(h_4)$ and $h_6 \notin M(h_4)$ it follows that $u_{h_4}(M) \leq 9$. Now $h_4$ has j-envy for $\alpha_{j}$ since $u_{h_4}(M) \leq 9 < 12 = u_{h_4}(\{ h_2, h_6 \})$, $v_{h_2}(\alpha_{j}) = 0 < 6 = v_{h_2}(h_4)$, and $v_{h_6}(\alpha_{j}) = 0 < 6 = v_{h_6}(h_4)$.
    \item Suppose $\{ {i_1}, {i_2} \} = \{ 2, 7 \}$. If $u_{h_4}(M) \leq 6$ then $h_4$ has j-envy for $\alpha_{j}$, since $u_{h_4}(M) \leq 6 < 7 = u_{h_4}(\{ h_2, h_7 \})$, $v_{h_2}(\alpha_{j}) = 0 < 6 = v_{h_2}(h_4)$, and $v_{h_7}(\alpha_{j}) = 0 < 1 = v_{h_7}(h_4)$. It follows that $u_{h_4}(M) \geq 7$. By the design of $H$, it follows that $M(h_4)$ contains three agents in $H$. Since $t_4 = \{ h_2, h_7, \alpha_{j} \}$ it follows that $M(h_4) = \{ h_4, h_{i_3}, h_{i_4} \}$ where $\{ i_3, i_4 \} \subset \{ 1, 3, 5, 6, 8, 9, 10, 11 \}$. Recall that $v_{h_4}(h_1) = 2$, $v_{h_4}(h_3) = 5$, $v_{h_4}(h_5) = 4$, $v_{h_4}(h_6) = 6$, $v_{h_4}(h_8) = 1$, $v_{h_4}(h_9) = 3$, $v_{h_4}(h_{10}) = 1$, and $v_{h_4}(h_{11}) = 1$. Since we established $u_{h_4}(M) \geq 7$ it follows that there are $11$ possibilities: $\{ {i_3}, {i_4} \} = \{ 1, 3 \}$, $\{ {i_3}, {i_4} \} = \{ 1, 6 \}$, $\{ {i_3}, {i_4} \} = \{ 3, 5 \}$, $\{ {i_3}, {i_4} \} = \{ 3, 6 \}$, $\{ {i_3}, {i_4} \} = \{ 3, 9 \}$, $\{ {i_3}, {i_4} \} = \{ 5, 6 \}$, $\{ {i_3}, {i_4} \} = \{ 5, 9 \}$, $\{ {i_3}, {i_4} \} = \{ 6, 8 \}$, $\{ {i_3}, {i_4} \} = \{ 6, 9 \}$, $\{ {i_3}, {i_4} \} = \{ 6, 10 \}$, and $\{ {i_3}, {i_4} \} = \{ 6, 11 \}$, which we shall now consider.
    \begin{itemize}
        \item Suppose $\{ {i_3}, {i_4} \} = \{ 1, 3 \}$. It follows that $h_2$ has j-envy for $h_1$ since $u_{h_2}(M) = 3 < 10 = u_{h_2}(\{ h_3, h_4 \})$, $v_{h_3}(h_1) = 2 < 4 = v_{h_3}(h_2)$, and $v_{h_4}(h_1) = 2 < 6 = v_{h_4}(h_2)$.
        \item Suppose $\{ {i_3}, {i_4} \} = \{ 1, 6 \}$. Consider $h_5$. It must be that $u_{h_5}(M) \leq 6$. Now $h_5$ has j-envy for $h_1$ since $u_{h_5}(M) \leq 6 < 9 = u_{h_5}(\{ h_4, h_6 \})$, $v_{h_4}(h_1) = 2 < 4 = v_{h_4}(h_5)$, and $v_{h_6}(h_1) = 2 < 5 = v_{h_6}(h_5)$.
        \item Suppose $\{ {i_3}, {i_4} \} = \{ 3, 5 \}$. It follows that $h_2$ has j-envy for $h_5$ since $u_{h_2}(M) = 3 < 10 = u_{h_2}(\{ h_3, h_4 \})$, $v_{h_3}(h_5) = 3 < 4 = v_{h_3}(h_2)$, and $v_{h_4}(h_5) = 4 < 6 = v_{h_4}(h_2)$.
        \item Suppose $\{ {i_3}, {i_4} \} = \{ 3, 6 \}$. In this case, consider $M(h_1)$. Since $h_1 \notin t_4$ it follows that $M(h_1)$ contains three agents in $H$. Suppose $M(h_1) = \{ h_1, h_{i_5}, h_{i_6} \}$ where $2 \leq {i_5}, {i_6} \leq 11$. Since we have established $M(h_2) = \{ h_2, h_7, \alpha_{j} \}$ and $M(h_4) = \{ h_3, h_4, h_6 \}$ it follows that $\{ h_{i_5}, h_{i_6} \} \subset \{ 5, 8, 9, 10, 11 \}$. Thus there are $\binom{5}{2}=10$ possible assignments of $\{ h_{i_5}, h_{i_6} \}$, which we shall now consider.
        \begin{itemize}
            \item If $\{ h_{i_5}, h_{i_6} \} = \{ 5, 8 \}$ then $h_7$ has j-envy for $h_1$ since $u_{h_7}(M) = 3 < 8 = u_{h_7}(\{ h_5, h_8 \})$, $v_{h_5}(h_1) = 2 < 3 = v_{h_5}(h_7)$, and $v_{h_8}(h_1) = 2 < 5 = v_{h_8}(h_7)$.
            \item If $\{ h_{i_5}, h_{i_6} \} = \{ 5, 9 \}$ then $h_7$ has j-envy for $h_1$ since $u_{h_7}(M) = 3 < 6 = u_{h_7}(\{ h_5, h_9 \})$, $v_{h_5}(h_1) = 2 < 3 = v_{h_5}(h_7)$, and $v_{h_9}(h_1) = 2 < 3 = v_{h_9}(h_7)$.
            \item If $\{ h_{i_5}, h_{i_6} \} = \{ 5, 10 \}$ then $u_{h_{10}}(M) = 5$. It follows that $h_{10}$ has j-envy for $\alpha_{j}$ since $u_{h_{10}}(M) = 5 < 7 = u_{h_{10}}(\{ h_2, h_7 \})$, $v_{h_2}(\alpha_{j}) = 0 < 6 = v_{h_2}(h_{10})$, and $v_{h_7}(\alpha_{j}) = 0 < 1 = v_{h_7}(h_{10})$.
            \item If $\{ h_{i_5}, h_{i_6} \} = \{ 5, 11 \}$ then $u_{h_{11}}(M) = 3$. It follows that $h_{11}$ has j-envy for $\alpha_{j}$ since $u_{h_{11}}(M) = 3 < 6 = u_{h_{11}}(\{ h_2, h_7 \})$, $v_{h_2}(\alpha_{j}) = 0 < 5 = v_{h_2}(h_{11})$, and $v_{h_7}(\alpha_{j}) = 0 < 1 = v_{h_7}(h_{11})$.
            \item If $\{ h_{i_5}, h_{i_6} \} = \{ 8, 9 \}$ then $h_7$ has j-envy for $h_1$ since $u_{h_7}(M) = 3 < 8 = u_{h_7}(\{ h_8, h_9 \})$, $v_{h_8}(h_1) = 2 < 5 = v_{h_8}(h_7)$, and $v_{h_9}(h_1) = 2 < 3 = v_{h_9}(h_7)$.
            \item If $\{ h_{i_5}, h_{i_6} \} = \{ 8, 10 \}$ then it remains that $M(h_9) = \{ h_5, h_9, h_{11} \}$ and thus $u_{h_9}(M)=u_{\{ h_{5}, h_{11} \}}=4$. It follows that $h_9$ has j-envy for $h_1$ since $u_{h_9}(M) = 4 < 9 = u_{h_9}(\{ h_8, h_{10} \})$, $v_{h_8}(h_1) = 2 < 4 = v_{h_8}(h_9)$, and $v_{h_{10}}(h_1) = 2 < 5 = v_{h_{10}}(h_9)$.
            \item If $\{ h_{i_5}, h_{i_6} \} = \{ 8, 11 \}$ then $u_{h_{11}}(M) = 3$. It follows that $h_{11}$ has j-envy for $\alpha_{j}$ since $u_{h_{11}}(M) = 3 < 6 = u_{h_{11}}(\{ h_2, h_7 \})$, $v_{h_2}(\alpha_{j}) = 0 < 5 = v_{h_2}(h_{11})$, and $v_{h_7}(\alpha_{j}) = 0 < 1 = v_{h_7}(h_{11})$.
            \item If $\{ h_{i_5}, h_{i_6} \} = \{ 9, 10 \}$ then it remains that $M(h_9) = \{ h_5, h_8, h_{11} \}$ and thus $u_{h_{11}}(M)=u_{\{ h_{5}, h_{8} \}}=2$. It follows that $h_{11}$ has j-envy for $\alpha_{j}$ since $u_{h_{11}}(M) = 2 < 6 = u_{h_{11}}(\{ h_2, h_7 \})$, $v_{h_2}(\alpha_{j}) = 0 < 5 = v_{h_2}(h_{11})$, and $v_{h_7}(\alpha_{j}) = 0 < 1 = v_{h_7}(h_{11})$.
            \item If $\{ h_{i_5}, h_{i_6} \} = \{ 9, 11 \}$ then $u_{h_{11}}(M) = 5$. It follows that $h_{11}$ has j-envy for $\alpha_{j}$ since $u_{h_{11}}(M) = 5 < 6 = u_{h_{11}}(\{ h_2, h_7 \})$, $v_{h_2}(\alpha_{j}) = 0 < 5 = v_{h_2}(h_{11})$, and $v_{h_7}(\alpha_{j}) = 0 < 1 = v_{h_7}(h_{11})$.
            \item If $\{ h_{i_5}, h_{i_6} \} = \{ 10, 11 \}$ then $h_2$ has j-envy for $h_1$ since $u_{h_2}(M) = 3 < 11 = u_{h_2}(\{ h_{10}, h_{11} \})$, $v_{h_{10}}(h_1) = 2 < 6 = v_{h_{10}}(h_2)$, and $v_{h_{11}}(h_1) = 2 < 5 = v_{h_{11}}(h_2)$.
        \end{itemize}
        \item Suppose $\{ {i_3}, {i_4} \} = \{ 3, 9 \}$. It follows that $h_2$ has j-envy for $h_9$ since $u_{h_2}(M) = 3 < 10 = u_{h_2}(\{ h_3, h_4 \})$, $v_{h_3}(h_9) = 1 < 4 = v_{h_3}(h_2)$, and $v_{h_4}(h_9) = 3 < 6 = v_{h_4}(h_2)$.
        \item Suppose $\{ {i_3}, {i_4} \} = \{ 5, 6 \}$. Consider $h_3$. If $u_{h_3}(M) \leq 4$ then $h_3$ has j-envy for $\alpha_{j}$, since $u_{h_3}(M) \leq 4 < 5 = u_{h_3}(\{ h_2, h_7 \})$, $v_{h_2}(\alpha_{j}) = 0 < 4 = v_{h_2}(h_3)$, and $v_{h_7}(\alpha_{j}) = 0 < 1 = v_{h_7}(h_3)$. It follows that $u_{h_3}(M) \geq 5$. We have established that $h_4 \notin M(h_3)$, $h_5 \notin M(h_3)$ and $h_2 \notin M(h_3)$ so, by the design of $H$, there are three possibilities: either $M(h_3) = \{ h_1, h_3, h_8 \}$, $M(h_3) = \{ h_1, h_3, h_{11} \}$, or $M(h_3) = \{ h_3, h_8, h_{11} \}$.
        \begin{itemize}
            \item If $M(h_3) = \{ h_1, h_3, h_8 \}$ then $u_{h_8}(M) = 5$. It follows that $h_8$ has j-envy for $\alpha_{j}$, since $u_{h_8}(M) = 5 < 6 = u_{h_8}(\{ h_2, h_7 \})$, $v_{h_2}(\alpha_{j}) = 0 < 1 = v_{h_2}(h_8)$, and $v_{h_7}(\alpha_{j}) = 0 < 5 = v_{h_7}(h_8)$.
            \item If $M(h_3) = \{ h_1, h_3, h_{11} \}$ then $u_{h_{11}}(M) = 5$. It follows that $h_{11}$ has j-envy for $\alpha_{j}$, since $u_{h_{11}}(M) = 5 < 6 = u_{h_{11}}(\{ h_2, h_7 \})$, $v_{h_2}(\alpha_{j}) = 0 < 5 = v_{h_2}(h_{11})$, and $v_{h_7}(\alpha_{j}) = 0 < 1 = v_{h_7}(h_{11})$.
            \item If $M(h_3) = \{ h_3, h_8, h_{11} \}$ then $u_{h_8}(M) = 4$. It follows that $h_8$ has j-envy for $\alpha_{j}$, since $u_{h_8}(M) = 4 < 6 = u_{h_8}(\{ h_2, h_7 \})$, $v_{h_2}(\alpha_{j}) = 0 < 1 = v_{h_2}(h_8)$, and $v_{h_7}(\alpha_{j}) = 0 < 5 = v_{h_7}(h_8)$.
        \end{itemize}
        \item Suppose $\{ {i_3}, {i_4} \} = \{ 5, 9 \}$. Consider $h_6$. It must be that $u_{h_6}(M) \leq 9$. Now $h_6$ has j-envy for $h_9$ since $u_{h_6}(M) \leq 9 < 11 = u_{h_6}(\{ h_4, h_5 \})$, $v_{h_4}(h_9) = 3 < 6 = v_{h_4}(h_6)$, and $v_{h_5}(h_9) = 1 < 5 = v_{h_5}(h_6)$.
        \item Suppose $\{ {i_3}, {i_4} \} = \{ 6, 8 \}$. Consider $h_{10}$. If $u_{h_{10}}(M) \leq 6$ then $h_{10}$ has j-envy for $\alpha_{j}$, since $u_{h_{10}}(M) \leq 6 < 7 = u_{h_{10}}(\{ h_2, h_7 \})$, $v_{h_2}(\alpha_{j}) = 0 < 6 = v_{h_2}(h_{10})$, and $v_{h_7}(\alpha_{j}) = 0 < 1 = v_{h_7}(h_{10})$. It follows that $u_{h_{10}}(M) \geq 7$. We have established that $h_2 \notin M(h_{10})$ and $h_8 \notin M(h_{10})$ so, by the design of $H$, there are four possibilities: either $M(h_{10}) = \{ h_1, h_9, h_{10} \}$, $M(h_{10}) = \{ h_5, h_9, h_{10} \}$, $M(h_{10}) = \{ h_5, h_{10}, h_{11} \}$, or $M(h_3) = \{ h_9, h_{10}, h_{11} \}$.
        \begin{itemize}
            \item If $M(h_{10}) = \{ h_1, h_9, h_{10} \}$ then $h_8$ has j-envy for $h_1$, since $u_{h_8}(M) = 7 < 10 = u_{h_8}(\{ h_9, h_{10} \})$, $v_{h_9}(h_1) = 2 < 4 = v_{h_9}(h_8)$, and $v_{h_{10}}(h_1) = 2 < 6 = v_{h_{10}}(h_8)$.
            \item If $M(h_{10}) = \{ h_5, h_9, h_{10} \}$ then $h_8$ has j-envy for $h_5$, since $u_{h_8}(M) = 7 < 10 = u_{h_8}(\{ h_9, h_{10} \})$, $v_{h_9}(h_5) = 1 < 4 = v_{h_9}(h_8)$, and $v_{h_{10}}(h_5) = 3 < 6 = v_{h_{10}}(h_8)$.
            \item If $M(h_{10}) = \{ h_5, h_{10}, h_{11} \}$ then $u_{h_{11}}(M) = 5$. It follows that $h_{11}$ has j-envy for $\alpha_{j}$, since $u_{h_{11}}(M) = 5 < 6 = u_{h_{11}}(\{ h_2, h_7 \})$, $v_{h_2}(\alpha_{j}) = 0 < 5 = v_{h_2}(h_{11})$, and $v_{h_7}(\alpha_{j}) = 0 < 1 = v_{h_7}(h_{11})$.
            \item If $M(h_3) = \{ h_9, h_{10}, h_{11} \}$ then $h_8$ has j-envy for $h_{11}$, since $u_{h_8}(M) = 7 < 10 = u_{h_8}(\{ h_9, h_{10} \})$, $v_{h_9}(h_{11}) = 3 < 4 = v_{h_9}(h_8)$, and $v_{h_{10}}(h_{11}) = 4 < 6 = v_{h_{10}}(h_8)$.
        \end{itemize}
        \item Suppose $\{ {i_3}, {i_4} \} = \{ 6, 9 \}$. It must be that $u_{h_5}(M) \leq 6$. Now $h_5$ has j-envy for $h_9$ since $u_{h_5}(M) \leq 6 < 9 = u_{h_5}(\{ h_4, h_6 \})$, $v_{h_4}(h_9) = 3 < 4 = v_{h_4}(h_5)$, and $v_{h_6}(h_9) = 1 < 5 = v_{h_6}(h_5)$.
        \item Suppose $\{ {i_3}, {i_4} \} = \{ 6, 10 \}$. It must be that $u_{h_5}(M) \leq 5$. Now $h_5$ has j-envy for $h_{10}$ since $u_{h_5}(M) \leq 5 < 9 = u_{h_5}(\{ h_4, h_6 \})$, $v_{h_4}(h_{10}) = 1 < 4 = v_{h_4}(h_5)$, and $v_{h_6}(h_{10}) = 1 < 5 = v_{h_6}(h_5)$.
        \item Suppose $\{ {i_3}, {i_4} \} = \{ 6, 11 \}$. It must be that $u_{h_5}(M) \leq 6$. Now $h_5$ has j-envy for $h_{11}$ since $u_{h_5}(M) \leq 6 < 9 = u_{h_5}(\{ h_4, h_6 \})$, $v_{h_4}(h_{11}) = 1 < 4 = v_{h_4}(h_5)$, and $v_{h_6}(h_{11}) = 3 < 5 = v_{h_6}(h_5)$.
    \end{itemize}
\item Suppose $\{ {i_1}, {i_2} \} = \{ 3, 4 \}$. If $u_{h_2}(M) \leq 9$ then $h_2$ has j-envy for $\alpha_{j}$, since $u_{h_2}(M) \leq 9 < 10 = u_{h_2}(\{ h_3, h_4 \})$, $v_{h_3}(\alpha_{j}) = 0 < 4 = v_{h_3}(h_2)$, and $v_{h_4}(\alpha_{j}) = 0 < 6 = v_{h_4}(h_2)$. It follows that $u_{h_2}(M) \geq 10$. The only possibility is that $M(h_2) = \{ h_2, h_{10}, h_{11} \}$. Now consider $h_5$. If $u_{h_5}(M) \leq 6$ then $h_5$ has j-envy for $\alpha_{j}$, since $u_{h_5}(M) \leq 6 < 7 = u_{h_5}(\{ h_3, h_4 \})$, $v_{h_3}(\alpha_{j}) = 0 < 3 = v_{h_3}(h_5)$, and $v_{h_4}(\alpha_{j}) = 0 < 4 = v_{h_4}(h_5)$. It follows that $u_{h_5}(M) \geq 7$. Since we have established $M(h_2) = \{ h_2, h_{10}, h_{11} \}$ and $t_4 = \{ h_3, h_4, \alpha_{j} \}$, there are just two possibilities: either $M(h_5) = \{ h_1, h_5, h_6 \}$ or $M(h_5) = \{ h_5, h_6, h_7 \}$.
\begin{itemize}
    \item If $M(h_5) = \{ h_1, h_5, h_6 \}$ then $h_4$ has j-envy for $h_1$ since $u_{h_4}(M) = 5 < 10 = u_{h_4}(\{ h_5, h_6 \})$, $v_{h_5}(h_1) = 2 < 4 = v_{h_5}(h_4)$, and $v_{h_6}(h_1) = 2 < 6 = v_{h_6}(h_4)$.
    \item If $M(h_5) = \{ h_5, h_6, h_7 \}$ then $h_4$ has j-envy for $h_7$ since $u_{h_4}(M) = 5 < 10 = u_{h_4}(\{ h_5, h_6 \})$, $v_{h_5}(h_7) = 3 < 4 = v_{h_5}(h_4)$, and $v_{h_6}(h_7) = 4 < 6 = v_{h_6}(h_4)$.
\end{itemize}
\item Suppose $\{ {i_1}, {i_2} \} = \{ 3, 5 \}$. If $u_{h_4}(M) \leq 8$ then $h_4$ has j-envy for $\alpha_{j}$, since $u_{h_4}(M) \leq 8 < 9 = u_{h_4}(\{ h_3, h_5 \})$, $v_{h_3}(\alpha_{j}) = 0 < 5 = v_{h_3}(h_4)$, and $v_{h_5}(\alpha_{j}) = 0 < 4 = v_{h_5}(h_4)$. It follows that $u_{h_4}(M) \geq 9$. There are three possibilities: either $M(h_4) = \{ h_2, h_4, h_6 \}$, $M(h_4) = \{ h_2, h_4, h_9 \}$, or $M(h_4) = \{ h_4, h_6, h_9 \}$.
\begin{itemize}
    \item Suppose $M(h_4) = \{ h_2, h_4, h_6 \}$. In this case, consider $M(h_1)$. Since $h_1 \notin t_4$ it follows that $M(h_1)$ contains three agents in $H$. Suppose $M(h_1) = \{ h_1, h_{i_5}, h_{i_6} \}$ where $1 \leq {i_5}, {i_6} \leq 11$. Since we have established $M(h_3) = \{ h_3, h_5, \alpha_{j} \}$ and $M(h_2) = \{ h_2, h_4, h_6 \}$ it follows that $\{ h_{i_5}, h_{i_6} \} \subset \{ 7, 8, 9, 10, 11 \}$. Thus there are $\binom{5}{2}=10$ possible assignments of $\{ h_{i_5}, h_{i_6} \}$, which we shall now consider.
    \begin{itemize}
        \item If $\{ h_{i_5}, h_{i_6} \} = \{ 7, 8 \}$ then $h_6$ has j-envy for $h_1$, since $u_{h_6}(M) = 7 < 10 = u_{h_6}(\{ h_7, h_8 \})$, $v_{h_7}(h_1) = 2 < 4 = v_{h_7}(h_6)$, and $v_{h_8}(h_1) = 2 < 6 = v_{h_8}(h_6)$.
        \item If $\{ h_{i_5}, h_{i_6} \} = \{ 7, 9 \}$ then it remains that $M(h_8) = \{ h_8, h_{10}, h_{11} \}$. It follows that $h_8$ has j-envy for $h_1$, since $u_{h_8}(M) = 7 < 9 = u_{h_8}(\{ h_7, h_9 \})$, $v_{h_7}(h_1) = 2 < 5 = v_{h_7}(h_8)$, and $v_{h_9}(h_1) = 2 < 4 = v_{h_9}(h_8)$.
        \item If $\{ h_{i_5}, h_{i_6} \} = \{ 7, 10 \}$ then $h_7$ has j-envy for $\alpha_{j}$, since $u_{h_7}(M) = 3 < 4 = u_{h_7}(\{ h_3, h_5 \})$, $v_{h_3}(\alpha_{j}) = 0 < 1 = v_{h_3}(h_7)$, and $v_{h_5}(\alpha_{j}) = 0 < 3 = v_{h_5}(h_7)$.
        \item If $\{ h_{i_5}, h_{i_6} \} = \{ 7, 11 \}$ then $h_7$ has j-envy for $\alpha_{j}$, since $u_{h_7}(M) = 3 < 4 = u_{h_7}(\{ h_3, h_5 \})$, $v_{h_3}(\alpha_{j}) = 0 < 1 = v_{h_3}(h_7)$, and $v_{h_5}(\alpha_{j}) = 0 < 3 = v_{h_5}(h_7)$.
        \item If $\{ h_{i_5}, h_{i_6} \} = \{ 8, 9 \}$ then it remains that $M(h_7) = \{ h_7, h_{10}, h_{11} \}$. It follows that $h_7$ has j-envy for $h_1$, since $u_{h_7}(M) = 2 < 8 = u_{h_7}(\{ h_8, h_9 \})$, $v_{h_8}(h_1) = 2 < 5 = v_{h_8}(h_7)$, and $v_{h_9}(h_1) = 2 < 3 = v_{h_9}(h_7)$.
        \item If $\{ h_{i_5}, h_{i_6} \} = \{ 8, 10 \}$ then it remains that $M(h_9) = \{ h_7, h_9, h_{11} \}$. It follows that $h_9$ has j-envy for $h_1$, since $u_{h_9}(M) = 6 < 9 = u_{h_9}(\{ h_8, h_{10} \})$, $v_{h_8}(h_1) = 2 < 4 = v_{h_8}(h_9)$, and $v_{h_{10}}(h_1) = 2 < 5 = v_{h_{10}}(h_9)$.
        \item If $\{ h_{i_5}, h_{i_6} \} = \{ 8, 11 \}$ then $h_{11}$ has j-envy for $\alpha_{j}$, since $u_{h_{11}}(M) = 3 < 4 = u_{h_{11}}(\{ h_3, h_5 \})$, $v_{h_3}(\alpha_{j}) = 0 < 3 = v_{h_3}(h_{11})$, and $v_{h_5}(\alpha_{j}) = 0 < 1 = v_{h_5}(h_{11})$.
        \item If $\{ h_{i_5}, h_{i_6} \} = \{ 9, 10 \}$ then it remains that $M(h_{11}) = \{ h_7, h_8, h_{11} \}$. It follows that $h_{11}$ has j-envy for $h_1$, since $u_{h_{11}}(M) = 2 < 7 = u_{h_{11}}(\{ h_9, h_{10} \})$, $v_{h_9}(h_1) = 2 < 3 = v_{h_9}(h_{11})$, and $v_{h_{10}}(h_1) = 2 < 4 = v_{h_{10}}(h_{11})$.
        \item If $\{ h_{i_5}, h_{i_6} \} = \{ 9, 11 \}$ then it remains that $M(h_{10}) = \{ h_7, h_8, h_{10} \}$. It follows that $h_{10}$ has j-envy for $h_1$, since $u_{h_{10}}(M) = 7 < 9 = u_{h_{10}}(\{ h_9, h_{11} \})$, $v_{h_9}(h_1) = 2 < 5 = v_{h_9}(h_{10})$, and $v_{h_{11}}(h_1) = 2 < 4 = v_{h_{11}}(h_{10})$.
        \item If $\{ h_{i_5}, h_{i_6} \} = \{ 10, 11 \}$ then $h_2$ has j-envy for $h_1$, since $u_{h_2}(M) = 7 < 11 = u_{h_2}(\{ h_{10}, h_{11} \})$, $v_{h_{10}}(h_1) = 2 < 6 = v_{h_{10}}(h_2)$, and $v_{h_{11}}(h_1) = 2 < 5 = v_{h_{11}}(h_2)$.
    \end{itemize}
    \item Suppose $M(h_4) = \{ h_2, h_4, h_9 \}$. It follows that $h_3$ has j-envy for $h_9$, since $u_{h_3}(M) = 3 < 9 = u_{h_3}(\{ h_2, h_4 \})$, $v_{h_2}(h_9) = 1 < 4 = v_{h_2}(h_3)$, and $v_{h_4}(h_9) = 3 < 5 = v_{h_4}(h_3)$.
    \item Suppose $M(h_4) = \{ h_4, h_6, h_9 \}$. It follows that $h_5$ has j-envy for $h_9$, since $u_{h_5}(M) = 3 < 9 = u_{h_5}(\{ h_4, h_6 \})$, $v_{h_4}(h_9) = 3 < 4 = v_{h_4}(h_5)$, and $v_{h_6}(h_9) = 1 < 5 = v_{h_6}(h_5)$.
\end{itemize}
\item Suppose $\{ {i_1}, {i_2} \} = \{ 3, 6 \}$. It must be that $u_{h_4}(M) \leq 10$. It follows that $h_4$ has j-envy for $\alpha_{j}$ since $u_{h_4}(M) \leq 10 < 11 = u_{h_4}(\{ h_3, h_6 \})$, $v_{h_3}(\alpha_{j}) = 0 < 5 = v_{h_3}(h_4)$, and $v_{h_6}(\alpha_{j}) = 0 < 6 = v_{h_6}(h_4)$.
\item Suppose $\{ {i_1}, {i_2} \} = \{ 3, 7 \}$. If $u_{h_8}(M) \leq 7$ then $h_8$ has j-envy for $\alpha_{j}$, since $u_{h_8}(M) \leq 7 < 8 = u_{h_8}(\{ h_3, h_7 \})$, $v_{h_3}(\alpha_{j}) = 0 < 3 = v_{h_3}(h_8)$, and $v_{h_7}(\alpha_{j}) = 0 < 5 = v_{h_7}(h_8)$. It follows that $u_{h_8}(M) \geq 8$. By the design of $H$, it follows that $M(h_8)$ contains three agents in $H$. Since $t_4 = \{ h_3, h_7, \alpha_{j} \}$ it must be that $M(h_8) = \{ h_8, h_{i_3}, h_{i_4} \}$ where $\{ i_3, i_4 \} \subset \{ 1, 2, 4, 5, 6, 9, 10, 11 \}$. Recall that $v_{h_8}(h_1) = 2$, $v_{h_8}(h_2) = 1$, $v_{h_8}(h_4) = 1$, $v_{h_8}(h_5) = 1$, $v_{h_8}(h_6) = 6$, $v_{h_8}(h_9) = 4$, $v_{h_8}(h_{10}) = 6$, and $v_{h_8}(h_{11}) = 1$. Since we established $u_{h_8}(M) \geq 8$ it follows that there are five possibilities: $\{ {i_3}, {i_4} \} = \{ 1, 6 \}$, $\{ {i_3}, {i_4} \} = \{ 1, 10 \}$, $\{ {i_3}, {i_4} \} = \{ 6, 9 \}$, $\{ {i_3}, {i_4} \} = \{ 6, 10 \}$, and $\{ {i_3}, {i_4} \} = \{ 9, 10 \}$, which we shall now consider.
\begin{itemize}
    \item Suppose $\{ {i_3}, {i_4} \} = \{ 1, 6 \}$. It follows that $h_7$ has j-envy for $h_1$, since $u_{h_7}(M) = 1 < 9 = u_{h_7}(\{ h_6, h_8 \})$, $v_{h_6}(h_1) = 2 < 4 = v_{h_6}(h_7)$, and $v_{h_8}(h_1) = 2 < 5 = v_{h_8}(h_7)$.
    \item Suppose $\{ {i_3}, {i_4} \} = \{ 1, 10 \}$. Consider $h_9$. It follows that $u_{h_9}(M) \leq 6$. Now $h_9$ has j-envy for $h_1$, since $u_{h_9}(M) \leq 6 < 9 = u_{h_9}(\{ h_8, h_{10} \})$, $v_{h_8}(h_1) = 2 < 4 = v_{h_8}(h_9)$, and $v_{h_{10}}(h_1) = 2 < 5 = v_{h_{10}}(h_9)$.
    \item Suppose $\{ {i_3}, {i_4} \} = \{ 6, 9 \}$. It follows that $h_7$ has j-envy for $h_9$, since $u_{h_7}(M) = 1 < 9 = u_{h_7}(\{ h_6, h_8 \})$, $v_{h_6}(h_9) = 1 < 4 = v_{h_6}(h_7)$, and $v_{h_8}(h_9) = 4 < 5 = v_{h_8}(h_7)$.
    \item Suppose $\{ {i_3}, {i_4} \} = \{ 6, 10 \}$. In this case, consider $M(h_1)$. Since $h_1 \notin t_4$ it follows that $M(h_1)$ contains three agents in $H$. Suppose $M(h_1) = \{ h_1, h_{i_5}, h_{i_6} \}$ where $2 \leq i_5, i_6 \leq 11$. Since we have established $M(h_3) = \{ h_3, h_7, \alpha_{j} \}$ and $M(h_6) = \{ h_6, h_8, h_{10} \}$ it follows that $\{ h_{i_5}, h_{i_6} \} \subset \{ 2, 4, 5, 9, 11 \}$. Thus there are $\binom{5}{2}=10$ possible assignments of $\{ h_{i_5}, h_{i_6} \}$, which we shall now consider.
    \begin{itemize}
        \item If $\{ h_{i_5}, h_{i_6} \} = \{ 2, 4 \}$ then $h_3$ has j-envy for $h_1$, since $u_{h_3}(M) = 1 < 9 = u_{h_3}(\{ h_2, h_4 \})$, $v_{h_2}(h_1) = 2 < 4 = v_{h_2}(h_3)$, and $v_{h_4}(h_1) = 2 < 5 = v_{h_4}(h_3)$.
        \item If $\{ h_{i_5}, h_{i_6} \} = \{ 2, 5 \}$ then $h_3$ has j-envy for $h_1$, since $u_{h_3}(M) = 1 < 7 = u_{h_3}(\{ h_2, h_5 \})$, $v_{h_2}(h_1) = 2 < 4 = v_{h_2}(h_3)$, and $v_{h_5}(h_1) = 2 < 3 = v_{h_5}(h_3)$.
        \item If $\{ h_{i_5}, h_{i_6} \} = \{ 2, 9 \}$ then it remains that $M(h_{11}) = \{ h_4, h_5, h_{11} \}$. Now $h_3$ has j-envy for $h_{11}$, since $u_{h_3}(M) = 1 < 8 = u_{h_3}(\{ h_4, h_5 \})$, $v_{h_4}(h_{11}) = 1 < 5 = v_{h_4}(h_3)$, and $v_{h_5}(h_{11}) = 1 < 3 = v_{h_5}(h_3)$.
        \item If $\{ h_{i_5}, h_{i_6} \} = \{ 2, 11 \}$ then $h_3$ has j-envy for $h_1$, since $u_{h_3}(M) = 1 < 7 = u_{h_3}(\{ h_2, h_{11} \})$, $v_{h_2}(h_1) = 2 < 4 = v_{h_2}(h_3)$, and $v_{h_{11}}(h_1) = 2 < 3 = v_{h_{11}}(h_3)$.
        \item If $\{ h_{i_5}, h_{i_6} \} = \{ 4, 5 \}$ then $h_3$ has j-envy for $h_1$, since $u_{h_3}(M) = 1 < 8 = u_{h_3}(\{ h_5, h_5 \})$, $v_{h_4}(h_1) = 2 < 5 = v_{h_4}(h_3)$, and $v_{h_5}(h_1) = 2 < 3 = v_{h_5}(h_3)$.
        \item If $\{ h_{i_5}, h_{i_6} \} = \{ 4, 9 \}$ then $h_4$ has j-envy for $\alpha_{j}$, since $u_{h_4}(M) = 5 < 6 = u_{h_4}(\{ h_3, h_7 \})$, $v_{h_3}(\alpha_{j}) = 0 < 5 = v_{h_3}(h_4)$, and $v_{h_7}(\alpha_{j}) = 0 < 1 = v_{h_7}(h_4)$.
        \item If $\{ h_{i_5}, h_{i_6} \} = \{ 4, 11 \}$ then $h_3$ has j-envy for $h_1$, since $u_{h_3}(M) = 1 < 8 = u_{h_3}(\{ h_4, h_{11} \})$, $v_{h_4}(h_1) = 2 < 5 = v_{h_4}(h_3)$, and $v_{h_{11}}(h_1) = 2 < 3 = v_{h_{11}}(h_3)$.
        \item If $\{ h_{i_5}, h_{i_6} \} = \{ 5, 9 \}$ then $h_5$ has j-envy for $\alpha_{j}$, since $u_{h_5}(M) = 3 < 6 = u_{h_5}(\{ h_3, h_7 \})$, $v_{h_3}(\alpha_{j}) = 0 < 3 = v_{h_3}(h_5)$, and $v_{h_7}(\alpha_{j}) = 0 < 3 = v_{h_7}(h_5)$.
        \item If $\{ h_{i_5}, h_{i_6} \} = \{ 5, 11 \}$ then $h_3$ has j-envy for $h_1$, since $u_{h_3}(M) = 1 < 6 = u_{h_3}(\{ h_5, h_{11} \})$, $v_{h_5}(h_1) = 2 < 3 = v_{h_5}(h_3)$, and $v_{h_{11}}(h_1) = 2 < 3 = v_{h_{11}}(h_3)$.
        \item If $\{ h_{i_5}, h_{i_6} \} = \{ 9, 11 \}$ then $h_{10}$ has j-envy for $h_1$, since $u_{h_{10}}(M) = 7 < 9 = u_{h_{10}}(\{ h_9, h_{11} \})$, $v_{h_9}(h_1) = 2 < 5 = v_{h_9}(h_{10})$, and $v_{h_{11}}(h_1) = 2 < 4 = v_{h_{11}}(h_{10})$.
    \end{itemize}
    \item Suppose $\{ {i_3}, {i_4} \} = \{ 9, 10 \}$. Consider $h_2$. If $u_{h_2}(M) \leq 6$ then $h_2$ has j-envy for $\alpha_{j}$, since $u_{h_2}(M) \leq 6 < 7 = u_{h_2}(\{ h_3, h_7 \})$, $v_{h_3}(\alpha_{j}) = 0 < 4 = v_{h_3}(h_2)$, and $v_{h_7}(\alpha_{j}) = 0 < 3 = v_{h_7}(h_2)$. It follows that $u_{h_2}(M) \geq 7$. We have established that $h_3 \notin M(h_2)$, $h_7 \notin M(h_2)$, and $h_{10} \notin M(h_2)$ so, by the design of $H$, there are three possibilities: either $M(h_2) = \{ h_1, h_2, h_4 \}$, $M(h_2) = \{ h_1, h_2, h_{11} \}$, or $M(h_2) = \{ h_2, h_4, h_{11} \}$.
    \begin{itemize}
        \item If $M(h_2) = \{ h_1, h_2, h_4 \}$ then $h_3$ has j-envy for $h_1$, since $u_{h_3}(M) = 1 < 9 = u_{h_3}(\{ h_2, h_4 \})$, $v_{h_2}(h_1) = 2 < 4 = v_{h_2}(h_3)$, and $v_{h_4}(h_1) = 2 < 5 = v_{h_4}(h_3)$.
        \item If $M(h_2) = \{ h_1, h_2, h_{11} \}$ then $h_3$ has j-envy for $h_1$, since $u_{h_3}(M) = 1 < 7 = u_{h_3}(\{ h_2, h_{11} \})$, $v_{h_2}(h_1) = 2 < 4 = v_{h_2}(h_3)$, and $v_{h_{11}}(h_1) = 2 < 3 = v_{h_{11}}(h_3)$.
        \item If $M(h_2) = \{ h_2, h_4, h_{11} \}$ then it remains that $M(h_1) = \{ h_1, h_5, h_6 \}$. Now $h_7$ has j-envy for $h_1$, since $u_{h_7}(M) = 1 < 7 = u_{h_7}(\{ h_5, h_6 \})$, $v_{h_5}(h_1) = 2 < 3 = v_{h_5}(h_7)$, and $v_{h_6}(h_1) = 2 < 4 = v_{h_6}(h_7)$.
    \end{itemize}
\end{itemize}
\end{itemize}
\end{proof}

\begin{lem}
\label{lem:threed_efr_as_jef_3332_case}
If $(N, V)$ contains a j-envy-free matching $M$ and $\sigma(H, M) = 4$ then $H$ has an open configuration in $M$.
\end{lem}
\begin{proof}
Suppose $\sigma(H, M) = 4$. Consider $u_{h_1}(M)$. By the design of $H$, it must be that $2 \leq u_{h_1}(M) \leq 4$. If $u_{h_1}(M) < 4$ then Lemma~\ref{lem:threed_efr_as_jef_3332_case_part1} shows that $H$ has an open configuration in $M$. If $u_{h_1}(M) = 4$ then Lemma~\ref{lem:threed_efr_as_jef_3332_case_part2} shows that $H$ has an open configuration in $M$.
\end{proof}

\begin{lem}
\label{lem:threed_efr_as_jef_32222_case}
If $(N, V)$ contains a j-envy-free matching $M$ and $\sigma(H, M) = 5$ then $H$ has an open configuration in $M$.
\end{lem}
\begin{proof}
Suppose, to the contrary, that $\sigma(H, M) = 5$ and $H$ has a closed configuration in $M$. Then it must be that four triples in $M$ each contain exactly two agents in $H$ and one triple in $M$ contains exactly three agents in $H$. Suppose $t_1, t_2, t_3, t_4 \in M$ each contain exactly two agents in $H$ and $t_5 \in M$ contains exactly three agents in $H$, where $t_1 = \{ h_{i_1}, h_{i_2}, \alpha_{j_1} \}$ where $1\leq i_1, i_2 \leq 11$ and $\alpha_{j_1} \in N \setminus H$. We use a case analysis on $v_{h_{i_1}}(h_{i_2})$ to prove a contradiction. Note that by the design of $H$ it must be that $1 \leq v_{h_{i_1}}(h_{i_2}) \leq 6$.
\begin{itemize}
    \item Suppose $v_{h_{i_1}}(h_{i_2}) = 6$. By the symmetry of $H$, assume without loss of generality that $\{ i_1, i_2 \} = \{ 2, 4 \}$. It follows that $u_{h_3}(M) \leq 6$. Now $h_3$ has j-envy for $\alpha_{j_1}$ since $u_{h_3}(M) \leq 6 < 9 = u_{h_3}(\{ h_2, h_4 \})$, $v_{h_2}(\alpha_{j_1}) = 0 < 4 = v_{h_2}(h_3)$, and $v_{h_4}(\alpha_{j_1}) = 0 < 5 = v_{h_4}(h_3)$. This contradicts the supposition that $M$ is j-envy-free. We shall use a similar technique to prove a contradiction when considering the other cases of $v_{h_{i_1}}(h_{i_2})$.
    \item Suppose $v_{h_{i_1}}(h_{i_2}) = 5$. Assume without loss of generality that $\{ i_1, i_2 \} = \{ 3, 4 \}$. If $u_{h_2}(M) \leq 9$ then $h_2$ has j-envy for $\alpha_{j_1}$ since $u_{h_2}(M) \leq 9 < 10 = u_{h_2}(\{ h_3, h_4 \})$, $v_{h_3}(\alpha_{j_1}) = 0 < 4 = v_{h_3}(h_2)$, and $v_{h_4}(\alpha_{j_1}) = 0 < 6 = v_{h_4}(h_2)$. It follows that $u_{h_2}(M) \geq 10$. The only possibility is that $M(h_2) = \{ h_2, h_{10}, h_{11} \}$. It must be that $M(h_2) = t_5$. We now consider $h_6$. Since $h_6 \notin t_5$ it must be that either $h_6 \in t_2$, $h_6 \in t_3$, or $h_6 \in t_4$. Assume without loss of generality that $h_6 \in t_2$ and that $t_2 = \{ h_6, h_{i_3}, \alpha_{j_2} \}$ where $1\leq i_3 \leq 11$ and $\alpha_{j_2} \in N \setminus H$. If $u_{h_6}(M) \leq 6$ then $h_6$ has j-envy for $\alpha_{j_1}$ since $u_{h_6}(M) \leq 6 < 7 = u_{h_6}(\{ h_3, h_4 \})$, $v_{h_3}(\alpha_{j_1}) = 0 < 1 = v_{h_3}(h_6)$, and $v_{h_4}(\alpha_{j_1}) = 0 < 6 = v_{h_4}(h_6)$. It follows that $u_{h_6}(M) \geq 7$.  Since $v_{h_6}(\alpha_{j_2})=0$ it follows that $v_{h_6}(h_{i_3}) = u_{h_6}(M) \geq 7$, which is a contradiction.
    \item Suppose $v_{h_{i_1}}(h_{i_2}) = 4$. Assume without loss of generality that $\{ i_1, i_2 \} = \{ 2, 3 \}$. If $u_{h_4}(M) \leq 10$ then $h_4$ has j-envy for $\alpha_{j_1}$ since $u_{h_4}(M) \leq 10 < 11 = u_{h_4}(\{ h_2, h_3 \})$, $v_{h_2}(\alpha_{j_1}) = 0 < 6 = v_{h_2}(h_4)$, and $v_{h_3}(\alpha_{j_1}) = 0 < 5 = v_{h_3}(h_4)$. It follows that $u_{h_4}(M) \geq 11$ which, since $h_2 \notin M(h_4)$, is impossible. This contradicts our supposition that $v_{h_{i_1}}(h_{i_2}) = 4$.
    \item Suppose $v_{h_{i_1}}(h_{i_2}) = 3$. We assume without loss of generality that either $\{ i_1, i_2 \} = \{ 3, 11 \}$ or $\{ i_1, i_2 \} = \{ 2, 7 \}$.
\begin{itemize}
    \item Suppose $\{ i_1, i_2 \} = \{ 3, 11 \}$. If $u_{h_2}(M) \leq 8$ then $h_2$ has j-envy for $\alpha_{j_1}$ since $u_{h_2}(M) \leq 8 < 9 = u_{h_2}(\{ h_3, h_{11} \})$, $v_{h_3}(\alpha_{j_1}) = 0 < 4 = v_{h_3}(h_2)$, and $v_{h_{11}}(\alpha_{j_1}) = 0 < 5 = v_{h_{11}}(h_2)$. It follows that $u_{h_2}(M) \geq 9$. By the design of $H$, it follows that $M(h_2) = t_5$. Now consider $h_1$. Since $h_1 \notin t_5$ it must be that either $h_1 \in t_2$, $h_1 \in t_3$, or $h_1 \in t_4$. Assume without loss of generality that $h_1 \in t_2$ and that $t_2 = \{ h_1, h_{i_3}, \alpha_{j_2} \}$ where $1\leq i_3 \leq 11$ and $\alpha_{j_2} \in N \setminus H$. Since $v_{h_1}(\alpha_{j_2})=0$ it follows that $u_{h_1}(M) = v_{h_1}(h_{i_3})$. By the design of $H$, it must be that $v_{h_1}(h_{i_3}) = 2$ so $u_{h_1}(M) = 2$. Now $h_1$ has j-envy for $\alpha_{j_1}$, since $u_{h_1}(M) = 2 < 4 = u_{h_1}(\{ h_3, h_{11} \})$, $v_{h_3}(\alpha_{j_1}) = 0 < 2 = v_{h_3}(h_1)$, and $v_{h_{11}}(\alpha_{j_1}) = 0 < 2 = v_{h_{11}}(h_1)$. This is a contradiction. 
    \item Suppose $\{ i_1, i_2 \} = \{ 2, 7 \}$. Consider $h_1$. As before, if $u_{h_1}(M) \leq 3$ then $h_1$ has j-envy for $\alpha_{j_1}$, since $u_{h_1}(M) \leq 3 < 4 = u_{h_1}(\{ h_2, h_7 \})$, $v_{h_2}(\alpha_{j_1}) = 0 < 2 = v_{h_2}(h_1)$, and $v_{h_7}(\alpha_{j_1}) = 0 < 2 = v_{h_7}(h_1)$. It follows that $u_{h_1}(M) \geq 4$. By the design of $H$, it must be that $M(h_1) = \{ h_1, h_{i_3}, h_{i_4} \}$ where $2\leq i_3, i_4 \leq 11$. It follows that $M(h_1) = t_5$. Now consider $h_4$. If $u_{h_4}(M) \leq 6$ then $h_4$ has j-envy for $\alpha_{j_1}$, since $u_{h_4}(M) \leq 6 < 7 = u_{h_1}(\{ h_2, h_7 \})$, $v_{h_2}(\alpha_{j_1}) = 0 < 6 = v_{h_2}(h_4)$, and $v_{h_7}(\alpha_{j_1}) = 0 < 1 = v_{h_7}(h_4)$. It follows that $u_{h_4}(M) \geq 7$ so, similarly, $M(h_4)$ must contain three agents in $H$ and thus $h_4 \in t_5$. Assume without loss of generality that $h_4 = h_{i_3}$ so $t_5 = \{ h_1, h_4, h_{i_4} \}$. Since $u_{h_4}(M) \geq 7$ and $v_{h_4}(h_1)=2$ it must be that $v_{h_4}(h_{i_4}) \geq 5$ and thus that either $i_4 = 3$ or $i_4 = 6$. Consider $h_{10}$. If $u_{h_{10}}(M) \leq 6$ then $h_{10}$ has j-envy for $\alpha_{j_1}$, since $u_{h_{10}}(M) \leq 6 < 7 = u_{h_{10}}(\{ h_2, h_7 \})$, $v_{h_2}(\alpha_{j_1}) = 0 < 6 = v_{h_2}(h_{10})$, and $v_{h_7}(\alpha_{j_1}) = 0 < 1 = v_{h_7}(h_{10})$. It follows that $u_{h_{10}}(M) \geq 7$. Since $h_{10} \notin t_5$ it must be that either $h_{10} \in t_2$, $h_{10} \in t_3$, or $h_{10} \in t_4$. Assume without loss of generality that $h_{10} \in t_2$ and that $t_2 = \{ h_{10}, h_{i_5}, \alpha_{j_2} \}$ where $1\leq i_5 \leq 11$ and $\alpha_{j_2} \in N \setminus H$. Since $v_{h_{10}}(\alpha_{j_2})=0$ it follows that $v_{h_{10}}(h_{i_5}) = u_{h_{10}}(M) \geq 7$, which is a contradiction.
\end{itemize}
\item Suppose $v_{h_{i_1}}(h_{i_2}) = 2$. It follows that either $i_1 = 1$ or $i_2 = 1$. Assume without loss of generality that $i_1 = 1$. Note that $2 \leq i_2 \leq 11$. Consider $h_{i_2}$. Note that since $v_{h_{i_2}}(\alpha_{j_1})=0$ it must be that $u_{h_{i_2}}(M) = v_{h_{i_2}}(h_1) = 2$. By the design of $H$, for each possible assignment of $i_2$, namely $2 \leq i_2 \leq 11$, there exist five agents $h_{i_3}, h_{i_4}, h_{i_5}, h_{i_6}, h_{i_7}$ such that $v_{h_{i_2}}(h_{i_k}) > 2$ for $3 \leq k \leq 7$. A counting argument shows that at least one of these five agents does not belong to $t_5$ and hence must belong to either $t_2$, $t_3$, or $t_4$. Assume without loss of generality that $h_{i_3} \in t_2$ and $t_2 = \{ h_{i_3}, h_{i_8}, \alpha_{j_2} \}$ where $2 \leq i_8 \leq 11$ and $\alpha_{j_2} \in N \setminus H$. Recall that $v_{h_{i_2}}(h_{i_3}) > 2$. By the design of $H$ it follows that $u_{h_{i_2}}(\{ h_{i_3}, h_{i_8} \}) > 3$. Now $h_{i_2}$ has j-envy for $\alpha_{j_2}$ since $u_{h_{i_2}}(M) = 2 < 3 < u_{h_{i_2}}(\{ h_{i_3}, h_{i_8} \})$, $v_{h_{i_3}}(\alpha_{j_2}) = 0 < 1 \leq v_{h_{i_3}}(h_{i_2})$, and $v_{h_{i_8}}(\alpha_{j_2}) = 0 < 1 \leq v_{h_{i_8}}(h_{i_2})$.
\item Suppose $v_{h_{i_1}}(h_{i_2}) = 1$. Without loss of generality we assume that either $\{ i_1, i_2 \} = \{ 2, 5 \}$ or $\{ i_1, i_2 \} = \{ 2, 6 \}$.
\begin{itemize}
    \item Suppose $\{ i_1, i_2 \} = \{ 2, 5 \}$. If $u_{h_{4}}(M) \leq 9$ then $h_{4}$ has j-envy for $\alpha_{j_1}$, since $u_{h_{4}}(M) \leq 9 < 10 = u_{h_{4}}(\{ h_2, h_5 \})$, $v_{h_2}(\alpha_{j_1}) = 0 < 6 = v_{h_2}(h_{4})$, and $v_{h_5}(\alpha_{j_1}) = 0 < 4 = v_{h_5}(h_{4})$. It follows that $u_{h_{4}}(M) \geq 10$. The only possibility is that $M(h_4) = \{ h_3, h_4, h_{6} \}$ and hence $M(h_4) = t_5$. Consider $h_1$. If $u_{h_1}(M) \leq 3$ then $h_1$ has j-envy for $\alpha_{j_1}$, since $u_{h_1}(M) \leq 3 < 4 = u_{h_1}(\{ h_2, h_5 \})$, $v_{h_2}(\alpha_{j_1}) = 0 < 2 = v_{h_2}(h_1)$, and $v_{h_5}(\alpha_{j_1}) = 0 < 2 = v_{h_5}(h_1)$. It follows that $u_{h_1}(M) \geq 4$. By the design of $H$, it follows that $M(h_1)$ must contain three agents in $H$, which is a contradiction since $h_1 \notin t_5$.
    \item Suppose $\{ i_1, i_2 \} = \{ 2, 6 \}$. It follows that $u_{h_4}(M) \leq 9$ so $h_4$ has j-envy for $\alpha_{j_1}$ since $u_{h_4}(M) \leq 9 < 12 = u_{h_4}(\{ h_2, h_6 \})$, $v_{h_2}(\alpha_{j_1}) = 0 < 6 = v_{h_2}(h_4)$, and $v_{h_6}(\alpha_{j_1}) = 0 < 6 = v_{h_6}(h_4)$.
\end{itemize}
\end{itemize}
\end{proof}

\begin{lem}
\label{lem:threed_efr_as_jef_hopen}
If $(N, V)$ contains a j-envy-free matching $M$ then $H$ has an open configuration in $M$.
\end{lem}
\begin{proof}
By definition, $4 \leq \sigma(H, M) \leq 11$. If $\sigma(H, M) \leq 5$ then $H$ has an open configuration in $M$, by Lemmas~\ref{lem:threed_efr_as_jef_3332_case} and~\ref{lem:threed_efr_as_jef_32222_case}. If $6 \leq \sigma(H, M) \leq 11$ then, by a counting argument, at least one triple in $M$ must contain exactly one agent in $H$. In other words, $H$ has an open configuration in $M$.
\end{proof}

We have shown, in Lemma~\ref{lem:threed_efr_as_jef_hopen}, that if $(N, V)$ contains a j-envy-free matching $M$ then $H$ has an open configuration in $M$. By definition, some triple $t_{\beta}$ in $M$ contains exactly one agent in $H$. Since $|H|=11$, if $t_{\beta}$ is the only triple in $M$ to contain exactly one agent in $H$ then there must exist some triple in $M$ that contains exactly two agents in $H$. By Lemma~\ref{lem:threed_efr_as_jef_two_and_one_in_h}, this is a contradiction. It follows that at least two triples in $M$ exist that each contain exactly one agent in $H$. Suppose $t_{\beta}, t_{\gamma} \in M$ are two such triples and $t_{\beta} = \{ h_{a_1}, \alpha_{b_1}, \alpha_{b_2} \}$ and $t_{\gamma} = \{ h_{a_2}, \alpha_{b_3}, \alpha_{b_4} \}$.

\begin{lem}
\label{lem:threed_efr_as_jef_lequalsthefourisolated}
If $(N, V)$ contains a j-envy-free matching then $\{ \alpha_{b_1}, \alpha_{b_2}, \alpha_{b_3}, \alpha_{b_4} \} = L$.
\end{lem}
\begin{proof}
Suppose for a contradiction that $\{ \alpha_{b_1}, \alpha_{b_2}, \alpha_{b_3}, \alpha_{b_4} \} \neq L$.

By definition, $\{ \alpha_{b_1}, \alpha_{b_2}, \alpha_{b_3}, \alpha_{b_4} \} \cap H = \varnothing$ and $\{ \alpha_{b_1}, \alpha_{b_2}, \alpha_{b_3}, \alpha_{b_4} \} \neq L$ it must be that at least one agent in $\{ \alpha_{b_1}, \alpha_{b_2}, \alpha_{b_3}, \alpha_{b_4} \}$ belongs to $C$. Assume without loss of generality that $\alpha_{b_1} \in C$.

We have already shown that $t_{\beta}$ contains exactly one agent in $H$. Since $v_{\alpha_{b_1}}(h_{a_1}) = 0$, by the design of the instance it must be that $u_{\alpha_{b_1}}(M) = v_{\alpha_{b_1}}(\alpha_{b_2}) \leq 3$. By the design of the instance $v_{\alpha_{b_1}}(\alpha_{b_3}) \geq 2$ and $v_{\alpha_{b_1}}(\alpha_{b_4}) \geq 2$ so $u_{\alpha_{b_1}}(\{ \alpha_{b_3}, \alpha_{b_4} \}) \geq 4$. Now $\alpha_{b_1}$ has j-envy for $h_{a_2}$ since $u_{\alpha_{b_1}}(M) \leq 3 < 4 \leq u_{\alpha_{b_1}}(\{ \alpha_{b_3}, \alpha_{b_4} \})$, $v_{\alpha_{b_3}}(h_{a_2}) = 0 < 2 \leq v_{\alpha_{b_3}}(\alpha_{b_1})$, and $v_{\alpha_{b_4}}(h_{a_2}) = 0 < 2 \leq v_{\alpha_{b_4}}(\alpha_{b_1})$.
\end{proof}

\begin{lem}
\label{lem:threed_efr_as_jef_structureofL}
If $(N, V)$ contains a j-envy-free matching then $\{ \{ \alpha_{b_1}, \alpha_{b_2} \}, \{ \alpha_{b_3}, \alpha_{b_4} \} \} = \{ \{ l_1, l_2 \}, \{ l_3, l_4 \} \}$.
\end{lem}
\begin{proof}
By Lemma~\ref{lem:threed_efr_as_jef_lequalsthefourisolated}, $\{ \alpha_{b_1}, \alpha_{b_2}, \alpha_{b_3}, \alpha_{b_4} \} = L$. There are now three possibilities: first that $\{ \{ \alpha_{b_1}, \alpha_{b_2} \}, \{ \alpha_{b_3}, \alpha_{b_4} \} \} = \{ \{ l_1, l_3 \}, \{ l_2, l_4 \} \}$, second that $\{ \{ \alpha_{b_1}, \alpha_{b_2} \}, \{ \alpha_{b_3}, \alpha_{b_4} \} \} = \{ \{ l_1, l_4 \}, \{ l_2, l_3 \} \}$, and third that $\{ \{ \alpha_{b_1}, \alpha_{b_2} \}, \{ \alpha_{b_3}, \alpha_{b_4} \} \} = \{ \{ l_1, l_2 \}, \{ l_3, l_4 \} \}$.

First suppose $\{ \{ \alpha_{b_1}, \alpha_{b_2} \}, \{ \alpha_{b_3}, \alpha_{b_4} \} \} = \{ \{ l_1, l_3 \}, \{ l_2, l_4 \} \}$. Now $l_1$ has j-envy for $h_{a_2}$ since $u_{l_1}(\{ h_{a_1}, l_3 \}) = 1 < 3 \leq u_{l_1}(\{ l_2, l_4 \})$, $v_{l_2}(h_{a_2}) = 0 < 2 = v_{l_2}(l_1)$, and $v_{l_4}(h_{a_2}) = 0 < 1 \leq v_{l_4}(l_1)$.

Second suppose $\{ \{ \alpha_{b_1}, \alpha_{b_2} \}, \{ \alpha_{b_3}, \alpha_{b_4} \} \} = \{ \{ l_1, l_4 \}, \{ l_2, l_3 \} \}$. As before, $l_1$ has j-envy for $h_{a_2}$ since $u_{l_1}(\{ h_{a_1}, l_4 \}) = 1 < 3 \leq u_{l_1}(\{ l_2, l_3 \})$, $v_{l_2}(h_{a_2}) = 0 < 2 = v_{l_2}(l_1)$, and $v_{l_3}(h_{a_2}) = 0 < 1 \leq v_{l_3}(l_1)$.

It remains that $\{ \{ \alpha_{b_1}, \alpha_{b_2} \}, \{ \alpha_{b_3}, \alpha_{b_4} \} \} = \{ \{ l_1, l_2 \}, \{ l_3, l_4 \} \}$.
\end{proof}

By Lemma~\ref{lem:threed_efr_as_jef_structureofL}, either $\{ \alpha_{b_1}, \alpha_{b_2} \} = \{ l_1, l_2 \}$ or $\{ \alpha_{b_1}, \alpha_{b_2} \} = \{ l_3, l_4 \}$. Without loss of generality assume that $\{ \alpha_{b_1}, \alpha_{b_2} \} = \{ l_1, l_2 \}$.

\begin{lem}
\label{lem:threed_efr_as_jef_eachpigets6}
If $(N, V)$ contains a j-envy-free matching then $u_{c_i}(M) = 6$ for each $i$ where $1\leq i \leq 3q$.
\end{lem}
\begin{proof}
Suppose to the contrary that some $1\leq i \leq 3q$ exists where $u_{c_i}(M) < 6$. Then $c_i$ has j-envy for $h_{a_1}$ since $u_{c_i}(M) \leq 5 < 6 \leq u_{c_i}(\{ l_1, l_2 \})$, $v_{l_1}(h_{a_1}) = 0 < 3 = v_{l_1}(c_i)$, and $v_{l_2}(h_{a_1}) = 0 < 3 = v_{l_2}(c_i)$. This contradicts our supposition that $M$ is j-envy-free.
\end{proof}

\begin{lem}
\label{lem:threed_efr_as_jef_second_direction}
If $(N, V)$ contains a j-envy-free matching then $G$ contains a partition into triangles.
\end{lem}
\begin{proof}
Suppose $(N, V)$ contains a j-envy-free partition into triangles $M$. Lemma~\ref{lem:threed_efr_as_jef_eachpigets6} shows that $u_{c_i}(M) = 6$ for each $i$ where $1\leq i \leq 3q$. By construction, it follows that $M(c_i)$ contains two agents $c_j, c_k$ such that $v_{c_i}(c_j) = v_{c_i}(c_k) = 3$. By construction, $c_j$ and $c_k$ therefore correspond to vertices $w_j, w_k \in W$ where $\{ w_i, w_j \} \in E$ and $\{ w_i, w_k \} \in E$. It follows thus that there are exactly $q$ triples in $M$ each containing three agents $\{ c_i, c_j, c_k \}$, where the three corresponding vertices $w_i, w_j, w_k$ are pairwise adjacent in $G$. From these triples a partition into triangles $X$ can be easily constructed.
\end{proof}

% \paragraph{Correctness of the reduction: conclusion}

We have now shown that the 3DR-AS instance $(N, V)$ contains a j-envy-free matching if and only if the PIT instance $G$ contains a partition into triangles. This shows that the reduction is correct.

\begin{thm}
\label{thm:threed_efr_as_jef_symmetric_6_npcomplete}
Deciding if a given instance of 3DR-AS contains a j-envy-free matching is $\NP$-complete, even when preferences are symmetric and the maximum possible valuation is~$6$.
\end{thm}
\begin{proof}
It is straightforward to show that this decision problem belongs to $\NP$, since for any two agents $\alpha_i, \alpha_j \in N$ we can test if $\alpha_i$ j-envies $\alpha_j$ in constant time. 

We have presented a polynomial-time reduction from Partition Into Triangles (PIT, Problem~\ref{prob:pit}), which is $\NP$-complete \cite{GJ79}. Given a graph $G$, the reduction constructs an instance $(N, V)$ of 3DR-AS with symmetric preferences in which the maximum valuation is $6$. Lemmas~\ref{lem:threed_efr_as_jef_first_direction} and~\ref{lem:threed_efr_as_jef_second_direction} show that $(N, V)$ contains a j-envy-free matching if and only if $G$ contains a partition into triangles and thus that this decision problem is $\NP$-hard.
\end{proof}