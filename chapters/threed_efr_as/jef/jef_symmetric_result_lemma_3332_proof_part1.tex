Suppose to the contrary that $\sigma(H, M) = 4$, $u_{h_1}(M) < 4$, and $H$ has a closed configuration in $M$. Since $\sigma(H, M) = 4$ it must be that three triples in $M$ each contain exactly three agents in $H$ and one triple in $M$ contains exactly two agents in $H$. Suppose then that $t_1, t_2, t_3 \in M$ each contain exactly three agents in $H$ and $t_4 \in M$ contains exactly two agents in $H$. Since $u_{h_1}(M) < 4$ by assumption, by the design of $H$ it follows that $M(h_1)$ contains at most one agent in $H \setminus \{ h_1 \}$ and therefore $h_1 \in t_4$. It follows that $t_4 = \{ h_1, h_{i_1}, \alpha_{j} \}$ where $2 \leq i_1 \leq 11$ and $\alpha_{j} \in N \setminus H$. We use a case analysis to prove a contradiction occurs for each possible assignment of $i_1$. 

\begin{itemize}
    \item Suppose first $i_1 = 2$. If $u_{h_4}(M) \leq 7$ then $h_4$ has j-envy for $\alpha_{j}$, since $u_{h_4}(M) \leq 7 < 8 = u_{h_4}(\{ h_1, h_2 \})$, $v_{h_1}(\alpha_{j}) = 0 < 2 = v_{h_1}(h_4)$, and $v_{h_2}(\alpha_{j}) = 0 < 6 = v_{h_2}(h_4)$. It follows that $u_{h_4}(M) \geq 8$. By the assumptions regarding the structure of $M$, it must be that $M(h_4)$ contains three agents in $H$. Since $t_4 = \{ h_1, h_2, \alpha_{j} \}$ it must be that $M(h_4) = \{ h_4, h_{i_2}, h_{i_3} \}$ where $\{ i_2, i_3 \} \subset \{ 3, 5, 6, 7, 8, 9, 10, 11 \}$. Recall that $v_{h_4}(h_3)=5$, $v_{h_4}(h_5)=4$, $v_{h_4}(h_6)=6$, $v_{h_4}(h_7)=1$, $v_{h_4}(h_8)=1$, $v_{h_4}(h_9)=3$, $v_{h_4}(h_{10})=1$, and $v_{h_4}(h_{11})=1$. Since we established $u_{h_4}(M) \geq 8$ it follows that there are 5 possibilities: $\{ i_2, i_3 \} = \{ 3, 9 \}$, $\{ i_2, i_3 \} = \{ 3, 5 \}$, $\{ i_2, i_3 \} = \{ 3, 6 \}$, $\{ i_2, i_3 \} = \{ 5, 6 \}$, and $\{ i_2, i_3 \} = \{ 6, 9 \}$, which we shall now consider.
\begin{itemize}
    \item Suppose $\{ i_2, i_3 \} = \{ 3, 9 \}$. It follows that $h_2$ has j-envy for $h_9$, since $u_{h_2}(M) = 2 < 10 = u_{h_2}(\{ h_3, h_4 \})$, $v_{h_3}(h_9) = 1 < 4 = v_{h_3}(h_2)$, and $v_{h_4}(h_9) = 3 < 6 = v_{h_4}(h_2)$. This contradicts the supposition that $M$ is j-envy-free.
    \item Suppose $\{ i_2, i_3 \} = \{ 3, 5 \}$. It follows that $h_2$ has j-envy for $h_5$, since $u_{h_2}(M) = 2 < 10 = u_{h_2}(\{ h_3, h_4 \})$, $v_{h_3}(h_5) = 3 < 4 = v_{h_3}(h_2)$, and $v_{h_4}(h_5) = 4 < 6 = v_{h_4}(h_2)$. 
    \item Suppose $\{ i_2, i_3 \} = \{ 3, 6 \}$. Consider $h_{11}$. If $u_{h_{11}}(M) \leq 6$ then $h_{11}$ has j-envy for $\alpha_{j}$, since $u_{h_{11}}(M) \leq 6 < 7 = u_{h_{11}}(\{ h_1, h_2 \})$, $v_{h_1}(\alpha_{j}) = 0 < 2 = v_{h_1}(h_{11})$, and $v_{h_2}(\alpha_{j}) = 0 < 5 = v_{h_2}(h_{11})$. It follows that $u_{h_{11}}(M) \geq 7$. We have established that $h_2 \notin M(h_{11})$, $h_3 \notin M(h_{11})$, and $h_6 \notin M(h_{11})$ so, by the design of $H$, it must be that $M(h_{11}) = \{ h_9, h_{10}, h_{11} \}$. Now $h_2$ has j-envy for $h_9$, since $u_{h_2}(M) = 2 < 11 = u_{h_2}(\{ h_{10}, h_{11} \})$, $v_{h_{10}}(h_9) = 5 < 6 = v_{h_{10}}(h_2)$, and $v_{h_{11}}(h_9) = 3 < 5 = v_{h_{11}}(h_2)$.
    \item Suppose $\{ i_2, i_3 \} = \{ 5, 6 \}$. Consider $h_3$. If $u_{h_3}(M) \leq 5$ then $h_3$ has j-envy for $\alpha_{j}$, since $u_{h_3}(M) \leq 5 < 6 = u_{h_3}(\{ h_1, h_2 \})$, $v_{h_1}(\alpha_{j}) = 0 < 2 = v_{h_1}(h_3)$, and $v_{h_2}(\alpha_{j}) = 0 < 4 = v_{h_2}(h_3)$. It follows that $u_{h_3}(M) \geq 6$. We have established that $h_2 \notin M(h_3)$, $h_4 \notin M(h_3)$, and $h_5 \notin M(h_3)$ so, by the design of $H$, it must be that $M(h_3) = \{ h_3, h_8, h_{11} \}$. Now $h_{11}$ has j-envy for $\alpha_{j}$, since $u_{h_{11}}(M) = 4 < 7 = u_{h_{11}}(\{ h_1, h_2 \})$, $v_{h_1}(\alpha_{j}) = 0 < 2 = v_{h_1}(h_{11})$, and $v_{h_2}(\alpha_{j}) = 0 < 5 = v_{h_2}(h_{11})$.
    \item Suppose $\{ i_2, i_3 \} = \{ 6, 9 \}$. Consider $h_{11}$. If $u_{h_{11}}(M) \leq 6$ then $h_{11}$ has j-envy for $\alpha_{j}$, since $u_{h_{11}}(M) \leq 6 < 7 = u_{h_{11}}(\{ h_1, h_2 \})$, $v_{h_1}(\alpha_{j}) = 0 < 2 = v_{h_1}(h_{11})$, and $v_{h_2}(\alpha_{j}) = 0 < 5 = v_{h_2}(h_{11})$. It follows that $u_{h_{11}}(M) \geq 7$. We have established that $h_2 \notin M(h_{11})$, $h_6 \notin M(h_{11})$, and $h_9 \notin M(h_{11})$ so, by the design of $H$, it must be that $M(h_{11}) = \{ h_3, h_{10}, h_{11} \}$. Now $h_2$ has j-envy for $h_{10}$, since $u_{h_2}(M) = 2 < 9 = u_{h_2}(\{ h_3, h_{11} \})$, $v_{h_3}(h_{10}) = 1 < 4 = v_{h_3}(h_2)$, and $v_{h_{11}}(h_{10}) = 4 < 5 = v_{h_{11}}(h_2)$.
\end{itemize}
\item Suppose next $i_1 = 3$. If $u_{h_4}(M) \leq 6$ then $h_4$ has j-envy for $\alpha_{j}$, since $u_{h_4}(M) \leq 6 < 7 = u_{h_4}(\{ h_1, h_3 \})$, $v_{h_1}(\alpha_{j}) = 0 < 2 = v_{h_1}(h_4)$, and $v_{h_3}(\alpha_{j}) = 0 < 5 = v_{h_3}(h_4)$. It follows that $u_{h_4}(M) \geq 7$. Since $M(h_4) \neq t_4$ it must be that $M(h_4) = \{ h_4, h_{i_2}, h_{i_3} \}$ where $\{ i_2, i_3 \} \subset \{ 2, 5, 6, 7, 8, 9, 10, 11 \}$. Recall that $v_{h_4}(h_2)=6$, $v_{h_4}(h_5)=4$, $v_{h_4}(h_6)=6$, $v_{h_4}(h_7)=1$, $v_{h_4}(h_8)=1$, $v_{h_4}(h_9)=3$, $v_{h_4}(h_{10})=1$, and $v_{h_4}(h_{11})=1$. Since we established $u_{h_4}(M) \geq 7$ it follows that there are $14$ possibilities: $\{ i_2, i_3 \} = \{ 2, 5 \}$, $\{ i_2, i_3 \} = \{ 2, 6 \}$, $\{ i_2, i_3 \} = \{ 2, 7 \}$, $\{ i_2, i_3 \} = \{ 2, 8 \}$, $\{ i_2, i_3 \} = \{ 2, 9 \}$, $\{ i_2, i_3 \} = \{ 2, 10 \}$, $\{ i_2, i_3 \} = \{ 2, 11 \}$, $\{ i_2, i_3 \} = \{ 5, 6 \}$, $\{ i_2, i_3 \} = \{ 5, 9 \}$, $\{ i_2, i_3 \} = \{ 6, 7 \}$, $\{ i_2, i_3 \} = \{ 6, 8 \}$, $\{ i_2, i_3 \} = \{ 6, 9 \}$, $\{ i_2, i_3 \} = \{ 6, 10 \}$, and $\{ i_2, i_3 \} = \{ 6, 11 \}$, which we shall now consider.
\begin{itemize}
    \item Suppose $\{ i_2, i_3 \} = \{ 2, 5 \}$. It follows that $h_3$ has j-envy for $h_5$, since $u_{h_3}(M) = 2 < 9 = u_{h_3}(\{ h_2, h_4 \})$, $v_{h_2}(h_5) = 1 < 4 = v_{h_2}(h_3)$, and $v_{h_4}(h_5) = 4 < 5 = v_{h_4}(h_3)$.
    \item Suppose $\{ i_2, i_3 \} = \{ 2, 6 \}$. If $u_{h_5}(M) \leq 4$ then $h_5$ has j-envy for $\alpha_{j}$, since $u_{h_5}(M) \leq 4 < 5 = u_{h_5}(\{ h_1, h_3 \})$, $v_{h_1}(\alpha_{j}) = 0 < 2 = v_{h_1}(h_5)$, and $v_{h_3}(\alpha_{j}) = 0 < 3 = v_{h_3}(h_5)$. It follows that $u_{h_5}(M) \geq 5$. We have established that $h_3 \notin M(h_5)$, $h_4 \notin M(h_5)$, and $h_6 \notin M(h_5)$ so, by the design of $H$, it must be that $M(h_5) = \{ h_5, h_7, h_{10} \}$. It remains that $M(h_{11}) = \{ h_8, h_9, h_{11} \}$. Now $h_{11}$ has j-envy for $\alpha_{j}$, since $u_{h_{11}}(M) = 4 < 5 = u_{h_{11}}(\{ h_1, h_3 \})$, $v_{h_1}(\alpha_{j}) = 0 < 2 = v_{h_1}(h_{11})$, and $v_{h_3}(\alpha_{j}) = 0 < 3 = v_{h_3}(h_{11})$.
    \item Suppose $\{ i_2, i_3 \} = \{ 2, 7 \}$. It follows that $h_3$ has j-envy for $h_7$, since $u_{h_3}(M) = 2 < 9 = u_{h_3}(\{ h_2, h_4 \})$, $v_{h_2}(h_7) = 3 < 4 = v_{h_2}(h_3)$, and $v_{h_4}(h_7) = 1 < 5 = v_{h_4}(h_3)$.
    \item Suppose $\{ i_2, i_3 \} = \{ 2, 8 \}$. It follows that $h_3$ has j-envy for $h_8$, since $u_{h_3}(M) = 2 < 9 = u_{h_3}(\{ h_2, h_4 \})$, $v_{h_2}(h_8) = 1 < 4 = v_{h_2}(h_3)$, and $v_{h_4}(h_8) = 1 < 5 = v_{h_4}(h_3)$.
    \item Suppose $\{ i_2, i_3 \} = \{ 2, 9 \}$. It follows that $h_3$ has j-envy for $h_9$, since $u_{h_3}(M) = 2 < 9 = u_{h_3}(\{ h_2, h_4 \})$, $v_{h_2}(h_9) = 1 < 4 = v_{h_2}(h_3)$, and $v_{h_4}(h_9) = 3 < 5 = v_{h_4}(h_3)$.
    \item Suppose $\{ i_2, i_3 \} = \{ 2, 10 \}$. If $u_{h_{11}}(M) \leq 4$ then $h_{11}$ has j-envy for $\alpha_{j}$, since $u_{h_{11}}(M) \leq 4 < 5 = u_{h_5}(\{ h_1, h_3 \})$, $v_{h_1}(\alpha_{j}) = 0 < 2 = v_{h_1}(h_{11})$, and $v_{h_3}(\alpha_{j}) = 0 < 3 = v_{h_3}(h_{11})$. It follows that $u_{h_{11}}(M) \geq 5$. We have established that $h_2 \notin M(h_{11})$, $h_3 \notin M(h_{11})$, and $h_{10} \notin M(h_{11})$ so, by the design of $H$, it must be that $M(h_{11}) = \{ h_6, h_9, h_{11} \}$. Since $h_5 \notin t_4$, it must be that $M(h_5)$ contains three agents in $H$ and thus $M(h_5) = \{ h_5, h_7, h_8 \}$. Now $h_6$ has j-envy for $h_5$, since $u_{h_6}(M) = 4 < 10 = u_{h_6}(\{ h_7, h_8 \})$, $v_{h_7}(h_5) = 3 < 4 = v_{h_7}(h_6)$, and $v_{h_8}(h_5) = 1 < 6 = v_{h_8}(h_6)$.
    \item Suppose $\{ i_2, i_3 \} = \{ 2, 11 \}$. If $u_{h_5}(M) \leq 4$ then $h_5$ has j-envy for $\alpha_{j}$, since $u_{h_5}(M) \leq 4 < 5 = u_{h_5}(\{ h_1, h_3 \})$, $v_{h_1}(\alpha_{j}) = 0 < 2 = v_{h_1}(h_5)$, and $v_{h_3}(\alpha_{j}) = 0 < 3 = v_{h_3}(h_5)$. It follows that $u_{h_5}(M) \geq 5$. We have established that $h_3 \notin M(h_5)$ and $h_4 \notin M(h_5)$ so, by the design of $H$, there are three possibilities: either $M(h_5) = \{ h_5, h_6, h_7 \}$, $M(h_5) = \{ h_5, h_6, h_{10} \}$, or $M(h_5) = \{ h_5, h_7, h_{10} \}$.
    \begin{itemize}
        \item If $M(h_5) = \{ h_5, h_6, h_7 \}$ then $h_4$ has j-envy for $h_7$, since $u_{h_4}(M) = 7 < 10 = u_{h_4}(\{ h_5, h_6 \})$, $v_{h_5}(h_7) = 3 < 4 = v_{h_5}(h_4)$, and $v_{h_6}(h_7) = 4 < 6 = v_{h_6}(h_4)$.
        \item If $M(h_5) = \{ h_5, h_6, h_{10} \}$ then $h_4$ has j-envy for $h_{10}$, since $u_{h_4}(M) = 7 < 10 = u_{h_4}(\{ h_5, h_6 \})$, $v_{h_5}(h_{10}) = 3 < 4 = v_{h_5}(h_4)$, and $v_{h_6}(h_{10}) = 1 < 6 = v_{h_6}(h_4)$.
        \item If $M(h_5) = \{ h_5, h_7, h_{10} \}$ then it remains that $M(h_6) = \{ h_6, h_8, h_9 \}$. Now $h_6$ has j-envy for $h_{10}$, since $u_{h_6}(M) = 7 < 9 = u_{h_6}(\{ h_5, h_7 \})$, $v_{h_5}(h_{10}) = 3 < 5 = v_{h_5}(h_6)$, and $v_{h_7}(h_{10}) = 1 < 4 = v_{h_7}(h_6)$.
    \end{itemize}
    \item Suppose $\{ i_2, i_3 \} = \{ 5, 6 \}$. If $u_{h_2}(M) \leq 5$ then $h_2$ has j-envy for $\alpha_{j}$, since $u_{h_2}(M) \leq 5 < 6 = u_{h_2}(\{ h_1, h_3 \})$, $v_{h_1}(\alpha_{j}) = 0 < 2 = v_{h_1}(h_2)$, and $v_{h_3}(\alpha_{j}) = 0 < 4 = v_{h_3}(h_2)$. It follows that $u_{h_2}(M) \geq 6$. Since $h_2 \notin t_4$ it must be that $M(h_2)$ contains three agents in $H$. Since $t_4 = \{ h_1, h_3, \alpha_{j} \}$ and $M(h_4) = \{ h_4, h_5, h_6 \}$ it follows that $M(h_2) = \{ h_2, h_{i_4}, h_{i_5} \}$ where $\{ i_4, i_5 \} \subset \{ 7, 8, 9, 10, 11 \}$. Recall that $v_{h_2}(h_7)=3$, $v_{h_2}(h_8)=1$, $v_{h_2}(h_9)=1$, $v_{h_2}(h_{10})=6$, and $v_{h_2}(h_{11})=5$. Since we established $u_{h_2}(M) \geq 6$ it follows that there are $7$ possibilities: $\{ h_{i_4}, h_{i_5} \} = \{ 7, 10 \}$, $\{ h_{i_4}, h_{i_5} \} = \{ 7, 11 \}$, $\{ h_{i_4}, h_{i_5} \} = \{ 8, 10 \}$, $\{ h_{i_4}, h_{i_5} \} = \{ 8, 11 \}$, $\{ h_{i_4}, h_{i_5} \} = \{ 9, 10 \}$, $\{ h_{i_4}, h_{i_5} \} = \{ 9, 11 \}$, and $\{ h_{i_4}, h_{i_5} \} = \{ 10, 11 \}$, which we shall now consider.
    \begin{itemize}
        \item If $\{ h_{i_4}, h_{i_5} \} = \{ 7, 10 \}$ then it remains that $M(h_{11}) = \{ h_8, h_9, h_{11} \}$. Now $h_{11}$ has j-envy for $h_7$, since $u_{h_{11}}(M) = 4 < 9 = u_{h_{11}}(\{ h_2, h_{10} \})$, $v_{h_2}(h_7) = 3 < 5 = v_{h_2}(h_{11})$, and $v_{h_{10}}(h_7) = 1 < 4 = v_{h_{10}}(h_{11})$.
        \item If $\{ h_{i_4}, h_{i_5} \} = \{ 7, 11 \}$ then $h_3$ has j-envy for $h_7$, since $u_{h_3}(M) = 2 < 7 = u_{h_3}(\{ h_2, h_{11} \})$, $v_{h_2}(h_7) = 3 < 4 = v_{h_2}(h_3)$, and $v_{h_{11}}(h_7) = 1 < 3 = v_{h_{11}}(h_3)$.
        \item If $\{ h_{i_4}, h_{i_5} \} = \{ 8, 10 \}$ then it remains that $M(h_7) = \{ h_7, h_9, h_{11} \}$. Now $h_8$ has j-envy for $h_{11}$, since $u_{h_8}(M) = 7 < 9 = u_{h_8}(\{ h_7, h_9 \})$, $v_{h_7}(h_{11}) = 1 < 5 = v_{h_7}(h_8)$, and $v_{h_9}(h_{11}) = 3 < 4 = v_{h_9}(h_8)$.
        \item If $\{ h_{i_4}, h_{i_5} \} = \{ 8, 11 \}$ then $h_3$ has j-envy for $h_8$, since $u_{h_3}(M) = 2 < 7 = u_{h_3}(\{ h_2, h_{11} \})$, $v_{h_2}(h_8) = 1 < 4 = v_{h_2}(h_3)$, and $v_{h_{11}}(h_8) = 1 < 3 = v_{h_{11}}(h_3)$.
        \item If $\{ h_{i_4}, h_{i_5} \} = \{ 9, 10 \}$ then it remains that $M(h_7) = \{ h_7, h_8, h_{11} \}$. Now $h_9$ has j-envy for $h_{11}$, since $u_{h_9}(M) = 6 < 7 = u_{h_9}(\{ h_7, h_8 \})$, $v_{h_7}(h_{11}) = 1 < 3 = v_{h_7}(h_9)$, and $v_{h_8}(h_{11}) = 1 < 4 = v_{h_8}(h_9)$.
        \item If $\{ h_{i_4}, h_{i_5} \} = \{ 9, 11 \}$ then it remains that $M(h_{10}) = \{ h_7, h_8, h_{10} \}$. Now $h_{10}$ has j-envy for $h_9$, since $u_{h_{10}}(M) = 7 < 10 = u_{h_{10}}(\{ h_2, h_{11} \})$, $v_{h_2}(h_9) = 1 < 6 = v_{h_2}(h_{10})$, and $v_{h_{11}}(h_9) = 3 < 4 = v_{h_{11}}(h_{10})$.
        \item If $\{ h_{i_4}, h_{i_5} \} = \{ 10, 11 \}$ then it remains that $M(h_7) = \{ h_7, h_8, h_9 \}$. Now $h_{10}$ has j-envy for $h_7$, since $u_{h_{10}}(M) = 10 < 11 = u_{h_{10}}(\{ h_8, h_9 \})$, $v_{h_8}(h_7) = 5 < 6 = v_{h_8}(h_{10})$, and $v_{h_9}(h_7) = 3 < 5 = v_{h_9}(h_{10})$. 
    \end{itemize}
    \item Suppose $\{ i_2, i_3 \} = \{ 5, 9 \}$. It follows that $h_3$ has j-envy for $h_9$, since $u_{h_3}(M) = 2 < 8 = u_{h_3}(\{ h_4, h_5 \})$, $v_{h_4}(h_9) = 3 < 5 = v_{h_4}(h_3)$, and $v_{h_5}(h_9) = 1 < 3 = v_{h_5}(h_3)$.
    \item Suppose $\{ i_2, i_3 \} = \{ 6, 7 \}$. Consider $h_5$. Since $h_3 \notin M(h_5)$, $h_4 \notin M(h_5)$, $h_6 \notin M(h_5)$, and $h_7 \notin M(h_5)$ by the design of $H$ it must be that $u_{h_5}(M) \leq 4$. It follows that $h_5$ has j-envy for $h_7$, since $u_{h_5}(M) \leq 4 < 9 = u_{h_5}(\{ h_4, h_6 \})$, $v_{h_4}(h_7) = 1 < 4 = v_{h_4}(h_5)$, and $v_{h_6}(h_7) = 4 < 5 = v_{h_6}(h_5)$.
    \item Suppose $\{ i_2, i_3 \} = \{ 6, 8 \}$. Consider $h_5$. If $u_{h_5}(M) \leq 4$ then $h_5$ has j-envy for $\alpha_{j}$, since $u_{h_5}(M) \leq 4 < 5 = u_{h_2}(\{ h_1, h_3 \})$, $v_{h_1}(\alpha_{j}) = 0 < 2 = v_{h_1}(h_5)$, and $v_{h_3}(\alpha_{j}) = 0 < 3 = v_{h_3}(h_5)$. It follows that $u_{h_5}(M) \geq 5$. Since $h_3 \notin M(h_5)$, $h_4 \notin M(h_5)$, and $h_6 \notin M(h_5)$, the only possibility is that $M(h_5) = \{ h_5, h_7, h_{10} \}$. It remains that $M(h_2) = \{ h_2, h_9, h_{11} \}$. Now $h_{10}$ has j-envy for $h_9$, since $u_{h_{10}}(M) = 4 < 10 = u_{h_{10}}(\{ h_2, h_{11} \})$, $v_{h_2}(h_9) = 1 < 6 = v_{h_2}(h_{10})$, and $v_{h_{11}}(h_9) = 3 < 4 = v_{h_{11}}(h_{10})$. 
    \item Suppose $\{ i_2, i_3 \} = \{ 6, 9 \}$. Consider $h_5$. Since $h_3 \notin M(h_5)$, $h_4 \notin M(h_5)$, and $h_6 \notin M(h_5)$ by the design of $H$ it must be that $u_{h_5}(M) \leq 6$. It follows that $h_5$ has j-envy for $h_9$, since $u_{h_5}(M) \leq 6 < 9 = u_{h_5}(\{ h_4, h_6 \})$, $v_{h_4}(h_9) = 3 < 4 = v_{h_4}(h_5)$, and $v_{h_6}(h_9) = 1 < 5 = v_{h_6}(h_5)$.
    \item Suppose $\{ i_2, i_3 \} = \{ 6, 10 \}$. Consider $h_5$. Since $h_3 \notin M(h_5)$, $h_4 \notin M(h_5)$, $h_6 \notin M(h_5)$, and $h_{10} \notin M(h_5)$ by the design of $H$ it must be that $u_{h_5}(M) \leq 4$. It follows that $h_5$ has j-envy for $h_{10}$, since $u_{h_5}(M) \leq 4 < 9 = u_{h_5}(\{ h_4, h_6 \})$, $v_{h_4}(h_{10}) = 1 < 4 = v_{h_4}(h_5)$, and $v_{h_6}(h_{10}) = 1 < 5 = v_{h_6}(h_5)$.
    \item Suppose $\{ i_2, i_3 \} = \{ 6, 11 \}$. Consider $h_5$. Since $h_3 \notin M(h_5)$, $h_4 \notin M(h_5)$, and $h_6 \notin M(h_5)$ by the design of $H$ it must be that $u_{h_5}(M) \leq 6$. It follows that $h_5$ has j-envy for $h_{11}$, since $u_{h_5}(M) \leq 6 < 9 = u_{h_5}(\{ h_4, h_6 \})$, $v_{h_4}(h_{11}) = 1 < 4 = v_{h_4}(h_5)$, and $v_{h_6}(h_{11}) = 3 < 5 = v_{h_6}(h_5)$.
\end{itemize}
\end{itemize}

