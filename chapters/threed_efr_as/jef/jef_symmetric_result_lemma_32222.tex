Suppose, to the contrary, that $\sigma(H, M) = 5$ and $H$ has a closed configuration in $M$. Then it must be that four triples in $M$ each contain exactly two agents in $H$ and one triple in $M$ contains exactly three agents in $H$. Suppose $t_1, t_2, t_3, t_4 \in M$ each contain exactly two agents in $H$ and $t_5 \in M$ contains exactly three agents in $H$, where $t_1 = \{ h_{i_1}, h_{i_2}, \alpha_{j_1} \}$ where $1\leq i_1, i_2 \leq 11$ and $\alpha_{j_1} \in N \setminus H$. We use a case analysis on $v_{h_{i_1}}(h_{i_2})$ to prove a contradiction. Note that by the design of $H$ it must be that $1 \leq v_{h_{i_1}}(h_{i_2}) \leq 6$.
\begin{itemize}
    \item Suppose $v_{h_{i_1}}(h_{i_2}) = 6$. By the symmetry of $H$, assume without loss of generality that $\{ i_1, i_2 \} = \{ 2, 4 \}$. It follows that $u_{h_3}(M) \leq 6$. Now $h_3$ has j-envy for $\alpha_{j_1}$ since $u_{h_3}(M) \leq 6 < 9 = u_{h_3}(\{ h_2, h_4 \})$, $v_{h_2}(\alpha_{j_1}) = 0 < 4 = v_{h_2}(h_3)$, and $v_{h_4}(\alpha_{j_1}) = 0 < 5 = v_{h_4}(h_3)$. This contradicts the supposition that $M$ is j-envy-free. We shall use a similar technique to prove a contradiction when considering the other cases of $v_{h_{i_1}}(h_{i_2})$.
    \item Suppose $v_{h_{i_1}}(h_{i_2}) = 5$. Assume without loss of generality that $\{ i_1, i_2 \} = \{ 3, 4 \}$. If $u_{h_2}(M) \leq 9$ then $h_2$ has j-envy for $\alpha_{j_1}$ since $u_{h_2}(M) \leq 9 < 10 = u_{h_2}(\{ h_3, h_4 \})$, $v_{h_3}(\alpha_{j_1}) = 0 < 4 = v_{h_3}(h_2)$, and $v_{h_4}(\alpha_{j_1}) = 0 < 6 = v_{h_4}(h_2)$. It follows that $u_{h_2}(M) \geq 10$. The only possibility is that $M(h_2) = \{ h_2, h_{10}, h_{11} \}$. It must be that $M(h_2) = t_5$. We now consider $h_6$. Since $h_6 \notin t_5$ it must be that either $h_6 \in t_2$, $h_6 \in t_3$, or $h_6 \in t_4$. Assume without loss of generality that $h_6 \in t_2$ and that $t_2 = \{ h_6, h_{i_3}, \alpha_{j_2} \}$ where $1\leq i_3 \leq 11$ and $\alpha_{j_2} \in N \setminus H$. If $u_{h_6}(M) \leq 6$ then $h_6$ has j-envy for $\alpha_{j_1}$ since $u_{h_6}(M) \leq 6 < 7 = u_{h_6}(\{ h_3, h_4 \})$, $v_{h_3}(\alpha_{j_1}) = 0 < 1 = v_{h_3}(h_6)$, and $v_{h_4}(\alpha_{j_1}) = 0 < 6 = v_{h_4}(h_6)$. It follows that $u_{h_6}(M) \geq 7$.  Since $v_{h_6}(\alpha_{j_2})=0$ it follows that $v_{h_6}(h_{i_3}) = u_{h_6}(M) \geq 7$, which is a contradiction.
    \item Suppose $v_{h_{i_1}}(h_{i_2}) = 4$. Assume without loss of generality that $\{ i_1, i_2 \} = \{ 2, 3 \}$. If $u_{h_4}(M) \leq 10$ then $h_4$ has j-envy for $\alpha_{j_1}$ since $u_{h_4}(M) \leq 10 < 11 = u_{h_4}(\{ h_2, h_3 \})$, $v_{h_2}(\alpha_{j_1}) = 0 < 6 = v_{h_2}(h_4)$, and $v_{h_3}(\alpha_{j_1}) = 0 < 5 = v_{h_3}(h_4)$. It follows that $u_{h_4}(M) \geq 11$ which, since $h_2 \notin M(h_4)$, is impossible. This contradicts our supposition that $v_{h_{i_1}}(h_{i_2}) = 4$.
    \item Suppose $v_{h_{i_1}}(h_{i_2}) = 3$. We assume without loss of generality that either $\{ i_1, i_2 \} = \{ 3, 11 \}$ or $\{ i_1, i_2 \} = \{ 2, 7 \}$.
\begin{itemize}
    \item Suppose $\{ i_1, i_2 \} = \{ 3, 11 \}$. If $u_{h_2}(M) \leq 8$ then $h_2$ has j-envy for $\alpha_{j_1}$ since $u_{h_2}(M) \leq 8 < 9 = u_{h_2}(\{ h_3, h_{11} \})$, $v_{h_3}(\alpha_{j_1}) = 0 < 4 = v_{h_3}(h_2)$, and $v_{h_{11}}(\alpha_{j_1}) = 0 < 5 = v_{h_{11}}(h_2)$. It follows that $u_{h_2}(M) \geq 9$. By the design of $H$, it follows that $M(h_2) = t_5$. Now consider $h_1$. Since $h_1 \notin t_5$ it must be that either $h_1 \in t_2$, $h_1 \in t_3$, or $h_1 \in t_4$. Assume without loss of generality that $h_1 \in t_2$ and that $t_2 = \{ h_1, h_{i_3}, \alpha_{j_2} \}$ where $1\leq i_3 \leq 11$ and $\alpha_{j_2} \in N \setminus H$. Since $v_{h_1}(\alpha_{j_2})=0$ it follows that $u_{h_1}(M) = v_{h_1}(h_{i_3})$. By the design of $H$, it must be that $v_{h_1}(h_{i_3}) = 2$ so $u_{h_1}(M) = 2$. Now $h_1$ has j-envy for $\alpha_{j_1}$, since $u_{h_1}(M) = 2 < 4 = u_{h_1}(\{ h_3, h_{11} \})$, $v_{h_3}(\alpha_{j_1}) = 0 < 2 = v_{h_3}(h_1)$, and $v_{h_{11}}(\alpha_{j_1}) = 0 < 2 = v_{h_{11}}(h_1)$. This is a contradiction. 
    \item Suppose $\{ i_1, i_2 \} = \{ 2, 7 \}$. Consider $h_1$. As before, if $u_{h_1}(M) \leq 3$ then $h_1$ has j-envy for $\alpha_{j_1}$, since $u_{h_1}(M) \leq 3 < 4 = u_{h_1}(\{ h_2, h_7 \})$, $v_{h_2}(\alpha_{j_1}) = 0 < 2 = v_{h_2}(h_1)$, and $v_{h_7}(\alpha_{j_1}) = 0 < 2 = v_{h_7}(h_1)$. It follows that $u_{h_1}(M) \geq 4$. By the design of $H$, it must be that $M(h_1) = \{ h_1, h_{i_3}, h_{i_4} \}$ where $2\leq i_3, i_4 \leq 11$. It follows that $M(h_1) = t_5$. Now consider $h_4$. If $u_{h_4}(M) \leq 6$ then $h_4$ has j-envy for $\alpha_{j_1}$, since $u_{h_4}(M) \leq 6 < 7 = u_{h_1}(\{ h_2, h_7 \})$, $v_{h_2}(\alpha_{j_1}) = 0 < 6 = v_{h_2}(h_4)$, and $v_{h_7}(\alpha_{j_1}) = 0 < 1 = v_{h_7}(h_4)$. It follows that $u_{h_4}(M) \geq 7$ so, similarly, $M(h_4)$ must contain three agents in $H$ and thus $h_4 \in t_5$. Assume without loss of generality that $h_4 = h_{i_3}$ so $t_5 = \{ h_1, h_4, h_{i_4} \}$. Since $u_{h_4}(M) \geq 7$ and $v_{h_4}(h_1)=2$ it must be that $v_{h_4}(h_{i_4}) \geq 5$ and thus that either $i_4 = 3$ or $i_4 = 6$. Consider $h_{10}$. If $u_{h_{10}}(M) \leq 6$ then $h_{10}$ has j-envy for $\alpha_{j_1}$, since $u_{h_{10}}(M) \leq 6 < 7 = u_{h_{10}}(\{ h_2, h_7 \})$, $v_{h_2}(\alpha_{j_1}) = 0 < 6 = v_{h_2}(h_{10})$, and $v_{h_7}(\alpha_{j_1}) = 0 < 1 = v_{h_7}(h_{10})$. It follows that $u_{h_{10}}(M) \geq 7$. Since $h_{10} \notin t_5$ it must be that either $h_{10} \in t_2$, $h_{10} \in t_3$, or $h_{10} \in t_4$. Assume without loss of generality that $h_{10} \in t_2$ and that $t_2 = \{ h_{10}, h_{i_5}, \alpha_{j_2} \}$ where $1\leq i_5 \leq 11$ and $\alpha_{j_2} \in N \setminus H$. Since $v_{h_{10}}(\alpha_{j_2})=0$ it follows that $v_{h_{10}}(h_{i_5}) = u_{h_{10}}(M) \geq 7$, which is a contradiction.
\end{itemize}
\item Suppose $v_{h_{i_1}}(h_{i_2}) = 2$. It follows that either $i_1 = 1$ or $i_2 = 1$. Assume without loss of generality that $i_1 = 1$. Note that $2 \leq i_2 \leq 11$. Consider $h_{i_2}$. Note that since $v_{h_{i_2}}(\alpha_{j_1})=0$ it must be that $u_{h_{i_2}}(M) = v_{h_{i_2}}(h_1) = 2$. By the design of $H$, for each possible assignment of $i_2$, namely $2 \leq i_2 \leq 11$, there exist five agents $h_{i_3}, h_{i_4}, h_{i_5}, h_{i_6}, h_{i_7}$ such that $v_{h_{i_2}}(h_{i_k}) > 2$ for $3 \leq k \leq 7$. A counting argument shows that at least one of these five agents does not belong to $t_5$ and hence must belong to either $t_2$, $t_3$, or $t_4$. Assume without loss of generality that $h_{i_3} \in t_2$ and $t_2 = \{ h_{i_3}, h_{i_8}, \alpha_{j_2} \}$ where $2 \leq i_8 \leq 11$ and $\alpha_{j_2} \in N \setminus H$. Recall that $v_{h_{i_2}}(h_{i_3}) > 2$. By the design of $H$ it follows that $u_{h_{i_2}}(\{ h_{i_3}, h_{i_8} \}) > 3$. Now $h_{i_2}$ has j-envy for $\alpha_{j_2}$ since $u_{h_{i_2}}(M) = 2 < 3 < u_{h_{i_2}}(\{ h_{i_3}, h_{i_8} \})$, $v_{h_{i_3}}(\alpha_{j_2}) = 0 < 1 \leq v_{h_{i_3}}(h_{i_2})$, and $v_{h_{i_8}}(\alpha_{j_2}) = 0 < 1 \leq v_{h_{i_8}}(h_{i_2})$.
\item Suppose $v_{h_{i_1}}(h_{i_2}) = 1$. Without loss of generality we assume that either $\{ i_1, i_2 \} = \{ 2, 5 \}$ or $\{ i_1, i_2 \} = \{ 2, 6 \}$.
\begin{itemize}
    \item Suppose $\{ i_1, i_2 \} = \{ 2, 5 \}$. If $u_{h_{4}}(M) \leq 9$ then $h_{4}$ has j-envy for $\alpha_{j_1}$, since $u_{h_{4}}(M) \leq 9 < 10 = u_{h_{4}}(\{ h_2, h_5 \})$, $v_{h_2}(\alpha_{j_1}) = 0 < 6 = v_{h_2}(h_{4})$, and $v_{h_5}(\alpha_{j_1}) = 0 < 4 = v_{h_5}(h_{4})$. It follows that $u_{h_{4}}(M) \geq 10$. The only possibility is that $M(h_4) = \{ h_3, h_4, h_{6} \}$ and hence $M(h_4) = t_5$. Consider $h_1$. If $u_{h_1}(M) \leq 3$ then $h_1$ has j-envy for $\alpha_{j_1}$, since $u_{h_1}(M) \leq 3 < 4 = u_{h_1}(\{ h_2, h_5 \})$, $v_{h_2}(\alpha_{j_1}) = 0 < 2 = v_{h_2}(h_1)$, and $v_{h_5}(\alpha_{j_1}) = 0 < 2 = v_{h_5}(h_1)$. It follows that $u_{h_1}(M) \geq 4$. By the design of $H$, it follows that $M(h_1)$ must contain three agents in $H$, which is a contradiction since $h_1 \notin t_5$.
    \item Suppose $\{ i_1, i_2 \} = \{ 2, 6 \}$. It follows that $u_{h_4}(M) \leq 9$ so $h_4$ has j-envy for $\alpha_{j_1}$ since $u_{h_4}(M) \leq 9 < 12 = u_{h_4}(\{ h_2, h_6 \})$, $v_{h_2}(\alpha_{j_1}) = 0 < 6 = v_{h_2}(h_4)$, and $v_{h_6}(\alpha_{j_1}) = 0 < 6 = v_{h_6}(h_4)$.
\end{itemize}
\end{itemize}