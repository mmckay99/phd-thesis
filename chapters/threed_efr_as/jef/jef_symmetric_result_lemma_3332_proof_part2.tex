Suppose to the contrary that $\sigma(H, M) = 4$, $u_{h_1}(M) = 4$, and $H$ has an open configuration in $M$. Since $\sigma(H, M) = 4$ it must be that three triples in $M$ each contain exactly three agents in $H$ and one triple in $M$ contains exactly two agents in $H$. Suppose $t_1, t_2, t_3 \in M$ each contain exactly three agents in $H$ and $t_4 \in M$ contains exactly two agents in $H$. Since $u_{h_1}(M) = 4$, by the design of $H$ it follows that $M(h_1)$ contains two agents in $H \setminus \{ h_1 \}$ and therefore $h_1 \notin t_4$. It follows that $t_4 = \{ h_{i_1}, h_{i_2}, \alpha_{j} \}$ where $2 \leq i_1, i_2 \leq 11$ and $\alpha_{j} \in (N \setminus H)$.  We use a case analysis to prove a contradiction occurs for each possible assignment of $\{ h_{i_1}, h_{i_2} \}$ where $2\leq i_1, i_2 \leq 11$.

As before, in the proof of Lemma~\ref{lem:threed_efr_as_jef_3332_case_part1}, the symmetries of $H$ allow us to shorten the case analysis. Recall that the structure of the valuations between agents $H \setminus \{ h_1 \}$ has five symmetries. It follows that for any possible assignment of $\{ {i_1}, {i_2} \}$ there exist in total five symmetric assignments, where the case analysis for one assignment is symmetric to the case analysis for the other four assignments. Since there are $\binom{10}{2}=45$ possible assignments of $\{ {i_1}, {i_2} \}$ where $2\leq i_1, i_2 \leq 11$ and five symmetries we need only consider the nine assignments $\{ 2, 3 \}$, $\{ 2, 4 \}$, $\{ 2, 5 \}$, $\{ 2, 6 \}$, $\{ 2, 7 \}$, $\{ 3, 4 \}$, $\{ 3, 5 \}$, $\{ 3, 6 \}$, $\{ 3, 7 \}$, of which no two are symmetric.
\begin{itemize}
    \item Suppose $\{ {i_1}, {i_2} \} = \{ 2, 3 \}$. Since $h_2 \notin M(h_4)$ and $h_3 \notin M(h_4)$ it must be that $u_{h_4}(M) \leq 10$. It follows that $h_4$ has j-envy for $\alpha_{j}$ since $u_{h_4}(M) \leq 10 < 12 = u_{h_4}(\{ h_2, h_3 \})$, $v_{h_2}(\alpha_{j}) = 0 < 6 = v_{h_2}(h_4)$, and $v_{h_3}(\alpha_{j}) = 0 < 5 = v_{h_3}(h_4)$. This contradicts the supposition that $M$ is j-envy-free. The following proofs concerning other possible assignments of $\{ {i_1}, {i_2} \}$ are similar in technique although in some cases we must make further deductions about the utilities of agents in $H$.
    \item Suppose $\{ {i_1}, {i_2} \} = \{ 2, 4 \}$. Since $h_2 \notin M(h_3)$ and $h_4 \notin M(h_3)$ it follows that $u_{h_3}(M) \leq 6$. It then follows that $h_3$ has j-envy for $\alpha_{j}$ since $u_{h_3}(M) \leq 6 < 9 = u_{h_3}(\{ h_2, h_4 \})$, $v_{h_2}(\alpha_{j}) = 0 < 4 = v_{h_2}(h_3)$, and $v_{h_4}(\alpha_{j}) = 0 < 5 = v_{h_4}(h_3)$.
    \item Suppose $\{ {i_1}, {i_2} \} = \{ 2, 5 \}$. If $u_{h_4}(M) \leq 9$ then $h_4$ has j-envy for $\alpha_{j}$, since $u_{h_4}(M) \leq 9 < 10 = u_{h_4}(\{ h_2, h_5 \})$, $v_{h_2}(\alpha_{j}) = 0 < 6 = v_{h_2}(h_4)$, and $v_{h_5}(\alpha_{j}) = 0 < 4 = v_{h_5}(h_4)$. It follows that $u_{h_4}(M) \geq 10$. The only possibility is that $M(h_4) = \{ h_3, h_4, h_6 \}$. Now $h_3$ has j-envy for $\alpha_{j}$ since $u_{h_3}(M) = 6 < 7 = u_{h_3}(\{ h_2, h_5 \})$, $v_{h_2}(\alpha_{j}) = 0 < 4 = v_{h_2}(h_3)$, and $v_{h_5}(\alpha_{j}) = 0 < 3 = v_{h_5}(h_3)$.
    \item Suppose $\{ {i_1}, {i_2} \} = \{ 2, 6 \}$. Since $h_2 \notin M(h_4)$ and $h_6 \notin M(h_4)$ it follows that $u_{h_4}(M) \leq 9$. Now $h_4$ has j-envy for $\alpha_{j}$ since $u_{h_4}(M) \leq 9 < 12 = u_{h_4}(\{ h_2, h_6 \})$, $v_{h_2}(\alpha_{j}) = 0 < 6 = v_{h_2}(h_4)$, and $v_{h_6}(\alpha_{j}) = 0 < 6 = v_{h_6}(h_4)$.
    \item Suppose $\{ {i_1}, {i_2} \} = \{ 2, 7 \}$. If $u_{h_4}(M) \leq 6$ then $h_4$ has j-envy for $\alpha_{j}$, since $u_{h_4}(M) \leq 6 < 7 = u_{h_4}(\{ h_2, h_7 \})$, $v_{h_2}(\alpha_{j}) = 0 < 6 = v_{h_2}(h_4)$, and $v_{h_7}(\alpha_{j}) = 0 < 1 = v_{h_7}(h_4)$. It follows that $u_{h_4}(M) \geq 7$. By the design of $H$, it follows that $M(h_4)$ contains three agents in $H$. Since $t_4 = \{ h_2, h_7, \alpha_{j} \}$ it follows that $M(h_4) = \{ h_4, h_{i_3}, h_{i_4} \}$ where $\{ i_3, i_4 \} \subset \{ 1, 3, 5, 6, 8, 9, 10, 11 \}$. Recall that $v_{h_4}(h_1) = 2$, $v_{h_4}(h_3) = 5$, $v_{h_4}(h_5) = 4$, $v_{h_4}(h_6) = 6$, $v_{h_4}(h_8) = 1$, $v_{h_4}(h_9) = 3$, $v_{h_4}(h_{10}) = 1$, and $v_{h_4}(h_{11}) = 1$. Since we established $u_{h_4}(M) \geq 7$ it follows that there are $11$ possibilities: $\{ {i_3}, {i_4} \} = \{ 1, 3 \}$, $\{ {i_3}, {i_4} \} = \{ 1, 6 \}$, $\{ {i_3}, {i_4} \} = \{ 3, 5 \}$, $\{ {i_3}, {i_4} \} = \{ 3, 6 \}$, $\{ {i_3}, {i_4} \} = \{ 3, 9 \}$, $\{ {i_3}, {i_4} \} = \{ 5, 6 \}$, $\{ {i_3}, {i_4} \} = \{ 5, 9 \}$, $\{ {i_3}, {i_4} \} = \{ 6, 8 \}$, $\{ {i_3}, {i_4} \} = \{ 6, 9 \}$, $\{ {i_3}, {i_4} \} = \{ 6, 10 \}$, and $\{ {i_3}, {i_4} \} = \{ 6, 11 \}$, which we shall now consider.
    \begin{itemize}
        \item Suppose $\{ {i_3}, {i_4} \} = \{ 1, 3 \}$. It follows that $h_2$ has j-envy for $h_1$ since $u_{h_2}(M) = 3 < 10 = u_{h_2}(\{ h_3, h_4 \})$, $v_{h_3}(h_1) = 2 < 4 = v_{h_3}(h_2)$, and $v_{h_4}(h_1) = 2 < 6 = v_{h_4}(h_2)$.
        \item Suppose $\{ {i_3}, {i_4} \} = \{ 1, 6 \}$. Consider $h_5$. It must be that $u_{h_5}(M) \leq 6$. Now $h_5$ has j-envy for $h_1$ since $u_{h_5}(M) \leq 6 < 9 = u_{h_5}(\{ h_4, h_6 \})$, $v_{h_4}(h_1) = 2 < 4 = v_{h_4}(h_5)$, and $v_{h_6}(h_1) = 2 < 5 = v_{h_6}(h_5)$.
        \item Suppose $\{ {i_3}, {i_4} \} = \{ 3, 5 \}$. It follows that $h_2$ has j-envy for $h_5$ since $u_{h_2}(M) = 3 < 10 = u_{h_2}(\{ h_3, h_4 \})$, $v_{h_3}(h_5) = 3 < 4 = v_{h_3}(h_2)$, and $v_{h_4}(h_5) = 4 < 6 = v_{h_4}(h_2)$.
        \item Suppose $\{ {i_3}, {i_4} \} = \{ 3, 6 \}$. In this case, consider $M(h_1)$. Since $h_1 \notin t_4$ it follows that $M(h_1)$ contains three agents in $H$. Suppose $M(h_1) = \{ h_1, h_{i_5}, h_{i_6} \}$ where $2 \leq {i_5}, {i_6} \leq 11$. Since we have established $M(h_2) = \{ h_2, h_7, \alpha_{j} \}$ and $M(h_4) = \{ h_3, h_4, h_6 \}$ it follows that $\{ h_{i_5}, h_{i_6} \} \subset \{ 5, 8, 9, 10, 11 \}$. Thus there are $\binom{5}{2}=10$ possible assignments of $\{ h_{i_5}, h_{i_6} \}$, which we shall now consider.
        \begin{itemize}
            \item If $\{ h_{i_5}, h_{i_6} \} = \{ 5, 8 \}$ then $h_7$ has j-envy for $h_1$ since $u_{h_7}(M) = 3 < 8 = u_{h_7}(\{ h_5, h_8 \})$, $v_{h_5}(h_1) = 2 < 3 = v_{h_5}(h_7)$, and $v_{h_8}(h_1) = 2 < 5 = v_{h_8}(h_7)$.
            \item If $\{ h_{i_5}, h_{i_6} \} = \{ 5, 9 \}$ then $h_7$ has j-envy for $h_1$ since $u_{h_7}(M) = 3 < 6 = u_{h_7}(\{ h_5, h_9 \})$, $v_{h_5}(h_1) = 2 < 3 = v_{h_5}(h_7)$, and $v_{h_9}(h_1) = 2 < 3 = v_{h_9}(h_7)$.
            \item If $\{ h_{i_5}, h_{i_6} \} = \{ 5, 10 \}$ then $u_{h_{10}}(M) = 5$. It follows that $h_{10}$ has j-envy for $\alpha_{j}$ since $u_{h_{10}}(M) = 5 < 7 = u_{h_{10}}(\{ h_2, h_7 \})$, $v_{h_2}(\alpha_{j}) = 0 < 6 = v_{h_2}(h_{10})$, and $v_{h_7}(\alpha_{j}) = 0 < 1 = v_{h_7}(h_{10})$.
            \item If $\{ h_{i_5}, h_{i_6} \} = \{ 5, 11 \}$ then $u_{h_{11}}(M) = 3$. It follows that $h_{11}$ has j-envy for $\alpha_{j}$ since $u_{h_{11}}(M) = 3 < 6 = u_{h_{11}}(\{ h_2, h_7 \})$, $v_{h_2}(\alpha_{j}) = 0 < 5 = v_{h_2}(h_{11})$, and $v_{h_7}(\alpha_{j}) = 0 < 1 = v_{h_7}(h_{11})$.
            \item If $\{ h_{i_5}, h_{i_6} \} = \{ 8, 9 \}$ then $h_7$ has j-envy for $h_1$ since $u_{h_7}(M) = 3 < 8 = u_{h_7}(\{ h_8, h_9 \})$, $v_{h_8}(h_1) = 2 < 5 = v_{h_8}(h_7)$, and $v_{h_9}(h_1) = 2 < 3 = v_{h_9}(h_7)$.
            \item If $\{ h_{i_5}, h_{i_6} \} = \{ 8, 10 \}$ then it remains that $M(h_9) = \{ h_5, h_9, h_{11} \}$ and thus $u_{h_9}(M)=u_{\{ h_{5}, h_{11} \}}=4$. It follows that $h_9$ has j-envy for $h_1$ since $u_{h_9}(M) = 4 < 9 = u_{h_9}(\{ h_8, h_{10} \})$, $v_{h_8}(h_1) = 2 < 4 = v_{h_8}(h_9)$, and $v_{h_{10}}(h_1) = 2 < 5 = v_{h_{10}}(h_9)$.
            \item If $\{ h_{i_5}, h_{i_6} \} = \{ 8, 11 \}$ then $u_{h_{11}}(M) = 3$. It follows that $h_{11}$ has j-envy for $\alpha_{j}$ since $u_{h_{11}}(M) = 3 < 6 = u_{h_{11}}(\{ h_2, h_7 \})$, $v_{h_2}(\alpha_{j}) = 0 < 5 = v_{h_2}(h_{11})$, and $v_{h_7}(\alpha_{j}) = 0 < 1 = v_{h_7}(h_{11})$.
            \item If $\{ h_{i_5}, h_{i_6} \} = \{ 9, 10 \}$ then it remains that $M(h_9) = \{ h_5, h_8, h_{11} \}$ and thus $u_{h_{11}}(M)=u_{\{ h_{5}, h_{8} \}}=2$. It follows that $h_{11}$ has j-envy for $\alpha_{j}$ since $u_{h_{11}}(M) = 2 < 6 = u_{h_{11}}(\{ h_2, h_7 \})$, $v_{h_2}(\alpha_{j}) = 0 < 5 = v_{h_2}(h_{11})$, and $v_{h_7}(\alpha_{j}) = 0 < 1 = v_{h_7}(h_{11})$.
            \item If $\{ h_{i_5}, h_{i_6} \} = \{ 9, 11 \}$ then $u_{h_{11}}(M) = 5$. It follows that $h_{11}$ has j-envy for $\alpha_{j}$ since $u_{h_{11}}(M) = 5 < 6 = u_{h_{11}}(\{ h_2, h_7 \})$, $v_{h_2}(\alpha_{j}) = 0 < 5 = v_{h_2}(h_{11})$, and $v_{h_7}(\alpha_{j}) = 0 < 1 = v_{h_7}(h_{11})$.
            \item If $\{ h_{i_5}, h_{i_6} \} = \{ 10, 11 \}$ then $h_2$ has j-envy for $h_1$ since $u_{h_2}(M) = 3 < 11 = u_{h_2}(\{ h_{10}, h_{11} \})$, $v_{h_{10}}(h_1) = 2 < 6 = v_{h_{10}}(h_2)$, and $v_{h_{11}}(h_1) = 2 < 5 = v_{h_{11}}(h_2)$.
        \end{itemize}
        \item Suppose $\{ {i_3}, {i_4} \} = \{ 3, 9 \}$. It follows that $h_2$ has j-envy for $h_9$ since $u_{h_2}(M) = 3 < 10 = u_{h_2}(\{ h_3, h_4 \})$, $v_{h_3}(h_9) = 1 < 4 = v_{h_3}(h_2)$, and $v_{h_4}(h_9) = 3 < 6 = v_{h_4}(h_2)$.
        \item Suppose $\{ {i_3}, {i_4} \} = \{ 5, 6 \}$. Consider $h_3$. If $u_{h_3}(M) \leq 4$ then $h_3$ has j-envy for $\alpha_{j}$, since $u_{h_3}(M) \leq 4 < 5 = u_{h_3}(\{ h_2, h_7 \})$, $v_{h_2}(\alpha_{j}) = 0 < 4 = v_{h_2}(h_3)$, and $v_{h_7}(\alpha_{j}) = 0 < 1 = v_{h_7}(h_3)$. It follows that $u_{h_3}(M) \geq 5$. We have established that $h_4 \notin M(h_3)$, $h_5 \notin M(h_3)$ and $h_2 \notin M(h_3)$ so, by the design of $H$, there are three possibilities: either $M(h_3) = \{ h_1, h_3, h_8 \}$, $M(h_3) = \{ h_1, h_3, h_{11} \}$, or $M(h_3) = \{ h_3, h_8, h_{11} \}$.
        \begin{itemize}
            \item If $M(h_3) = \{ h_1, h_3, h_8 \}$ then $u_{h_8}(M) = 5$. It follows that $h_8$ has j-envy for $\alpha_{j}$, since $u_{h_8}(M) = 5 < 6 = u_{h_8}(\{ h_2, h_7 \})$, $v_{h_2}(\alpha_{j}) = 0 < 1 = v_{h_2}(h_8)$, and $v_{h_7}(\alpha_{j}) = 0 < 5 = v_{h_7}(h_8)$.
            \item If $M(h_3) = \{ h_1, h_3, h_{11} \}$ then $u_{h_{11}}(M) = 5$. It follows that $h_{11}$ has j-envy for $\alpha_{j}$, since $u_{h_{11}}(M) = 5 < 6 = u_{h_{11}}(\{ h_2, h_7 \})$, $v_{h_2}(\alpha_{j}) = 0 < 5 = v_{h_2}(h_{11})$, and $v_{h_7}(\alpha_{j}) = 0 < 1 = v_{h_7}(h_{11})$.
            \item If $M(h_3) = \{ h_3, h_8, h_{11} \}$ then $u_{h_8}(M) = 4$. It follows that $h_8$ has j-envy for $\alpha_{j}$, since $u_{h_8}(M) = 4 < 6 = u_{h_8}(\{ h_2, h_7 \})$, $v_{h_2}(\alpha_{j}) = 0 < 1 = v_{h_2}(h_8)$, and $v_{h_7}(\alpha_{j}) = 0 < 5 = v_{h_7}(h_8)$.
        \end{itemize}
        \item Suppose $\{ {i_3}, {i_4} \} = \{ 5, 9 \}$. Consider $h_6$. It must be that $u_{h_6}(M) \leq 9$. Now $h_6$ has j-envy for $h_9$ since $u_{h_6}(M) \leq 9 < 11 = u_{h_6}(\{ h_4, h_5 \})$, $v_{h_4}(h_9) = 3 < 6 = v_{h_4}(h_6)$, and $v_{h_5}(h_9) = 1 < 5 = v_{h_5}(h_6)$.
        \item Suppose $\{ {i_3}, {i_4} \} = \{ 6, 8 \}$. Consider $h_{10}$. If $u_{h_{10}}(M) \leq 6$ then $h_{10}$ has j-envy for $\alpha_{j}$, since $u_{h_{10}}(M) \leq 6 < 7 = u_{h_{10}}(\{ h_2, h_7 \})$, $v_{h_2}(\alpha_{j}) = 0 < 6 = v_{h_2}(h_{10})$, and $v_{h_7}(\alpha_{j}) = 0 < 1 = v_{h_7}(h_{10})$. It follows that $u_{h_{10}}(M) \geq 7$. We have established that $h_2 \notin M(h_{10})$ and $h_8 \notin M(h_{10})$ so, by the design of $H$, there are four possibilities: either $M(h_{10}) = \{ h_1, h_9, h_{10} \}$, $M(h_{10}) = \{ h_5, h_9, h_{10} \}$, $M(h_{10}) = \{ h_5, h_{10}, h_{11} \}$, or $M(h_3) = \{ h_9, h_{10}, h_{11} \}$.
        \begin{itemize}
            \item If $M(h_{10}) = \{ h_1, h_9, h_{10} \}$ then $h_8$ has j-envy for $h_1$, since $u_{h_8}(M) = 7 < 10 = u_{h_8}(\{ h_9, h_{10} \})$, $v_{h_9}(h_1) = 2 < 4 = v_{h_9}(h_8)$, and $v_{h_{10}}(h_1) = 2 < 6 = v_{h_{10}}(h_8)$.
            \item If $M(h_{10}) = \{ h_5, h_9, h_{10} \}$ then $h_8$ has j-envy for $h_5$, since $u_{h_8}(M) = 7 < 10 = u_{h_8}(\{ h_9, h_{10} \})$, $v_{h_9}(h_5) = 1 < 4 = v_{h_9}(h_8)$, and $v_{h_{10}}(h_5) = 3 < 6 = v_{h_{10}}(h_8)$.
            \item If $M(h_{10}) = \{ h_5, h_{10}, h_{11} \}$ then $u_{h_{11}}(M) = 5$. It follows that $h_{11}$ has j-envy for $\alpha_{j}$, since $u_{h_{11}}(M) = 5 < 6 = u_{h_{11}}(\{ h_2, h_7 \})$, $v_{h_2}(\alpha_{j}) = 0 < 5 = v_{h_2}(h_{11})$, and $v_{h_7}(\alpha_{j}) = 0 < 1 = v_{h_7}(h_{11})$.
            \item If $M(h_3) = \{ h_9, h_{10}, h_{11} \}$ then $h_8$ has j-envy for $h_{11}$, since $u_{h_8}(M) = 7 < 10 = u_{h_8}(\{ h_9, h_{10} \})$, $v_{h_9}(h_{11}) = 3 < 4 = v_{h_9}(h_8)$, and $v_{h_{10}}(h_{11}) = 4 < 6 = v_{h_{10}}(h_8)$.
        \end{itemize}
        \item Suppose $\{ {i_3}, {i_4} \} = \{ 6, 9 \}$. It must be that $u_{h_5}(M) \leq 6$. Now $h_5$ has j-envy for $h_9$ since $u_{h_5}(M) \leq 6 < 9 = u_{h_5}(\{ h_4, h_6 \})$, $v_{h_4}(h_9) = 3 < 4 = v_{h_4}(h_5)$, and $v_{h_6}(h_9) = 1 < 5 = v_{h_6}(h_5)$.
        \item Suppose $\{ {i_3}, {i_4} \} = \{ 6, 10 \}$. It must be that $u_{h_5}(M) \leq 5$. Now $h_5$ has j-envy for $h_{10}$ since $u_{h_5}(M) \leq 5 < 9 = u_{h_5}(\{ h_4, h_6 \})$, $v_{h_4}(h_{10}) = 1 < 4 = v_{h_4}(h_5)$, and $v_{h_6}(h_{10}) = 1 < 5 = v_{h_6}(h_5)$.
        \item Suppose $\{ {i_3}, {i_4} \} = \{ 6, 11 \}$. It must be that $u_{h_5}(M) \leq 6$. Now $h_5$ has j-envy for $h_{11}$ since $u_{h_5}(M) \leq 6 < 9 = u_{h_5}(\{ h_4, h_6 \})$, $v_{h_4}(h_{11}) = 1 < 4 = v_{h_4}(h_5)$, and $v_{h_6}(h_{11}) = 3 < 5 = v_{h_6}(h_5)$.
    \end{itemize}
\item Suppose $\{ {i_1}, {i_2} \} = \{ 3, 4 \}$. If $u_{h_2}(M) \leq 9$ then $h_2$ has j-envy for $\alpha_{j}$, since $u_{h_2}(M) \leq 9 < 10 = u_{h_2}(\{ h_3, h_4 \})$, $v_{h_3}(\alpha_{j}) = 0 < 4 = v_{h_3}(h_2)$, and $v_{h_4}(\alpha_{j}) = 0 < 6 = v_{h_4}(h_2)$. It follows that $u_{h_2}(M) \geq 10$. The only possibility is that $M(h_2) = \{ h_2, h_{10}, h_{11} \}$. Now consider $h_5$. If $u_{h_5}(M) \leq 6$ then $h_5$ has j-envy for $\alpha_{j}$, since $u_{h_5}(M) \leq 6 < 7 = u_{h_5}(\{ h_3, h_4 \})$, $v_{h_3}(\alpha_{j}) = 0 < 3 = v_{h_3}(h_5)$, and $v_{h_4}(\alpha_{j}) = 0 < 4 = v_{h_4}(h_5)$. It follows that $u_{h_5}(M) \geq 7$. Since we have established $M(h_2) = \{ h_2, h_{10}, h_{11} \}$ and $t_4 = \{ h_3, h_4, \alpha_{j} \}$, there are just two possibilities: either $M(h_5) = \{ h_1, h_5, h_6 \}$ or $M(h_5) = \{ h_5, h_6, h_7 \}$.
\begin{itemize}
    \item If $M(h_5) = \{ h_1, h_5, h_6 \}$ then $h_4$ has j-envy for $h_1$ since $u_{h_4}(M) = 5 < 10 = u_{h_4}(\{ h_5, h_6 \})$, $v_{h_5}(h_1) = 2 < 4 = v_{h_5}(h_4)$, and $v_{h_6}(h_1) = 2 < 6 = v_{h_6}(h_4)$.
    \item If $M(h_5) = \{ h_5, h_6, h_7 \}$ then $h_4$ has j-envy for $h_7$ since $u_{h_4}(M) = 5 < 10 = u_{h_4}(\{ h_5, h_6 \})$, $v_{h_5}(h_7) = 3 < 4 = v_{h_5}(h_4)$, and $v_{h_6}(h_7) = 4 < 6 = v_{h_6}(h_4)$.
\end{itemize}
\item Suppose $\{ {i_1}, {i_2} \} = \{ 3, 5 \}$. If $u_{h_4}(M) \leq 8$ then $h_4$ has j-envy for $\alpha_{j}$, since $u_{h_4}(M) \leq 8 < 9 = u_{h_4}(\{ h_3, h_5 \})$, $v_{h_3}(\alpha_{j}) = 0 < 5 = v_{h_3}(h_4)$, and $v_{h_5}(\alpha_{j}) = 0 < 4 = v_{h_5}(h_4)$. It follows that $u_{h_4}(M) \geq 9$. There are three possibilities: either $M(h_4) = \{ h_2, h_4, h_6 \}$, $M(h_4) = \{ h_2, h_4, h_9 \}$, or $M(h_4) = \{ h_4, h_6, h_9 \}$.
\begin{itemize}
    \item Suppose $M(h_4) = \{ h_2, h_4, h_6 \}$. In this case, consider $M(h_1)$. Since $h_1 \notin t_4$ it follows that $M(h_1)$ contains three agents in $H$. Suppose $M(h_1) = \{ h_1, h_{i_5}, h_{i_6} \}$ where $1 \leq {i_5}, {i_6} \leq 11$. Since we have established $M(h_3) = \{ h_3, h_5, \alpha_{j} \}$ and $M(h_2) = \{ h_2, h_4, h_6 \}$ it follows that $\{ h_{i_5}, h_{i_6} \} \subset \{ 7, 8, 9, 10, 11 \}$. Thus there are $\binom{5}{2}=10$ possible assignments of $\{ h_{i_5}, h_{i_6} \}$, which we shall now consider.
    \begin{itemize}
        \item If $\{ h_{i_5}, h_{i_6} \} = \{ 7, 8 \}$ then $h_6$ has j-envy for $h_1$, since $u_{h_6}(M) = 7 < 10 = u_{h_6}(\{ h_7, h_8 \})$, $v_{h_7}(h_1) = 2 < 4 = v_{h_7}(h_6)$, and $v_{h_8}(h_1) = 2 < 6 = v_{h_8}(h_6)$.
        \item If $\{ h_{i_5}, h_{i_6} \} = \{ 7, 9 \}$ then it remains that $M(h_8) = \{ h_8, h_{10}, h_{11} \}$. It follows that $h_8$ has j-envy for $h_1$, since $u_{h_8}(M) = 7 < 9 = u_{h_8}(\{ h_7, h_9 \})$, $v_{h_7}(h_1) = 2 < 5 = v_{h_7}(h_8)$, and $v_{h_9}(h_1) = 2 < 4 = v_{h_9}(h_8)$.
        \item If $\{ h_{i_5}, h_{i_6} \} = \{ 7, 10 \}$ then $h_7$ has j-envy for $\alpha_{j}$, since $u_{h_7}(M) = 3 < 4 = u_{h_7}(\{ h_3, h_5 \})$, $v_{h_3}(\alpha_{j}) = 0 < 1 = v_{h_3}(h_7)$, and $v_{h_5}(\alpha_{j}) = 0 < 3 = v_{h_5}(h_7)$.
        \item If $\{ h_{i_5}, h_{i_6} \} = \{ 7, 11 \}$ then $h_7$ has j-envy for $\alpha_{j}$, since $u_{h_7}(M) = 3 < 4 = u_{h_7}(\{ h_3, h_5 \})$, $v_{h_3}(\alpha_{j}) = 0 < 1 = v_{h_3}(h_7)$, and $v_{h_5}(\alpha_{j}) = 0 < 3 = v_{h_5}(h_7)$.
        \item If $\{ h_{i_5}, h_{i_6} \} = \{ 8, 9 \}$ then it remains that $M(h_7) = \{ h_7, h_{10}, h_{11} \}$. It follows that $h_7$ has j-envy for $h_1$, since $u_{h_7}(M) = 2 < 8 = u_{h_7}(\{ h_8, h_9 \})$, $v_{h_8}(h_1) = 2 < 5 = v_{h_8}(h_7)$, and $v_{h_9}(h_1) = 2 < 3 = v_{h_9}(h_7)$.
        \item If $\{ h_{i_5}, h_{i_6} \} = \{ 8, 10 \}$ then it remains that $M(h_9) = \{ h_7, h_9, h_{11} \}$. It follows that $h_9$ has j-envy for $h_1$, since $u_{h_9}(M) = 6 < 9 = u_{h_9}(\{ h_8, h_{10} \})$, $v_{h_8}(h_1) = 2 < 4 = v_{h_8}(h_9)$, and $v_{h_{10}}(h_1) = 2 < 5 = v_{h_{10}}(h_9)$.
        \item If $\{ h_{i_5}, h_{i_6} \} = \{ 8, 11 \}$ then $h_{11}$ has j-envy for $\alpha_{j}$, since $u_{h_{11}}(M) = 3 < 4 = u_{h_{11}}(\{ h_3, h_5 \})$, $v_{h_3}(\alpha_{j}) = 0 < 3 = v_{h_3}(h_{11})$, and $v_{h_5}(\alpha_{j}) = 0 < 1 = v_{h_5}(h_{11})$.
        \item If $\{ h_{i_5}, h_{i_6} \} = \{ 9, 10 \}$ then it remains that $M(h_{11}) = \{ h_7, h_8, h_{11} \}$. It follows that $h_{11}$ has j-envy for $h_1$, since $u_{h_{11}}(M) = 2 < 7 = u_{h_{11}}(\{ h_9, h_{10} \})$, $v_{h_9}(h_1) = 2 < 3 = v_{h_9}(h_{11})$, and $v_{h_{10}}(h_1) = 2 < 4 = v_{h_{10}}(h_{11})$.
        \item If $\{ h_{i_5}, h_{i_6} \} = \{ 9, 11 \}$ then it remains that $M(h_{10}) = \{ h_7, h_8, h_{10} \}$. It follows that $h_{10}$ has j-envy for $h_1$, since $u_{h_{10}}(M) = 7 < 9 = u_{h_{10}}(\{ h_9, h_{11} \})$, $v_{h_9}(h_1) = 2 < 5 = v_{h_9}(h_{10})$, and $v_{h_{11}}(h_1) = 2 < 4 = v_{h_{11}}(h_{10})$.
        \item If $\{ h_{i_5}, h_{i_6} \} = \{ 10, 11 \}$ then $h_2$ has j-envy for $h_1$, since $u_{h_2}(M) = 7 < 11 = u_{h_2}(\{ h_{10}, h_{11} \})$, $v_{h_{10}}(h_1) = 2 < 6 = v_{h_{10}}(h_2)$, and $v_{h_{11}}(h_1) = 2 < 5 = v_{h_{11}}(h_2)$.
    \end{itemize}
    \item Suppose $M(h_4) = \{ h_2, h_4, h_9 \}$. It follows that $h_3$ has j-envy for $h_9$, since $u_{h_3}(M) = 3 < 9 = u_{h_3}(\{ h_2, h_4 \})$, $v_{h_2}(h_9) = 1 < 4 = v_{h_2}(h_3)$, and $v_{h_4}(h_9) = 3 < 5 = v_{h_4}(h_3)$.
    \item Suppose $M(h_4) = \{ h_4, h_6, h_9 \}$. It follows that $h_5$ has j-envy for $h_9$, since $u_{h_5}(M) = 3 < 9 = u_{h_5}(\{ h_4, h_6 \})$, $v_{h_4}(h_9) = 3 < 4 = v_{h_4}(h_5)$, and $v_{h_6}(h_9) = 1 < 5 = v_{h_6}(h_5)$.
\end{itemize}
\item Suppose $\{ {i_1}, {i_2} \} = \{ 3, 6 \}$. It must be that $u_{h_4}(M) \leq 10$. It follows that $h_4$ has j-envy for $\alpha_{j}$ since $u_{h_4}(M) \leq 10 < 11 = u_{h_4}(\{ h_3, h_6 \})$, $v_{h_3}(\alpha_{j}) = 0 < 5 = v_{h_3}(h_4)$, and $v_{h_6}(\alpha_{j}) = 0 < 6 = v_{h_6}(h_4)$.
\item Suppose $\{ {i_1}, {i_2} \} = \{ 3, 7 \}$. If $u_{h_8}(M) \leq 7$ then $h_8$ has j-envy for $\alpha_{j}$, since $u_{h_8}(M) \leq 7 < 8 = u_{h_8}(\{ h_3, h_7 \})$, $v_{h_3}(\alpha_{j}) = 0 < 3 = v_{h_3}(h_8)$, and $v_{h_7}(\alpha_{j}) = 0 < 5 = v_{h_7}(h_8)$. It follows that $u_{h_8}(M) \geq 8$. By the design of $H$, it follows that $M(h_8)$ contains three agents in $H$. Since $t_4 = \{ h_3, h_7, \alpha_{j} \}$ it must be that $M(h_8) = \{ h_8, h_{i_3}, h_{i_4} \}$ where $\{ i_3, i_4 \} \subset \{ 1, 2, 4, 5, 6, 9, 10, 11 \}$. Recall that $v_{h_8}(h_1) = 2$, $v_{h_8}(h_2) = 1$, $v_{h_8}(h_4) = 1$, $v_{h_8}(h_5) = 1$, $v_{h_8}(h_6) = 6$, $v_{h_8}(h_9) = 4$, $v_{h_8}(h_{10}) = 6$, and $v_{h_8}(h_{11}) = 1$. Since we established $u_{h_8}(M) \geq 8$ it follows that there are five possibilities: $\{ {i_3}, {i_4} \} = \{ 1, 6 \}$, $\{ {i_3}, {i_4} \} = \{ 1, 10 \}$, $\{ {i_3}, {i_4} \} = \{ 6, 9 \}$, $\{ {i_3}, {i_4} \} = \{ 6, 10 \}$, and $\{ {i_3}, {i_4} \} = \{ 9, 10 \}$, which we shall now consider.
\begin{itemize}
    \item Suppose $\{ {i_3}, {i_4} \} = \{ 1, 6 \}$. It follows that $h_7$ has j-envy for $h_1$, since $u_{h_7}(M) = 1 < 9 = u_{h_7}(\{ h_6, h_8 \})$, $v_{h_6}(h_1) = 2 < 4 = v_{h_6}(h_7)$, and $v_{h_8}(h_1) = 2 < 5 = v_{h_8}(h_7)$.
    \item Suppose $\{ {i_3}, {i_4} \} = \{ 1, 10 \}$. Consider $h_9$. It follows that $u_{h_9}(M) \leq 6$. Now $h_9$ has j-envy for $h_1$, since $u_{h_9}(M) \leq 6 < 9 = u_{h_9}(\{ h_8, h_{10} \})$, $v_{h_8}(h_1) = 2 < 4 = v_{h_8}(h_9)$, and $v_{h_{10}}(h_1) = 2 < 5 = v_{h_{10}}(h_9)$.
    \item Suppose $\{ {i_3}, {i_4} \} = \{ 6, 9 \}$. It follows that $h_7$ has j-envy for $h_9$, since $u_{h_7}(M) = 1 < 9 = u_{h_7}(\{ h_6, h_8 \})$, $v_{h_6}(h_9) = 1 < 4 = v_{h_6}(h_7)$, and $v_{h_8}(h_9) = 4 < 5 = v_{h_8}(h_7)$.
    \item Suppose $\{ {i_3}, {i_4} \} = \{ 6, 10 \}$. In this case, consider $M(h_1)$. Since $h_1 \notin t_4$ it follows that $M(h_1)$ contains three agents in $H$. Suppose $M(h_1) = \{ h_1, h_{i_5}, h_{i_6} \}$ where $2 \leq i_5, i_6 \leq 11$. Since we have established $M(h_3) = \{ h_3, h_7, \alpha_{j} \}$ and $M(h_6) = \{ h_6, h_8, h_{10} \}$ it follows that $\{ h_{i_5}, h_{i_6} \} \subset \{ 2, 4, 5, 9, 11 \}$. Thus there are $\binom{5}{2}=10$ possible assignments of $\{ h_{i_5}, h_{i_6} \}$, which we shall now consider.
    \begin{itemize}
        \item If $\{ h_{i_5}, h_{i_6} \} = \{ 2, 4 \}$ then $h_3$ has j-envy for $h_1$, since $u_{h_3}(M) = 1 < 9 = u_{h_3}(\{ h_2, h_4 \})$, $v_{h_2}(h_1) = 2 < 4 = v_{h_2}(h_3)$, and $v_{h_4}(h_1) = 2 < 5 = v_{h_4}(h_3)$.
        \item If $\{ h_{i_5}, h_{i_6} \} = \{ 2, 5 \}$ then $h_3$ has j-envy for $h_1$, since $u_{h_3}(M) = 1 < 7 = u_{h_3}(\{ h_2, h_5 \})$, $v_{h_2}(h_1) = 2 < 4 = v_{h_2}(h_3)$, and $v_{h_5}(h_1) = 2 < 3 = v_{h_5}(h_3)$.
        \item If $\{ h_{i_5}, h_{i_6} \} = \{ 2, 9 \}$ then it remains that $M(h_{11}) = \{ h_4, h_5, h_{11} \}$. Now $h_3$ has j-envy for $h_{11}$, since $u_{h_3}(M) = 1 < 8 = u_{h_3}(\{ h_4, h_5 \})$, $v_{h_4}(h_{11}) = 1 < 5 = v_{h_4}(h_3)$, and $v_{h_5}(h_{11}) = 1 < 3 = v_{h_5}(h_3)$.
        \item If $\{ h_{i_5}, h_{i_6} \} = \{ 2, 11 \}$ then $h_3$ has j-envy for $h_1$, since $u_{h_3}(M) = 1 < 7 = u_{h_3}(\{ h_2, h_{11} \})$, $v_{h_2}(h_1) = 2 < 4 = v_{h_2}(h_3)$, and $v_{h_{11}}(h_1) = 2 < 3 = v_{h_{11}}(h_3)$.
        \item If $\{ h_{i_5}, h_{i_6} \} = \{ 4, 5 \}$ then $h_3$ has j-envy for $h_1$, since $u_{h_3}(M) = 1 < 8 = u_{h_3}(\{ h_5, h_5 \})$, $v_{h_4}(h_1) = 2 < 5 = v_{h_4}(h_3)$, and $v_{h_5}(h_1) = 2 < 3 = v_{h_5}(h_3)$.
        \item If $\{ h_{i_5}, h_{i_6} \} = \{ 4, 9 \}$ then $h_4$ has j-envy for $\alpha_{j}$, since $u_{h_4}(M) = 5 < 6 = u_{h_4}(\{ h_3, h_7 \})$, $v_{h_3}(\alpha_{j}) = 0 < 5 = v_{h_3}(h_4)$, and $v_{h_7}(\alpha_{j}) = 0 < 1 = v_{h_7}(h_4)$.
        \item If $\{ h_{i_5}, h_{i_6} \} = \{ 4, 11 \}$ then $h_3$ has j-envy for $h_1$, since $u_{h_3}(M) = 1 < 8 = u_{h_3}(\{ h_4, h_{11} \})$, $v_{h_4}(h_1) = 2 < 5 = v_{h_4}(h_3)$, and $v_{h_{11}}(h_1) = 2 < 3 = v_{h_{11}}(h_3)$.
        \item If $\{ h_{i_5}, h_{i_6} \} = \{ 5, 9 \}$ then $h_5$ has j-envy for $\alpha_{j}$, since $u_{h_5}(M) = 3 < 6 = u_{h_5}(\{ h_3, h_7 \})$, $v_{h_3}(\alpha_{j}) = 0 < 3 = v_{h_3}(h_5)$, and $v_{h_7}(\alpha_{j}) = 0 < 3 = v_{h_7}(h_5)$.
        \item If $\{ h_{i_5}, h_{i_6} \} = \{ 5, 11 \}$ then $h_3$ has j-envy for $h_1$, since $u_{h_3}(M) = 1 < 6 = u_{h_3}(\{ h_5, h_{11} \})$, $v_{h_5}(h_1) = 2 < 3 = v_{h_5}(h_3)$, and $v_{h_{11}}(h_1) = 2 < 3 = v_{h_{11}}(h_3)$.
        \item If $\{ h_{i_5}, h_{i_6} \} = \{ 9, 11 \}$ then $h_{10}$ has j-envy for $h_1$, since $u_{h_{10}}(M) = 7 < 9 = u_{h_{10}}(\{ h_9, h_{11} \})$, $v_{h_9}(h_1) = 2 < 5 = v_{h_9}(h_{10})$, and $v_{h_{11}}(h_1) = 2 < 4 = v_{h_{11}}(h_{10})$.
    \end{itemize}
    \item Suppose $\{ {i_3}, {i_4} \} = \{ 9, 10 \}$. Consider $h_2$. If $u_{h_2}(M) \leq 6$ then $h_2$ has j-envy for $\alpha_{j}$, since $u_{h_2}(M) \leq 6 < 7 = u_{h_2}(\{ h_3, h_7 \})$, $v_{h_3}(\alpha_{j}) = 0 < 4 = v_{h_3}(h_2)$, and $v_{h_7}(\alpha_{j}) = 0 < 3 = v_{h_7}(h_2)$. It follows that $u_{h_2}(M) \geq 7$. We have established that $h_3 \notin M(h_2)$, $h_7 \notin M(h_2)$, and $h_{10} \notin M(h_2)$ so, by the design of $H$, there are three possibilities: either $M(h_2) = \{ h_1, h_2, h_4 \}$, $M(h_2) = \{ h_1, h_2, h_{11} \}$, or $M(h_2) = \{ h_2, h_4, h_{11} \}$.
    \begin{itemize}
        \item If $M(h_2) = \{ h_1, h_2, h_4 \}$ then $h_3$ has j-envy for $h_1$, since $u_{h_3}(M) = 1 < 9 = u_{h_3}(\{ h_2, h_4 \})$, $v_{h_2}(h_1) = 2 < 4 = v_{h_2}(h_3)$, and $v_{h_4}(h_1) = 2 < 5 = v_{h_4}(h_3)$.
        \item If $M(h_2) = \{ h_1, h_2, h_{11} \}$ then $h_3$ has j-envy for $h_1$, since $u_{h_3}(M) = 1 < 7 = u_{h_3}(\{ h_2, h_{11} \})$, $v_{h_2}(h_1) = 2 < 4 = v_{h_2}(h_3)$, and $v_{h_{11}}(h_1) = 2 < 3 = v_{h_{11}}(h_3)$.
        \item If $M(h_2) = \{ h_2, h_4, h_{11} \}$ then it remains that $M(h_1) = \{ h_1, h_5, h_6 \}$. Now $h_7$ has j-envy for $h_1$, since $u_{h_7}(M) = 1 < 7 = u_{h_7}(\{ h_5, h_6 \})$, $v_{h_5}(h_1) = 2 < 3 = v_{h_5}(h_7)$, and $v_{h_6}(h_1) = 2 < 4 = v_{h_6}(h_7)$.
    \end{itemize}
\end{itemize}
\end{itemize}