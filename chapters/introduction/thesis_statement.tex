% Many algorithmic problems related to Hedonic Games are believed to be computationally intractable while others are solvable in polynomial time. 
% In the context of hedonic games, imposing constraints on the possible sizes of coalitions significantly alters the structure and complexity of associated algorithmic problems. 
% V2
% Computational problems of to coalition formation with fixed size coalitions do appear to be hard in general. Nevertheless, some problems are solvable in polynomial time if either the preferences are sufficiently restricted or the solution concept is sufficiently weak. In fact, for successively weaker solution concepts, existence and polynomial-time solvability hold under successively weaker preference restrictions. 

% Although most computational problems in which agents must be assigned to coalitions of fixed size are $\NP$-hard in general, many of them can be solved in polynomial time if either the agents' preferences are sufficiently restricted or the solution concept is sufficiently weak.
% In fact, for successively weaker solution concepts, existence and polynomial-time solvability hold under successively weaker preference restrictions.
Computational problems involving agents that form coalitions are generally $\NP$-hard, and in particular when those coalitions must have a fixed size. Nevertheless, there exist natural models of fixed-size coalition formation in which optimal or near-optimal matchings can be found using efficient algorithms.