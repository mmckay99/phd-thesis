Our main contribution relates to a family of models of fixed-size coalition formation known as \emph{Three-Dimensional Roommates} (\mysymbolfirstusedefinition{symboldef:threedr}{3DR}). The defining characteristic of a model of 3DR is that there are a set of $3n$ agents that must be partitioned into $n$ triples, which we call a \emph{matching}, and each agent has some kind of preference between the triples to which they may belong. Working within the framework of 3DR, we systematically study a set of fixed-size coalition formation problems.

Many real-world scenarios involve coalitions of restricted size, which provides a strong rationale for our study \cite{Sless18}. We motivate our study of 3DR in particular with the fact that results relating to coalitions of size three often generalise directly to problems involving other restrictions on coalition size. Even when this is not the case, it may be that they at least give an indication of what to expect in more general models. For this reason, much of the existing research also focused specifically on models in which coalitions must have size three.

We begin by reviewing the literature in and around coalition formation, including both models involving restricted coalition sizes and models in which coalitions need not have a fixed size. We tie related concepts and theory together from across computing science, economics, and game theory.

In our first technical contribution, we consider a variant of 3DR known as \emph{Three-Dimensional Stable Matching with Cyclic preferences} (3-DSM-CYC), which has been of independent interest. In 3-DSM-CYC, each agent has one of three types, sometimes termed \emph{man}, \emph{woman}, and \emph{dog}, which have a cyclic order. Any feasible triple must contain exactly one agent of each type. There are $n$ agents of each type and each agent has a preference list only over the agents of the next type. A matching is \emph{stable} if there exists no triple $t$ in which each agent in $t$ prefers the triple $t$ to their assigned triple in the matching. It was recently shown, contrary to previous conjectures, that a given instance of 3-DSM-CYC need not contain a stable matching and that the associated decision problem is $\NP$-complete~\cite{Plaxton3DSMCYCJournal}. We thus consider the approximability of a closely related optimisation problem, in which the objective is to construct a matching that is, in terms of a specific measure, as stable as possible. To our knowledge, this work is the first investigation into the approximability of 3-DSM-CYC. We first show that an existing algorithm for another three-dimensional matching problem, which is closely related to 3-DSM-CYC, can be used to construct a $9/4$-approximation algorithm. Improving this approximation, we then present a $6/5$-approximation algorithm based on serial dictatorship. We then consider a restriction of 3-DSM-CYC in which the preferences of all agents of at least one type are in some way similar, using a specific similarity measure. Specifically, we suppose the preference lists of all agents of at least one type are derived from a master preference list and consider the maximum \emph{Kendall tau distance} \cite{KendallTauCitation} between the master list and the list of any agent of that type. We show that if this distance is sufficiently small then as it is further reduced the approximation ratio decreases from $6/5$ to $1$.

Next, we define two new models of 3DR that involve so-called $\mathscr{B}$- and $\mathscr{W}$-preferences, which we name 3DR-B and 3DR-W. Using $\mathscr{B}$- ($\mathscr{W}$-) preferences, each agent has a strict preference list over the other agents and compares two triples based only on the most-preferred (least-preferred) member of each triple. We consider in both 3DR-B and 3DR-W the existence of matchings that are stable. We first show that it is $\NP$-complete to decide if a given instance (of agents and preferences) of either model contains a stable matching. Interestingly, this contrasts with the existence of polynomial-time algorithms in two analogous models in which coalitions may have any size \cite{CR01, CH04}. For both 3DR-B and 3DR-W, we also consider a closely related optimisation problem in which the objective is to construct a matching that is, in terms of a specific measure, as stable as possible. We show that an existing result leads to a $9/4$-approximation algorithm in both models and a simple algorithm based on serial dictatorship gives a $3/2$-approximation for 3DR-B.

We then formalise a model of 3DR with additively separable preferences, which we call 3DR-AS. In this model, each agent provides a numerical valuation of every other agent and compares triples based on the sum of the valuations of the other two agents in each triple. We investigate in 3DR-AS the existence of stable matchings as well as matchings that are envy-free, meaning there exists no pair of agents where the one would prefer to swap places with the other. In fact, we consider three successively weaker formalisms of this notion, namely envy-freeness, weakly justified envy-freeness, and justified envy-freeness. We consider the computational problems of deciding if such a matching exists, and constructing one if so. In particular, we study these problems in a setting where the agents' valuations are restricted. We consider various restrictions involving binary, ternary, and symmetric valuations. We provide a full complexity classification and identify dichotomies in terms of these restrictions. Interestingly, we identify a general trend that shows, for successively weaker solution concepts, existence and polynomial-time solvability hold under three successively weaker restrictions on the agents' preferences. 

Building on our new result that any instance of 3DR-AS with binary and symmetric preferences must contain a stable matching, we also consider a related optimisation problem in which the objective is to construct a stable matching in such an instance with maximum utilitarian welfare, i.e.\ the total sum of agents' valuations of their assigned partners is maximised. We devise a $2$-approximation algorithm for this optimisation problem.

Finally, we consider a problem in graph theory that generalises the notion of assigning agents to coalitions of a fixed size. Rather than partitioning a set of agents, the problem is to find a maximum-cardinality set of $r$-cliques in an undirected graph, subject to that set being either vertex disjoint or edge disjoint, for a fixed integer $r \geq 3$. This general problem is known as the \emph{$K_r$-packing problem}. Here we study the restriction of this problem in which the graph has a fixed maximum degree $\Delta$. It is known for $r=3$ that the vertex-disjoint (edge-disjoint) variant is solvable in linear time if $\Delta=3$ ($\Delta=4$) but $\APX$-hard if $\Delta \geq 4$ ($\Delta \geq 5$) \cite{caprara_packing_2002}. In other words, there exists some fixed constants $\varepsilon > 1$ and $\varepsilon' > 1$ such that no polynomial-time $\varepsilon$-approximation algorithm exists for the vertex-disjoint variant if $\Delta \geq 4$; and no polynomial-time $\varepsilon'$-approximation algorithm exists for the edge-disjoint variant if $\Delta \geq 5$. We generalise these results to an arbitrary but fixed $r \geq 3$, and provide a full complexity classification for both the vertex- and edge-disjoint variants in graphs of maximum degree $\Delta$, for every $r \geq 3$. Specifically, we show that the vertex-disjoint problem is solvable in linear time if $\Delta < 3r/2 - 1$, solvable in polynomial time if $\Delta < 5r/3 - 1$, and $\APX$-hard if $\Delta \geq \lceil 5r/3 \rceil - 1$. We also show that if $r\geq 6$ then these implications also hold for the edge-disjoint problem. If $r < 6$, then the edge-disjoint problem is solvable in linear time if $\Delta < 3r/2 - 1$, solvable in polynomial time if $\Delta \leq 2r - 2$, and $\APX$-hard if $\Delta > 2r - 2$.
